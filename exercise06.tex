\documentclass[11pt]{article}
%% \def\hidesolutions{}
%%%%%%%%%%% SET MARGINS
\setlength{\textheight}{20cm}
\setlength{\topmargin}{-0.5cm}
\setlength{\oddsidemargin}{+0cm}
\setlength{\textwidth}{16.3cm}
%\setlength{\parskip}{6pt}
\setlength{\parindent}{0pt}

%%%%%%%%%%% PACKAGES
\usepackage{amsmath}
\usepackage{amssymb}
\usepackage{amsfonts}
%\usepackage{a4wide}
\usepackage{graphicx}
\usepackage{color}
\usepackage[normalem]{ulem}
\usepackage{enumitem}
\usepackage{capt-of}
\usepackage{float}
\usepackage{amsmath}
\usepackage{listings}
\definecolor{mygreen}{RGB}{28,172,0} % color values Red, Green, Blue
\definecolor{mylilas}{RGB}{170,55,241}
\usepackage{empheq}
\usepackage[ruled]{algorithm2e}
\usepackage{mathrsfs}
\usepackage{datetime}
\usepackage{subcaption}

% TODO: combine the two package lists and reduce redundancies 
\usepackage{mathtools}
\usepackage{nicefrac}
\usepackage{hyperref}
\usepackage{url}
\usepackage{amsmath,amssymb,amsfonts}
\usepackage{a4wide}
\usepackage{graphicx}
\usepackage{color}
\usepackage[normalem]{ulem}
\usepackage{capt-of}
\usepackage{float}
\usepackage[ruled]{algorithm2e}
\usepackage{amsmath,amssymb,amsfonts}
\usepackage{a4wide}
\usepackage{graphicx}
\usepackage{color}
\usepackage[normalem]{ulem}
\usepackage{capt-of}
\usepackage{float}
\usepackage[ruled]{algorithm2e}
\usepackage{mathrsfs}







\newcommand{\Lc}[2]{{\color{blue} \sout{#1} } \textcolor{red}{#2}}
\newcommand{\La}[1]{\textcolor{red}{#1}}
\newcommand{\lh}{\mathscr{L}_h}
\newcommand{\cl}{\mathscr{L}}
\newcommand{\cf}{\mathscr{F}}
\newcommand{\dx}{dx}
\newcommand{\ltn}{\mathscr{l}^2}
\newcommand{\bbR}{\mathbb{R}}
\newcommand{\Rset}{\mathbb{R}}
\newcommand{\Nset}{\mathbb{N}}
\newcommand{\scL}{\mathcal{L}}
\newcommand{\xx}{\mathbf{x}}
\newcommand{\norm}[1]{\|{#1}\|}
\newcommand{\yy}{\mathbf{y}}
\newcommand{\at}[1]{\big|_{#1}}
\renewcommand{\div}{\mathrm{div}}
\newcommand{\divergence}{\mathrm{div}}
\newcommand{\cp}[1]{\textcolor{blue}{#1}}

\newcommand{\FF}{\texttt{FreeFem++ }}
\newcommand{\FFns}{\texttt{FreeFem++}}
\newcommand{\FFfull}{\texttt{FreeFem++-x11}}
\newcommand{\cmd}[1]{ \medskip \noindent \texttt{#1} \medskip}
\newcommand{\incmd}[1]{\texttt{#1}}
\newcommand{\shrinkitems}{\addtolength{\itemsep}{-0.5\baselineskip}}
\newcommand{\mtt}[1]{\mathtt{#1}}
\newcommand{\ML}{\texttt{Matlab }}

\newcommand{\bb}{\mathbf{b}}
\newcommand{\nn}{\mathbf{n}}
\newcommand{\vecA}{\vec{A}}
\newcommand{\vecB}{\vec{B}}


\newcommand{\mesh}{\mathcal{T}_h}
\newcommand{\refel}{\widehat{K}}
\newcommand{\ver}{\mathbf{a}}
\newcommand{\refver}{\widehat{\mathbf{a}}}
\newcommand{\grad}{\nabla}
\newcommand{\refgrad}{\widehat{\nabla}}
\newcommand{\refu}{\widehat{u}}
\newcommand{\refbasis}{\widehat{\varphi}}
\newcommand{\refxx}{\widehat{\xx}}
\newcommand{\refx}{\widehat{x}}
\newcommand{\refy}{\widehat{y}}
\newcommand{\refrho}{\widehat{\rho}}
\newcommand{\refh}{\widehat{h}}






% For typesetting Python code
\newcommand{\matlab}{{\sc Matlab}\xspace}
\usepackage{listings}
\lstloadlanguages{Python}
\lstloadlanguages{csh}%
\definecolor{MyDarkGreen}{rgb}{0.0,0.4,0.0}
\definecolor{purple}{rgb}{0.58,0,0.82}
\lstset{language=Python,                    % Use Python
	%frame=single,                          % Single frame around code
	basicstyle=\ttfamily\footnotesize\color{black},
	keywordstyle=[1]\color{blue}\bf,        % Python functions bold and blue
	keywordstyle=[2]\color{purple},         % Python function arguments purple
	keywordstyle=[3]\color{red}\underbar,   % User functions underlined and blue
	commentstyle=\usefont{T1}{pcr}{m}{sl}\color{MyDarkGreen}\small,
	stringstyle=\color{purple},
	showstringspaces=false,                 % Don't put marks in string spaces
	tabsize=3,                              % 5 spaces per tab
	morekeywords={xlim,ylim,var,alpha,factorial,poissrnd,normpdf,normcdf},
	morecomment=[l][\color{blue}]{...},
	breaklines=true,
	breakatwhitespace=true,
	emptylines=1,
	mathescape=true,
	xleftmargin=0ex,
	emphstyle=\bfseries\color{red}
}





%%%%%%%%%%% MACROS NAMES
\newcommand{\lecturername}{Martin Licht}
% \newcommand{\assistantnamea}{Jochen Hinz}
% \newcommand{\assistantnameb}{Ivan Bioli}
\newcommand{\semestername}{Winter Semester 2023}
\newcommand{\lecturename}{Analysis III - 202(c)}
\DeclarePairedDelimiter\floor{\lfloor}{\rfloor}

%%%%%%%%%%% HEADER
\newdateformat{yeardate}{\THEYEAR}
\newcommand{\exsheet}[3] % input is the number of the session and the day TODO What's that
{\clearpage

	\begin{center}
		{\Large \textbf{\lecturename}}\\[2ex]
		\semestername
	\end{center}

	% \vspace{2ex}
	% \lecturername

	\vspace{2ex}
	{\Large Session #1: #3\,#2, \yeardate\today}
	%\hfill
	%{\Large EPF Lausanne}

	\hrulefill
}





\usepackage{comment}

\newtheorem{exercise}{Exercise}
\newtheorem{solutionenv}{Solution}

\newboolean{hide_solution}
\ifx\hidesolutions\undefined
\newenvironment{solution}{\begin{solutionenv}}{\end{solutionenv}}
\setboolean{hide_solution}{false}
\else
\excludecomment{solution}
\setboolean{hide_solution}{true}
\fi

\newcommand{\ifnotsolution}[1]{\ifthenelse{\boolean{hide_solution}}{#1}{}}
\newcommand{\ifsolution}[1]{\ifthenelse{\boolean{hide_solution}}{}{#1}}








\allowdisplaybreaks

\begin{document}
\exsheet{6}{17}{October} % parameters are the number of the session and the day



\begin{exercise}
    We have the vector field 
    \begin{gather*}
        \vec F( x_1, x_2 ) = \left( x_1 x_2, x_2^{2} \right).
    \end{gather*}
    and the domains 
    \begin{align*}
        \Omega_1 &:= \left\{ (x_1,x_2) \in \bbR^{2} \suchthat* x_1^{2} + x_2^{2} < 1 \right\},
        \\
        \Omega_2 &:= \left\{ (x_1,x_2) \in \bbR^{2} \suchthat* x_1^{2} + x_2^{2} < 1, \; x_1 > 0 \right\},
     \end{align*}
    Verify Green's theorem for the vector field $\vec F$ with the domains $\Omega_1$ and $\Omega_2$. 
    \textit{You need to find parameterizations of the boundary first.}
\end{exercise}
\begin{solution}   
   Green's theorem over some domain $\Omega$ is given by 
   \begin{gather*}
        \int_{\partial \Omega} \vec{F} \cdot \;d\vec{s} 
        = 
        \int \int_{\Omega} \frac{\partial F_2}{\partial x_1} - \frac{\partial F_1}{\partial x_2} \;d \Omega
        .
    \end{gather*}
    Here, $\partial\Omega$ is a curve parameterized in counterclockwise direction. 
    
    We consider the domain $\Omega_1$. 
    A counterclockwise parameterisation for the boundary is given by 
    $$
        \gamma: [0,2\pi] \mapsto \mathbb{R}^2, \quad t\mapsto  (\cos t, \sin t)
        .
    $$
    \begin{align*}
        \int_{\partial \Omega} \vec{F} \cdot \;d\vec{s} 
        &= 
        \int_0^ {2\pi} 
        \begin{pmatrix} \cos t \sin t\\ \sin^2 t \end{pmatrix} 
        \cdot 
        \begin{pmatrix} -\sin t \\ \cos t \end{pmatrix} \; d t 
        \\&= 
        \int_0^{2\pi} -\sin^2 t\cos t + \sin^ 2 t \cos t \;dt = 0
    \end{align*}
    On the other hand, 
    \begin{align*}
        \int \int_{\Omega} \frac{\partial F_2}{\partial x_1} - \frac{\partial F_1}{\partial x_2} \;d \Omega &= \int \int_{\Omega} -x_1 \; d\Omega.
    \end{align*}
    There are different ways to compute the last integral. 
	For example, since $\Omega$ is symmetric along the $x_2$ axis but $-x_1$ is obviously an odd function,
	we can conclude that 
	\begin{align*}
        \int \int_{\Omega} -x_1 \; d\Omega = 0.
    \end{align*}
	Alternatively, we can use radial coordinates for the disk: 
    \begin{align*}
        \int \int_{\Omega} -x_1 \; d\Omega
		= 
		\int_0^ {2\pi} \int_0^1 -r\cos \theta r \;d\theta\;dr
        = 
		-\int_0^ {2\pi} \cos \theta \;d \theta \int_0^1 r^ 2 \;d r = 0,
    \end{align*}
    where we have used that integrating the cosine function over a period is $0$. 
	Yet another possibility is 
	\begin{align*}
        \int \int_{\Omega} -x_1 \; d\Omega
		= 
		-
		\int_{-1}^{1} \int_{-\sqrt{1-x_1^2}}^{\sqrt{1-x_1^2}}
		x_1 \;dx_2 \;dx_1
        = 
		\int_{-1}^{1} 
		2x_1 \sqrt{1-x_1^2} \;dx_1
        =
		0
		,
    \end{align*}
	where we have used that the last function is odd. 
	In any case, the integral of the domain and the integral along the boundary give the same value, 
	thereby verifying Green's theorem.
    
    We consider domain $\Omega_2$. A parameterisation for the boundary is given by two seperate curves. For the circular arc we have $\gamma: [-\frac{\pi}{2},\frac{\pi}{2}] \mapsto \mathbb{R}^2, \quad t\mapsto  (\cos t, \sin t)$ and for the vertical line we have  $\gamma: [1,-1] \mapsto \mathbb{R}^2, \quad t\mapsto  (0, t)$. 
    \begin{align*}
        \int_{\partial \Omega} \vec{F} \cdot \;d\vec{s} 
        &= 
        \int_{-\frac{\pi}{2}}^{\frac{\pi}{2}} 
        \begin{pmatrix}\sin t \cos t\\ -\sin^2 t \end{pmatrix} 
        \cdot 
        \begin{pmatrix}-\sin t \\ \cos t \end{pmatrix} \; d t 
        + 
        \int_1^ {-1} 
        \begin{pmatrix} 0 \\ t^2 \end{pmatrix}
        \cdot 
        \begin{pmatrix} 0 \\ 1 \end{pmatrix}
        \\&
        = 
        \int_0^{2\pi} -\sin^2 t\cos t + \sin^ 2 t \cos t \;dt + \int_{1}^ {-1} t^ 2 \;d t
        \\&
        = \left[\frac{1}{3} t^ 3\right]_1^ {-1}
        = -\frac{2}{3}
        .
    \end{align*}
    Thus, 
    \begin{align*}
        \int \int_{\Omega} \frac{\partial F_2}{\partial x_1} - \frac{\partial F_1}{\partial x_2} \;d \Omega &= \int \int_{\Omega} -x_1 \; d\Omega
        \\&= \int_{-\frac{\pi}{2}}^ {\frac{\pi}{2}} \int_0^1 -r\cos \theta r \;d\theta\;dr
        \\&= -\int_{-\frac{\pi}{2}}^ {\frac{\pi}{2}} \cos \theta \;d \theta \int_0^1 r^ 2 \;d r
        = \left[\sin \theta\right]_{-\frac{\pi}{2}}^ {\frac{\pi}{2}} \left[-\frac{1}{3}r^ 3\right]_0^{1} = -\frac{2}{3}
        .
    \end{align*}
    Both integrals give the same value thereby verifying Green's theorem.
\end{solution}





\begin{exercise}
    Consider the triangle domain 
    \begin{gather*}
        T := \left\{ (x_1,x_2) \in \bbR^{2} \suchthat* x_1 > 0, \; x_2 > 0, \; x_1+x_2 < 1 \right\}.
    \end{gather*}
    and the vector field 
    \begin{gather*}
        \vec F(x_1,x_2) = \left( x_1x_2 + \frac{x_1}{1+x_1^2+x_2^2}, x_1x_2 + \frac{x_2}{1+x_1^2+x_2^2} \right)
    \end{gather*}
    Find the curve integral of $\vec F$ along the boundary of $T$ using Green's theorem.
\end{exercise}
\begin{solution}
\begin{align*}
    \int_{\partial \Omega} \vec{F} \cdot \; d \vec{s}
    &= 
    \int\int_{\Omega} \frac{\partial F_2}{\partial x_1} - \frac{\partial F_1}{\partial x_2}\; d \Omega
    \\
    &= 
    \int\int_{\Omega} x_2 + \frac{-2x_1x_2}{(1+x_1^ 2 + x_2^ 2)^2} - x_1 - \frac{-2x_1x_2}{(1+x_1^ 2 + x_2^ 2)^2} \; d \Omega
    \\
    &= 
    \int\int_{\Omega} x_2  - x_1\; d \Omega
    \\
    &= 
    \int_0^ 1\int_0^ {1-x_1} x_2 - x_1 \;dx_2\;dx_1
    \\
    &= 
    \int_0^ 1 \frac{1}{2}(1-x_1)^ 2 - x_1(1-x_1) \;d x
    \\
    &= 
    \left[ -\frac{1}{6}(1-x_1)^ 3 - \frac{1}{2} x_1^2 + \frac{1}{3}x_1^ 3 \right]_0^1 = -\frac{1}{2} + \frac{1}{3} - \frac{-1}{6} = 0
\end{align*}
\end{solution}
















\begin{exercise}
    Consider the parabolic arc 
    \begin{gather*}
        \Gamma := \left\{ (x_1,x_2) \in \bbR^{2} \suchthat* -1 < x_1 < 1, \; x_2 = 3 ( 1 - x_1^2 ) \right\}.
    \end{gather*}
    Find the curve integral $\int_\Gamma \vec F \cdot \vec{n} \;dl$, where 
    \begin{gather*}
        \vec F(x_1,x_2) 
        = 
        \left( 
            x_1 ( 2 - \cos(x_1x_2)^{2} )
            , 
            x_2 ( 2 + \cos(x_1x_2)^{2} ) 
        \right)
    \end{gather*}
    and where $\vec{n}$ is the unit vector along $\Gamma$, perpendicular to $\Gamma$ and having non-negative $x_2$ component.
\end{exercise}
\begin{solution}
    % Draw a line between arc endpoints to enclose a domain.
    % The divergence of the vector field is constant.
    % Compute the integral along the bottom line and the integrate the divergence over the domain
    % We just need the area of the domain 
    The Idea is to make use of divergence theorem. 
    We do this by closing the curve $\gamma$ by introducing the curve 
    \begin{gather*}
        \delta: [-1,1] \mapsto \mathbb{R}^2, \quad t \mapsto  (t,0).
    \end{gather*}
    This allows us to reformulate the problem as follows:
    \begin{gather*}
        \int_{\Gamma} \vec{F} \cdot \vec{n} \; d\ell
        =
        \int_{\Gamma\cup \Delta} \vec{F} \cdot \vec{n} \; d\ell
        -
        \int_{\Delta} \vec{F}\cdot \vec{n} \;d \ell
    \end{gather*}
    For the first integral on the right-hand side we use the divergence theorem
    \begin{align*}
        \int_{\Gamma\cup\Delta} \vec{F} \cdot \vec{n} \; d\ell 
        = 
        \int_{\Omega} \nabla\cdot\vec{F}\;d \Omega
        =
        \int_{\Omega} 4\;d \Omega
        .
    \end{align*}
    We need to compute the area of the domain $\Omega$ and proceed as follows: 
    \begin{align*}
        4\int_{\Omega} 1\;d \Omega
        &=
        4 \int_{-1}^1\int_{0}^{3(1-x_1^2)} 1 \;d x_2 \;d x_1
        \\&=
        4 \int_{-1}^1\left[x_2\right]_{0}^{3(1-x_1^2)} \;d x_1
        \\&=
        4 \int_{-1}^1 3( 1-x_1^2)\;d x_1
        \\&=
        4 \left[ 3 x_1 - x_1^3 \right]_{-1}^1
        =
        4 \left( \left[ 3 - 1 \right] - \left[ -3 + 1 \right] \right)
        = 
        16
        .
    \end{align*}
    The value of the second integral is given by the integral of $\vec{F}$ against the vector $(0,-1)$:
    \begin{align*}
        \int_{\Delta} \vec{F} \cdot \binom{0}{-1} \; dl
        =
        \int_{-1}^1 0 \; dx_1
        =
        0
        .
    \end{align*}
    We conclude that:
    \begin{align*}
        \int_{\Gamma} \vec{F} \cdot \vec{n} \; dl
        = 
        \int_{\Gamma\cup \Delta} \vec{F} \cdot \vec{n} \; dl
        =
        \left[ 12 x_1 - 4 x_1^3 \right]_{-1}^1
        =
        ( 12 - 4 ) - ( -12 + 4 )
        = 
        16.
    \end{align*}
\end{solution}

\begin{exercise}
    Find the tangential vector $\dot\gamma(t)$, the unit tangential vector $\vec\tau$ and the unit normal $\vec n$ of the simple closed curve 
    \begin{align*}
        \gamma : [0,2\pi] \to \bbR^{2}, \quad t \mapsto ( \cos(t), \sin(t)( 1 + \sin(2t)^2 ) ).
    \end{align*}
    Find the values of $\gamma$ and $\vec\tau$ for a few values of $t \in [0,2\pi]$, such as $t = \frac{\pi}{4}, 2 \frac{\pi}{4}, 3 \frac{\pi}{4}, \dots, 7 \frac{\pi}{4}$
\end{exercise}
\begin{solution}
    Some standard calculations show that 
    \begin{align*}
        \dot\gamma(t) 
        &= 
        \left( \sin(t), \cos(t) + \cos(t) \sin(2t)^{2} + \sin(t) 2 \sin(2t) \cos(2t) 2 \right)
        \\&= 
        \left( \sin(t), \cos(t) + \cos(t) \sin(2t)^{2} + 4 \sin(t) \sin(2t) \cos(2t) \right)
    \end{align*}
    Hence,{\scriptsize
    \begin{align*}
        &
        |\dot\gamma(t)|
        \\&=
        \sqrt{
            \sin(t)^{2}
            +
            \cos(t)^{2} + \cos(t)^{2} \sin(2t)^{4} + 16 \sin(t)^{2} \sin(2t)^{2} \cos(2t)^{2}
            +
            2\cos(t) \sin(2t)^{2} + 2\cos(t) 4 \sin(t) \sin(2t) \cos(2t) + 2 \sin(2t)^{2} 4 \sin(t) \sin(2t) \cos(2t)
        }
        \\&
        =
        \sqrt{
            \frac{1}{16} (6 \cos(t)+3 \cos(3 t)-5 \cos(5t))^2+\sin(t)^2
        }
    \end{align*}
    }

    \begin{tikzpicture}
        \begin{axis}[
            domain=0:360,
            xmin=-2,
            xmax=2,
            ymin=-2,
            ymax=2,
            xlabel={x},
            ylabel={y}
        ]
        \addplot[blue, variable=\t, samples=100] ({cos(\t)}, {sin(\t)* ( 1 + 0.5 * sin(2*\t)*sin(2*\t) )});
        \end{axis}
    \end{tikzpicture}
\end{solution}





\begin{exercise}
    Find the area of the graph of $\phi$ over $\Omega = [0,1] \times [0,1]$, where 
    \begin{gather*}
        \phi(s,t) := \sqrt{ s^2 + t^2 }
    \end{gather*}
    Find the integral of the function 
    \[
        f(x_1,x_2,x_3) := x_1 x_2 x_3
    \]
    over the graph of $\phi$ over $\Omega$. 
\end{exercise}
\begin{solution}
    We compute the partial derivatives of the function:
    \[
        \partial_s \phi(s,t) = s ( s^2 + t^2 )^{- \frac 1 2},
        \quad 
        \partial_t \phi(s,t) = t ( s^2 + t^2 )^{- \frac 1 2}.
    \]
    The area of the graph over $\Omega$ is given by the integral 
    \[
        \int_0^1 \int_0^1 \sqrt{ 1 + \frac{ s^2 }{ s^2 + t^2 } + \frac{ t^2 }{ s^2 + t^2 } } \;dsdt = \sqrt{2}.
    \]
    The integral of $f$ over the graph $S$ reads as follows:
    \begin{align*}
        \iint_S f \;d\sigma
        &=
        \int_0^1 \int_0^1 s \cdot t \cdot \sqrt{ s^2 + t^2 } \cdot \sqrt{ 1 + \frac{ s^2 }{ s^2 + t^2 } + \frac{ t^2 }{ s^2 + t^2 } } \;dsdt
        \\&=
        \sqrt{2}
        \int_0^1 \int_0^1 s \cdot t \cdot \sqrt{ s^2 + t^2 } \;ds \;dt.
    \end{align*}
    We simplify it further:
    \begin{align*}
        &
        \int_0^1 \int_0^1 s \cdot t \cdot \left( s^2 + t^2 \right)^{\frac 1 2} \;ds \;dt
        \\&
        =
        \int_0^1 t \cdot \frac 1 3 \int_0^1 \partial_s ( s^2 + t^2 )^{\frac 3 2} \;ds \;dt
        \\&
        =
        \int_0^1 t \cdot \frac 1 3 \left[ ( s^2 + t^2 )^{\frac 3 2} \right]_{s=0}^{s=1} \;dt
        \\&
        =
        \int_0^1 t \cdot \frac 1 3 \left( ( 1 + t^2 )^{\frac 3 2} - t^{3} \right) \;dt
        =
        \frac 1 3 
        \int_0^1 t \cdot ( 1 + t^2 )^{\frac 3 2} - t^{4} \;dt
    \end{align*}
    Next, 
    \begin{align*}
        &
        \int_0^1 t \cdot ( 1 + t^2 )^{\frac 3 2} - t^{4} \;dt
        \\&
        =
        \int_0^1 
        \frac 1 5 \partial_t \left( \left( 1 + t^2 \right)^{\frac 5 2} \right) - \partial_t \left( \frac 1 5 t^{5} \right) 
        \;dt
        \\&
        =
        \frac 1 5
        \left[ \left( 1 + t^2 \right)^{\frac 5 2} \right]_{t=0}^{t=1}
        -
        \frac 1 5
        \left[ t^5 \right]_{t=0}^{t=1}
        =
        \frac 1 5 \left( \sqrt{2}^{5} - 1 - 1 \right)
        .
    \end{align*}
    In summary, 
    \begin{align*}
        \iint_S f \;d\sigma
        &=
        \sqrt{2}
        \cdot 
        \frac 1 3 
        \cdot 
        \frac 1 5 \left( \sqrt{2}^{5} - 1 - 1 \right)
        =
        \frac{\sqrt 2}{15}
        \cdot 
        \left( \sqrt{2}^{5} - 2 \right)
        =
        \frac{2}{15}
        \cdot 
        \left( 4 - \sqrt 2 \right)
        .
    \end{align*}
\end{solution}




\begin{exercise}
    The parameterization 
    \begin{align*}
        \Phi : [0,2\pi) \times (0,1) \rightarrow \bbR^{3}, 
        \quad 
        (\theta,z) \mapsto \left( ( 1 + z ) \cos(\theta), ( 1 + z ) \sin(\theta), z \right)
    \end{align*}
    describes a surface $S$. Find the surface area of $S$.
\end{exercise}
\begin{solution}
    We compute the derivatives of $\Phi$:
    \begin{align*}
        \partial_{\theta} \Phi(\theta,z)
        :=
        \begin{pmatrix}
         - ( 1 + z ) \sin(\theta)
         \\
         ( 1 + z ) \cos(\theta)
         \\
         0
        \end{pmatrix}
        \quad 
        \partial_{z} \Phi(\theta,z)
        :=
        \begin{pmatrix}
         \cos(\theta)
         \\
         \sin(\theta)
         \\
         1
        \end{pmatrix}
        .
    \end{align*}
    The cross product $\partial_{\theta} \Phi(\theta,z) \times \partial_{z} \Phi(\theta,z)$ equals 
    \begin{align*}
        \begin{pmatrix}
         - ( 1 + z ) \sin(\theta)
         \\
         ( 1 + z ) \cos(\theta)
         \\
         0
        \end{pmatrix}
        \times
        \begin{pmatrix}
         \cos(\theta)
         \\
         \sin(\theta)
         \\
         1
        \end{pmatrix}
        =
        \begin{pmatrix}
         ( 1 + z ) \cos(\theta)
         \\
         ( 1 + z ) \sin(\theta)
         \\
         - ( 1 + z )
        \end{pmatrix}
    \end{align*}
    The norm of cross product is 
    \begin{align*}
     \| \partial_{\theta} \Phi(\theta,z) \times \partial_{z} \Phi(\theta,z) \|
     &
     =
     \| \left( ( 1 + z ) \cos(\theta), ( 1 + z ) \sin(\theta), -( 1 + z ) \right) \|
     \\&
     =
     \sqrt{ ( 1 + z )^{2} \cos^{2}(\theta) + ( 1 + z )^{2} \sin^{2}(\theta) + ( 1 + z )^{2} }
     \\&
     =
     \sqrt{ 2 ( 1 + z )^{2} }
     \\&
     =
     \sqrt{ 2 } \cdot ( 1 + z )
     .
    \end{align*}
    We can now compute the surface area:
    \begin{align*}
        \iint_{S} 1 d\sigma
        &=
        \int_{0}^{1} \int_{0}^{2\pi} 1 \cdot \sqrt{2} ( 1 + z ) d\theta dz
        \\&
        =
        2\pi \cdot \sqrt{2} \cdot \left[ z + \frac 1 2 z^{2} \right]_{z=0}^{z=1}
        \\&
        =
        2\pi \cdot \sqrt{2} \cdot \frac 3 2 
        \\&
        =
        3\sqrt{2} \cdot \pi
        .
    \end{align*}
\end{solution}








\begin{exercise}
    Let $f(x_1,x_2,x_3) = x_1 x_2 + x_3^2$ and consider the surface 
    \begin{gather*}
        S := \left\{ \; (x_1,x_2,x_3) \in \mathbb R^3 \suchthat* 0 < x_3 < 1, \; x_1^2 + x_2^2 = x_3^2 \;  \right\}
    \end{gather*}
    Find a parameterization of $S$ and compute the surface integral $\iint_S f \;d\sigma$. 
\end{exercise}
\begin{solution}
    We define the following parameterisation for the surface S:
    \[
    \Phi: [0,2\pi) \times (0,1) \mapsto S: (\theta,z)\mapsto (z\cos\theta, z\sin\theta,z)
    \]
    with
    \[
        \left\|\partial_{\theta}\Phi\times \partial_z\Phi\right\|  
        = 
        \left\|\begin{pmatrix} -z\sin\theta \\ z\cos\theta\\0 \end{pmatrix} \times \begin{pmatrix} \cos\theta \\ \sin\theta \\1 \end{pmatrix}\right\|
        = 
        \left\|\begin{pmatrix} z\cos\theta \\ z\sin\theta \\ -z \end{pmatrix}\right\| 
        = 
        \sqrt{2}z
    \]
    Now the surface integral is given by:
    \begin{align*}
        \iint_{S} f \;d S
        &
        = \int_0^{2\pi} \int_0^1 \left(z^2\cos\theta \sin\theta + z^2 \right)\sqrt{2}z\;d z \;d \theta
        \\&
        = \int_0^{2\pi} \int_0^1\sqrt{2}z^{3} \;d z \;d \theta + \int_0^{2\pi} \cos\theta \sin\theta \;d \theta \int_0^1  \sqrt{2}z^{3}\;d z 
        \\&
        =
        2\pi \left[\frac{\sqrt{2}}{4} z^{4}\right]_0^{1} + \int_0^{2\pi} \frac{1}{2}\sin2\theta \;d \theta 2\pi \left[\frac{\sqrt{2}}{4} z^{4}\right]_0^{1}
        \\&
        =
        \pi \frac{2\sqrt{2}}{4}+ \left[-\frac{1}{4}\cos2\theta\right]_{0}^{2\pi} \pi \frac{2\sqrt{2}}{4} 
        \\&
        = 
        \pi \frac{2\sqrt{2}}{4}+ \left(-\frac{1}{4} -  - \frac{1}{4}\right) \pi\frac{2\sqrt{2}}{4}
        = 
        \pi\frac{\sqrt{2}}{2}
        .
    \end{align*}
\end{solution}














































\end{document}
