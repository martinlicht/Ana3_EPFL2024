\documentclass[11pt]{article}
% \def\hidesolutions{}
%%%%%%%%%%% SET MARGINS
\setlength{\textheight}{20cm}
\setlength{\topmargin}{-0.5cm}
\setlength{\oddsidemargin}{+0cm}
\setlength{\textwidth}{16.3cm}
%\setlength{\parskip}{6pt}
\setlength{\parindent}{0pt}

%%%%%%%%%%% PACKAGES
\usepackage{amsmath}
\usepackage{amssymb}
\usepackage{amsfonts}
%\usepackage{a4wide}
\usepackage{graphicx}
\usepackage{color}
\usepackage[normalem]{ulem}
\usepackage{enumitem}
\usepackage{capt-of}
\usepackage{float}
\usepackage{amsmath}
\usepackage{listings}
\definecolor{mygreen}{RGB}{28,172,0} % color values Red, Green, Blue
\definecolor{mylilas}{RGB}{170,55,241}
\usepackage{empheq}
\usepackage[ruled]{algorithm2e}
\usepackage{mathrsfs}
\usepackage{datetime}
\usepackage{subcaption}

% TODO: combine the two package lists and reduce redundancies 
\usepackage{mathtools}
\usepackage{nicefrac}
\usepackage{hyperref}
\usepackage{url}
\usepackage{amsmath,amssymb,amsfonts}
\usepackage{a4wide}
\usepackage{graphicx}
\usepackage{color}
\usepackage[normalem]{ulem}
\usepackage{capt-of}
\usepackage{float}
\usepackage[ruled]{algorithm2e}
\usepackage{amsmath,amssymb,amsfonts}
\usepackage{a4wide}
\usepackage{graphicx}
\usepackage{color}
\usepackage[normalem]{ulem}
\usepackage{capt-of}
\usepackage{float}
\usepackage[ruled]{algorithm2e}
\usepackage{mathrsfs}







\newcommand{\Lc}[2]{{\color{blue} \sout{#1} } \textcolor{red}{#2}}
\newcommand{\La}[1]{\textcolor{red}{#1}}
\newcommand{\lh}{\mathscr{L}_h}
\newcommand{\cl}{\mathscr{L}}
\newcommand{\cf}{\mathscr{F}}
\newcommand{\dx}{dx}
\newcommand{\ltn}{\mathscr{l}^2}
\newcommand{\bbR}{\mathbb{R}}
\newcommand{\Rset}{\mathbb{R}}
\newcommand{\Nset}{\mathbb{N}}
\newcommand{\scL}{\mathcal{L}}
\newcommand{\xx}{\mathbf{x}}
\newcommand{\norm}[1]{\|{#1}\|}
\newcommand{\yy}{\mathbf{y}}
\newcommand{\at}[1]{\big|_{#1}}
\renewcommand{\div}{\mathrm{div}}
\newcommand{\divergence}{\mathrm{div}}
\newcommand{\cp}[1]{\textcolor{blue}{#1}}

\newcommand{\FF}{\texttt{FreeFem++ }}
\newcommand{\FFns}{\texttt{FreeFem++}}
\newcommand{\FFfull}{\texttt{FreeFem++-x11}}
\newcommand{\cmd}[1]{ \medskip \noindent \texttt{#1} \medskip}
\newcommand{\incmd}[1]{\texttt{#1}}
\newcommand{\shrinkitems}{\addtolength{\itemsep}{-0.5\baselineskip}}
\newcommand{\mtt}[1]{\mathtt{#1}}
\newcommand{\ML}{\texttt{Matlab }}

\newcommand{\bb}{\mathbf{b}}
\newcommand{\nn}{\mathbf{n}}
\newcommand{\vecA}{\vec{A}}
\newcommand{\vecB}{\vec{B}}


\newcommand{\mesh}{\mathcal{T}_h}
\newcommand{\refel}{\widehat{K}}
\newcommand{\ver}{\mathbf{a}}
\newcommand{\refver}{\widehat{\mathbf{a}}}
\newcommand{\grad}{\nabla}
\newcommand{\refgrad}{\widehat{\nabla}}
\newcommand{\refu}{\widehat{u}}
\newcommand{\refbasis}{\widehat{\varphi}}
\newcommand{\refxx}{\widehat{\xx}}
\newcommand{\refx}{\widehat{x}}
\newcommand{\refy}{\widehat{y}}
\newcommand{\refrho}{\widehat{\rho}}
\newcommand{\refh}{\widehat{h}}






% For typesetting Python code
\newcommand{\matlab}{{\sc Matlab}\xspace}
\usepackage{listings}
\lstloadlanguages{Python}
\lstloadlanguages{csh}%
\definecolor{MyDarkGreen}{rgb}{0.0,0.4,0.0}
\definecolor{purple}{rgb}{0.58,0,0.82}
\lstset{language=Python,                    % Use Python
	%frame=single,                          % Single frame around code
	basicstyle=\ttfamily\footnotesize\color{black},
	keywordstyle=[1]\color{blue}\bf,        % Python functions bold and blue
	keywordstyle=[2]\color{purple},         % Python function arguments purple
	keywordstyle=[3]\color{red}\underbar,   % User functions underlined and blue
	commentstyle=\usefont{T1}{pcr}{m}{sl}\color{MyDarkGreen}\small,
	stringstyle=\color{purple},
	showstringspaces=false,                 % Don't put marks in string spaces
	tabsize=3,                              % 5 spaces per tab
	morekeywords={xlim,ylim,var,alpha,factorial,poissrnd,normpdf,normcdf},
	morecomment=[l][\color{blue}]{...},
	breaklines=true,
	breakatwhitespace=true,
	emptylines=1,
	mathescape=true,
	xleftmargin=0ex,
	emphstyle=\bfseries\color{red}
}





%%%%%%%%%%% MACROS NAMES
\newcommand{\lecturername}{Martin Licht}
% \newcommand{\assistantnamea}{Jochen Hinz}
% \newcommand{\assistantnameb}{Ivan Bioli}
\newcommand{\semestername}{Winter Semester 2023}
\newcommand{\lecturename}{Analysis III - 202(c)}
\DeclarePairedDelimiter\floor{\lfloor}{\rfloor}

%%%%%%%%%%% HEADER
\newdateformat{yeardate}{\THEYEAR}
\newcommand{\exsheet}[3] % input is the number of the session and the day TODO What's that
{\clearpage

	\begin{center}
		{\Large \textbf{\lecturename}}\\[2ex]
		\semestername
	\end{center}

	% \vspace{2ex}
	% \lecturername

	\vspace{2ex}
	{\Large Session #1: #3\,#2, \yeardate\today}
	%\hfill
	%{\Large EPF Lausanne}

	\hrulefill
}





\usepackage{comment}

\newtheorem{exercise}{Exercise}
\newtheorem{solutionenv}{Solution}

\newboolean{hide_solution}
\ifx\hidesolutions\undefined
\newenvironment{solution}{\begin{solutionenv}}{\end{solutionenv}}
\setboolean{hide_solution}{false}
\else
\excludecomment{solution}
\setboolean{hide_solution}{true}
\fi

\newcommand{\ifnotsolution}[1]{\ifthenelse{\boolean{hide_solution}}{#1}{}}
\newcommand{\ifsolution}[1]{\ifthenelse{\boolean{hide_solution}}{}{#1}}








\allowdisplaybreaks

\begin{document}
\exsheet{6}{24}{October} % parameters are the number of the session and the day



\begin{exercise}
    Find the area of the graph of $\phi$ over $\Omega = [0,1] \times [0,1]$, where 
    \begin{gather*}
        \phi(s,t) := \sqrt{ s^2 + t^2 }
    \end{gather*}
    Find the integral of the function 
    \[
        f(x_1,x_2,x_3) := x_1 x_2 x_3
    \]
    over the graph of $\phi$ over $\Omega$. 
\end{exercise}
\begin{solution}
    We compute the partial derivatives of the function:
    \[
        \partial_s \phi(s,t) = s ( s^2 + t^2 )^{- \frac 1 2},
        \quad 
        \partial_t \phi(s,t) = t ( s^2 + t^2 )^{- \frac 1 2}.
    \]
    The area of the graph over $\Omega$ is given by the integral 
    \[
        \int_0^1 \int_0^1 \sqrt{ 1 + \frac{ s^2 }{ s^2 + t^2 } + \frac{ t^2 }{ s^2 + t^2 } } \;dsdt = \sqrt{2}.
    \]
    The integral of $f$ over the graph $S$ reads as follows:
    \begin{align*}
        \iint_S f \;d\sigma
        &=
        \int_0^1 \int_0^1 s \cdot t \cdot \sqrt{ s^2 + t^2 } \cdot \sqrt{ 1 + \frac{ s^2 }{ s^2 + t^2 } + \frac{ t^2 }{ s^2 + t^2 } } \;dsdt
        \\&=
        \sqrt{2}
        \int_0^1 \int_0^1 s \cdot t \cdot \sqrt{ s^2 + t^2 } \;ds \;dt.
    \end{align*}
    We simplify it further:
    \begin{align*}
        &
        \int_0^1 \int_0^1 s \cdot t \cdot \left( s^2 + t^2 \right)^{\frac 1 2} \;ds \;dt
        \\&
        =
        \int_0^1 t \cdot \frac 1 3 \int_0^1 \partial_s ( s^2 + t^2 )^{\frac 3 2} \;ds \;dt
        \\&
        =
        \int_0^1 t \cdot \frac 1 3 \left[ ( s^2 + t^2 )^{\frac 3 2} \right]_{s=0}^{s=1} \;dt
        \\&
        =
        \int_0^1 t \cdot \frac 1 3 \left( ( 1 + t^2 )^{\frac 3 2} - t^{3} \right) \;dt
        =
        \frac 1 3 
        \int_0^1 t \cdot ( 1 + t^2 )^{\frac 3 2} - t^{4} \;dt
    \end{align*}
    Next, 
    \begin{align*}
        &
        \int_0^1 t \cdot ( 1 + t^2 )^{\frac 3 2} - t^{4} \;dt
        \\&
        =
        \int_0^1 
        \frac 1 5 \partial_t \left( \left( 1 + t^2 \right)^{\frac 5 2} \right) - \partial_t \left( \frac 1 5 t^{5} \right) 
        \;dt
        \\&
        =
        \frac 1 5
        \left[ \left( 1 + t^2 \right)^{\frac 5 2} \right]_{t=0}^{t=1}
        -
        \frac 1 5
        \left[ t^5 \right]_{t=0}^{t=1}
        =
        \frac 1 5 \left( \sqrt{2}^{5} - 1 - 1 \right)
        .
    \end{align*}
    In summary, 
    \begin{align*}
        \iint_S f \;d\sigma
        &=
        \sqrt{2}
        \cdot 
        \frac 1 3 
        \cdot 
        \frac 1 5 \left( \sqrt{2}^{5} - 1 - 1 \right)
        =
        \frac{\sqrt 2}{15}
        \cdot 
        \left( \sqrt{2}^{5} - 2 \right)
        =
        \frac{2}{15}
        \cdot 
        \left( 4 - \sqrt 2 \right)
        .
    \end{align*}
\end{solution}




\begin{exercise}
    The parameterization 
    \begin{align*}
        \Phi : [0,2\pi) \times (0,1) \rightarrow \bbR^{3}, 
        \quad 
        (\theta,z) \mapsto \left( ( 1 + z ) \cos(\theta), ( 1 + z ) \sin(\theta), z \right)
    \end{align*}
    describes a surface $S$. Find the surface area of $S$.
\end{exercise}
\begin{solution}
    We compute the derivatives of $\Phi$:
    \begin{align*}
        \partial_{\theta} \Phi(\theta,z)
        :=
        \begin{pmatrix}
         - ( 1 + z ) \sin(\theta)
         \\
         ( 1 + z ) \cos(\theta)
         \\
         0
        \end{pmatrix}
        \quad 
        \partial_{z} \Phi(\theta,z)
        :=
        \begin{pmatrix}
         \cos(\theta)
         \\
         \sin(\theta)
         \\
         1
        \end{pmatrix}
        .
    \end{align*}
    The cross product $\partial_{\theta} \Phi(\theta,z) \times \partial_{z} \Phi(\theta,z)$ equals 
    \begin{align*}
        \begin{pmatrix}
         - ( 1 + z ) \sin(\theta)
         \\
         ( 1 + z ) \cos(\theta)
         \\
         0
        \end{pmatrix}
        \times
        \begin{pmatrix}
         \cos(\theta)
         \\
         \sin(\theta)
         \\
         1
        \end{pmatrix}
        =
        \begin{pmatrix}
         ( 1 + z ) \cos(\theta)
         \\
         ( 1 + z ) \sin(\theta)
         \\
         - ( 1 + z )
        \end{pmatrix}
    \end{align*}
    The norm of cross product is 
    \begin{align*}
     \| \partial_{\theta} \Phi(\theta,z) \times \partial_{z} \Phi(\theta,z) \|
     &
     =
     \| \left( ( 1 + z ) \cos(\theta), ( 1 + z ) \sin(\theta), -( 1 + z ) \right) \|
     \\&
     =
     \sqrt{ ( 1 + z )^{2} \cos^{2}(\theta) + ( 1 + z )^{2} \sin^{2}(\theta) + ( 1 + z )^{2} }
     \\&
     =
     \sqrt{ 2 ( 1 + z )^{2} }
     \\&
     =
     \sqrt{ 2 } \cdot ( 1 + z )
     .
    \end{align*}
    We can now compute the surface area:
    \begin{align*}
        \iint_{S} 1 d\sigma
        &=
        \int_{0}^{1} \int_{0}^{2\pi} 1 \cdot \sqrt{2} ( 1 + z ) d\theta dz
        \\&
        =
        2\pi \cdot \sqrt{2} \cdot \left[ z + \frac 1 2 z^{2} \right]_{z=0}^{z=1}
        \\&
        =
        2\pi \cdot \sqrt{2} \cdot \frac 3 2 
        \\&
        =
        3\sqrt{2} \cdot \pi
        .
    \end{align*}
\end{solution}








\begin{exercise}
    Let $f(x_1,x_2,x_3) = x_1 x_2 + x_3^2$ and consider the surface 
    \begin{gather*}
        S := \left\{ \; (x_1,x_2,x_3) \in \mathbb R^3 \suchthat* 0 < x_3 < 1, \; x_1^2 + x_2^2 = x_3^2 \;  \right\}
    \end{gather*}
    Find a parameterization of $S$ and compute the surface integral $\iint_S f \;d\sigma$. 
\end{exercise}
\begin{solution}
    We define the following parameterisation for the surface S:
    \[
    \Phi: [0,2\pi) \times (0,1) \mapsto S: (\theta,z)\mapsto (z\cos\theta, z\sin\theta,z)
    \]
    with
    \[
        \left\|\partial_{\theta}\Phi\times \partial_z\Phi\right\|  
        = 
        \left\|\begin{pmatrix} -z\sin\theta \\ z\cos\theta\\0 \end{pmatrix} \times \begin{pmatrix} \cos\theta \\ \sin\theta \\1 \end{pmatrix}\right\|
        = 
        \left\|\begin{pmatrix} z\cos\theta \\ z\sin\theta \\ -z \end{pmatrix}\right\| 
        = 
        \sqrt{2}z
    \]
    Now the surface integral is given by:
    \begin{align*}
        \iint_{S} f \;d S
        &
        = \int_0^{2\pi} \int_0^1 \left(z^2\cos\theta \sin\theta + z^2 \right)\sqrt{2}z\;d z \;d \theta
        \\&
        = \int_0^{2\pi} \int_0^1\sqrt{2}z^{2\frac{1}{2}} \;d z \;d \theta + \int_0^{2\pi} \cos\theta \sin\theta \;d \theta \int_0^1  \sqrt{2}z^{2\frac{1}{2}}\;d z 
        \\&
        =
        2\pi \left[\frac{\sqrt{2}}{3\frac{1}{2}} z^{2\frac{1}{2}}\right]_0^{1} + \int_0^{2\pi} \frac{1}{2}\sin2\theta \;d \theta 2\pi \left[\frac{\sqrt{2}}{3\frac{1}{2}} z^{2\frac{1}{2}}\right]_0^{1}
        \\&
        =
        \pi \frac{4\sqrt{2}}{7}+ \left[-\frac{1}{4}\cos2\theta\right]_{0}^{2\pi} \pi \frac{4\sqrt{2}}{7} 
        \\&
        = 
        \pi \frac{4\sqrt{2}}{7}+ \left(-\frac{1}{4} -  - \frac{1}{4}\right) \pi \frac{4\sqrt{2}}{7} 
        = 
        \frac{4\sqrt{2}}{7} 
        .
    \end{align*}
\end{solution}












\begin{exercise}
    Consider the volume  
    \begin{gather*}
        V := \left\{\; (x_1,x_2,x_3) \in \mathbb R^3 \suchthat* x_1^2 + x_2^2 < x_3 < 1 \;\right\}
    \end{gather*}
    Find the the boundary $S$ of this volume and compute its surface area. 
\end{exercise}
\begin{solution} 
    The boundary of this domain consist of two surfaces:
    \begin{align*}
        S_a =\left\{ (x_1,x_2,x_3) \in \mathbb R^3 \suchthat* x_1^2 + x_2^2 < 1, x_3 = 1 \right\} 
        ,
        \\
        S_b =\left\{ (x_1,x_2,x_3) \in \mathbb R^3 \suchthat* x_1^2 + x_2^2 = x_3, 0<x_3<1 \right\} 
    \end{align*}
    We have that $\partial V = \partial \Omega_a \cup \partial V_b$. 
    To calculate the area we need a parameterisation of both surfaces:
    \begin{align*}
        \Phi_a: [0,2\pi) \times (0,1) \mapsto S_a: (\theta,r)\mapsto (r\cos\theta, r\sin\theta,1),
        \\
        \Phi_b: [0,2\pi) \times (0,1) \mapsto S_b: (\theta,z)\mapsto (\sqrt{z}\cos\theta, \sqrt{z}\sin\theta,z).
    \end{align*}
    Moreover we have that
    \[
        \left\|\partial_{\theta}\Phi_{a}\times \partial_r\Phi_{a}\right\| 
        = 
        \left\|\begin{pmatrix} -r\sin\theta \\ r\cos\theta\\0 \end{pmatrix} \times \begin{pmatrix} \cos\theta \\ \sin\theta \\0 \end{pmatrix}\right\| 
        = 
        \left\|\begin{pmatrix} 0 \\ 0 \\ -r \end{pmatrix}\right\| 
        = 
        r
    \]
    and 
    \[
        \left\|\partial_{\theta}\Phi_{b}\times \partial_z\Phi_{b}\right\|  
        = 
        \left\|\begin{pmatrix} -\sqrt{z}\sin\theta \\ \sqrt{z}\cos\theta\\0 \end{pmatrix} \times \begin{pmatrix} \frac{1}{2\sqrt{z}}\cos\theta \\  \frac{1}{2\sqrt{z}}\sin\theta \\1 \end{pmatrix}\right\| 
        = 
        \left\|\begin{pmatrix} \sqrt{z}\sin\theta \\ \sqrt{z}\cos\theta\\-\frac{1}{2} \end{pmatrix}\right\| 
        = 
        \sqrt{z+\frac{1}{4}}
    \]
    Now the area is given by:
    \begin{align*}
        \iint_{\partial\Omega} 1 \;d (\partial \Omega)
        &
        =
        \iint_{\partial\Omega_a} 1 \;dS_a + 	\iint_{\partial\Omega_b} 1 \;dS_b
        \\&
        = \int_0^{2\pi} \int_0^1 r \;d r\;d\theta + \int_0^{2\pi} \int_0^1\sqrt{z+\frac{1}{4}}\;d z \;d \theta
        \\&
        = 
        2\pi\left[\frac{1}{2}r^2\right]_0^1 + \left[\frac{2}{3}\left(z+\frac{1}{4}\right)^{\frac{3}{2}}\right]_0^1
        \\&
        = 
        \pi + \frac{4\pi}{3}\left(\left(\frac{5}{4}\right)^{\frac{3}{2}} - \left(\frac{1}{4}\right)^{\frac{3}{2}}\right)
        = 
        \pi + \pi \frac{5\sqrt{5} - 1}{6}
        = 
        5\pi \frac{1 + \sqrt{5}}{6}
        .
    \end{align*}
\end{solution}















\begin{exercise}
    Verify the divergence theorem for the following vector field $\vec F$ and volume $V$:
    \begin{gather*}
        \vec F(x_1,x_2,x_3) := ( x_1 x_3, x_2, x_2 ),
        \qquad 
        V := \left\{\; (x_1,x_2,x_3) \in \mathbb R^3 \suchthat* x_1^2 + x_2^2 + x_3^2 < 1 \;\right\}
        .
    \end{gather*}
    Note that $V$ is just the three-dimensional unit ball. 
\end{exercise}
\begin{solution}    
    We have to show that the followig identity holds: 
    \[
    \iiint_{V}\nabla \cdot \vec{F}\;dV =  \oint \vec{F}\cdot \vec{n} \;d(\partial V)
    \] 
    We start with the volume integral:
    \begin{align*}
        &
        \iiint_{V}\nabla \cdot \vec{F}\;dV
        \\&
        =
        \iiint_{V}x_3 + 1\;dV
        \\&
        =
        \int_{0}^{2\pi}\int_0^{\pi}\int_0^ 1 (r\cos\phi + 1)r^ 2\sin\phi\;d r\;d\phi\;d\theta
        \\&
        =
        2\pi\int_0^1 r^3\;d r\int_0^{\pi}\cos\phi\sin\phi\;d \phi + 2\pi\int_0^1 r^2\;d r\int_0^{\pi}\sin\phi\;d \phi 
        \\&
        =
        2\pi\left[ \frac{1}{4}r^ 4 \right]_0^ {1} \left[-\frac{1}{4}\cos2\phi\right]_0^{\pi} + 2\pi\left[ \frac{1}{3}r^ 3 \right]_0^ {1} \left[-\cos\phi\right]_0^{\pi}
        \\&
        =
        2\pi\left(\frac{1}{4} - 0\right)\left(\frac{1}{4} - -\frac{1}{4}\right) + 2\pi\left(\frac{1}{3} - 0\right)\left(1 - - 1\right) = \frac{4\pi}{3}
        .
    \end{align*}
    Next we consider the surface integral. We need a parameterisation for the surface. One can consider:
    \[
    \Phi: [0,2\pi) \times (0,\pi) \mapsto S: (\theta,\phi)\mapsto (\cos\theta \sin\phi, \sin\theta\sin\phi,\cos\phi)
    \]
    with 
    \[
    \partial_{\theta}\Phi\times \partial_{\phi}\Phi  = \begin{pmatrix} -\sin\theta\sin\phi\\ \cos\theta\sin\phi \\0 \end{pmatrix} \times \begin{pmatrix} -\cos\theta\cos\phi\\ \sin\theta\cos\phi \\-\sin\phi \end{pmatrix} = -\begin{pmatrix} \cos\theta\sin^2\phi\\ \sin\theta\sin^2\phi \\\sin\phi\cos\phi \end{pmatrix} 
    \]
    Note this is inward pointing. Therefore we need an extra minus sign.
    \begin{align*}
        &
        \oint \vec{F}\cdot \vec{n} \;dS
        \\&
        =
        \int_{0}^{2\pi}\int_0^{\pi} \begin{pmatrix}\cos\theta\sin\phi\cos\phi\\ \sin\theta\sin\phi \\ \sin\theta\sin\phi \end{pmatrix}\cdot\begin{pmatrix} \cos\theta\sin^2\phi\\ \sin\theta\sin^2\phi \\\sin\phi\cos\phi \end{pmatrix} \;d\phi\;d\theta
        \\&
        =
        \int_{0}^{2\pi}\cos^2\theta\;d\theta\int_0^{\pi}\sin^3\phi\cos\phi\;d\phi + \int_{0}^{2\pi}\sin^2\theta\;d\theta\int_0^{\pi}\sin^3\phi\;d\phi +  \int_{0}^{2\pi}\sin\theta\;d\theta\int_0^{\pi}\sin^2\phi\cos\phi\;d\phi
        .
    \end{align*}
    The last term is zero due to the fact that the integral of $\sin\theta$ over 1 period is equal to $0$. The other integrals are evaluated seperately, where we use the substiitution $u = \cos\phi$ several times:
    \begin{align*}
        &
        \int_0^{\pi}\sin^3\phi\cos\phi\;d\phi 
        \\&
        = \int_0^{\pi}(1 - \cos^2\phi)\sin\phi\cos\phi\;d\phi 
        \\&
        = \int_0^{\pi}\sin\phi\cos\phi - \cos^3\phi\sin\phi\;d\phi 	
        \\&
        = \int_{-1}^1 u + u^3 \;d u  
        \\&
        = \left[\frac{1}{2} u^2 +\frac{1}{4}u^4\right]_{-1}^{1} = 0
    \end{align*}
    Now, 
    \begin{align*}
        &
        \int_0^{2\pi}\sin^2\theta\;d\theta
        \\&
        = \int_0^{2\pi}\frac{1}{2}(1-\cos2\theta)\;d\theta 
        \\&
        = \left[\frac{1}{2} \theta -\frac{1}{4}\sin2\theta\right]_{0}^{2\pi} = \pi
    \end{align*}
    Next, 
    \begin{align*}
        &
        \int_0^{\pi}\sin^3\phi\;d\phi 
        \\&
        = \int_0^{\pi}(1 - \cos^2\phi)\sin\phi\;d\phi 
        \\&
        = \int_{-1}^1 1 - u^2 \;d u  
        \\&
        = \left[ u - \frac{1}{3}u^3\right]_{-1}^{1} = \frac{4}{3}
    \end{align*}
    If we put everything together we get:
    \begin{align*}
        &
        \oint \vec{F}\cdot \vec{n} \;d(\partial \Omega)
        \\&
        =
        \int_{0}^{2\pi}\cos^2\theta\;d\theta\int_0^{\pi}\sin^3\phi\cos\phi\;d\phi + \int_{0}^{2\pi}\sin^2\theta\;d\theta\int_0^{\pi}\sin^3\phi\;d\phi 
        \\&
        =\int_{0}^{2\pi}\cos^2\theta\;d\theta \cdot 0+ \pi\frac{4}{3} = \frac{4\pi}{3}
        .
    \end{align*}
    We conclude that both integrals give the same value thereby verifying the divergence theorem.
\end{solution}

    


















\begin{exercise}
    Find a regular parameterization $\Phi(s,t)$ of the surface 
    \begin{gather*}
        S := \left\{\; (x_1,x_2,x_3) \in \mathbb R^3 \suchthat* 0 < x_3 < 1, \; x_1^2 + x_2^2 = 1 + x_3^2 \;\right\}
    \end{gather*}
    Compute cross product $\partial_s \Phi(s,t) \times \partial_t \Phi(s,t)$ and its norm. 
\end{exercise}
\begin{solution}     
    We define the following parameterisation for the surface $S$:
    \[
        \Phi: [0,2\pi) \times (0,1) \mapsto S,
        \quad 
        (s,t) \mapsto \left( \sqrt{1+t^{2}}\cos s, \sqrt{1+t^{2}}\sin s, t \right)
    \]
    \[
        \partial_s \Phi(s,t) \times \partial_t \Phi(s,t) 
        = 
        \begin{pmatrix} 
            -\sqrt{1+t^2}\sin s
            \\ 
            \sqrt{1+t^2}\cos s
            \\
            0 
        \end{pmatrix} 
        \times 
        \begin{pmatrix} 
            \frac{- t}{\sqrt{1+t^2}}\cos s 
            \\ 
            \frac{- t}{\sqrt{1+t^2}}\sin s 
            \\
            1 
        \end{pmatrix} 
        = 
        \begin{pmatrix} 
            \sqrt{1+t^2}\cos s 
            \\ 
            \sqrt{1+t^2}\sin s
            \\ 
            t
        \end{pmatrix} 
    \]
    \[
        \left\|
        \partial_s \Phi(s,t) \times \partial_t \Phi(s,t)
        \right\| 
        = 
        \left\| 
        \begin{pmatrix} 
            \sqrt{1+t^2}\cos s 
            \\ 
            \sqrt{1+t^2}\sin s
            \\ 
            t
        \end{pmatrix} \right\| 
        = 
        \sqrt{1+2t^2}
    \]
\end{solution}





















\end{document}
