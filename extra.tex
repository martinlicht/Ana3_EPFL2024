\documentclass[11pt]{article}
% \def\hidesolutions{}
%%%%%%%%%%% SET MARGINS
\setlength{\textheight}{20cm}
\setlength{\topmargin}{-0.5cm}
\setlength{\oddsidemargin}{+0cm}
\setlength{\textwidth}{16.3cm}
%\setlength{\parskip}{6pt}
\setlength{\parindent}{0pt}

%%%%%%%%%%% PACKAGES
\usepackage{amsmath}
\usepackage{amssymb}
\usepackage{amsfonts}
%\usepackage{a4wide}
\usepackage{graphicx}
\usepackage{color}
\usepackage[normalem]{ulem}
\usepackage{enumitem}
\usepackage{capt-of}
\usepackage{float}
\usepackage{amsmath}
\usepackage{listings}
\definecolor{mygreen}{RGB}{28,172,0} % color values Red, Green, Blue
\definecolor{mylilas}{RGB}{170,55,241}
\usepackage{empheq}
\usepackage[ruled]{algorithm2e}
\usepackage{mathrsfs}
\usepackage{datetime}
\usepackage{subcaption}

% TODO: combine the two package lists and reduce redundancies 
\usepackage{mathtools}
\usepackage{nicefrac}
\usepackage{hyperref}
\usepackage{url}
\usepackage{amsmath,amssymb,amsfonts}
\usepackage{a4wide}
\usepackage{graphicx}
\usepackage{color}
\usepackage[normalem]{ulem}
\usepackage{capt-of}
\usepackage{float}
\usepackage[ruled]{algorithm2e}
\usepackage{amsmath,amssymb,amsfonts}
\usepackage{a4wide}
\usepackage{graphicx}
\usepackage{color}
\usepackage[normalem]{ulem}
\usepackage{capt-of}
\usepackage{float}
\usepackage[ruled]{algorithm2e}
\usepackage{mathrsfs}







\newcommand{\Lc}[2]{{\color{blue} \sout{#1} } \textcolor{red}{#2}}
\newcommand{\La}[1]{\textcolor{red}{#1}}
\newcommand{\lh}{\mathscr{L}_h}
\newcommand{\cl}{\mathscr{L}}
\newcommand{\cf}{\mathscr{F}}
\newcommand{\dx}{dx}
\newcommand{\ltn}{\mathscr{l}^2}
\newcommand{\bbR}{\mathbb{R}}
\newcommand{\Rset}{\mathbb{R}}
\newcommand{\Nset}{\mathbb{N}}
\newcommand{\scL}{\mathcal{L}}
\newcommand{\xx}{\mathbf{x}}
\newcommand{\norm}[1]{\|{#1}\|}
\newcommand{\yy}{\mathbf{y}}
\newcommand{\at}[1]{\big|_{#1}}
\renewcommand{\div}{\mathrm{div}}
\newcommand{\divergence}{\mathrm{div}}
\newcommand{\cp}[1]{\textcolor{blue}{#1}}

\newcommand{\FF}{\texttt{FreeFem++ }}
\newcommand{\FFns}{\texttt{FreeFem++}}
\newcommand{\FFfull}{\texttt{FreeFem++-x11}}
\newcommand{\cmd}[1]{ \medskip \noindent \texttt{#1} \medskip}
\newcommand{\incmd}[1]{\texttt{#1}}
\newcommand{\shrinkitems}{\addtolength{\itemsep}{-0.5\baselineskip}}
\newcommand{\mtt}[1]{\mathtt{#1}}
\newcommand{\ML}{\texttt{Matlab }}

\newcommand{\bb}{\mathbf{b}}
\newcommand{\nn}{\mathbf{n}}
\newcommand{\vecA}{\vec{A}}
\newcommand{\vecB}{\vec{B}}


\newcommand{\mesh}{\mathcal{T}_h}
\newcommand{\refel}{\widehat{K}}
\newcommand{\ver}{\mathbf{a}}
\newcommand{\refver}{\widehat{\mathbf{a}}}
\newcommand{\grad}{\nabla}
\newcommand{\refgrad}{\widehat{\nabla}}
\newcommand{\refu}{\widehat{u}}
\newcommand{\refbasis}{\widehat{\varphi}}
\newcommand{\refxx}{\widehat{\xx}}
\newcommand{\refx}{\widehat{x}}
\newcommand{\refy}{\widehat{y}}
\newcommand{\refrho}{\widehat{\rho}}
\newcommand{\refh}{\widehat{h}}






% For typesetting Python code
\newcommand{\matlab}{{\sc Matlab}\xspace}
\usepackage{listings}
\lstloadlanguages{Python}
\lstloadlanguages{csh}%
\definecolor{MyDarkGreen}{rgb}{0.0,0.4,0.0}
\definecolor{purple}{rgb}{0.58,0,0.82}
\lstset{language=Python,                    % Use Python
	%frame=single,                          % Single frame around code
	basicstyle=\ttfamily\footnotesize\color{black},
	keywordstyle=[1]\color{blue}\bf,        % Python functions bold and blue
	keywordstyle=[2]\color{purple},         % Python function arguments purple
	keywordstyle=[3]\color{red}\underbar,   % User functions underlined and blue
	commentstyle=\usefont{T1}{pcr}{m}{sl}\color{MyDarkGreen}\small,
	stringstyle=\color{purple},
	showstringspaces=false,                 % Don't put marks in string spaces
	tabsize=3,                              % 5 spaces per tab
	morekeywords={xlim,ylim,var,alpha,factorial,poissrnd,normpdf,normcdf},
	morecomment=[l][\color{blue}]{...},
	breaklines=true,
	breakatwhitespace=true,
	emptylines=1,
	mathescape=true,
	xleftmargin=0ex,
	emphstyle=\bfseries\color{red}
}





%%%%%%%%%%% MACROS NAMES
\newcommand{\lecturername}{Martin Licht}
% \newcommand{\assistantnamea}{Jochen Hinz}
% \newcommand{\assistantnameb}{Ivan Bioli}
\newcommand{\semestername}{Winter Semester 2023}
\newcommand{\lecturename}{Analysis III - 202(c)}
\DeclarePairedDelimiter\floor{\lfloor}{\rfloor}

%%%%%%%%%%% HEADER
\newdateformat{yeardate}{\THEYEAR}
\newcommand{\exsheet}[3] % input is the number of the session and the day TODO What's that
{\clearpage

	\begin{center}
		{\Large \textbf{\lecturename}}\\[2ex]
		\semestername
	\end{center}

	% \vspace{2ex}
	% \lecturername

	\vspace{2ex}
	{\Large Session #1: #3\,#2, \yeardate\today}
	%\hfill
	%{\Large EPF Lausanne}

	\hrulefill
}





\usepackage{comment}

\newtheorem{exercise}{Exercise}
\newtheorem{solutionenv}{Solution}

\newboolean{hide_solution}
\ifx\hidesolutions\undefined
\newenvironment{solution}{\begin{solutionenv}}{\end{solutionenv}}
\setboolean{hide_solution}{false}
\else
\excludecomment{solution}
\setboolean{hide_solution}{true}
\fi

\newcommand{\ifnotsolution}[1]{\ifthenelse{\boolean{hide_solution}}{#1}{}}
\newcommand{\ifsolution}[1]{\ifthenelse{\boolean{hide_solution}}{}{#1}}








\allowdisplaybreaks

\begin{document}
\exsheet{14}{19}{December} % parameters are the number of the session and the day


\begin{exercise}
    Recall the ReLU function 
    \begin{align}
        \ReLU(x) = \max( 0, x ).
    \end{align}
    A shallow neural network with one output neuron, $m$ internal neurons, and $n$ input neurons has the general form 
    \begin{align}
        f(x_1,\dots,x_{n}) = \sum_{k=1}^{m} \ReLU\left( \sum_{i=1}^{n} A_{ki} x_{i} + b_{k} \right)
    \end{align}
    where $A_{ki}$ are weights and where $b_{k}$ is a shift parameter. 
    \begin{itemize}
        \item Most training algorithms require the gradient of this network. Compute the derivatives $\partial_{i} f$
        \item There is interest in training algorithms that use the Hessian matrix. Compute the partial derivatives $\partial^{2}_{ij} f$.
    \end{itemize}
\end{exercise}
\begin{solution}
    We begin with the first derivatives. Obviously,
    \begin{align}
        \partial_{i} f(x_1,\dots,x_{n}) = \sum_{k=1}^{m} \partial_{i} \ReLU\left( \sum_{i=1}^{n} A_{ki} x_{i} + b_{k} \right)
    \end{align}
    
    The ReLU function can only be differentiated in the sense of distributions. Recalling the Heaviside function,
    \begin{align}
        H(x) = \begin{cases} 1 & x > 0 \\ 0 & x \leq 0 \end{cases},
    \end{align}
    it is now possible to write the derivative as 
    \begin{align}
        \partial_{i} f(x_1,\dots,x_{n}) 
        &= 
        \sum_{k=1}^{m} H\left( \sum_{i=1}^{n} A_{ki} x_{i} \right) A_{ii} 
        \\&
        =
        \sum_{k=1}^{m} A_{ki} H\left( \sum_{i=1}^{n} A_{ki} x_{i} \right)
        .
    \end{align}
    This computes the first derivatives.\footnote{Remark: the derivative does not have a meaningful value if one of the arguments of the Heaviside function is zero (or close to zero within the range of rounding errors). When training a neural network via gradient descent, this is ``justified'' by the assumption that these arguments being close to zero is very unlikely to happen in practice.}
    
    We continue with the second derivatives. Recall that the Heaviside function has a derivative in the sense of distributions, which is the Dirac Delta. Hence 
    \begin{align}
        \partial^{2}_{ji} f(x_1,\dots,x_{n}) 
        &
        = 
        \sum_{k=1}^{m} A_{ki} \partial_{j} H\left( \sum_{i=1}^{n} A_{ki} x_{i} \right)
        \\&
        = 
        \sum_{k=1}^{m} A_{ki} A_{ji} \delta_{0}\left( \sum_{i=1}^{n} A_{ki} x_{i} \right)
        .
    \end{align}
    This computes the second derivatives.\footnote{Remark: the situation here is even worse than with the first derivatives. The Dirac Delta is zero everywhere except at the origin, and it equals a pointmass at the origin. Correspondingly, second-order training algorithms, such as, e.g., Newton's method, are not well-defined for such a network. This is a possible incentive to replace ReLU by other activation functions, such as $S(x) = \ln( 1 + e^{x} )$.}
\end{solution}


\begin{exercise}
    In standard models of elasticity, a long straight beam of elastic material, such as wood or metal, can be modeled as a one-dimensional interval. 
    When it is subject to an outside force $f$, such as gravity, than the deformation from the base is modeled by the beam equation 
    \begin{align}
        u''''(x) = f(x)
    \end{align}
    Here, the fourth derivative $u''''$ can be interpreted as the curvature of a curvature and $f$ describes the direction (upwards, downwards) and magnitude of the force.
    
    So-called non-local interactions are modeled via a convolutional term $k \star u$, where $k$ how parts of a beam are influenced by neighboring parts.
    With that in mind, we consider a generalized beam equation 
    \begin{align}
        u''''(x) + c u(x) + ( k \star u )(x) = f(x)
        .
    \end{align}
    This is a so-called integro-differential equation. 
    
    Compute the Fourier transform of this differential equation for general source terms,
    and write it down for the particular example
    \begin{align}
        k(x) = e^{-|y|},
        \quad 
        f(x) = e^{-y^{2}}.
    \end{align}
    \textit{You are not expected to solve this equation.}
\end{exercise}
\begin{solution}
    The Fourier transform of this equation reads
    \begin{align}
        (i\alpha)^{4} \hat u(\alpha) + c \hat u(\alpha) + \sqrt{2\pi} \hat k(\alpha) \hat u(\alpha) = \hat f(\alpha)
        .
    \end{align}
    We simplify this to:
    \begin{align}
        \alpha^{4} \hat u(\alpha) + c \hat u(\alpha) + \sqrt{2\pi} \hat k(\alpha) \hat u(\alpha) = \hat f(\alpha)
        .
    \end{align}
    
    We isolate $\hat u(\alpha)$, giving us:
    \begin{align}
        \hat u(\alpha)
        =
        \frac{\hat f(\alpha)}{ \alpha^{4} + c + \sqrt{2\pi} \hat k(\alpha) }
        .
    \end{align}
\end{solution}







\end{document}

