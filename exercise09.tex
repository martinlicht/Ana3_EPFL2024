\documentclass[11pt]{article}
% \def\hidesolutions{}
%%%%%%%%%%% SET MARGINS
\setlength{\textheight}{20cm}
\setlength{\topmargin}{-0.5cm}
\setlength{\oddsidemargin}{+0cm}
\setlength{\textwidth}{16.3cm}
%\setlength{\parskip}{6pt}
\setlength{\parindent}{0pt}

%%%%%%%%%%% PACKAGES
\usepackage{amsmath}
\usepackage{amssymb}
\usepackage{amsfonts}
%\usepackage{a4wide}
\usepackage{graphicx}
\usepackage{color}
\usepackage[normalem]{ulem}
\usepackage{enumitem}
\usepackage{capt-of}
\usepackage{float}
\usepackage{amsmath}
\usepackage{listings}
\definecolor{mygreen}{RGB}{28,172,0} % color values Red, Green, Blue
\definecolor{mylilas}{RGB}{170,55,241}
\usepackage{empheq}
\usepackage[ruled]{algorithm2e}
\usepackage{mathrsfs}
\usepackage{datetime}
\usepackage{subcaption}

% TODO: combine the two package lists and reduce redundancies 
\usepackage{mathtools}
\usepackage{nicefrac}
\usepackage{hyperref}
\usepackage{url}
\usepackage{amsmath,amssymb,amsfonts}
\usepackage{a4wide}
\usepackage{graphicx}
\usepackage{color}
\usepackage[normalem]{ulem}
\usepackage{capt-of}
\usepackage{float}
\usepackage[ruled]{algorithm2e}
\usepackage{amsmath,amssymb,amsfonts}
\usepackage{a4wide}
\usepackage{graphicx}
\usepackage{color}
\usepackage[normalem]{ulem}
\usepackage{capt-of}
\usepackage{float}
\usepackage[ruled]{algorithm2e}
\usepackage{mathrsfs}







\newcommand{\Lc}[2]{{\color{blue} \sout{#1} } \textcolor{red}{#2}}
\newcommand{\La}[1]{\textcolor{red}{#1}}
\newcommand{\lh}{\mathscr{L}_h}
\newcommand{\cl}{\mathscr{L}}
\newcommand{\cf}{\mathscr{F}}
\newcommand{\dx}{dx}
\newcommand{\ltn}{\mathscr{l}^2}
\newcommand{\bbR}{\mathbb{R}}
\newcommand{\Rset}{\mathbb{R}}
\newcommand{\Nset}{\mathbb{N}}
\newcommand{\scL}{\mathcal{L}}
\newcommand{\xx}{\mathbf{x}}
\newcommand{\norm}[1]{\|{#1}\|}
\newcommand{\yy}{\mathbf{y}}
\newcommand{\at}[1]{\big|_{#1}}
\renewcommand{\div}{\mathrm{div}}
\newcommand{\divergence}{\mathrm{div}}
\newcommand{\cp}[1]{\textcolor{blue}{#1}}

\newcommand{\FF}{\texttt{FreeFem++ }}
\newcommand{\FFns}{\texttt{FreeFem++}}
\newcommand{\FFfull}{\texttt{FreeFem++-x11}}
\newcommand{\cmd}[1]{ \medskip \noindent \texttt{#1} \medskip}
\newcommand{\incmd}[1]{\texttt{#1}}
\newcommand{\shrinkitems}{\addtolength{\itemsep}{-0.5\baselineskip}}
\newcommand{\mtt}[1]{\mathtt{#1}}
\newcommand{\ML}{\texttt{Matlab }}

\newcommand{\bb}{\mathbf{b}}
\newcommand{\nn}{\mathbf{n}}
\newcommand{\vecA}{\vec{A}}
\newcommand{\vecB}{\vec{B}}


\newcommand{\mesh}{\mathcal{T}_h}
\newcommand{\refel}{\widehat{K}}
\newcommand{\ver}{\mathbf{a}}
\newcommand{\refver}{\widehat{\mathbf{a}}}
\newcommand{\grad}{\nabla}
\newcommand{\refgrad}{\widehat{\nabla}}
\newcommand{\refu}{\widehat{u}}
\newcommand{\refbasis}{\widehat{\varphi}}
\newcommand{\refxx}{\widehat{\xx}}
\newcommand{\refx}{\widehat{x}}
\newcommand{\refy}{\widehat{y}}
\newcommand{\refrho}{\widehat{\rho}}
\newcommand{\refh}{\widehat{h}}






% For typesetting Python code
\newcommand{\matlab}{{\sc Matlab}\xspace}
\usepackage{listings}
\lstloadlanguages{Python}
\lstloadlanguages{csh}%
\definecolor{MyDarkGreen}{rgb}{0.0,0.4,0.0}
\definecolor{purple}{rgb}{0.58,0,0.82}
\lstset{language=Python,                    % Use Python
	%frame=single,                          % Single frame around code
	basicstyle=\ttfamily\footnotesize\color{black},
	keywordstyle=[1]\color{blue}\bf,        % Python functions bold and blue
	keywordstyle=[2]\color{purple},         % Python function arguments purple
	keywordstyle=[3]\color{red}\underbar,   % User functions underlined and blue
	commentstyle=\usefont{T1}{pcr}{m}{sl}\color{MyDarkGreen}\small,
	stringstyle=\color{purple},
	showstringspaces=false,                 % Don't put marks in string spaces
	tabsize=3,                              % 5 spaces per tab
	morekeywords={xlim,ylim,var,alpha,factorial,poissrnd,normpdf,normcdf},
	morecomment=[l][\color{blue}]{...},
	breaklines=true,
	breakatwhitespace=true,
	emptylines=1,
	mathescape=true,
	xleftmargin=0ex,
	emphstyle=\bfseries\color{red}
}





%%%%%%%%%%% MACROS NAMES
\newcommand{\lecturername}{Martin Licht}
% \newcommand{\assistantnamea}{Jochen Hinz}
% \newcommand{\assistantnameb}{Ivan Bioli}
\newcommand{\semestername}{Winter Semester 2023}
\newcommand{\lecturename}{Analysis III - 202(c)}
\DeclarePairedDelimiter\floor{\lfloor}{\rfloor}

%%%%%%%%%%% HEADER
\newdateformat{yeardate}{\THEYEAR}
\newcommand{\exsheet}[3] % input is the number of the session and the day TODO What's that
{\clearpage

	\begin{center}
		{\Large \textbf{\lecturename}}\\[2ex]
		\semestername
	\end{center}

	% \vspace{2ex}
	% \lecturername

	\vspace{2ex}
	{\Large Session #1: #3\,#2, \yeardate\today}
	%\hfill
	%{\Large EPF Lausanne}

	\hrulefill
}





\usepackage{comment}

\newtheorem{exercise}{Exercise}
\newtheorem{solutionenv}{Solution}

\newboolean{hide_solution}
\ifx\hidesolutions\undefined
\newenvironment{solution}{\begin{solutionenv}}{\end{solutionenv}}
\setboolean{hide_solution}{false}
\else
\excludecomment{solution}
\setboolean{hide_solution}{true}
\fi

\newcommand{\ifnotsolution}[1]{\ifthenelse{\boolean{hide_solution}}{#1}{}}
\newcommand{\ifsolution}[1]{\ifthenelse{\boolean{hide_solution}}{}{#1}}








\allowdisplaybreaks

\begin{document}
\exsheet{9}{14}{November} % parameters are the number of the session and the day

\begin{exercise}
    Consider the following functions with period $T$:
    \begin{itemize}
        \item A function $f$ with period $T = 1$ such that 
        \begin{align*}
            f(x) = 
            \left\{\begin{array}{ll}
                1 & \text{if } 0 \leq x < 0.5 \\
                0 & \text{if } 0.5 \leq x \leq 1
            \end{array}\right.
        \end{align*}
        \item A function $g$ with period $T = 2\pi$ such that 
        \begin{align*}
            g(x) = 
            \left\{\begin{array}{ll}
                x & \text{if } 0 \leq x < \pi \\
                2\pi - x & \text{if } \pi \leq x \leq 2\pi
            \end{array}\right.
        \end{align*}
        \item A function $h$ with period $T = 1$ such that 
        \begin{align*}
            h(x) = -x \text{ if } 0 \leq x < 1.
        \end{align*}
    \end{itemize}
    Find the Fourier coefficients of the Fourier series of these functions. 
    \textit{You can modify the functions seen in the lecture to get the coefficients.}
\end{exercise}

\begin{solution}     
\begin{itemize}
\item As seen in the lecture we know the fourier coefficients of a square wave,
 $$
l(x)= \begin{cases}1, & \text { if } 0 \leq x<0.5 \\ -1, & \text { if } 0.5 \leq x<2 \pi\end{cases},
$$

are given by:

$$
a_n = 0\quad \text{for } n \geq 0, \quad b_n = \begin{cases}0, & \text { if n is odd} \\ \frac{4}{n\pi}, & \text{ if n is even}\end{cases}
$$

We can express the function $f(x)$ in terms of $l(x)$ as follows:

$$
f(x) = \frac{1}{2} + \frac{1}{2}l(x)
$$

therefore the fourier coefficients of $f$ are given by:

$$
a_n = \begin{cases}\frac{1}{2}, & \text { if n} = 0 \\0,& \text{ if n }>0\end{cases}, \quad b_n = \begin{cases}0, & \text { if n is odd} \\ \frac{2}{n\pi}, & \text{ if n is even}\end{cases}
$$
\item  As seen in the lecture we know the fourier coefficients of a triangle wave,
$$
m(x)= \begin{cases}2 x, & \text { if } 0 \leq x<0.5 \\ 2(1-x), & \text { if } 0.5 \leq x<1\end{cases}
$$

are given by:

$$
a_0 = \frac{1}{2}, \quad a_n = \begin{cases}-\frac{4}{n^2\pi^2}, & \text { if n is odd} \\ 0, & \text{ if n is even}\end{cases}, \quad b_n = 0 \text{ for } n > 0
$$

We can express the function $g(x)$ in terms of $m(x)$ as follows:

$$
g(x) = \pi m(\frac{x}{2\pi})
$$

therefore the fourier coefficients of $g$ are given by:

$$
a_0 = \frac{\pi}{2}, \quad a_n = \begin{cases}-\frac{4}{n^2\pi}, & \text { if n is odd} \\ 0, & \text{ if n is even}\end{cases}, \quad b_n = 0 \text{ for } n > 0
$$
\item As seen in the lecture we know the fourier coefficients of a sawtooth wave,
$$
n(x)=x \text { for } 0 \leq x<1
$$

are given by:

$$
a_0 = \frac{1}{2}, \quad a_n = 0 \text{ for } n > 0,  \quad a_n = -\frac{1}{n\pi} \text{ for } n > 0,
$$

We can express the function $h(x)$ in terms of $n(x)$ as follows:

$$
h(x) = -n(x)
$$

therefore the fourier coefficients of $h$ are given by:

$$
a_0 = -\frac{1}{2}, \quad a_n = 0 \text{ for } n > 0,  \quad a_n = \frac{1}{n\pi} \text{ for } n > 0,
$$
\end{itemize}
\end{solution}

\begin{exercise}
    Explicitly write down the coefficients $a_n$ and $b_n$ and the periods of the following Fourier series:
    \begin{align*}
        f(x) &= \sum_{n=1}^{\infty} \frac{1}{(2n+1)^2}\sin(2\pi (2n+1) x),
        \\
        g(x) &= \sum_{n=1}^{\infty} \frac{(-1)^n}{(2n-1)^3}\cos(2\pi (2n-1) x),
        \\
        h(x) &= \frac \pi 3 + \sum_{n=1}^{\infty} \frac{(-1)^{n+1}}{n^2}\cos(\pi n x),
        .
    \end{align*}
    Determine the Fourier coefficients of the derivatives of those functions. 
\end{exercise}
\begin{solution}     
\begin{itemize}
\item We substitute $m = 2n+1$ to obtain 
$$
        f(x) = \sum_{m=3|m \text{ is odd}}^{\infty} \frac{1}{m^2}\cos(2\pi m x),
$$

therefore the fourier coefficients are given by 
$$
a_m = 0 \text{ for } m \geq 0, \quad b_m = \begin{cases}\frac{1}{m^2}, & \text { if m is odd and }m \geq 3 \\ 0, & \text{ if m is even and } m \geq 3\\ 0 & \text{ if } 0< m < 3  \end{cases} 
$$

and the period is $T = 1$. The derivative of $f(x)$ is given by:
$$
f'(x) = \sum_{m=3|m \text{ is odd}}^{\infty} \frac{2\pi}{m}\sin(2\pi m x),
$$

therefore the fourier coefficients of the derivative of $f(x)$ are given by 

$$
b_m = 0 \text{ for } m \geq 0, \quad a_m = \begin{cases}\frac{2\pi}{m}, & \text { if m is odd and }m \geq 3 \\ 0, & \text{ if m is even and } m \geq 3\\ 0 & \text{ if } 0< m < 3  \end{cases} 
$$

\item We substitute $m = 2n-1$ to obtain 
$$
       g(x) = \sum_{m=1|m \text{ is odd}}^{\infty} \frac{(-1)^{\frac{m+1}{2}}}{m^3}\cos(2\pi m x),
$$

therefore the fourier coefficients are given by 
$$
b_m = 0 \text{ for } m >0, \quad a_m = \begin{cases}\frac{(-1)^{\frac{m+1}{2}}}{m^3}, & \text { if m is odd and }m \geq 1 \\ 0, & \text{ if m is even and } m \geq 1\\ 0 & \text{ if } m = 0  \end{cases} 
$$

and the period is $T = 1$. The derivative of $g(x)$ is given by:
$$
g'(x) = \sum_{m=1|m \text{ is odd}}^{\infty} -\frac{2\pi (-1)^{\frac{m+1}{2}}}{m^2}\sin(2\pi m x),
$$

therefore the fourier coefficients of the derivative of $g(x)$ are given by 
$$
a_m = 0 \text{ for } m \geq 0, \quad b_m = \begin{cases}-\frac{2\pi (-1)^{\frac{m+1}{2}}}{m^2}, & \text { if m is odd and }m \geq 1 \\ 0, & \text{ if m is even and } m \geq 1\\ 0 & \text{ if } m = 0  \end{cases} 
$$
\item 
the fourier coefficients are given by 
$$
b_n = 0 \text{ for } n >0, \quad a_n = \begin{cases}\frac{\pi}{3}, & \text { if }n = 0 \\ -\frac{(-1)^n}{n^2}, & \text{ if } n > 1 \end{cases} 
$$

and the period $T = 2$. The derivative of $h(x)$ is given by:
$$
h'(x) =\sum_{n=1}^{\infty} -\frac{\pi(-1)^{n+1}}{n}\sin(\pi n x),
$$

therefore the fourier coefficients of the derivative of $h(x)$ are given by 
$$
a_n = 0 \text{ for } n \geq b, \quad b_n = -\frac{\pi(-1)^{n+1}}{n},\quad \text{for }n > 0
$$
\end{itemize}
\end{solution}

\begin{exercise}
    Compute the Fourier coefficients of the function $f$ that has period $T = 2\pi$ and satisfies
    \[
        f(x) = \left\{\begin{array}{ll} 1 & 0 \leq x < \pi \\ 0 & \pi < x \leq 2\pi \end{array}\right.
    \]
\end{exercise}
\begin{solution}     
    We notice that the period $T$ equals $T = 2\pi$. 
    We have defined the first coefficient to be just the average:
    \begin{align*}
        a_0 
        &= 
        \frac{1}{2\pi} \int_0^{2\pi} f(x) \, dx 
        = 
        \frac{1}{2\pi} \int_0^\pi \, dx  = \frac 1 2
        .
    \end{align*}
    For the other coefficients, we use the formulas from the lecture:
    \begin{align*}
        a_n 
        &= 
        \frac{2}{2\pi} \int_0^{2\pi} f(x) \cos\left(\frac{2\pi n x}{T}\right) \, dx 
        \\&= 
        \frac{2}{2\pi} \int_0^\pi \cos\left(\frac{2\pi n x}{T}\right) \, dx 
        \\
        &= 
        \frac{2}{2\pi} \left[ \frac{T}{2\pi n} \sin\left(\frac{2\pi n x}{T}\right) \right]_{x=0}^{x=\pi}
        \\&= 
        \frac{2}{2\pi} \frac{T}{2\pi n} \left( \sin\left(\frac{2\pi n \pi}{T}\right) - \sin(0) \right) 
        \\&= 
        \frac{2}{2\pi} \frac{2\pi}{2\pi n} \left( \sin\left(n \pi \right) - \sin(0) \right) 
        \\&= 
        0
    \end{align*}
    because $\sin$ equals zero at integer multiples of $\pi$.
    For the sine mode coefficients, we calculate similarly:
    \begin{align*}
        b_n 
        &= 
        \frac{2}{2\pi}
        \int_0^{2\pi} f(x) \sin(nx) \, dx 
        \\
        &= 
        \frac{1}{\pi}
        \int_0^\pi \sin(nx) \, dx 
        \\
        &= 
        \frac{1}{\pi}
        \left[-\frac{\cos(nx)}{n} \right]_{x=0}^{x=\pi} 
        \\
        &= 
        -
        \frac{1}{\pi}
        \frac{1}{n} \left[ \cos(n\pi) - \cos(0) \right] 
        \\
        &= 
        -
        \frac{1}{n\pi}
        \left[ (-1)^n - 1 \right] 
        \\
        &= 
        \frac{1}{n\pi}
        \begin{cases}
            0 & \text{if } n \text{ is even} 
            \\
            2 & \text{if } n \text{ is odd}
        \end{cases}
    \end{align*}
    Thus we find $a_0 = 0.5$ and $a_1 = a_2 = \dots = 0$,
    and either $b_n = 0$ if $n$ is even and $b_n = \frac{2}{n\pi}$ if $n$ is odd. 
\end{solution}


\begin{exercise}
    Compute the Fourier coefficients of the following functions, which have period $T = 1$ and have the given values over the interval $[0,1)$:
    \begin{itemize}
     \item 
     \[
        f(x) = x^{2}
     \]
     \item 
     \[
        g(x) = (1-x)x
     \]
     \item 
     \[
        h(x) = |\sin( 2 \pi x )|
     \]
     %
    \end{itemize}
\end{exercise}
\begin{solution}     
    \begin{itemize}
        \item 
        To compute the Fourier coefficients of $f(x) = x^2$ over the interval $[0,1)$ with a period of $1$, use the formulas from the lecture:
        \[
            a_0 = \int_{0}^{1} f(x) dx  = \int_{0}^{1} x^2 dx  = \left[ \frac{x^3}{3} \right]_{0}^{1} = \frac 1 3.
        \]
        For other coefficients, we apply integration by parts twice, and we use that $\sin$ vanishes at multiples of $\pi$:
        \begin{align*}
            a_n 
            &
            = 
            \frac{2}{T}
            \int_{0}^{1} x^2 \cos(2\pi n x) dx
            \\&
            =
            \frac{2}{T}
            \left[ x^2 \frac{ \sin(2\pi n x) }{ 2\pi n } \right]_{x=0}^{x=1}
            -
            \frac{2}{T}
            \int_{0}^{1} \frac{ 2x \sin(2\pi n x) }{ 2\pi n } dx
            \\&
            =
            \frac{2}{T}
            \left[ x^2 \frac{ \sin(2\pi n x) }{ 2\pi n } \right]_{x=0}^{x=1}
            -
            \frac{4}{T}
            \int_{0}^{1} x \frac{ \sin(2\pi n x) }{ 2\pi n } dx
            \\&
            =
            \frac{2}{T}
            \left[ x^2 \frac{ \sin(2\pi n x) }{ 2\pi n } \right]_{x=0}^{x=1}
            -
            \frac{4}{T}
            \left[ x \frac{ -\cos(2\pi n x) }{ 4\pi^2 n^2 } \right]_{x=0}^{x=1}
            +
            \frac{4}{T}
            \int_{0}^{1} \frac{ -\cos(2\pi n x) }{ 4\pi^2 n^2 } dx
            \\&
            =
            \frac{2}{T}
            \left[ x^2 \frac{ \sin(2\pi n x) }{ 2\pi n } \right]_{x=0}^{x=1}
            -
            \frac{4}{T}
            \left[ x \frac{ -\cos(2\pi n x) }{ 4\pi^2 n^2 } \right]_{x=0}^{x=1}
            +
            \frac{4}{T}
            \left[ \frac{ -\sin(2\pi n x) }{ 8\pi^3 n^3 } \right]_{x=0}^{x=1}
            \\&
            = 
            \frac{1}{\pi^2 n^2}
        \end{align*}
        \begin{align*}
            b_n 
            &
            = 
            \frac{2}{T}
            \int_{0}^{1} x^2 \sin(2\pi n x) dx
            \\&
            =
            \frac{2}{T}
            \left[- x^2 \frac{ \cos(2\pi n x) }{ 2\pi n } \right]_{x=0}^{x=1}
            +
            \frac{2}{T}
            \int_{0}^{1} \frac{ 2x \cos(2\pi n x) }{ 2\pi n } dx
            \\&
            =
            \frac{2}{T}
            \left[- x^2 \frac{ \cos(2\pi n x) }{ 2\pi n } \right]_{x=0}^{x=1}
            +
            \frac{4}{T}
            \int_{0}^{1} \frac{ x \cos(2\pi n x) }{ 2\pi n } dx
            \\&
            =
            \frac{2}{T}
            \left[ -x^2 \frac{ \cos(2\pi n x) }{ 2\pi n } \right]_{x=0}^{x=1}
            +
            \frac{4}{T}
            \left[ x \frac{ \sin(2\pi n x) }{ 4\pi^2 n^2 } \right]_{x=0}^{x=1}
            -
            \frac{4}{T}
            \int_{0}^{1} \frac{ \sin(2\pi n x) }{ 4\pi^2 n^2 } dx
            \\&
            =
            \frac{2}{T}
            \left[ -x^2 \frac{ \cos(2\pi n x) }{ 2\pi n } \right]_{x=0}^{x=1}
            +
            \frac{4}{T}
            \left[ x \frac{ \sin(2\pi n x) }{ 4\pi^2 n^2 } \right]_{x=0}^{x=1}
            +
            \frac{4}{T}
            \left[ \frac{ \cos(2\pi n x) }{ 8\pi^3 n^3 } \right]_{x=0}^{x=1}
            \\&
            = 
            -\frac{2}{T}\frac{1}{2\pi n} + \frac{4}{T}\frac{1}{8\pi^3n^3} -  \frac{4}{T}\frac{1}{8\pi^3n^3} = -\frac{1}{\pi n}
        \end{align*}
        \item 
        We notice that the function $g(x) = (1-x)x$ can be written $g(x) = x - f(x)$ over the interval $[0,1)$. 
        In other words, $g$ is the difference of the sawtooth function (which equals $x$ over $[0,1)$ and repeats with period $T=1$),
        and the periodic function $f(x)$ seen earlier.
        
        We will only need to determine the Fourier coefficients of $\tilde{g}(x) = x$ since we already computed the Fourier coefficients of $f(x)$. 
        We calculate the average 
        \[
            a_0^{\tilde g} = \int_{0}^{1} \tilde{g}(x) dx  = \int_{0}^{1} x dx  = \left[ \frac{x^2}{2} \right]_{0}^{1} = \frac 1 2.
        \]
        We calculate 
        \begin{align*}
            a_n^{\tilde g} 
            &
            = 
            \frac{2}{T}
            \int_{0}^{1} x \cos(2\pi n x) dx
            \\&
            =
            \frac{2}{T}
            \left[ x \frac{ \sin(2\pi n x) }{ 2\pi n } \right]_{x=0}^{x=1}
            -
            \frac{2}{T}
            \int_{0}^{1} \frac{ \sin(2\pi n x) }{ 2\pi n } dx
            \\&
            =
            \frac{2}{T}
            \left[ x \frac{ \sin(2\pi n x) }{ 2\pi n } \right]_{x=0}^{x=1}
            +
            \frac{2}{T}
            \left[ \frac{ \cos(2\pi n x) }{ 4\pi^2 n^2 } \right]_{x=0}^{x=1}
            \\&
            =
            \frac{2}{T}
            \frac{1}{2\pi n}
            \left( \sin(2\pi n) - \sin(0) \right)
            +
            \frac{2}{T}
            \frac{1}{ 4\pi^2 n^2 }
            \left( \cos(2\pi n) - \cos(0) \right)
            \\&
            =
            0
        \end{align*}	
        Lastly, 
        \begin{align*}
            b_n^{\tilde g} 
            &
            = 
            \frac{2}{T}
            \int_{0}^{1} x \sin(2\pi n x) dx
            \\&
            =
            \frac{2}{T}
            \left[- x \frac{ \cos(2\pi n x) }{ 2\pi n } \right]_{x=0}^{x=1}
            +
            \frac{2}{T}
            \int_{0}^{1} \frac{ \cos(2\pi n x) }{ 2\pi n } dx
            \\&
            =
           \frac{2}{T}
            \left[- x \frac{ \cos(2\pi n x) }{ 2\pi n } \right]_{x=0}^{x=1}
            +
            \frac{2}{T}
            \left[ \frac{ \sin(2\pi n x) }{ 4\pi^2 n^2 } \right]_{x=0}^{x=1}
            \\&
            = 
            -\frac{2}{T}\frac{1}{2\pi n} =  -\frac{1}{\pi n} 
        \end{align*}	
        Now the Fourier coefficients of $g(x)$ are given by
                \begin{align*}
                    &b_n = -\frac{1}{\pi n} - - \frac{1}{\pi n} = 0
                    \\& 
            a_n = 0 - \frac{1}{2\pi^2 n^2} =  - \frac{1}{\pi^2 n^2} 
                    \\&
            a_0 = \frac{1}{2} - \frac{1}{3} = \frac{1}{6}
        \end{align*}	
        \item
        Note that $h(x)$ is an even function.
        Therefore $b_n = 0$. 
        We calculate the average 
        \[
            a_0 
            = 
            \frac{1}{T} \int_{0}^{1} |\sin 2\pi x| dx  
            = 
            \frac{2}{T} \int_{0}^{\frac 1 2} \sin 2\pi x dx  
            =  
            \frac 2 T
            \left[ \frac{-1}{2\pi} \cos{2\pi x} \right]_{0}^{\frac 1 2} 
            =
            \frac{2}{\pi} 
            .
        \]
        The coefficients $a_n$ for the cosine modes with $n \geq 1$ are:
        \begin{align*}
            a_n 
            &
            = 
            \frac{4}{T}
            \int_{0}^{\frac 1 2} \sin 2\pi x \cos(2\pi n x) dx
            \\&
            =
            \frac{4}{T}
            \int_{0}^{\frac 1 2} \frac 1 2 \sin{2\pi(1-n)x} + \frac 1 2 \sin{2\pi(n+1)x} dx
            \\&
		=
            \frac{4}{T}
            \int_{0}^{\frac 1 2} -\frac 1 2 \sin{2\pi(n-1)x} + \frac 1 2 \sin{2\pi(n+1)x} dx
            \\&
            =
            \left[\frac{ \cos2\pi(n-1)x }{ \pi(n-1) } \right]_{x=0}^{x=\frac 1 2}
            +
            \left[- \frac{ \cos2\pi(n+1)x }{ \pi(n+1) } \right]_{x=0}^{x=\frac 1 2}
            \\&
            = 
            \frac{(-1)^{n-1}}{\pi(n-1)} -  \frac{1}{\pi(n-1)}  - \frac{(-1)^{n+1}}{\pi(n+1)} + \frac{1}{\pi(n+1)}
            \\&
            = -\frac{1+(-1)^{n}}{\pi(n-1)} +  \frac{1-(-1)^{n+1}}{\pi(n+1)} 
            \\&
            = -\frac{1+(-1)^{n}}{\pi(n-1)} +  \frac{1+(-1)^{n}}{\pi(n+1)} 
        \end{align*}	
        Therefore, when $n \geq 1$:
        \[a_n = 
        \begin{cases}
        0 ,\quad &\text{if } n\text{ is odd,}\\
        -\frac{2}{\pi(n-1)} + \frac{2}{\pi(n+1)}  ,\quad &\text{if } n\text{ is even,}
        \end{cases}
        \]
       \end{itemize}
\end{solution}

\begin{exercise}
    Compute the Fourier coefficients of the following functions, which have period $T = 2\pi$ and have the given values over the interval $[0,2\pi)$:
    \begin{itemize}
     \item 
     \[
        f(x) = e^{x - \pi}
     \]
     \item 
     \[
        g(x) = {(x - \pi)^3}
     \]
     \item 
     \[
        h(x) = \left\{\begin{array}{ll}
            \sin(x) & 0 \leq x < \frac \pi 2
            \\
            0   & \frac \pi 2 \leq x < \frac{3\pi}{2} 
            \\
            \sin(x) & \frac {3\pi} 2 \leq x < 2\pi
        \end{array}\right.
     \]
     %
    \end{itemize}
\end{exercise}
\begin{solution}     
    \begin{itemize}
    \item To compute the Fourier coefficients of $f(x)$ with a period of $2\pi$, use the formulas from the lecture:
    \[
        a_0 
        = 
        \frac{1}{2\pi}\int_{0}^{2\pi} f(x) dx  = \frac{1}{2\pi}\int_{0}^{2\pi} e^{x-\pi} dx  = \frac{1}{2\pi}\left[ e^{x-\pi} \right]_{0}^{2\pi} = \frac{1}{\pi}\left(e^{\pi} - e^{-\pi}\right)
    \]
    For other coefficients, we apply integration by parts twice, and see that we obtain the same integral again:
    \begin{align*}
        a_n 
        &
        = 
        \frac{1}{\pi}
        \int_{0}^{2\pi} e^{x - \pi} \cos( n x) dx
        \\&
        =
        \frac{1}{\pi}
        \left[ e^{x - \pi} \frac{ \sin(n x) }{ n } \right]_{x=0}^{x=2\pi}
        -
        \frac{1}{\pi}
        \int_{0}^{2\pi} e^{x-\pi}\frac{ \sin( n x) }{ n } dx
        \\&
        =
        \frac{1}{\pi}
        \left[ e^{x - \pi} \frac{ \sin(n x) }{ n } \right]_{x=0}^{x=2\pi}
        -
        \frac{1}{\pi}
        \left[ e^{x - \pi} \frac{ -\cos(n x) }{ n^2 } \right]_{x=0}^{x=2\pi}
        \frac{1}{\pi}
        \int_{0}^{2\pi} e^{x-\pi}\frac{ -\cos(n x) }{ n^2 } dx
        \\&
        =
        \frac{1}{\pi}
        \left( e^{\pi} \frac{ \sin(2\pi x) }{ n } - e^{0 - \pi} \frac{ \sin(n 0) }{ n } \right)
        -
        \frac{1}{\pi}
        \left( e^{\pi} \frac{ -\cos(n 2\pi) }{ n^2 } - e^{- \pi} \frac{ -\cos(n 0) }{ n^2 } \right)
        \\&\qquad\qquad
        +
        \frac{1}{\pi}
        \int_{0}^{2\pi} e^{x-\pi}\frac{ -\cos(n x) }{ n^2 } dx
        \\&
        =
        \frac{1}{\pi}\frac{e^{\pi} - e^{-\pi}}{n^2}
        +
        \frac{1}{\pi}
        \int_{0}^{2\pi} e^{x-\pi}\frac{ -\cos(n x) }{ n^2 } dx
        \\&
        =
        \frac{1}{\pi}\frac{e^{\pi} - e^{-\pi}}{n^2}
        -
        \frac{1}{\pi}
        \int_{0}^{2\pi} e^{x-\pi}\frac{ \cos(n x) }{ n^2 } dx
        \end{align*}
        We rearrange that 
        \begin{align*}
            \int_{0}^{2\pi} e^{x-\pi} \cos(n x) dx
            &=
            \frac{e^{\pi} - e^{-\pi}}{ n^2}
            -
            \frac{1}{n^2}
            \int_{0}^{2\pi} e^{x-\pi} \cos(n x) dx
            \\
            \implies\qquad 
            \left( 1 + \frac 1 {n^2} \right)
            \int_{0}^{2\pi} e^{x-\pi} \cos(n x) dx
            &=
            \frac{e^{\pi} - e^{-\pi}}{ n^2}
            \\
            \implies\qquad 
            \int_{0}^{2\pi} e^{x-\pi} \cos(n x) dx
            &=
            \left( 1 + \frac 1 {n^2} \right)^{-1}
            \frac{e^{\pi} - e^{-\pi}}{ n^2}
            .   
        \end{align*}
        We conclude that 
        \begin{align*}
            a_n := 
            \frac{1}{\pi}
            \int_{0}^{2\pi} e^{x - \pi} \cos( n x) dx 
            &= 
            \left( 1 + \frac 1 {n^2} \right)^{-1}
            \frac{e^{\pi} - e^{-\pi}}{\pi n^2}
            \\&= 
            \left( \frac {n^2+ 1} {n^2} \right)^{-1}
            \frac{e^{\pi} - e^{-\pi}}{\pi n^2}
            \\&= 
            \frac{n^2}{1+n^2}
            \frac{e^{\pi} - e^{-\pi}}{\pi n^2}
            \\&= 
            \frac{e^{\pi} - e^{-\pi}}{(1+n^2)\pi}
        \end{align*}
        The coefficients of the sine modes are:
        \begin{align*}
            b_n 
            &
            = 
            \frac{1}{\pi}
            \int_{0}^{2\pi} e^{x - \pi} \sin(n x) dx
            \\&
            =
            \frac{1}{\pi}
            \left[- e^{x - \pi} \frac{ \cos( n x) }{ n } \right]_{x=0}^{x=2\pi}
            +
            \frac{1}{\pi}
            \int_{0}^{2\pi} \frac{ e^{x - \pi} \cos( n x) }{ n } dx
            \\&
            =
            \frac{1}{\pi}
             \left[- e^{x - \pi} \frac{ \cos( n x) }{ n } \right]_{x=0}^{x=2\pi}
            +
            \frac{1}{\pi}
            \left[ e^{x-\pi} \frac{ \sin(n x) }{ n^2 } \right]_{x=0}^{x=2\pi}
            -
            \frac{1}{\pi}
            \int_{0}^{2\pi} e^{x-\pi}\frac{ \sin(n x) }{n^2 } dx
            \\&
            =
            \frac{1}{\pi}\frac{e^{-\pi} - e^{\pi}}{n}
            -
            \frac{1}{\pi}
            \int_{0}^{2\pi} e^{x-\pi}\frac{ \sin(n x) }{n^2 } dx
        \end{align*}
        Similar as above, it follows that 
        \begin{align*}
            b_n:= \frac{1}{\pi}
                    \int_{0}^{2\pi} e^{x - \pi} \sin(n x) dx = \frac{1}{1 + \frac{1}{n^2}} \frac{1}{\pi}\frac{e^{-\pi} - e^{\pi}}{n} = \frac{1}{\pi}  \frac{e^{-\pi} - e^{\pi}}{n + \frac{1}{n}}
        \end{align*}
    %subquestion 2
    \item
% <<<<<<< HEAD
    We compute the average 
    \begin{align*}
        a_0
        :=
        \frac{1}{2\pi} \int_{0}^{2\pi} (x-\pi)^{3} dx
        =
        \frac{1}{2\pi} \int_{-\pi}^{\pi} z^{3} dz
        =
        0
        .
    \end{align*}
    We compute the coefficients of the cosine modes: for $n \geq 1$, we find 
    \begin{align*}
        a_n
        &:=
        \frac{2}{2\pi} \int_{0}^{2\pi} (x-\pi)^{3} \cos(nx) dx
        \\&=
        \frac{1}{\pi} \int_{0}^{2\pi} (x-\pi)^{3} \cos(nx) dx
        \\&=
        \frac{1}{\pi} \int_{-\pi}^{\pi} z^{3} \cos(n(z+\pi)) dz
        \\&=
        \frac{1}{\pi} \int_{-\pi}^{\pi} z^{3} \cos(nz+n\pi)) dz
        \\&=
        \frac{1}{\pi} \int_{-\pi}^{\pi} z^{3} \cos(nz)) dz
        \cdot \left\{\begin{array}{ll} 1 & \text{ if $n$ is even } \\ -1 & \text{ if $n$ is odd } \end{array}\right.
        .
    \end{align*}
    However, $z^3 \cos(nz)$ is the product of an odd and an even function, and so the integral vanishes. 
    Explicitly:
    \begin{align*}
        \int_{-\pi}^{\pi} z^{3} \cos(nz)) dz
        &=
        \int_{-\pi}^{0} z^{3} \cos(nz)) dz
        + 
        \int_{0}^{\pi} z^{3} \cos(nz)) dz
        \\&=
        - 
        \int_{0}^{\pi} w^{3} \cos(nw)) dw
        + 
        \int_{0}^{\pi} z^{3} \cos(nz)) dz
        = 
        0.
    \end{align*}
    Finally, we compute the coefficients of the sine modes:
    \begin{align*}
        b_n
        &:=
        \frac{2}{2\pi} \int_{0}^{2\pi} (x-\pi)^{3} \sin(nx) dx
        \\&=
        \frac{1}{\pi} \int_{0}^{2\pi} (x-\pi)^{3} \sin(nx) dx
        \\&=
        \frac{1}{\pi} \int_{-\pi}^{\pi} z^{3} \sin(n(z+\pi)) dx
        \\&=
        \frac{1}{\pi} \int_{-\pi}^{\pi} z^{3} \sin(nz) dz
        \cdot \left\{\begin{array}{ll} 1 & \text{ if $n$ is even } \\ -1 & \text{ if $n$ is odd } \end{array}\right.
        .
    \end{align*}
    In either case, since $z^3$ and $\sin(nz)$ are odd functions,
    \begin{align*}
        \int_{-\pi}^{\pi} z^{3} \sin(nz) dz
        &=
        2\int_{0}^{\pi} z^{3} \sin(nz) dz
        .
    \end{align*}
    We perform integration by parts several times:
    \begin{align*}
        &
        \int_{0}^{\pi} z^{3} \sin(nz) dz
        \\&=
        -
        \frac 1 {n} 
        \left[ z^{3} \cos(nz) \right]_{z=0}^{z=\pi}
        +
        \frac 3 {n}
        \int_{0}^{\pi} z^{2} \cos(nz) dz
        \\&=
        -
        \frac 1 {n} 
        \left[ z^{3} \cos(nz) \right]_{z=0}^{z=\pi}
        +
        \frac 3 {n^2} 
        \left[ z^{2} \sin(nz) \right]_{z=0}^{z=\pi}
        -
        \frac 6 {n^2} 
        \int_{0}^{\pi} z \sin(nz) dz
        \\&=
        - \frac 1 {n} 
        \left[ z^{3} \cos(nz) \right]_{z=0}^{z=\pi}
        + 
        \frac 3 {n^2}
        \left[ z^{2} \sin(nz) \right]_{z=0}^{z=\pi}
        +
        \frac 6 {n^3} 
        \left[ z \cos(nz) \right]_{z=0}^{z=\pi}
        -
        \frac 6 {n^3} 
        \int_{0}^{\pi} \cos(nz) dz
        \\&=
        - \frac 1 {n} 
        \left[ z^{3} \cos(nz) \right]_{z=0}^{z=\pi}
        + 
        \frac 3 {n^2}
        \left[ z^{2} \sin(nz) \right]_{z=0}^{z=\pi}
        +
        \frac 6 {n^3} 
        \left[ z \cos(nz) \right]_{z=0}^{z=\pi}
        +
        \frac 6 {n^4} 
        \left[ \sin(nz) \right]_{z=0}^{z=\pi}
        \\&=
	-\frac{\pi^3(-1)^n}{n}+\frac{6\pi(-1)^n}{n^3} 
	\\&\implies
            b_n = 
            \begin{cases}
            \frac{2\pi^2}{n}-\frac{12}{n^3}  ,\quad &\text{if } n\text{ is odd,}\\
            -\frac{2\pi^2}{n}+\frac{12}{n^3}   ,\quad &\text{if } n\text{ is even,}
            \end{cases}\cdot \left\{\begin{array}{ll} -1 & \text{ if $n$ is odd } \\ 1 & \text{ if $n$ is even } \end{array}\right. =  -\frac{2\pi^2}{n}+\frac{12}{n^3}
    \end{align*}
% 
% 
% =======
% To compute the Fourier coefficients of $g(x)$ over the interval $[0,2\pi)$ with a period of $2\pi$, we use the formulas from the lecture:
%         \[
%             a_0 = \frac{1}{2\pi}\int_{0}^{2\pi} f(x) dx  = \frac{1}{2\pi}\int_{0}^{2\pi} (x-\pi)^3 dx  = \left[ \frac{(x-\pi)^4}{4} \right]_{0}^{2\pi} = \frac{1}{2\pi}\frac 1 4 \pi^4 - \frac{1}{2\pi}\frac 1 4 (-\pi)^4 = 0.
%         \]
%         Note that the function is odd therefore $a_n = 0$. For $b_n$, we apply integration by parts thrice, and we use that $\sin$ vanishes at multiples of $\pi$:
%         \begin{align*}
%             b_n 
%             &
%             = 
%             \frac{2}{2\pi}
%             \int_{0}^{2\pi} (x-\pi)^3 \sin(n x) dx
%             \\&
%             =
%             \frac{2}{\pi}
%             \int_{0}^{\pi} (x-\pi)^3 \sin(n x) dx
%             \\&
%             =
%             \frac{2}{\pi}
%             \left[- (x-\pi)^3 \frac{ \cos(n x) }{n } \right]_{x=0}^{x=\pi}
%             +
%             \frac{2}{\pi}
%             \int_{0}^{\pi} \frac{ 3(x-\pi)^2 \cos(n x) }{ n } dx
%             \\&
%             =
%  		\frac{2}{\pi}
%             \left[- (x-\pi)^3 \frac{ \cos(n x) }{n } \right]_{x=0}^{x=\pi}
%             +
%             \frac{2}{\pi}
%             \left[ 3(x-\pi)^2 \frac{ \sin(n x) }{ n^2 } \right]_{x=0}^{x=\pi}
%             -
%             \frac{2}{\pi}
%             \int_{0}^{\pi} 6(x-\pi)\frac{ \sin( n x) }{n^2 } dx
%             \\&
%             =
%             \frac{2}{\pi}
%             \left[- (x-\pi)^3 \frac{ \cos(n x) }{n } \right]_{x=0}^{x=\pi}
%             +
%             \frac{2}{\pi}
%             \left[ 3(x-\pi)^2 \frac{ \sin(n x) }{ n^2 } \right]_{x=0}^{x=\pi}
%             +
%             \frac{2}{\pi}
%             \left[ 6(x-\pi)\frac{ \cos( n x) }{ n^3 } \right]_{x=0}^{x=\pi}
% 		-
%             \frac{2}{\pi}
%             \int_{0}^{\pi} 6\frac{ \cos( n x) }{ n^3 } dx
%             \\&
%             =
%  		\frac{2}{\pi}
%             \left[- (x-\pi)^3 \frac{ \cos(n x) }{n } \right]_{x=0}^{x=\pi}
%             +
%             \frac{2}{\pi}
%             \left[ 3(x-\pi)^2 \frac{ \sin(n x) }{ n^2 } \right]_{x=0}^{x=\pi}
%             -
%             \frac{2}{\pi}
%             \int_{0}^{\pi} 6(x-\pi)\frac{ \sin( n x) }{n^2 } dx
%             \\&
%             =
%             \frac{2}{\pi}
%             \left[- (x-\pi)^3 \frac{ \cos(n x) }{n } \right]_{x=0}^{x=\pi}
%             +
%             \frac{2}{\pi}
%             \left[ 3(x-\pi)^2 \frac{ \sin(n x) }{ n^2 } \right]_{x=0}^{x=\pi}
%             +
%             \frac{2}{\pi}
%             \left[ 6(x-\pi)\frac{ \cos( n x) }{ n^3 } \right]_{x=0}^{x=\pi}
% 		-
%             \frac{2}{\pi}
%             \left[ 6\frac{ \sin( n x) }{ n^3 } \right]_{x=0}^{x=\pi}
% 		\\&
%             =\frac{2}{\pi}
%             \left[- (x-\pi)^3 \frac{ \cos(n x) }{n } \right]_{x=0}^{x=\pi}
%             +
%             \frac{2}{\pi}
%             \left[ 6(x-\pi)\frac{ \cos( n x) }{ n^3 } \right]_{x=0}^{x=\pi}
% 		\\&
% 		= -\frac{2\pi^2}{n}+\frac{12}{n^3}
%         \end{align*}
% >>>>>>> 602337bdc4aee219b2f971dc33945611c0773aed
    %subquestion 3
    \item 
    The average of $h(x)$ over $1$ period $T$ is $0$ therefore:
    \[
        a_0 = 0
    \]
    Moreover $h(x)$ is odd therefore $a_n = 0$. For $b_n$ we have:
    \begin{align*}
        b_n 
        &
        = 
        \frac{2}{2\pi}
        \int_{0}^{\frac{\pi}{2}} \sin(x) \sin(n x) dx
        +
        \frac{2}{2\pi}
        \int_{\frac{3\pi}{2}}^{2\pi} \sin(x) \sin(n x) dx
        \\&
        =
        \frac{2}{2\pi}
        \int_{-\frac{\pi}{2}}^{\frac{\pi}{2}} \sin(x) \sin(n x) dx
        \\&
        =
        \frac{1}{\pi}
        \int_{-\frac{\pi}{2}}^{\frac{\pi}{2}} \sin(x) \sin(n x) dx
        \\&
        =
        \frac{2}{\pi}
        \int_{0}^{\frac{\pi}{2}} \sin(x) \sin(n x) dx
        \\&
        =
        \frac{1}{\pi}
        \int_{0}^{\frac{\pi}{2}} \cos((n-1)x) - \cos((n+1)x) dx
        \\&
        =
        \frac{1}{\pi}
        \left[ \frac{\sin((n-1)x)}{n-1} - \frac{\sin((n+1)x)}{n+1}\right]_{x=0}^{x=\frac{\pi}{2}}
        \\&
        =
        \frac{\sin((n-1)\frac{\pi}{2})}{n-1} - \frac{\sin((n+1)\frac{\pi}{2})}{n+1}
    \end{align*}
    If $n$ is odd, then $n-1$ and $n+1$ are both even, and so $(n-1)\frac{\pi}{2}$ and $(n+1)\frac{\pi}{2}$ will be integer multiples of $\pi$. 
    Hence $b_n$ is zero if $n$ is odd. 
    For the even numbers, we have got another case distinction:
    if $n=1,5,9,\dots$, then 
    \begin{align*}
        b_n = \frac{1}{n-1} - \frac{-1}{n+1} = \frac{1}{n-1} + \frac{1}{n+1} = \frac{n+1+n-1}{n^2+1} = \frac{2n}{n^2+1}.
    \end{align*}
    if $n=3,7,11,\dots$, then 
    \begin{align*}
        b_n = \frac{-1}{n-1} - \frac{1}{n+1} = \frac{-n-1-n+1}{n^2+1} = \frac{-2n}{n^2+1}.
    \end{align*}
    

    
\end{itemize}
\end{solution}


\begin{exercise}
    Compute the Fourier coefficients of the function $f$ with period $T = 2\pi$ and which satisfies 
     \[
        f(x) = \sin(x) \text{ if } 0 \leq x  < \pi,
     \]
     and 
     \[
        f(x) = f(-x) \text{ for all } x \in \mathbb R.
     \]
\end{exercise}
\begin{solution}     
    Note that $f(x)$ is an even function,
    therefore $b_n = 0$. 
    The average is 
	\[
        a_0 = \frac{2}{2\pi} \int_{0}^{\pi} \sin x dx  =  \frac{1}{\pi}\left[ - \cos{x} \right]_{0}^{\pi} =\frac{2}{\pi} .
    \]
    The cosine mode coefficients are:
    \begin{align*}
            a_n 
            &
            = 
            \frac{2}{T}
            \int_{-\pi}^{\pi} f(x)   \cos(n x) dx
            \\&
            = 
            \frac{4}{T}
            \int_{0}^{\pi} \sin x \cos(n x) dx
            \\&
            =
            \frac{4}{T}
            \int_{0}^{\pi} \frac 1 2 \sin((1-n)x) + \frac 1 2 \sin((n+1)x) dx
            \\&
            =
            \frac{2}{T}
            \int_{0}^{\pi} \sin((1-n)x) + \sin((n+1)x) dx
            \\&
            =
            \frac{2}{T}
            \int_{0}^{\pi} -\sin((n-1)x) + \sin((n+1)x) dx
            \\&
            =
            \frac{2}{T}
            \left[\frac{ \cos((n-1)x) }{ (n-1) } \right]_{x=0}^{x=\pi}
            +
            \frac{2}{T}
            \left[- \frac{ \cos((n+1)x) }{ (n+1) } \right]_{x=0}^{x=\pi}
            \\&
            = 
            \frac 2 T 
            \left( 
                \frac{ \cos((n-1)\pi) }{(n-1)} 
                - 
                \frac{1}{(n-1)}
                -
                \frac{ \cos((n+1)\pi) }{(n+1)} 
                + 
                \frac{1}{(n+1)}
            \right)
            \\&
            = 
            \frac 2 T 
            \left( 
                \frac{ \cos( n \pi - \pi ) }{(n-1)} 
                - 
                \frac{1}{(n-1)}
                -
                \frac{ \cos( n \pi + \pi ) }{(n+1)} 
                + 
                \frac{1}{(n+1)}
            \right)
            \\&
            = 
            \frac 2 T 
            \left( 
                -
                \frac{ \cos(n\pi) }{(n-1)} 
                - 
                \frac{1}{(n-1)}
                +
                \frac{ \cos(n\pi) }{(n+1)} 
                + 
                \frac{1}{(n+1)}
            \right)
            \\&
            = 
            \frac 2 T 
            \left( 
                -
                \frac{ (-1)^n }{(n-1)} 
                - 
                \frac{1}{(n-1)}
                +
                \frac{ (-1)^n }{(n+1)} 
                + 
                \frac{1}{(n+1)}
            \right)
            \\&
            = 
            \frac 2 T 
            \left( 
                \frac{ 1 + (-1)^n }{(n+1)} 
                -
                \frac{ 1 + (-1)^n }{(n-1)} 
            \right)
            \\&
            = 
            \frac 1 \pi 
            \left( 
                \frac{ 1 + (-1)^n }{(n+1)} 
                -
                \frac{ 1 + (-1)^n }{(n-1)} 
            \right)
            .
        \end{align*}	
        Here, if $n$ is odd, the coefficient numerators are $1 + (-1) = 0$, and if $n$ is even, they are both equal $2$. Thus:
        \[
            a_n = 
            \begin{cases}
            0 ,\quad &\text{if } n\text{ is odd,}\\
            -\frac{2}{\pi(n-1)} + \frac{2}{\pi(n+1)}  ,\quad &\text{if } n\text{ is even,}
            \end{cases}
        \]
\end{solution}


\end{document}
