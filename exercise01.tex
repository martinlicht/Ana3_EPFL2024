\documentclass[11pt]{article}
%\def\hidesolutions{}
%%%%%%%%%%% SET MARGINS
\setlength{\textheight}{20cm}
\setlength{\topmargin}{-0.5cm}
\setlength{\oddsidemargin}{+0cm}
\setlength{\textwidth}{16.3cm}
%\setlength{\parskip}{6pt}
\setlength{\parindent}{0pt}

%%%%%%%%%%% PACKAGES
\usepackage{amsmath}
\usepackage{amssymb}
\usepackage{amsfonts}
%\usepackage{a4wide}
\usepackage{graphicx}
\usepackage{color}
\usepackage[normalem]{ulem}
\usepackage{enumitem}
\usepackage{capt-of}
\usepackage{float}
\usepackage{amsmath}
\usepackage{listings}
\definecolor{mygreen}{RGB}{28,172,0} % color values Red, Green, Blue
\definecolor{mylilas}{RGB}{170,55,241}
\usepackage{empheq}
\usepackage[ruled]{algorithm2e}
\usepackage{mathrsfs}
\usepackage{datetime}
\usepackage{subcaption}

% TODO: combine the two package lists and reduce redundancies 
\usepackage{mathtools}
\usepackage{nicefrac}
\usepackage{hyperref}
\usepackage{url}
\usepackage{amsmath,amssymb,amsfonts}
\usepackage{a4wide}
\usepackage{graphicx}
\usepackage{color}
\usepackage[normalem]{ulem}
\usepackage{capt-of}
\usepackage{float}
\usepackage[ruled]{algorithm2e}
\usepackage{amsmath,amssymb,amsfonts}
\usepackage{a4wide}
\usepackage{graphicx}
\usepackage{color}
\usepackage[normalem]{ulem}
\usepackage{capt-of}
\usepackage{float}
\usepackage[ruled]{algorithm2e}
\usepackage{mathrsfs}







\newcommand{\Lc}[2]{{\color{blue} \sout{#1} } \textcolor{red}{#2}}
\newcommand{\La}[1]{\textcolor{red}{#1}}
\newcommand{\lh}{\mathscr{L}_h}
\newcommand{\cl}{\mathscr{L}}
\newcommand{\cf}{\mathscr{F}}
\newcommand{\dx}{dx}
\newcommand{\ltn}{\mathscr{l}^2}
\newcommand{\bbR}{\mathbb{R}}
\newcommand{\Rset}{\mathbb{R}}
\newcommand{\Nset}{\mathbb{N}}
\newcommand{\scL}{\mathcal{L}}
\newcommand{\xx}{\mathbf{x}}
\newcommand{\norm}[1]{\|{#1}\|}
\newcommand{\yy}{\mathbf{y}}
\newcommand{\at}[1]{\big|_{#1}}
\renewcommand{\div}{\mathrm{div}}
\newcommand{\divergence}{\mathrm{div}}
\newcommand{\cp}[1]{\textcolor{blue}{#1}}

\newcommand{\FF}{\texttt{FreeFem++ }}
\newcommand{\FFns}{\texttt{FreeFem++}}
\newcommand{\FFfull}{\texttt{FreeFem++-x11}}
\newcommand{\cmd}[1]{ \medskip \noindent \texttt{#1} \medskip}
\newcommand{\incmd}[1]{\texttt{#1}}
\newcommand{\shrinkitems}{\addtolength{\itemsep}{-0.5\baselineskip}}
\newcommand{\mtt}[1]{\mathtt{#1}}
\newcommand{\ML}{\texttt{Matlab }}

\newcommand{\bb}{\mathbf{b}}
\newcommand{\nn}{\mathbf{n}}
\newcommand{\vecA}{\vec{A}}
\newcommand{\vecB}{\vec{B}}


\newcommand{\mesh}{\mathcal{T}_h}
\newcommand{\refel}{\widehat{K}}
\newcommand{\ver}{\mathbf{a}}
\newcommand{\refver}{\widehat{\mathbf{a}}}
\newcommand{\grad}{\nabla}
\newcommand{\refgrad}{\widehat{\nabla}}
\newcommand{\refu}{\widehat{u}}
\newcommand{\refbasis}{\widehat{\varphi}}
\newcommand{\refxx}{\widehat{\xx}}
\newcommand{\refx}{\widehat{x}}
\newcommand{\refy}{\widehat{y}}
\newcommand{\refrho}{\widehat{\rho}}
\newcommand{\refh}{\widehat{h}}






% For typesetting Python code
\newcommand{\matlab}{{\sc Matlab}\xspace}
\usepackage{listings}
\lstloadlanguages{Python}
\lstloadlanguages{csh}%
\definecolor{MyDarkGreen}{rgb}{0.0,0.4,0.0}
\definecolor{purple}{rgb}{0.58,0,0.82}
\lstset{language=Python,                    % Use Python
	%frame=single,                          % Single frame around code
	basicstyle=\ttfamily\footnotesize\color{black},
	keywordstyle=[1]\color{blue}\bf,        % Python functions bold and blue
	keywordstyle=[2]\color{purple},         % Python function arguments purple
	keywordstyle=[3]\color{red}\underbar,   % User functions underlined and blue
	commentstyle=\usefont{T1}{pcr}{m}{sl}\color{MyDarkGreen}\small,
	stringstyle=\color{purple},
	showstringspaces=false,                 % Don't put marks in string spaces
	tabsize=3,                              % 5 spaces per tab
	morekeywords={xlim,ylim,var,alpha,factorial,poissrnd,normpdf,normcdf},
	morecomment=[l][\color{blue}]{...},
	breaklines=true,
	breakatwhitespace=true,
	emptylines=1,
	mathescape=true,
	xleftmargin=0ex,
	emphstyle=\bfseries\color{red}
}





%%%%%%%%%%% MACROS NAMES
\newcommand{\lecturername}{Martin Licht}
% \newcommand{\assistantnamea}{Jochen Hinz}
% \newcommand{\assistantnameb}{Ivan Bioli}
\newcommand{\semestername}{Winter Semester 2023}
\newcommand{\lecturename}{Analysis III - 202(c)}
\DeclarePairedDelimiter\floor{\lfloor}{\rfloor}

%%%%%%%%%%% HEADER
\newdateformat{yeardate}{\THEYEAR}
\newcommand{\exsheet}[3] % input is the number of the session and the day TODO What's that
{\clearpage

	\begin{center}
		{\Large \textbf{\lecturename}}\\[2ex]
		\semestername
	\end{center}

	% \vspace{2ex}
	% \lecturername

	\vspace{2ex}
	{\Large Session #1: #3\,#2, \yeardate\today}
	%\hfill
	%{\Large EPF Lausanne}

	\hrulefill
}





\usepackage{comment}

\newtheorem{exercise}{Exercise}
\newtheorem{solutionenv}{Solution}

\newboolean{hide_solution}
\ifx\hidesolutions\undefined
\newenvironment{solution}{\begin{solutionenv}}{\end{solutionenv}}
\setboolean{hide_solution}{false}
\else
\excludecomment{solution}
\setboolean{hide_solution}{true}
\fi

\newcommand{\ifnotsolution}[1]{\ifthenelse{\boolean{hide_solution}}{#1}{}}
\newcommand{\ifsolution}[1]{\ifthenelse{\boolean{hide_solution}}{}{#1}}








\allowdisplaybreaks

\begin{document}
\exsheet{1}{19}{September} % parameters are the number of the session and the day


\begin{exercise}
	Compute the norms (that is, lengths), the $3$ scalar products and $3$ vector products of the following vectors. 
    Find the volume of the parallelepiped spanned by these vectors.
    \[
    \vec{a} = \left(\begin{array}{c} 1 \\ 2 \\ -1 \end{array}\right)
    ,
    \vec{b} = \left(\begin{array}{c} 2 \\ -1 \\ 0 \end{array}\right)
    ,
    \vec{c} = \left(\begin{array}{c} 1 \\ 1 \\ 3 \end{array}\right)
    \]
\end{exercise}

\begin{solution}
    \begin{gather*}
        \vec{a} = \sqrt{6}, \quad \vec{b} = \sqrt{5}, \quad \vec{c} = \sqrt{11}
        \\
        \vec{a} \cdot \vec{b} = 0,
        \quad 
        \vec{a} \cdot \vec{c} = 0,
        \quad 
        \vec{b} \cdot \vec{c} = 1,
        \\ 
        \vec{a} \times \vec{b} = \left(\begin{array}{c} -1 \\ -2 \\ -5 \end{array}\right),
        \quad 
        \vec{a} \times \vec{c} = \left(\begin{array}{c} 7 \\ -4 \\ -1 \end{array}\right),
        \quad 
        \vec{b} \times \vec{c} = \left(\begin{array}{c} -3 \\ -6 \\ 3 \end{array}\right),
    \end{gather*}
    The volume of the paralleliped spanned by the three vectors is $18$.
    
    
\end{solution}


\begin{exercise}
	Consider the function 
    \[
        f(x,y) = \sin\left( 2\pi x^2 + 2\pi y^2 \right)
    \]
    Find the first derivatives in $x$ and $y$.
    Sketch the level sets of $f$ for the values $0$ and $1$.
\end{exercise}

\begin{solution}
    We easily compute 
    \[
        \partial_x f(x,y) = \cos\left( 2\pi x^2 + 2\pi y^2 \right) \cdot 4\pi \cdot x,
        \quad 
        \partial_y f(x,y) = \cos\left( 2\pi x^2 + 2\pi y^2 \right) \cdot 4\pi \cdot y.
    \]
    The level sets are concentric circles centered at zero. One approach to understanding works as follows.
    First, we rewrite 
    \[
        f(x,y) = \sin\left( 2\pi ( x^2 + y^2 ) \right)
    \]
    We remember 
    that 
    \[
        \sin( 2\pi t ) = 0 \text{ for the positive values } t = 0, \frac 1 2, 1, \frac 3 2, \dots
    \] 
    and 
    that 
    \[
        \sin( 2\pi t ) = 1 \text{ for the positive values } t = \frac 1 4, \frac 5 4, \frac 9 4, \dots.
    \]
    Hence, whenever $x^2 + y^2$ hits one these values listed above, then we are on the level of $0$ or $1$.
    Specifically, $f(x,y) = 0$ if $(x,y)$ lies on a circle centered at zero and with radius
    \[
        0, \sqrt{\frac 1 2}, \sqrt{1}, \sqrt{\frac 3 2}, \sqrt{2}, \dots, 
    \]
    Similarly, $f(x,y) = 1$ if $(x,y)$ lies on a circle centered at zero and with radius
    \[
        0, \sqrt{\frac 1 4}, \sqrt{\frac 5 4}, \sqrt{\frac 9 2}, \sqrt{\frac {13} 4}, \dots. 
    \]
\end{solution}





\begin{exercise}
	Sketch the following vector field:
    \[
        f(x,y) = \left(\begin{array}{c} \sin(x) \\ sin(y) \end{array}\right)
    \]
\end{exercise}

\begin{solution}
    Please do this in the exercise session.
\end{solution}



\begin{exercise}
	Compute the following integrals:
    \[
        \int_0^1 \int_0^1 x^2 e^y \; dx dy,
        \quad 
        \int_0^1 x^2 \cos(x) \; dx.
    \]
\end{exercise}

\begin{solution}
    For the first integral, we can either split up the integral into two factors, or we compute one after the other.
    We find 
    \[
        \int_0^1 \int_0^1 x^2 e^y \; dx dy
        =
        \int_0^1 x^2 \; dx 
        \cdot 
        \int_0^1 e^y \; dy
        =
        \frac 1 3 \cdot ( e - 1).
    \]
    For the second integral, we use integration by parts. 
    \begin{align*}
        \int_0^1 x^2 \cos(x) \; dx
        &=
        x^2 \sin(x) |_0^1
        -
        \int_0^1 2x \sin(x) \; dx
        \\&=
        x^2 \sin(x) |_0^1
        -
        \left( 2x (-\cos(x)) |_0^1 - \int_0^1 2 (-\cos(x)) \; dx \right)
        \\&=
        x^2 \sin(x) |_0^1
        -
        2x (-\cos(x)) |_0^1 
        +
        \int_0^1 2 (-\cos(x)) \; dx
        \\&=
        x^2 \sin(x) |_0^1
        -
        2x (-\cos(x)) |_0^1 
        -
        2 \int_0^1 \cos(x) \; dx
        \\&=
        x^2 \sin(x) |_0^1
        -
        2x (-\cos(x)) |_0^1 
        -
        2 \left( \sin(1) - \sin(0) \right)
        \\&=
        \sin(1)
        -
        2 (-\cos(1)) 
        -
        2 \sin(1)
        \\&=
        2 \cos(1)
        - 
        \sin(1)
        .
    \end{align*}
\end{solution}




\end{document}
