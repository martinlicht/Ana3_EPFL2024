\documentclass[11pt]{article}
%%%%%%%%%%% SET MARGINS
\setlength{\textheight}{20cm}
\setlength{\topmargin}{-0.5cm}
\setlength{\oddsidemargin}{+0cm}
\setlength{\textwidth}{16.3cm}
%\setlength{\parskip}{6pt}
\setlength{\parindent}{0pt}

%%%%%%%%%%% PACKAGES
\usepackage{amsmath}
\usepackage{amssymb}
\usepackage{amsfonts}
%\usepackage{a4wide}
\usepackage{graphicx}
\usepackage{color}
\usepackage[normalem]{ulem}
\usepackage{enumitem}
\usepackage{capt-of}
\usepackage{float}
\usepackage{amsmath}
\usepackage{listings}
\definecolor{mygreen}{RGB}{28,172,0} % color values Red, Green, Blue
\definecolor{mylilas}{RGB}{170,55,241}
\usepackage{empheq}
\usepackage[ruled]{algorithm2e}
\usepackage{mathrsfs}
\usepackage{datetime}
\usepackage{subcaption}

% TODO: combine the two package lists and reduce redundancies 
\usepackage{mathtools}
\usepackage{nicefrac}
\usepackage{hyperref}
\usepackage{url}
\usepackage{amsmath,amssymb,amsfonts}
\usepackage{a4wide}
\usepackage{graphicx}
\usepackage{color}
\usepackage[normalem]{ulem}
\usepackage{capt-of}
\usepackage{float}
\usepackage[ruled]{algorithm2e}
\usepackage{amsmath,amssymb,amsfonts}
\usepackage{a4wide}
\usepackage{graphicx}
\usepackage{color}
\usepackage[normalem]{ulem}
\usepackage{capt-of}
\usepackage{float}
\usepackage[ruled]{algorithm2e}
\usepackage{mathrsfs}







\newcommand{\Lc}[2]{{\color{blue} \sout{#1} } \textcolor{red}{#2}}
\newcommand{\La}[1]{\textcolor{red}{#1}}
\newcommand{\lh}{\mathscr{L}_h}
\newcommand{\cl}{\mathscr{L}}
\newcommand{\cf}{\mathscr{F}}
\newcommand{\dx}{dx}
\newcommand{\ltn}{\mathscr{l}^2}
\newcommand{\bbR}{\mathbb{R}}
\newcommand{\Rset}{\mathbb{R}}
\newcommand{\Nset}{\mathbb{N}}
\newcommand{\scL}{\mathcal{L}}
\newcommand{\xx}{\mathbf{x}}
\newcommand{\norm}[1]{\|{#1}\|}
\newcommand{\yy}{\mathbf{y}}
\newcommand{\at}[1]{\big|_{#1}}
\renewcommand{\div}{\mathrm{div}}
\newcommand{\divergence}{\mathrm{div}}
\newcommand{\cp}[1]{\textcolor{blue}{#1}}

\newcommand{\FF}{\texttt{FreeFem++ }}
\newcommand{\FFns}{\texttt{FreeFem++}}
\newcommand{\FFfull}{\texttt{FreeFem++-x11}}
\newcommand{\cmd}[1]{ \medskip \noindent \texttt{#1} \medskip}
\newcommand{\incmd}[1]{\texttt{#1}}
\newcommand{\shrinkitems}{\addtolength{\itemsep}{-0.5\baselineskip}}
\newcommand{\mtt}[1]{\mathtt{#1}}
\newcommand{\ML}{\texttt{Matlab }}

\newcommand{\bb}{\mathbf{b}}
\newcommand{\nn}{\mathbf{n}}
\newcommand{\vecA}{\vec{A}}
\newcommand{\vecB}{\vec{B}}


\newcommand{\mesh}{\mathcal{T}_h}
\newcommand{\refel}{\widehat{K}}
\newcommand{\ver}{\mathbf{a}}
\newcommand{\refver}{\widehat{\mathbf{a}}}
\newcommand{\grad}{\nabla}
\newcommand{\refgrad}{\widehat{\nabla}}
\newcommand{\refu}{\widehat{u}}
\newcommand{\refbasis}{\widehat{\varphi}}
\newcommand{\refxx}{\widehat{\xx}}
\newcommand{\refx}{\widehat{x}}
\newcommand{\refy}{\widehat{y}}
\newcommand{\refrho}{\widehat{\rho}}
\newcommand{\refh}{\widehat{h}}






% For typesetting Python code
\newcommand{\matlab}{{\sc Matlab}\xspace}
\usepackage{listings}
\lstloadlanguages{Python}
\lstloadlanguages{csh}%
\definecolor{MyDarkGreen}{rgb}{0.0,0.4,0.0}
\definecolor{purple}{rgb}{0.58,0,0.82}
\lstset{language=Python,                    % Use Python
	%frame=single,                          % Single frame around code
	basicstyle=\ttfamily\footnotesize\color{black},
	keywordstyle=[1]\color{blue}\bf,        % Python functions bold and blue
	keywordstyle=[2]\color{purple},         % Python function arguments purple
	keywordstyle=[3]\color{red}\underbar,   % User functions underlined and blue
	commentstyle=\usefont{T1}{pcr}{m}{sl}\color{MyDarkGreen}\small,
	stringstyle=\color{purple},
	showstringspaces=false,                 % Don't put marks in string spaces
	tabsize=3,                              % 5 spaces per tab
	morekeywords={xlim,ylim,var,alpha,factorial,poissrnd,normpdf,normcdf},
	morecomment=[l][\color{blue}]{...},
	breaklines=true,
	breakatwhitespace=true,
	emptylines=1,
	mathescape=true,
	xleftmargin=0ex,
	emphstyle=\bfseries\color{red}
}





%%%%%%%%%%% MACROS NAMES
\newcommand{\lecturername}{Martin Licht}
% \newcommand{\assistantnamea}{Jochen Hinz}
% \newcommand{\assistantnameb}{Ivan Bioli}
\newcommand{\semestername}{Winter Semester 2023}
\newcommand{\lecturename}{Analysis III - 202(c)}
\DeclarePairedDelimiter\floor{\lfloor}{\rfloor}

%%%%%%%%%%% HEADER
\newdateformat{yeardate}{\THEYEAR}
\newcommand{\exsheet}[3] % input is the number of the session and the day TODO What's that
{\clearpage

	\begin{center}
		{\Large \textbf{\lecturename}}\\[2ex]
		\semestername
	\end{center}

	% \vspace{2ex}
	% \lecturername

	\vspace{2ex}
	{\Large Session #1: #3\,#2, \yeardate\today}
	%\hfill
	%{\Large EPF Lausanne}

	\hrulefill
}





\usepackage{comment}

\newtheorem{exercise}{Exercise}
\newtheorem{solutionenv}{Solution}

\newboolean{hide_solution}
\ifx\hidesolutions\undefined
\newenvironment{solution}{\begin{solutionenv}}{\end{solutionenv}}
\setboolean{hide_solution}{false}
\else
\excludecomment{solution}
\setboolean{hide_solution}{true}
\fi

\newcommand{\ifnotsolution}[1]{\ifthenelse{\boolean{hide_solution}}{#1}{}}
\newcommand{\ifsolution}[1]{\ifthenelse{\boolean{hide_solution}}{}{#1}}








\allowdisplaybreaks

\begin{document}
\exsheet{2}{26}{September} % parameters are the number of the session and the day
\def\hidesolutions{}

\begin{exercise}
	Sketch the level sets of the functions 
    \[
        f(x_1,x_2) = x_1 + x_2, \quad g(x_1,x_2) = x_1^2 x_2 
    \]
    for the values $0$, $1$, and $-1$.
    Compute the gradients $\nabla f$ and  $\nabla g$.
\end{exercise}

\begin{solution}
    \[
        \nabla f(x_1,x_2) = \begin{pmatrix} 1\\ 1 \end{pmatrix}, \quad \nabla g(x_1,x_2) =  \begin{pmatrix} 2x_1x_2\\ x_1^2 \end{pmatrix}
    \]


For the level sets of $f(x_1,x_2)$ consider:
    \[
        x_1 + x_2 = C \implies x_2 = -x_1 + C
    \]
therefore the level sets are lines that intersect the axis $x_2$ at $C$ with slope $-1$. One can use a similar approach for $g(x_1,x_2)$. Note that for $C =0$:
    \[
        x_1^2 x_2 = 0 \implies x_1 =0 \text{ or } x_2 = 0
    \]
which gives that the level sets for $C = 0$ are the two axis $x_1$ and $x_2$. 
\end{solution}


\begin{exercise}
	Compute the curl and the divergence of the following vector field:
    \[
        \vec{f}(x_1,x_2,x_3) = \left( \; x_1 x_2, \; x_3^2 x_1,\;  e^{x_1 - x_2} \; \right)
    \]
\end{exercise}

\begin{solution}
    \[
        \nabla \times \vec{f}(x_1,x_2,x_3) = \left( \; -e^{x_1-x_2} - 2x_1x_3, \; -e^{x_1-x_2},\;  x_3^2 - x_1 \; \right)
    \]
   \[
        \nabla \cdot \vec{f}(x_1,x_2,x_3) = x_2
    \]
\end{solution}





\begin{exercise}
	Show that for every three-dimensional vector field $\vec{f} \colon \mathbb{R}^3 \to \mathbb{R}^3$ we have 
    \[
        \nabla \cdot \left( \nabla \times \vec{f} \right) = 0
    \]
    Show that for every scalar field $g \colon \mathbb{R}^3 \to \mathbb{R}$ we have 
    \[
        \nabla \times \left( \nabla g \right) = 0
    \]
    Consider the following vector field:
    \[
        \vec{A}(x_1,x_2,x_3) := \left( \; \sin(x_2),  \; \sin(x_3), \; \sin(x_1) \; \right)
    \]
    Is this vector field a gradient of some scalar field?
\end{exercise}

\begin{solution}
Let $\vec{f} \colon \mathbb{R}^3 \to \mathbb{R}^3$ then:
\begin{align*}
     \nabla \cdot (\nabla \times \vec{f}(x_1,x_2,x_3)) &= (\partial_{x_1}, \partial_{x_2}, \partial_{x_3}) \cdot (\partial _{x_2} f_3 - \partial _{x_3} f_2,\partial _{x_3} f_1 - \partial _{x_1} f_3,\partial _{x_1} f_2 - \partial _{x_2} f_1)  \\
 &=  \partial_{x_1 x_2}f_3 - \partial_{x_1 x_3}f_2 + \partial _{x_2 x_3} f_1 - \partial _{x_2 x_1} f_3 + \partial _{x_3 x_1} f_2 - \partial _{x_3 x_2} f_1
\end{align*}

We can interchange the order of the partial derivatives, since $\vec{f} \in \mathrm{C}^2$ to arrive at:

\begin{align*}
 \nabla \cdot (\nabla \times \vec{f}(x_1,x_2,x_3)) &=	 \partial_{x_1 x_2}f_3 - \partial _{x_2 x_1} f_3 + \partial _{x_3 x_1} f_2 - \partial_{x_1 x_3}f_2 + \partial _{x_2 x_3} f_1  - \partial _{x_3 x_2} f_1\\
&=	 \partial_{x_2 x_1}f_3 - \partial _{x_2 x_1} f_3 + \partial _{x_1 x_3} f_2 - \partial_{x_1 x_3}f_2 + \partial _{x_3 x_2} f_1  - \partial _{x_3 x_2} f_1 = 0\\
&\implies  \nabla \cdot (\nabla \times \vec{f}(x_1,x_2,x_3)) =0
\end{align*}

Let  $g \colon \mathbb{R}^3 \to \mathbb{R}$ then: 
\begin{align*}
	\nabla \times (\nabla g(x_1,x_2,x_3)) &= \nabla \times \begin{pmatrix}\partial_{x_1} g \\ \partial_{x_2} g \\ \partial_{x_3} g \end{pmatrix} = \begin{pmatrix} \partial_{x_2x_3} g - \partial_{x_3x_2} g \\ \partial_{x_3x_1} g - \partial_{x_1x_3} g \\ \partial_{x_1x_2} g - \partial_{x_2x_1} g \end{pmatrix} = \begin{pmatrix} \partial_{x_2x_3} g - \partial_{x_2x_3} g \\ \partial_{x_1x_3} g - \partial_{x_1x_3} g \\ \partial_{x_1x_2} g - \partial_{x_1x_2} g \end{pmatrix} = \vec{0}\\
&\implies \nabla \times (\nabla g(x_1,x_2,x_3))  = 0
\end{align*}

    \[
        \vec{\nabla} \times \vec{A}(x_1,x_2,x_3)
        :=
        \left(\;
            -\cos(x_3),  \cos(x_1), -\cos(x_2)
        \;\right)
    \]
    The curl of this vector field is not zero. 
    If it were a gradient, then it would be zero. 
    So it cannot be a gradient of any function.
     
\end{solution}



\begin{exercise}
	Given scalar fields $f, g : \mathbb{R}^3 \to \mathbb{R}$ and vector fields $\vec{A}, \vec{B} : \mathbb{R}^3 \to \mathbb{R}^3$,
    show that 
    \begin{gather*}
        \nabla\left( f \cdot g \right)
        = 
        f \cdot \nabla g
        +
        g \cdot \nabla f
        ,
        \\
        \nabla\cdot\left( f \cdot \vec{A} \right)
        = 
        ( \nabla f ) \cdot \vec{A}
        +
        f ( \nabla \cdot \vec{A} )
        ,
        \\ 
        \nabla\cdot\left( \vec{A} \times \vec{B} \right)
        = 
        \left( \nabla\times\vec{A} \right) \cdot \vec{B}
        -
        \vec{A} \cdot \left( \nabla\times\vec{B} \right)
        ,
        \\ 
        \Delta ( f g )
        = 
        f \Delta g + 2 \nabla(f) \cdot \nabla( g ) + g \Delta f
        .
    \end{gather*}
    Suppose that we also have two real numbers $\alpha, \beta \in \mathbb R$.
    Show that 
    \begin{gather*}
        \nabla( \alpha f + \beta g )
        = 
        \alpha \nabla f + \beta \nabla g
        ,
        \\ 
        \nabla\cdot ( \alpha \vec{A} + \beta \vec{B} )
        = 
        \alpha \nabla\cdot \vec{A} + \beta \nabla\cdot \vec{B}
        ,
        \\ 
        \nabla\times ( \alpha \vec{A} + \beta \vec{B} )
        = 
        \alpha \nabla\times \vec{A} + \beta \nabla\times \vec{B}
        .
    \end{gather*}
\end{exercise}

\begin{solution}
    For the first identity, we use the product rule:
    \begin{align*}
        \nabla\left( f \cdot g \right)
        &= 
        \left( \partial_1 ( f \cdot g ), \cdots, \partial_n ( f \cdot g ) \right)
        \\&= 
        \left( \partial_1 f \cdot g + f \cdot \partial_1 g, \cdots, \partial_n f \cdot g + f \cdot \partial_n g \right)
        \\&= 
        \left( \partial_1 f \cdot g, \cdots, \partial_n f \cdot g \right)
        +
        \left( f \cdot \partial_1 g, \cdots, f \cdot \partial_n g \right)
        \\&= 
        g \left( \partial_1 f, \cdots, \partial_n f \right)
        +
        f \left( \partial_1 g, \cdots, \partial_n g \right)
        \\&= 
        f \cdot \nabla g
        +
        g \cdot \nabla f
        .
    \end{align*}
    The second identity, again, can be proved using the product rule:
    \begin{align*}
        \nabla\cdot\left( f \cdot \vec{A} \right)
        &= 
        \partial_1 ( f \cdot A_{1} ) + \cdots + \partial_n ( f \cdot A_{n} )
        \\&= 
        \partial_1 ( f \cdot A_{1} ) + \cdots + \partial_n ( f \cdot A_{n} )
        \\&= 
        ( \partial_1 f ) \cdot A_{1} ) + \cdots + ( \partial_n f ) \cdot A_{n}
        +
        f \cdot \partial_1 ( A_{1} ) + \cdots + f \cdot \partial_n ( A_{n} )
        \\&= 
        \nabla(f) \cdot \vecA
        +
        f \cdot \nabla \cdot \vecA
        ,
    \end{align*}
    As for the third identity:
    \begin{align*}
        \nabla\cdot\left( \vec{A} \times \vec{B} \right)
        &=
        \nabla\cdot\left( 
            A_2 B_3 - A_3 B_2,
          - A_1 B_3 + A_3 B_1,
            A_1 B_2 - A_2 B_1 
        \right)
        \\&=
        \partial_1\left(   A_2 B_3 - A_3 B_2 \right) + \partial_2\left( - A_1 B_3 + A_3 B_1 \right) + \partial_3\left(   A_1 B_2 - A_2 B_1 \right)
        \\&=
        + \partial_1\left( A_2 B_3 \right) - \partial_1\left( A_3 B_2 \right) 
        \\&\qquad
        - \partial_2\left( A_1 B_3 \right) + \partial_2\left( A_3 B_1 \right) 
        \\&\qquad
        + \partial_3\left( A_1 B_2 \right) - \partial_3\left( A_1 B_2 \right)
        \\&=
        + \partial_1\left( A_2 \right) B_3 + A_2 \partial_1\left( B_3 \right) - \partial_1\left( A_3 \right) B_2 - A_3 \partial_1\left( B_2 \right) 
        \\&\qquad
        - \partial_2\left( A_1 \right) B_3 - A_1 \partial_2\left( B_3 \right) + \partial_2\left( A_3 \right) B_1 + A_3 \partial_2\left( B_1 \right) 
        \\&\qquad
        + \partial_3\left( A_1 \right) B_2 + A_1 \partial_3\left( B_2 \right) - \partial_3\left( A_2 \right) B_1 - A_2 \partial_3\left( B_1 \right)
        \\&=
         + \partial_2\left( A_3 \right) B_1 - \partial_3\left( A_2 \right) B_1 
         - \partial_1\left( A_3 \right) B_2 + \partial_3\left( A_1 \right) B_2
         + \partial_1\left( A_2 \right) B_3 - \partial_2\left( A_1 \right) B_3 
        \\&\qquad
         + A_1 \partial_3\left( B_2 \right) - A_1 \partial_2\left( B_3 \right)
         - A_2 \partial_3\left( B_1 \right) + A_2 \partial_1\left( B_3 \right) 
         + A_3 \partial_2\left( B_1 \right) - A_3 \partial_1\left( B_2 \right) 
        \\&=
        ( \nabla\times\vecA ) \cdot \vecB - ( \nabla\times\vecB ) \cdot \vecA
        .
    \end{align*}
    We can prove the fourth identity again via direct computation, which is perfectly valid. 
    We can also use some of the identities already shown above to reduce the effort:
    \begin{align*}
        \Delta ( f g )
        &= 
        \nabla \cdot ( \nabla ( f g ) )
        \\&= 
        \nabla \cdot ( g \nabla(f) + f \nabla(g) )
        \\&= 
        \partial_1( g \nabla(f)_1 + f \nabla(g)_1 ) + \cdots + \partial_n( g \nabla(f)_n + f \nabla(g)_n )
        \\&= 
        \partial_1( g \nabla(f)_1 ) + \partial_1( f \nabla(g)_1 ) + \cdots + \partial_n( g \nabla(f)_n ) + \partial_n( f \nabla(g)_n )
        \\&= 
        \partial_1( g \nabla(f)_1 ) + \cdots + \partial_n( g \nabla(f)_n ) + \partial_1( f \nabla(g)_1 ) + \cdots + \partial_n( f \nabla(g)_n )
        \\&= 
        \nabla \cdot ( g \nabla(f) ) + \nabla \cdot ( f \nabla(g) )
        \\&= 
        \nabla(g) \cdot \nabla(f) + g \cdot \nabla\cdot(\nabla f) + \nabla(f) \cdot \nabla(g) + f \cdot \nabla\cdot(\nabla g)
        = 
        2 \nabla(f) \cdot \nabla( g ) + f \Delta g + g\Delta f
        .
    \end{align*}
    The last three identities describe that gradient, divergence and curl are \emph{linear},
    which is a concept that you have already seen in linear algebra. 
    We verify by direct computation:
    \begin{align*}
        \nabla( \alpha f + \beta g )
        &= 
        \left( \alpha \partial_1 f_1 + \beta \partial_1 g_1, \cdots, \alpha \partial_n f_n + \beta \partial_n g_n \right)
        \\&= 
        \left( \alpha \partial_1 f_1, \cdots, \alpha \partial_n f_n \right)
        +
        \left( \beta \partial_1 g_1,  \cdots,  \beta \partial_n g_n \right)
        \\&= 
        \alpha \left( \partial_1 f_1, \cdots,  \partial_n f_n \right)
        +
        \beta \left( \partial_1 g_1,  \cdots,  \partial_n g_n \right)
        .
    \end{align*}
    Similarly,
    \begin{align*}
        \nabla \cdot ( \alpha \vecA + \beta \vecB )
        &= 
        \nabla \cdot \left( \alpha A_1 + \beta B_1, \cdots, \alpha A_n + \beta B_n \right)
        \\&= 
        \alpha \partial_1 A_1 + \beta \partial_1 B_1 + \cdots + \alpha \partial_n A_n + \beta \partial_n B_n 
        \\&= 
        \alpha \partial_1 A_1 + \cdots + \alpha \partial_n A_n + \beta \partial_1 B_1 + \cdots + \beta \partial_n B_n 
        \\&= 
        \alpha \left( \partial_1 A_1 + \cdots + \partial_n A_n \right) + \beta \left( \partial_1 B_1 + \cdots + \partial_n B_n \right)
        \\&= 
        \alpha \nabla \cdot \vecA + \beta \nabla \vecB
        .
    \end{align*}
    Lastly,
    {\small
    \begin{align*}
        &
        \nabla \times ( \alpha \vecA + \beta \vecB )
        \\&= 
        \nabla \times \left( 
            \alpha A_1 + \beta B_1,
            \alpha A_2 + \beta B_2,
            \alpha A_3 + \beta B_3 
        \right)
        \\&= 
        \left( 
            \alpha \partial_2 A_3 + \beta \partial_2 B_3 - \alpha \partial_3 A_2 - \beta \partial_3 B_2,
          - \alpha \partial_1 A_3 - \beta \partial_1 B_3 + \alpha \partial_3 A_1 + \beta \partial_3 B_1,
            \alpha \partial_1 A_2 + \beta \partial_1 B_2 - \alpha \partial_2 A_1 - \beta \partial_2 B_1
        \right)        
        \\&= 
        \left( 
            \alpha \partial_2 A_3 - \alpha \partial_3 A_2 + \beta \partial_2 B_3 - \beta \partial_3 B_2,
          - \alpha \partial_1 A_3 + \alpha \partial_3 A_1 - \beta \partial_1 B_3 + \beta \partial_3 B_1,
            \alpha \partial_1 A_2 - \alpha \partial_2 A_1 + \beta \partial_1 B_2 - \beta \partial_2 B_1
        \right)        
        \\&= 
        \alpha \left( 
            \partial_2 A_3 - \partial_3 A_2,
          - \partial_1 A_3 + \partial_3 A_1,
            \partial_1 A_2 - \partial_2 A_1
        \right)        
        +
        \beta\left( 
            \partial_2 B_3 - \partial_3 B_2,
          - \partial_1 B_3 + \partial_3 B_1,
            \partial_1 B_2 - \partial_2 B_1
        \right)        
        \\&= 
        \alpha \nabla \times \vecA + \beta \nabla \times \vecB
        .
    \end{align*}
    }
\end{solution} 




\end{document}
