\documentclass[11pt]{article}
% \def\hidesolutions{}
%%%%%%%%%%% SET MARGINS
\setlength{\textheight}{20cm}
\setlength{\topmargin}{-0.5cm}
\setlength{\oddsidemargin}{+0cm}
\setlength{\textwidth}{16.3cm}
%\setlength{\parskip}{6pt}
\setlength{\parindent}{0pt}

%%%%%%%%%%% PACKAGES
\usepackage{amsmath}
\usepackage{amssymb}
\usepackage{amsfonts}
%\usepackage{a4wide}
\usepackage{graphicx}
\usepackage{color}
\usepackage[normalem]{ulem}
\usepackage{enumitem}
\usepackage{capt-of}
\usepackage{float}
\usepackage{amsmath}
\usepackage{listings}
\definecolor{mygreen}{RGB}{28,172,0} % color values Red, Green, Blue
\definecolor{mylilas}{RGB}{170,55,241}
\usepackage{empheq}
\usepackage[ruled]{algorithm2e}
\usepackage{mathrsfs}
\usepackage{datetime}
\usepackage{subcaption}

% TODO: combine the two package lists and reduce redundancies 
\usepackage{mathtools}
\usepackage{nicefrac}
\usepackage{hyperref}
\usepackage{url}
\usepackage{amsmath,amssymb,amsfonts}
\usepackage{a4wide}
\usepackage{graphicx}
\usepackage{color}
\usepackage[normalem]{ulem}
\usepackage{capt-of}
\usepackage{float}
\usepackage[ruled]{algorithm2e}
\usepackage{amsmath,amssymb,amsfonts}
\usepackage{a4wide}
\usepackage{graphicx}
\usepackage{color}
\usepackage[normalem]{ulem}
\usepackage{capt-of}
\usepackage{float}
\usepackage[ruled]{algorithm2e}
\usepackage{mathrsfs}







\newcommand{\Lc}[2]{{\color{blue} \sout{#1} } \textcolor{red}{#2}}
\newcommand{\La}[1]{\textcolor{red}{#1}}
\newcommand{\lh}{\mathscr{L}_h}
\newcommand{\cl}{\mathscr{L}}
\newcommand{\cf}{\mathscr{F}}
\newcommand{\dx}{dx}
\newcommand{\ltn}{\mathscr{l}^2}
\newcommand{\bbR}{\mathbb{R}}
\newcommand{\Rset}{\mathbb{R}}
\newcommand{\Nset}{\mathbb{N}}
\newcommand{\scL}{\mathcal{L}}
\newcommand{\xx}{\mathbf{x}}
\newcommand{\norm}[1]{\|{#1}\|}
\newcommand{\yy}{\mathbf{y}}
\newcommand{\at}[1]{\big|_{#1}}
\renewcommand{\div}{\mathrm{div}}
\newcommand{\divergence}{\mathrm{div}}
\newcommand{\cp}[1]{\textcolor{blue}{#1}}

\newcommand{\FF}{\texttt{FreeFem++ }}
\newcommand{\FFns}{\texttt{FreeFem++}}
\newcommand{\FFfull}{\texttt{FreeFem++-x11}}
\newcommand{\cmd}[1]{ \medskip \noindent \texttt{#1} \medskip}
\newcommand{\incmd}[1]{\texttt{#1}}
\newcommand{\shrinkitems}{\addtolength{\itemsep}{-0.5\baselineskip}}
\newcommand{\mtt}[1]{\mathtt{#1}}
\newcommand{\ML}{\texttt{Matlab }}

\newcommand{\bb}{\mathbf{b}}
\newcommand{\nn}{\mathbf{n}}
\newcommand{\vecA}{\vec{A}}
\newcommand{\vecB}{\vec{B}}


\newcommand{\mesh}{\mathcal{T}_h}
\newcommand{\refel}{\widehat{K}}
\newcommand{\ver}{\mathbf{a}}
\newcommand{\refver}{\widehat{\mathbf{a}}}
\newcommand{\grad}{\nabla}
\newcommand{\refgrad}{\widehat{\nabla}}
\newcommand{\refu}{\widehat{u}}
\newcommand{\refbasis}{\widehat{\varphi}}
\newcommand{\refxx}{\widehat{\xx}}
\newcommand{\refx}{\widehat{x}}
\newcommand{\refy}{\widehat{y}}
\newcommand{\refrho}{\widehat{\rho}}
\newcommand{\refh}{\widehat{h}}






% For typesetting Python code
\newcommand{\matlab}{{\sc Matlab}\xspace}
\usepackage{listings}
\lstloadlanguages{Python}
\lstloadlanguages{csh}%
\definecolor{MyDarkGreen}{rgb}{0.0,0.4,0.0}
\definecolor{purple}{rgb}{0.58,0,0.82}
\lstset{language=Python,                    % Use Python
	%frame=single,                          % Single frame around code
	basicstyle=\ttfamily\footnotesize\color{black},
	keywordstyle=[1]\color{blue}\bf,        % Python functions bold and blue
	keywordstyle=[2]\color{purple},         % Python function arguments purple
	keywordstyle=[3]\color{red}\underbar,   % User functions underlined and blue
	commentstyle=\usefont{T1}{pcr}{m}{sl}\color{MyDarkGreen}\small,
	stringstyle=\color{purple},
	showstringspaces=false,                 % Don't put marks in string spaces
	tabsize=3,                              % 5 spaces per tab
	morekeywords={xlim,ylim,var,alpha,factorial,poissrnd,normpdf,normcdf},
	morecomment=[l][\color{blue}]{...},
	breaklines=true,
	breakatwhitespace=true,
	emptylines=1,
	mathescape=true,
	xleftmargin=0ex,
	emphstyle=\bfseries\color{red}
}





%%%%%%%%%%% MACROS NAMES
\newcommand{\lecturername}{Martin Licht}
% \newcommand{\assistantnamea}{Jochen Hinz}
% \newcommand{\assistantnameb}{Ivan Bioli}
\newcommand{\semestername}{Winter Semester 2023}
\newcommand{\lecturename}{Analysis III - 202(c)}
\DeclarePairedDelimiter\floor{\lfloor}{\rfloor}

%%%%%%%%%%% HEADER
\newdateformat{yeardate}{\THEYEAR}
\newcommand{\exsheet}[3] % input is the number of the session and the day TODO What's that
{\clearpage

	\begin{center}
		{\Large \textbf{\lecturename}}\\[2ex]
		\semestername
	\end{center}

	% \vspace{2ex}
	% \lecturername

	\vspace{2ex}
	{\Large Session #1: #3\,#2, \yeardate\today}
	%\hfill
	%{\Large EPF Lausanne}

	\hrulefill
}





\usepackage{comment}

\newtheorem{exercise}{Exercise}
\newtheorem{solutionenv}{Solution}

\newboolean{hide_solution}
\ifx\hidesolutions\undefined
\newenvironment{solution}{\begin{solutionenv}}{\end{solutionenv}}
\setboolean{hide_solution}{false}
\else
\excludecomment{solution}
\setboolean{hide_solution}{true}
\fi

\newcommand{\ifnotsolution}[1]{\ifthenelse{\boolean{hide_solution}}{#1}{}}
\newcommand{\ifsolution}[1]{\ifthenelse{\boolean{hide_solution}}{}{#1}}








\allowdisplaybreaks

% \usepackage{enumitem}
\begin{document}
\exsheet{13}{12}{December} % parameters are the number of the session and the day






% 
% 
% \begin{exercise}
%     Draw the complex exponentials $e^{zt}$ in the complex plane for $t=0,1,2,3,4$, 
%     where $z$ is one of the following complex numbers:
%     \begin{gather*}
%         z_1 = \frac 1 2,
%         \quad 
%         z_2 = - \frac 1 2,
%         \quad 
%         z_3 = 0.2i,
%         \quad 
%         z_4 = -0.2i,
%         \quad 
%         z_3 = - \frac 1 2 + 0.2i
%         .
%     \end{gather*}
%     \textit{You may use a calculater.}
% \end{exercise}
% 
% \begin{solution}     
% \end{solution}




\begin{exercise}
    The Fourier transform of 
    \begin{align*}
        f(x) = e^{-5x^2}
    \end{align*}
    is the function 
    \begin{align*}
        \hat f(\alpha) = \frac{1}{\sqrt{10}} e^{-\frac{\alpha^2}{20}}.
    \end{align*}
    Find the Fourier transforms of 
    \begin{align*}
     f', \quad f'', \quad f''', \quad f'''', \quad g(x) = f(2x), \quad h(x) = f(x-3).
    \end{align*}
\end{exercise}
\begin{solution}  
    To evaluate the Fourier transform of the derivative of a function we will use the following identity from the lecture:
    \begin{gather*}
        \mathfrak{F}\left(f^{(n)}\right)(\alpha)=(i \alpha)^n \mathfrak{F}(f)(\alpha)
    \end{gather*}
    We then find 
    \begin{itemize}
    \item $\mathfrak{F}\left(f'    \right)(\alpha) = i \alpha \mathfrak{F}(f)(\alpha) = \frac{i \alpha}{\sqrt{10}} e^{-\frac{\alpha^2}{20}}$
    \item $\mathfrak{F}\left(f''   \right)(\alpha) = -\alpha^2 \mathfrak{F}(f)(\alpha) = -\frac{ \alpha^2}{\sqrt{10}} e^{-\frac{\alpha^2}{20}}$
    \item $\mathfrak{F}\left(f'''  \right)(\alpha) = -i \alpha^3 \mathfrak{F}(f)(\alpha) = \frac{-i \alpha^3}{\sqrt{10}} e^{-\frac{\alpha^2}{20}}$
    \item $\mathfrak{F}\left(f'''' \right)(\alpha) = \alpha^4 \mathfrak{F}(f)(\alpha) = \frac{ \alpha^4}{\sqrt{10}} e^{-\frac{\alpha^2}{20}}$
    \end{itemize}
    For the Fourier transform of $g(x)$ we use the following identity from the lecture: 
    $g(x)=e^{-i b x} f(a x)$ has the Fourier transform
    $\hat{g}(\alpha)=\frac{1}{|a|} \hat{f}\left(\frac{\alpha+b}{a}\right)$,
    consequently:
    \begin{align*}
        \mathfrak{F}(g(x))(\alpha) & =\mathfrak{F}(f(2 x))(\alpha) \\ & =\frac{1}{2} \mathfrak{F}(f(x))\left(\frac{\alpha}{2}\right) \\ & =\frac{1}{2 \sqrt{10}} e^{-\frac{\left(\frac{\alpha}{2}\right)^2}{20}} \\ & =\frac{1}{2 \sqrt{10}} e^{-\frac{\alpha^2}{80}}
    \end{align*}
    For the Fourier transform of $h(x)$ we use a change of variables:
    \begin{align*}
        \mathfrak{F}(h(x))(\alpha)
        &= 
        \frac{1}{\sqrt{2\pi}}
        \int_{-\infty}^{+\infty} e^{-5(x-3)^2} e^{ - \imath x \alpha } \;dx
        \\&= 
        \frac{1}{\sqrt{2\pi}}
        \int_{-\infty}^{+\infty} e^{-5x^2} e^{ - \imath (x + 3) \alpha } \;dx
        \\&= 
        \frac{1}{\sqrt{2\pi}}
        \int_{-\infty}^{+\infty} e^{-5x^2} e^{ - \imath x \alpha } e^{ - \imath 3 \alpha }\;dx
        \\&= 
        \frac{ e^{ - \imath 3 \alpha }}{\sqrt{2\pi}}
        \int_{-\infty}^{+\infty} e^{-5x^2} e^{ - \imath x \alpha }\;dx
        \\&= 
        e^{ - \imath 3 \alpha } 
        \mathfrak{F}(f(x))(\alpha)
        \\&= 
        \frac{e^{ - \imath 3 \alpha }}{\sqrt{10}} e^{-\frac{\alpha^2}{20}}.
        \\&= 
        \frac{1}{\sqrt{10}} e^{-\frac{\alpha^2}{20} - \imath 3 \alpha }
        .
    \end{align*}
\end{solution}






































\begin{exercise}
    Find $f : \mathbb R \to \mathbb R$ such that 
    \begin{align*}
        \hat f(\alpha) = \frac{3}{1+\alpha^{2}} + \frac{-1}{1+4\alpha^{2}} + \frac{ \sin( 4 \alpha + 3 ) }{ 4 \alpha + 3 }
    \end{align*}
\end{exercise}
\begin{solution}   Because of the linearity property we can treat the inverse of each term in seperately. Therefore 
    First,
    \begin{align*} 
        \mathfrak{F}^{-1}\left(\frac{\sin (4 \alpha+3)}{4 a r+3}\right)(x) & =\frac{\frac{1}{4}}{\frac{1}{4}} \mathfrak{F}^{-1}\left(\frac{\sin \left(\frac{\alpha+3 / 4}{\frac{1}{4}}\right)}{\frac{\alpha+\frac{3}{4}}{\frac{1}{4}}}\right)(x) \\ & =\frac{1}{4} e^{-i \frac{3}{4} x} \sqrt{\frac{\pi}{2}} \mathfrak{F}^{-1}\left(\sqrt{\frac{2}{\pi}} \frac{\sin (\alpha)}{\alpha}\right)\left(\frac{x}{4}\right) \\ & = \begin{cases}\frac{1}{4} \sqrt{\frac{\pi}{2}} e^{-i \frac{3}{4} x} & \text { if }-4 < x < 4 \\ 0 & \text { else }\end{cases} 
    \end{align*}
    Second,
    \begin{align*} 
        \mathfrak{F}^{-1}\left(\frac{3}{1+\alpha^2}\right)(x) & =3 \sqrt{\frac{\pi}{2}} \mathfrak{F}^{-1}\left(\sqrt{\frac{2}{\pi}} \frac{1}{1+a^2}\right)(x) \\ & =3 \sqrt{\frac{\pi}{2}} e^{-|x|}
    \end{align*}
    Lastly,
    \begin{align*} 
        \mathfrak{F}^{-1}\left(\frac{-1}{1+4 \alpha^2}\right)(x) 
        & =
        -\sqrt{\frac{\pi}{2}} \mathfrak{F}^{-1}\left(\sqrt{\frac{2}{\pi}} \frac{1}{1+(2 \alpha)^2}\right)(x) 
        \\ & 
        =
        -\sqrt{\frac{\pi}{2}} \frac{\frac{1}{2}}{\frac{1}{2}} \mathfrak{F}^{-1}\left(\sqrt{\frac{2}{\pi}} \frac{1}{1+\left(\frac{\alpha}{\frac{1}{2}}\right)^2}\right)(x \mid 
        \\ &
        =
        -\sqrt{\frac{\pi}{2}} \frac{1}{2} \mathfrak{F}^{-1}\left(\sqrt{\frac{2}{\pi}} \frac{1}{1+\alpha^2}\right)\left(\frac{x}{2}\right) 
        \\ & 
        =
        -\frac{1}{2} \sqrt{\frac{\pi}{2}} e^{-\left|\frac{x}{2}\right|}
        .
    \end{align*}
    Altogether this gives
    \begin{align*} 
        f(x) = \begin{cases}-\frac{1}{2} \sqrt{\frac{\pi}{2}} e^{-\left|\frac{x}{2}\right|}+3 \sqrt{\frac{\pi}{2}} e^{-|x|}+\frac{1}{4} \sqrt{\frac{\pi}{2}} e^{-i \frac{3}{4} x} & -4 \leq x \leq 4 \\ -\frac{1}{2} \sqrt{\frac{\pi}{2}} e^{-\left|\frac{x}{2}\right|}+3 \sqrt{\frac{\pi}{2}} e^{-|x|} & \text { else }\end{cases}
    \end{align*}
    
\end{solution}


















\begin{exercise}
    We consider the Poisson problem with Dirichlet boundary conditions over the interval $[0,L]$:
    \begin{gather*}
        - \Delta u(x) = x^2, \quad 0 < x < L,
        \\
        u(0) = 1, \quad u(L) = 2
    \end{gather*}
    \begin{itemize}
        \item Solve this problem directly. The solution is a polynomial of order $4$.
        \item Extend the right-hand side $f(x) = x^2$ to an odd function with period $2L$ and compute its Fourier coefficients.
        \item 
        Using these coefficients, find the Fourier series of the solution $u^f$,
        which solves 
        \begin{gather*}
            - \Delta u^f(x) = x^2, \quad 0 < x < L,
            \\
            u^f(0) = 0, \quad u^f(L) = 0.
        \end{gather*}
        How do you use the superposition principle to solve the full problem? 
    \end{itemize}
\end{exercise}
\begin{solution}  
    \begin{itemize}
        \item According to the hint, $u(x)$ must be of the form $u(x) = a x^4 + b x^3 + c x^2 + d x + e$ for some coefficients $a,b,c,d,e \in \mathbb R$.
        Inserting this ansatz into the Poisson equation and the boundary conditions, we find the system of equations
        \begin{align}
            u(0) &= e = 1, \label{eq:ex3:left bc}\\
            u(L) &= a L^4 + b L^3 + c L^2 + d L + e = 2, \label{eq:ex3:right bc}\\
            - \Delta u(x) &= - 12 a x^2 - 6 b x - 2 c = x^2. \label{eq:ex3:pde}
        \end{align}
        By matching the coefficients of the polynomials on the right- and left-hand side in \eqref{eq:ex3:pde}, we find
        \begin{align*}
            - 12 a &= 1,\\
            - 6 b &= 0,\\
            - 2 c &= 0.
        \end{align*}
        Hence, we have $a = -\frac{1}{12}, b = 0, c = 0$. Therefore, \eqref{eq:ex3:right bc} simplifies to
        \begin{align*}
            -\frac{1}{12} L^4 + dL + 1 = 2,
        \end{align*}
        from which we deduce that $d = \frac{1}{12} L^3 + \frac{1}{L}$. Thus, the solution to the Poisson problem is
        \begin{align*}
            u(x) = -\frac{1}{12} x^4 + \left(\frac{1}{12} L^3 + \frac{1}{L}\right) x + 1.
        \end{align*}
        \item We define the odd extension of $f(x) = x^2$ as the function $\Tilde{f}$ with period $2L$ that satisfies 
        \begin{align*}
            \Tilde{f}(x) = \begin{cases}
                x^2 & \text{if } 0 \leq x \leq L,\\
                -x^2 & \text{if } -L \leq x < 0.
            \end{cases}
        \end{align*}
        By construction, $\Tilde f$ is an odd function with period $2L$. Denote by $\Tilde{a}_n$ and $\Tilde{b}_n$ the Fourier coefficients of $\Tilde{f}$.
        We immediately have $\Tilde{a}_n = 0$ for all $n \in \mathbb N$ because the function is odd.
        Moreover, by some simple integration by parts 
        \begin{align*}
            \Tilde{b}_n &= \frac{1}{L} \int_{-L}^L \Tilde{f}(x) \sin\left(\frac{\pi n}{L} x\right) dx = \frac{2}{L} \int_{0}^L x^2 \sin\left(\frac{\pi n}{L} x\right) dx\\
            &= -\frac{2 L^2 (2 + (-1)^n ((n \pi)^2 - 2))}{(n\pi)^3}.
        \end{align*}
        We conclude that
        \begin{align*}
            \Tilde{f}(x) = \sum_{n \geq 1} \Tilde{b}_n \sin\left(\frac{\pi n}{L} x\right) = \sum_{n \geq 1} -\frac{2 L^2 (2 + (-1)^n ((n \pi)^2 - 2))}{(n\pi)^3} \sin\left(\frac{\pi n}{L} x\right).
        \end{align*}
        \item 
        To apply the superposition principle, 
        we split the problem into two Poisson problems: on the one hand,
        \begin{gather*}
            - \Delta u^f(x) = x^2, \quad 0 < x < L,
            \\
            u^f(0) = 0, \quad u^f(L) = 0, 
        \end{gather*}
        and on the other hand, 
        \begin{gather*}
            - \Delta u^g(x) = x^2, \quad 0 < x < L,
            \\
            u^g(0) = 1, \quad u^g(L) = 2.
        \end{gather*}
        The full solution will be the sum of $u^{f}$ and $u^{g}$.
        
        We define 
        \begin{align}
            u^{f}(x) 
            &
            = 
            - \sum_{n=1}^{\infty} \frac{\Tilde{b}_n}{ -\pi^{2} n^{2} / L^{2} } \sin\left( \pi n x / L \right)
            \\&
            = 
            \sum_{n=1}^{\infty} \frac{\Tilde{b}_n}{ \pi^{2} n^{2} / L^{2} } \sin\left( \pi n x / L \right)
            .
        \end{align}
        Differentation, distributed over the Fourier modes, now shows that 
        \begin{align}
            - \partial_{xx} u^{f}(x) = f(x).
        \end{align}
        Additionally, $u^{f}$ satisfies the homogeneous Dirichlet boundary conditions. 
        
        We can find $u^{g}$ in a manner similar to the previous step. 
        Since its second-derivative is zero, $u^{g}$ must be a linear function over $[0,L]$, having the form:
        \begin{align}
            u^{g}(x) = d x + e.
        \end{align}
        The boundary conditions now lead to $u^{g}(x) = \frac{1}{L} x + 1$
        
        By the superposition principle, the solution to the original problem is 
        \begin{align}
            u(x) = u^{f}(x) + u^{g}(x).
        \end{align}
%         
%         
% 
%         we can use the superposition principle to split the Poisson problem into $n$ Poisson problems with homogeneous Dirichlet boundary conditions. 
%         The the $n$-th problem is
%         \begin{align*}
%             \begin{cases}
%             -\Delta u_n(x) &= \Tilde{b}_n \sin\left(\frac{\pi n}{L} x\right), \quad 0 < x < L,\\
%             u_n(0) &= 0,\\
%             u_n(L) &= 0.
%             \end{cases}
%         \end{align*}
%         We choose a sine-wave ansatz for $u_n$, i.e. $u_n(x) = b_n \sin\left(\frac{\pi n}{L} x\right)$. Note that this ansatz satisfies the boundary conditions.
%          Inserting this ansatz into the Poisson equation, we find
%         \begin{align*}
%             \left(\frac{\pi n}{L} \right)^2 b_n \sin\left(\frac{\pi n}{L} x\right) &= \Tilde{b}_n \sin\left(\frac{\pi n}{L} x\right).
%         \end{align*}
%         Hence, we must have $b_n = \left(\frac{L}{\pi n}\right)^2 \Tilde{b}_n$. Therefore, the solution to the $n$-th Poisson problem is
%         \begin{align*}
%             u_n(x) = \left(\frac{L}{\pi n}\right)^2 \Tilde{b}_n \sin\left(\frac{\pi n}{L} x\right),
%         \end{align*}
%         and the solution to the original Poisson problem is
%         \begin{align*}
%             u^f(x) = \sum_{n \geq 1} u_n(x) = \sum_{n \geq 1} \left(\frac{L}{\pi n}\right)^2 \Tilde{b}_n \sin\left(\frac{\pi n}{L} x\right).
%         \end{align*}
    \end{itemize}
\end{solution}




























\begin{exercise}
    Directly compute the solution of the problem
    \begin{gather*}
        - \Delta u(x) = x^2, \quad 0 < x < L,
        \\
        u(0) = 0, \quad u(L) = 0,
    \end{gather*}
    using elementary analysis. Then find the Fourier coefficients of its odd extension to the interval $[-L, L]$. Compare this with the function $u^f$ from the previous exercise.
\end{exercise}
\begin{solution}
    As in the previous exercise, the solution must be of the form $u(x) = ax^4 + bx^3 + cx^2 + dx + e$ for some coefficients $a,b,c,d,e \in \mathbb R$.
%     We have already seen how to compute it:
%     \begin{align*}
%         u(x) = -\frac{1}{12} x^4 + \frac{1}{12} L^3 x.
%     \end{align*}
    Inserting this ansatz into the Poisson equation and the boundary conditions, we find the system of equations
    \begin{align}
        u(0) &= e = 0, \label{eq:ex4:left bc}\\
        u(L) &= a L^4 + b L^3 + c L^2 + d L + e = 0, \label{eq:ex4:right bc}\\
        - \Delta u(x) &= - 12 a x^2 - 6 b x - 2 c = x^2. \label{eq:ex4:pde}
    \end{align}
    By matching the coefficients of the polynomials on the right- and left-hand side in \eqref{eq:ex4:pde}, we find
    \begin{align*}
        - 12 a = 1, \quad - 6 b = 0, \quad - 2 c = 0.
    \end{align*}
    Hence, we have $a = -\frac{1}{12}$, $b = 0$, and $c = 0$. Therefore, \eqref{eq:ex4:right bc} simplifies to
    \begin{align*}
        -\frac{1}{12} L^4 + dL = 0,
    \end{align*}
    from which we deduce that $d = \frac{1}{12} L^3$. Thus, the solution to the Poisson problem is
    \begin{align*}
        u(x) = -\frac{1}{12} x^4 + \frac{1}{12} L^3 x.
    \end{align*}
    To check that this solution is equal to the one found in the previous exercise, 
    we compute the Fourier coefficients of the odd extension $\Tilde u$ of $u$ to the interval $[-L,L]$ defined by
    \begin{align*}
        \Tilde u(x) = \begin{cases}
            u(x) & \text{if } 0 \leq x \leq L,\\
            -u(-x) & \text{if } -L \leq x < 0.
        \end{cases}
    \end{align*}
    By definition, $\Tilde u$ is an odd function, and therefore the Fourier series takes the form 
    \begin{align*}
        \Tilde u(x) = \sum_{n \geq 1} \Tilde u_n \sin\left(\frac{\pi n}{L} x\right),
    \end{align*}
    where $\Tilde u_{n}$ is the Fourier coefficient of the $n$-th cosine mode. 
    We compute 
    \begin{align*}
        \Tilde u_n 
        &= 
        \frac{2}{L} \int_0^L u(x) \sin\left(\frac{\pi n}{L} x\right) dx
        \\&
        = 
        \frac{2}{L} \int_0^L \left(-\frac{1}{12} x^4 + \frac{1}{12} L^3 x\right) \sin\left(\frac{\pi n}{L} x\right) dx
        \\&
        = 
        -\frac{2 L^4 (2 + (-1)^n)((n \pi)^2 - 2)}{(n \pi)^5}
        .
    \end{align*}
    Note that we have $\Tilde u_n = \left(\frac{L}{\pi n}\right)^2 \Tilde{b}_n$, where $\Tilde b_n$ is as in the previous exercise.
\end{solution}



















\begin{exercise}
    We solve the Poisson problem with homogeneous Dirichlet boundary conditions over $[0,1]$:
    \begin{gather*}
        - \Delta u(x) = \left\{\begin{array}{ll} x & \text{ if } 0 < x \leq 0.5 \\ 0 & \text{ if } 0.5 < x \leq 1 \end{array}\right., \qquad 0 < x < 1,
        \\
        u(0) = 0, \quad u(1) = 0
    \end{gather*}
    \begin{itemize}
        \item Extend the right-hand side $f(x)$ to an odd function with period $2$ and compute its Fourier coefficients.
        \item Using these coefficients, find the solution $u$. Verify that the boundary condition $u(0) = u(1) = 0$ is satisfied.
    \end{itemize}
\end{exercise}
\begin{solution}
\begin{itemize}
    \item We define the odd extension of $f(x)$ as
    \begin{align*}
        \Tilde{f}(x) = \begin{cases}
            f(x) & \text{if } 0 \leq x < 1,\\
            -f(-x) & \text{if } -1 < x < 0.
        \end{cases}
    \end{align*}
    As in the previous exercises, we find that
    \begin{align*}
        \Tilde{f}(x) = \sum_{n \geq 1} \Tilde{f}_n \sin(\pi n x),
    \end{align*}
    using that $\Tilde f$ is odd. The Fourier coefficients of the sine modes are 
    \begin{align*}
        \Tilde{f}_n = 2 \int_0^1 f(x) \sin(\pi n x) dx = 2 \int_0^{\frac 1 2} x \sin(\pi n x) dx = \frac{- \pi n \cos (\frac{\pi n}{2}) + 2 \sin(\frac{\pi n}{2})}{(\pi n)^2}.
    \end{align*}
    One can check that 
    \begin{align*}
        \Tilde{f}_n = \begin{cases}
            \frac{2}{(\pi n)^2} (-1)^n, & \text{if } n \text{ is odd},\\
            -\frac{1}{\pi n}(-1)^{\frac n 2}, & \text{if } n \text{ is even}.
        \end{cases}
    \end{align*}
    \item 
    
    We define the solution in terms of its Fourier series:
    \begin{align}
        u(x) 
        &
        = 
        - \sum_{n=1}^{\infty} \frac{\Tilde{f}_n}{ -\pi^{2} n^{2} } \sin\left( \pi n x \right)
%         \\&
        = 
        \sum_{n=1}^{\infty} \frac{\Tilde{f}_n}{ \pi^{2} n^{2} } \sin\left( \pi n x \right)
        .
    \end{align}
    Clearly, $u$ satisfies the homogeneous Dirichlet boundary conditions,
    and differentiation provides the desired differential equation 
    \begin{align}
        - u''(x) = f(x).
    \end{align}
    This shows that $u$ solves the Poisson problem.
    Note that $u$ is not in the form of a Fourier series over the interval $[0, L]$ but in the form of the Fourier series of its odd extension over $[-L, L]$. 
    The boundary conditions are satisfied because $u$ is a Fourier sine series. 
    
    
%     Using the superposition principle, we set $u(x) = \sum_{n \geq 1} u_n(x)$, where $u_n(x)$ solves the Poisson problem
%     \begin{align*}
%         \begin{cases}
%             - \Delta u_n(x) = \Tilde{f}_n \sin(\pi n x), \quad 0 < x < 1,\\
%             u_n(0) = 0,\\
%             u_n(1) = 0.
%         \end{cases}
%     \end{align*}
%     We choose a sine-wave ansatz for $u_n$, i.e. $u_n(x) = b_n \sin(\pi n x)$. Note that this ansatz satisfies the boundary conditions.
%     Inserting this ansatz into the Poisson equation, we find
%     \begin{align*}
%         - \pi^2 n^2 b_n \sin(\pi n x) = \Tilde{f}_n \sin(\pi n x).
%     \end{align*}
%     Hence, we must have $b_n = -\frac{\Tilde{f}_n}{\pi^2 n^2}$. Therefore, the solution to the $n$-th Poisson problem is
%     \begin{align*}
%         u_n(x) = b_n \sin(\pi n x),
%     \end{align*}
%     with
%     \begin{align*}
%         b_n = \begin{cases}
%             \frac{2}{(\pi n)^4} (-1)^n, & \text{if } n \text{ is odd},\\
%             -\frac{1}{(\pi n)^3}(-1)^{\frac n 2}, & \text{if } n \text{ is even}.
%         \end{cases}
%     \end{align*}
%     Therefore, the solution to the original Poisson problem is
%     \begin{align*}
%         u(x) = \sum_{n \geq 1} u_n(x) = \sum_{n \geq 1} b_n \sin(\pi n x).
%     \end{align*}
%     Note that $u$ is not in the form of a Fourier series over the interval $[0, L]$ but in the form of the Fourier series of its odd extension over $[-L, L]$. 
\end{itemize}
\end{solution}














\begin{exercise}
    Consider the following Fourier sine series for two functions $u, f : \bbR \to \bbR$ with period $T$:
    \begin{align*}
        u(x) = \sum_{n=1}^{\infty} b_{n}^{u} \sin\left( \frac{2\pi n}{T} x \right)
        ,
        \quad 
        f(x) = \sum_{n=1}^{\infty} b_{n}^{f} \sin\left( \frac{2\pi n}{T} x \right)
        .
    \end{align*}
    Express the Fourier coefficients of $f$ by the Fourier coefficients of $u$, and vice-versa, express the Fourier coefficients of $u$ by the Fourier coefficients of $f$,
    if $u$ and $f$ are related by the following differential equations:
    \begin{enumerate}[label=(\alph*)]
     \item 
     $-u''(x) = f(x)$.
     \item 
     $u''''(x) = f(x)$.
     \item 
     $-u''(x) + 25 u(x) = f(x)$.
     \item 
     $u''''(x) - u''(x) = f(x)$.
     \item 
     $-u''(x) + \gamma \cdot u''(x) = f(x)$.
    \end{enumerate}
    In the last item, $\gamma \in \bbR$ is a constant. For which values of $\gamma$ can you always find a solution?
\end{exercise}
\begin{solution}   
    \begin{itemize}
     \item 
     We have 
     \begin{align*}
        -u''(x) = - \sum_{n=1}^{\infty} b_{n}^{u} \left( \frac{2\pi n}{T} \right)^{2} \sin\left( \frac{2\pi n}{T} x \right).
     \end{align*}
     Matching coefficients leads to the relationships
     \begin{align*}
        b_{n}^{u} \left( \frac{2\pi n}{T} \right)^{2} = b_{n}^{f},
        \quad 
        b_{n}^{u} = \left( \frac{2\pi n}{T} \right)^{-2} b_{n}^{f}.
     \end{align*}
     \item 
     We have 
     \begin{align*}
        u''''(x) = \sum_{n=1}^{\infty} b_{n}^{u} \left( \frac{2\pi n}{T} \right)^{4} \sin\left( \frac{2\pi n}{T} x \right).
     \end{align*}
     Matching coefficients leads to the relationships
     \begin{align*}
        b_{n}^{u} \left( \frac{2\pi n}{T} \right)^{4} = b_{n}^{f},
        \quad 
        b_{n}^{u} = \left( \frac{2\pi n}{T} \right)^{-4} b_{n}^{f}.
     \end{align*}
     \item 
     \begin{align*}
        -u''(x) + 25 u = \sum_{n=1}^{\infty} b_{n}^{u} \left( -\left( \frac{2\pi n}{T} \right)^{2} + 25 \right) \sin\left( \frac{2\pi n}{T} x \right).
     \end{align*}
     Matching coefficients leads to the relationships
     \begin{align*}
        b_{n}^{u} \left( -\left( \frac{2\pi n}{T} \right)^{2} + 25 \right) = b_{n}^{f},
        \quad 
        b_{n}^{u} = \left( -\left( \frac{2\pi n}{T} \right)^{2} + 25 \right)^{-1} b_{n}^{f}.
     \end{align*}
     \item 
     \begin{align*}
        u''''(x) - u''(x) = \sum_{n=1}^{\infty} b_{n}^{u} \left( \left( \frac{2\pi n}{T} \right)^{4} - \left( \frac{2\pi n}{T} \right)^{2} \right) \sin\left( \frac{2\pi n}{T} x \right).
     \end{align*}
     Matching coefficients leads to the relationships
     \begin{align*}
        b_{n}^{u} \left( \left( \frac{2\pi n}{T} \right)^{4} - \left( \frac{2\pi n}{T} \right)^{2} \right) = b_{n}^{f},
        \quad 
        b_{n}^{u} = \left( \left( \frac{2\pi n}{T} \right)^{4} - \left( \frac{2\pi n}{T} \right)^{2} \right)^{-1} b_{n}^{f}.
     \end{align*}
     \item 
     \begin{align*}
        u''''(x) - \gamma \cdot u''(x) = \sum_{n=1}^{\infty} b_{n}^{u} \left( \left( \frac{2\pi n}{T} \right)^{4} - \gamma \cdot \left( \frac{2\pi n}{T} \right)^{2} \right) \sin\left( \frac{2\pi n}{T} x \right).
     \end{align*}
     Matching coefficients leads to the relationships
     \begin{align*}
        b_{n}^{u} \left( \left( \frac{2\pi n}{T} \right)^{4} - \gamma \cdot \left( \frac{2\pi n}{T} \right)^{2} \right) = b_{n}^{f},
        \quad 
        b_{n}^{u} = \left( \left( \frac{2\pi n}{T} \right)^{4} - \gamma \cdot \left( \frac{2\pi n}{T} \right)^{2} \right)^{-1} b_{n}^{f}.
     \end{align*}
    \end{itemize}
    Remark: the second-order equation $-u'' = f$, the Poisson problem, appears in many applications. The fourth-order equation $u'''' = f$ appears in the mathematical theory of elasticity, for example, when modeling the bending of a one-dimensional stick under external force. 
\end{solution}





\begin{exercise}
    Recall the definition of the Fourier transform:
    \begin{align}
        \mathfrak{F}[f](\alpha) = \frac{1}{\sqrt{2\pi}} \int_{-\infty}^{+\infty} f(t) e^{-it\alpha} \ d\alpha
    \end{align}
    Let $a,b,c \in \mathbb R$ be real parameters with $a \neq 0$. Compute the Fourier transforms of the following:
    \begin{align}
        g(t) = f( a t ),
        \\
        h(t) = e^{-i t b} f(t),
        \\
        m(t) = f( t - c ).
    \end{align}
    \textit{Hint: } you have the first two in the lecture and in textbook. You can compute them using results from the lecture or via some standard integral manipulations. 
\end{exercise}

\begin{solution}
We write \( {\hat f}(\alpha) = \mathfrak{F}[f](\alpha) \) for the Fourier transform of $\alpha$. 
To compute the Fourier transforms of the given functions \( g(t) \), \( h(t) \), and \( m(t) \), we proceed as follows:
\begin{itemize}
\item 
Let \( {\hat g}(\alpha) = \mathfrak{F}[g](\alpha) \). By definition,
\begin{gather*}
    {\hat g}(\alpha) 
    = 
    \frac{1}{\sqrt{2\pi}} \int_{-\infty}^{+\infty} g(t) e^{-it\alpha} \, dt
    =
    \frac{1}{\sqrt{2\pi}} \int_{-\infty}^{+\infty} f(at) e^{-it\alpha} \, dt.
\end{gather*}
We perform a substitution \( u = at \), so \( du = |a| \, dt \) and \( t = \frac{u}{a} \). This gives
\begin{gather*}
    {\hat g}(\alpha) 
    = 
    \frac{1}{\sqrt{2\pi}} \int_{-\infty}^{+\infty} f(u) e^{-i \frac{u}{a} \alpha} \frac{du}{a}
    =
    \frac{1}{\sqrt{2\pi}} \cdot \frac{1}{a} \int_{-\infty}^{+\infty} f(u) e^{-i \frac{\alpha}{a} u} \, du.
\end{gather*}
Recognizing the integral as the Fourier transform of \( f(t) \), we find:
\begin{gather*}
    {\hat g}(\alpha) = \frac{1}{|a|} {\hat f}\left(\frac{\alpha}{a}\right),
\end{gather*}
\item 
Let \( {\hat h}(\alpha) = \mathfrak{F}[h](\alpha) \). Using definitions. 
\begin{align*}
    {\hat h}(\alpha) 
    &
    = 
    \frac{1}{\sqrt{2\pi}} \int_{-\infty}^{+\infty} h(t) e^{-it\alpha} \, dt
    \\&
    =
    \frac{1}{\sqrt{2\pi}} \int_{-\infty}^{+\infty} e^{-ibt} f(t) e^{-it\alpha} \, dt
    =
    \frac{1}{\sqrt{2\pi}} \int_{-\infty}^{+\infty} f(t) e^{-it(\alpha+b)} \, dt.
\end{align*}
We recognizing the integral as the Fourier transform of \( f(t) \). Thus, 
\begin{gather*}
    {\hat h}(\alpha) = {\hat f}(\alpha + b).
\end{gather*}
\item 
Let \( {\hat m}(\alpha) = \mathfrak{F}[m](\alpha) \). By definition,
\begin{gather*}
    {\hat m}(\alpha) 
    = 
    \frac{1}{\sqrt{2\pi}} \int_{-\infty}^{+\infty} m(t) e^{-it\alpha} \, dt
    =
    \frac{1}{\sqrt{2\pi}} \int_{-\infty}^{+\infty} f(t - c) e^{-it\alpha} \, dt.
\end{gather*}
We substitute \( u = t - c \), so \( du = dt \) and \( t = u + c \). This gives
\begin{gather*}
    {\hat m}(\alpha) 
    = 
    \frac{1}{\sqrt{2\pi}} \int_{-\infty}^{+\infty} f(u) e^{-i(u+c)\alpha} \, du
    =
    \frac{1}{\sqrt{2\pi}} \int_{-\infty}^{+\infty} f(u) e^{-iu\alpha} e^{-ic\alpha} \, du
    .
\end{gather*}
We pull out the factor out \( e^{-ic\alpha} \),
\begin{gather*}
    {\hat m}(\alpha) = e^{-ic\alpha} \cdot \frac{1}{\sqrt{2\pi}} \int_{-\infty}^{+\infty} f(u) e^{-iu\alpha} \, du,
\end{gather*}
and recognize the last integral as the Fourier transform of \( f(t) \). In summary, 
\begin{gather*}
    {\hat m}(\alpha) = e^{-ic\alpha} {\hat f}(\alpha).
\end{gather*}
\end{itemize}
\end{solution}




% \begin{exercise}
%     Consider the functions
%     \begin{align}
%         I_{c}( \omega ) = \int_{0}^{1} \cos( \omega x ) \ dx,
%         \quad 
%         I_{s}( \omega ) = \int_{0}^{1} \sin( \omega x ) \ dx,
%         .
%     \end{align}
%     Compute the derivatives of $I_{c}(\omega)$ and $I_{s}(\omega)$ in the variable $\omega$.
%     Use that to find formulas for 
%     \begin{align}
%         \int_{0}^{1} x^{k} \cos( \omega x ) \ dx
%         ,
%         \quad 
%         \int_{0}^{1} x^{k} \sin( \omega x ) \ dx
%         .
%     \end{align}
%     \textit{Hint: You can exchange differentiation and integration.}
% \end{exercise}
% \begin{solution}
%     We find that 
%     \begin{align}
%         \partial_\omega^{k} 
%         = 
%         \int_{0}^{1} \partial_\omega^{k} \cos( \omega x ) \ dx
%         =
%         \int_{0}^{1} \omega^{k} x^{k} \cos^{(k)}( \omega x ) \ dx
%         =
%         \omega^{k} \int_{0}^{1} x^{k} \cos^{(k)}( \omega x ) \ dx
%         .
%     \end{align}
%     Similarly,
%     \begin{align}
%         \partial_\omega^{k} 
%         = 
%         \int_{0}^{1} \partial_\omega^{k} \sin( \omega x ) \ dx
%         =
%         \int_{0}^{1} \omega^{k} x^{k} \sin^{(k)}( \omega x ) \ dx
%         =
%         \omega^{k} \int_{0}^{1} x^{k} \sin^{(k)}( \omega x ) \ dx
%         .
%     \end{align}
%     On the other hand,
%     \begin{align}
%         I_{c}( \omega ) = \int_{0}^{1} \cos( \omega x ) \ dx = \frac{\sin(\omega)}{\omega}
%         ,
%         \\ 
%         I_{s}( \omega ) = \int_{0}^{1} \sin( \omega x ) \ dx = \frac{1-\cos(\omega)}{\omega}
%         .
%     \end{align}
%     We explicitly compute the derivatives in $\omega$:
%     \begin{align}
%         \partial_\omega^{k} \frac{\sin(\omega)}{\omega}
%         =
%         \sum_{i=0}^{k} \tbinom{k}{i} ....
%     \end{align}
% \end{solution}










\end{document}
