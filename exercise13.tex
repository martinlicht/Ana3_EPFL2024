\documentclass[11pt]{article}
\def\hidesolutions{}
%%%%%%%%%%% SET MARGINS
\setlength{\textheight}{20cm}
\setlength{\topmargin}{-0.5cm}
\setlength{\oddsidemargin}{+0cm}
\setlength{\textwidth}{16.3cm}
%\setlength{\parskip}{6pt}
\setlength{\parindent}{0pt}

%%%%%%%%%%% PACKAGES
\usepackage{amsmath}
\usepackage{amssymb}
\usepackage{amsfonts}
%\usepackage{a4wide}
\usepackage{graphicx}
\usepackage{color}
\usepackage[normalem]{ulem}
\usepackage{enumitem}
\usepackage{capt-of}
\usepackage{float}
\usepackage{amsmath}
\usepackage{listings}
\definecolor{mygreen}{RGB}{28,172,0} % color values Red, Green, Blue
\definecolor{mylilas}{RGB}{170,55,241}
\usepackage{empheq}
\usepackage[ruled]{algorithm2e}
\usepackage{mathrsfs}
\usepackage{datetime}
\usepackage{subcaption}

% TODO: combine the two package lists and reduce redundancies 
\usepackage{mathtools}
\usepackage{nicefrac}
\usepackage{hyperref}
\usepackage{url}
\usepackage{amsmath,amssymb,amsfonts}
\usepackage{a4wide}
\usepackage{graphicx}
\usepackage{color}
\usepackage[normalem]{ulem}
\usepackage{capt-of}
\usepackage{float}
\usepackage[ruled]{algorithm2e}
\usepackage{amsmath,amssymb,amsfonts}
\usepackage{a4wide}
\usepackage{graphicx}
\usepackage{color}
\usepackage[normalem]{ulem}
\usepackage{capt-of}
\usepackage{float}
\usepackage[ruled]{algorithm2e}
\usepackage{mathrsfs}







\newcommand{\Lc}[2]{{\color{blue} \sout{#1} } \textcolor{red}{#2}}
\newcommand{\La}[1]{\textcolor{red}{#1}}
\newcommand{\lh}{\mathscr{L}_h}
\newcommand{\cl}{\mathscr{L}}
\newcommand{\cf}{\mathscr{F}}
\newcommand{\dx}{dx}
% \newcommand{\d}{d}
\newcommand{\ltn}{\mathscr{l}^2}
\newcommand{\bbR}{\mathbb{R}}
\newcommand{\Rset}{\mathbb{R}}
\newcommand{\Nset}{\mathbb{N}}
\newcommand{\scL}{\mathcal{L}}
\newcommand{\xx}{\mathbf{x}}
\newcommand{\norm}[1]{\|{#1}\|}
\newcommand{\yy}{\mathbf{y}}
\newcommand{\at}[1]{\big|_{#1}}
\renewcommand{\div}{\mathrm{div}}
\newcommand{\divergence}{\mathrm{div}}
\newcommand{\cp}[1]{\textcolor{blue}{#1}}

\newcommand{\FF}{\texttt{FreeFem++ }}
\newcommand{\FFns}{\texttt{FreeFem++}}
\newcommand{\FFfull}{\texttt{FreeFem++-x11}}
\newcommand{\cmd}[1]{ \medskip \noindent \texttt{#1} \medskip}
\newcommand{\incmd}[1]{\texttt{#1}}
\newcommand{\shrinkitems}{\addtolength{\itemsep}{-0.5\baselineskip}}
\newcommand{\mtt}[1]{\mathtt{#1}}
\newcommand{\ML}{\texttt{Matlab }}

\newcommand{\bb}{\mathbf{b}}
\newcommand{\nn}{\mathbf{n}}
\newcommand{\vecA}{\vec{A}}
\newcommand{\vecB}{\vec{B}}


\newcommand{\frakF}{\mathfrak{F}}


\newcommand{\mesh}{\mathcal{T}_h}
\newcommand{\refel}{\widehat{K}}
\newcommand{\ver}{\mathbf{a}}
\newcommand{\refver}{\widehat{\mathbf{a}}}
\newcommand{\grad}{\nabla}
\newcommand{\refgrad}{\widehat{\nabla}}
\newcommand{\refu}{\widehat{u}}
\newcommand{\refbasis}{\widehat{\varphi}}
\newcommand{\refxx}{\widehat{\xx}}
\newcommand{\refx}{\widehat{x}}
\newcommand{\refy}{\widehat{y}}
\newcommand{\refrho}{\widehat{\rho}}
\newcommand{\refh}{\widehat{h}}

\makeatletter
\newcommand\suchthat{%
 \@ifstar
  {\mathrel{}\middle|\mathrel{}}
  {\mid}%
}
\makeatother





% For typesetting Python code
\newcommand{\matlab}{{\sc Matlab}\xspace}
\usepackage{listings}
\lstloadlanguages{Python}
\lstloadlanguages{csh}%
\definecolor{MyDarkGreen}{rgb}{0.0,0.4,0.0}
\definecolor{purple}{rgb}{0.58,0,0.82}
\lstset{language=Python,                    % Use Python
	%frame=single,                          % Single frame around code
	basicstyle=\ttfamily\footnotesize\color{black},
	keywordstyle=[1]\color{blue}\bf,        % Python functions bold and blue
	keywordstyle=[2]\color{purple},         % Python function arguments purple
	keywordstyle=[3]\color{red}\underbar,   % User functions underlined and blue
	commentstyle=\usefont{T1}{pcr}{m}{sl}\color{MyDarkGreen}\small,
	stringstyle=\color{purple},
	showstringspaces=false,                 % Don't put marks in string spaces
	tabsize=3,                              % 5 spaces per tab
	morekeywords={xlim,ylim,var,alpha,factorial,poissrnd,normpdf,normcdf},
	morecomment=[l][\color{blue}]{...},
	breaklines=true,
	breakatwhitespace=true,
	emptylines=1,
	mathescape=true,
	xleftmargin=0ex,
	emphstyle=\bfseries\color{red}
}





%%%%%%%%%%% MACROS NAMES
\newcommand{\lecturername}{Martin Licht}
% \newcommand{\assistantnamea}{Jochen Hinz}
% \newcommand{\assistantnameb}{Ivan Bioli}
\newcommand{\semestername}{Winter Semester 2024}
\newcommand{\lecturename}{Analysis III - 203(d)}
\DeclarePairedDelimiter\floor{\lfloor}{\rfloor}

%%%%%%%%%%% HEADER
\newdateformat{yeardate}{\THEYEAR}
\newcommand{\exsheet}[3] % input is the number of the session and the day TODO What's that
{\clearpage

	\begin{center}
		{\Large \textbf{\lecturename}}\\[2ex]
		\semestername
	\end{center}

	% \vspace{2ex}
	% \lecturername

	\vspace{2ex}
	{\Large Session #1: #3\,#2, \yeardate\today}
	%\hfill
	%{\Large EPF Lausanne}

	\hrulefill
}





\usepackage{comment}

\newtheorem{exercise}{Exercise}
\newtheorem{solutionenv}{Solution}

\newboolean{hide_solution}
\ifx\hidesolutions\undefined
\newenvironment{solution}{\begin{solutionenv}}{\end{solutionenv}}
\setboolean{hide_solution}{false}
\else
\excludecomment{solution}
\setboolean{hide_solution}{true}
\fi

\newcommand{\ifnotsolution}[1]{\ifthenelse{\boolean{hide_solution}}{#1}{}}
\newcommand{\ifsolution}[1]{\ifthenelse{\boolean{hide_solution}}{}{#1}}


\usepackage{tikz}
\usepackage{pgfplots}





\allowdisplaybreaks

% \usepackage{enumitem}
\begin{document}
\exsheet{13}{5}{December} % parameters are the number of the session and the day


\begin{exercise}
    Consider the following Fourier sine series for two functions $u, f : \bbR \to \bbR$ with period $T$:
    \begin{align*}
        u(x) = \sum_{n=1}^{\infty} b_{n}^{u} \sin\left( \frac{2\pi n}{T} x \right)
        ,
        \quad 
        f(x) = \sum_{n=1}^{\infty} b_{n}^{f} \sin\left( \frac{2\pi n}{T} x \right)
        .
    \end{align*}
    Express the Fourier coefficients of $f$ by the Fourier coefficients of $u$, and vice-versa, express the Fourier coefficients of $u$ by the Fourier coefficients of $f$,
    in the following cases:
    \begin{enumerate}[label=(\alph*)]
     \item 
     $-u''(x) = f(x)$.
     \item 
     $u''''(x) = f(x)$.
     \item 
     $-u''(x) + 25 u(x) = f(x)$.
     \item 
     $u''''(x) - u''(x) = f(x)$.
     \item 
     $-u''(x) + \gamma \cdot u''(x) = f(x)$.
    \end{enumerate}
    In the last item, $\gamma \in \bbR$ is a constant. For which values of $\gamma$ can you always find a solution?
\end{exercise}
\begin{solution}     
\end{solution}

\begin{exercise}
    Consider the function 
    \begin{align*}
        f : [0,1] \to \bbR, \quad x \mapsto x^{3}.
    \end{align*}
    Extend this to an odd function with period $T = 2$. Sketch the graph of that function from $-2$ to $2$.
    Compute its Fourier coefficients in standard form. Compute the complex Fourier coefficients.
\end{exercise}
\begin{solution}     
\end{solution}




\begin{exercise}
    Given the following functions over an interval $[0,1)$, 
    \begin{enumerate}[label=(\alph*)]
        \item $f(x) = x$
        \item $g(x) = x^2$
        \item $h(x) = e^x$
        \item $s(x) = sin(\pi x)$
    \end{enumerate}
    sketch their extension to 
    \begin{itemize}
        \item a periodic function with period $1$,
        \item an even periodic function with period $2$,
        \item an odd periodic function with period $2$.
    \end{itemize}
    and state the formulas for these functions over $[-1,1]$.
\end{exercise}

\begin{solution}     
\end{solution}



\begin{exercise}[Fun with Neumann boundary conditions 1]
    Consider the Poisson problem with Neumann boundary conditions over the interval $[a,b]$ with constant right-hand side: 
    \begin{gather*}
        - u''(x) = c, \quad a < x < b,
        \\
        u'(a) = h_a, \quad u'(b) = h_b.
    \end{gather*}
    \begin{enumerate}[label=(\alph*)]
        \item
        What is the compatibility condition?
        \item 
        Since $u$ has a constant second derivative, it must be a quadratic function $u(x) = c_1 x^2 + c_2 x + c_3$. 
        Assuming the compatibility condition, find the solution(s).
    \end{enumerate}
\end{exercise}
\begin{solution}     
\end{solution}

\begin{exercise}[Fun with Neumann boundary conditions 2]
    Suppose we have got a Poisson problem with Neumann boundary conditions over the interval $[a,b]$. 
    \begin{gather*}
        - u''(x) = f(x), \quad a < x < b,
        \\
        u'(a) = h_a, \quad u'(b) = h_b.
    \end{gather*}
    We need the compatibility condition 
    \begin{gather*}
        \int_a^b f(x) dx + ( h_b - h_a ) = 0.
    \end{gather*}
    \begin{enumerate}[label=(\alph*)]
        \item
        Find a constant function $c^f : [a,b] \to \bbR$ such that $f - c_f$ has integral zero.
        \item
        Use this to split up the original problem into two subproblems: 
        one with homogeneous boundary conditions, one with constant right-hand side, such that both problems satisfy the compatibility condition.
    \end{enumerate}
\end{exercise}
\begin{solution}     
\end{solution}

\end{document}
