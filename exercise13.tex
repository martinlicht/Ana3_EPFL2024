\documentclass[11pt]{article}
% \def\hidesolutions{}
%%%%%%%%%%% SET MARGINS
\setlength{\textheight}{20cm}
\setlength{\topmargin}{-0.5cm}
\setlength{\oddsidemargin}{+0cm}
\setlength{\textwidth}{16.3cm}
%\setlength{\parskip}{6pt}
\setlength{\parindent}{0pt}

%%%%%%%%%%% PACKAGES
\usepackage{amsmath}
\usepackage{amssymb}
\usepackage{amsfonts}
%\usepackage{a4wide}
\usepackage{graphicx}
\usepackage{color}
\usepackage[normalem]{ulem}
\usepackage{enumitem}
\usepackage{capt-of}
\usepackage{float}
\usepackage{amsmath}
\usepackage{listings}
\definecolor{mygreen}{RGB}{28,172,0} % color values Red, Green, Blue
\definecolor{mylilas}{RGB}{170,55,241}
\usepackage{empheq}
\usepackage[ruled]{algorithm2e}
\usepackage{mathrsfs}
\usepackage{datetime}
\usepackage{subcaption}

% TODO: combine the two package lists and reduce redundancies 
\usepackage{mathtools}
\usepackage{nicefrac}
\usepackage{hyperref}
\usepackage{url}
\usepackage{amsmath,amssymb,amsfonts}
\usepackage{a4wide}
\usepackage{graphicx}
\usepackage{color}
\usepackage[normalem]{ulem}
\usepackage{capt-of}
\usepackage{float}
\usepackage[ruled]{algorithm2e}
\usepackage{amsmath,amssymb,amsfonts}
\usepackage{a4wide}
\usepackage{graphicx}
\usepackage{color}
\usepackage[normalem]{ulem}
\usepackage{capt-of}
\usepackage{float}
\usepackage[ruled]{algorithm2e}
\usepackage{mathrsfs}







\newcommand{\Lc}[2]{{\color{blue} \sout{#1} } \textcolor{red}{#2}}
\newcommand{\La}[1]{\textcolor{red}{#1}}
\newcommand{\lh}{\mathscr{L}_h}
\newcommand{\cl}{\mathscr{L}}
\newcommand{\cf}{\mathscr{F}}
\newcommand{\dx}{dx}
\newcommand{\ltn}{\mathscr{l}^2}
\newcommand{\bbR}{\mathbb{R}}
\newcommand{\Rset}{\mathbb{R}}
\newcommand{\Nset}{\mathbb{N}}
\newcommand{\scL}{\mathcal{L}}
\newcommand{\xx}{\mathbf{x}}
\newcommand{\norm}[1]{\|{#1}\|}
\newcommand{\yy}{\mathbf{y}}
\newcommand{\at}[1]{\big|_{#1}}
\renewcommand{\div}{\mathrm{div}}
\newcommand{\divergence}{\mathrm{div}}
\newcommand{\cp}[1]{\textcolor{blue}{#1}}

\newcommand{\FF}{\texttt{FreeFem++ }}
\newcommand{\FFns}{\texttt{FreeFem++}}
\newcommand{\FFfull}{\texttt{FreeFem++-x11}}
\newcommand{\cmd}[1]{ \medskip \noindent \texttt{#1} \medskip}
\newcommand{\incmd}[1]{\texttt{#1}}
\newcommand{\shrinkitems}{\addtolength{\itemsep}{-0.5\baselineskip}}
\newcommand{\mtt}[1]{\mathtt{#1}}
\newcommand{\ML}{\texttt{Matlab }}

\newcommand{\bb}{\mathbf{b}}
\newcommand{\nn}{\mathbf{n}}
\newcommand{\vecA}{\vec{A}}
\newcommand{\vecB}{\vec{B}}


\newcommand{\mesh}{\mathcal{T}_h}
\newcommand{\refel}{\widehat{K}}
\newcommand{\ver}{\mathbf{a}}
\newcommand{\refver}{\widehat{\mathbf{a}}}
\newcommand{\grad}{\nabla}
\newcommand{\refgrad}{\widehat{\nabla}}
\newcommand{\refu}{\widehat{u}}
\newcommand{\refbasis}{\widehat{\varphi}}
\newcommand{\refxx}{\widehat{\xx}}
\newcommand{\refx}{\widehat{x}}
\newcommand{\refy}{\widehat{y}}
\newcommand{\refrho}{\widehat{\rho}}
\newcommand{\refh}{\widehat{h}}






% For typesetting Python code
\newcommand{\matlab}{{\sc Matlab}\xspace}
\usepackage{listings}
\lstloadlanguages{Python}
\lstloadlanguages{csh}%
\definecolor{MyDarkGreen}{rgb}{0.0,0.4,0.0}
\definecolor{purple}{rgb}{0.58,0,0.82}
\lstset{language=Python,                    % Use Python
	%frame=single,                          % Single frame around code
	basicstyle=\ttfamily\footnotesize\color{black},
	keywordstyle=[1]\color{blue}\bf,        % Python functions bold and blue
	keywordstyle=[2]\color{purple},         % Python function arguments purple
	keywordstyle=[3]\color{red}\underbar,   % User functions underlined and blue
	commentstyle=\usefont{T1}{pcr}{m}{sl}\color{MyDarkGreen}\small,
	stringstyle=\color{purple},
	showstringspaces=false,                 % Don't put marks in string spaces
	tabsize=3,                              % 5 spaces per tab
	morekeywords={xlim,ylim,var,alpha,factorial,poissrnd,normpdf,normcdf},
	morecomment=[l][\color{blue}]{...},
	breaklines=true,
	breakatwhitespace=true,
	emptylines=1,
	mathescape=true,
	xleftmargin=0ex,
	emphstyle=\bfseries\color{red}
}





%%%%%%%%%%% MACROS NAMES
\newcommand{\lecturername}{Martin Licht}
% \newcommand{\assistantnamea}{Jochen Hinz}
% \newcommand{\assistantnameb}{Ivan Bioli}
\newcommand{\semestername}{Winter Semester 2023}
\newcommand{\lecturename}{Analysis III - 202(c)}
\DeclarePairedDelimiter\floor{\lfloor}{\rfloor}

%%%%%%%%%%% HEADER
\newdateformat{yeardate}{\THEYEAR}
\newcommand{\exsheet}[3] % input is the number of the session and the day TODO What's that
{\clearpage

	\begin{center}
		{\Large \textbf{\lecturename}}\\[2ex]
		\semestername
	\end{center}

	% \vspace{2ex}
	% \lecturername

	\vspace{2ex}
	{\Large Session #1: #3\,#2, \yeardate\today}
	%\hfill
	%{\Large EPF Lausanne}

	\hrulefill
}





\usepackage{comment}

\newtheorem{exercise}{Exercise}
\newtheorem{solutionenv}{Solution}

\newboolean{hide_solution}
\ifx\hidesolutions\undefined
\newenvironment{solution}{\begin{solutionenv}}{\end{solutionenv}}
\setboolean{hide_solution}{false}
\else
\excludecomment{solution}
\setboolean{hide_solution}{true}
\fi

\newcommand{\ifnotsolution}[1]{\ifthenelse{\boolean{hide_solution}}{#1}{}}
\newcommand{\ifsolution}[1]{\ifthenelse{\boolean{hide_solution}}{}{#1}}








\allowdisplaybreaks

% \usepackage{enumitem}
\begin{document}
\exsheet{13}{12}{December} % parameters are the number of the session and the day








\begin{exercise}
    Draw the complex exponentials $e^{zt}$ in the complex plane for $t=0,1,2,3,4$, 
    where $z$ is one of the following complex numbers:
    \begin{gather*}
        z_1 = \frac 1 2,
        \quad 
        z_2 = - \frac 1 2,
        \quad 
        z_3 = 0.2i,
        \quad 
        z_4 = -0.2i,
        \quad 
        z_3 = - \frac 1 2 + 0.2i
        .
    \end{gather*}
    \textit{You may use a calculater.}
\end{exercise}

\begin{solution}     
\end{solution}




\begin{exercise}
    The Fourier transform of 
    \begin{align*}
        f(x) = e^{-5x^2}
    \end{align*}
    is the function 
    \begin{align*}
        \hat f(x) = \frac{1}{\sqrt{10}} e^{-\frac{\alpha^2}{20}}.
    \end{align*}
    Find the Fourier transforms of 
    \begin{align*}
     f', \quad f'', \quad f''', \quad f'''', \quad g(x) = f(2x), \quad h(x) = f(x-3).
    \end{align*}
\end{exercise}
\begin{solution}  
    To evaluate the Fourier transform of the derivative of a function we will use the following identity from the lecture:
    \begin{gather*}
        \mathfrak{F}\left(f^{(n)}\right)(\alpha)=(i \alpha)^n \mathfrak{F}(f)(\alpha)
    \end{gather*}
    We then find 
    \begin{itemize}
    \item $\mathfrak{F}\left(f' \right)(\alpha) = i \alpha \mathfrak{F}(f)(\alpha) = \frac{i \alpha}{\sqrt{10}} e^{-\frac{\alpha^2}{20}}$
    \item $\mathfrak{F}\left(f'' \right)(\alpha) = -\alpha^2 \mathfrak{F}(f)(\alpha) = -\frac{ \alpha^2}{\sqrt{10}} e^{-\frac{\alpha^2}{20}}$
    \item $\mathfrak{F}\left(f' \right)(\alpha) = -i \alpha^3 \mathfrak{F}(f)(\alpha) = \frac{-i \alpha^3}{\sqrt{10}} e^{-\frac{\alpha^2}{20}}$
    \item $\mathfrak{F}\left(f' \right)(\alpha) = \alpha^4 \mathfrak{F}(f)(\alpha) = \frac{ \alpha^4}{\sqrt{10}} e^{-\frac{\alpha^2}{20}}$
    \end{itemize}
    For the Fourier transform of $g(x)$ we use the following identity from the lecture: 
    $g(x)=e^{-i b x} f(a x)$ has the Fourier transform
    $\hat{g}(\alpha)=\frac{1}{|a|} \hat{f}\left(\frac{\alpha+b}{a}\right)$,
    consequently:
    \begin{align*}
        \mathfrak{F}(g(x))(\alpha) & =\mathfrak{F}(f(2 x))(\alpha) \\ & =\frac{1}{2} \mathfrak{F}(f(x))\left(\frac{\alpha}{2}\right) \\ & =\frac{1}{2 \sqrt{10}} e^{-\frac{\left(\frac{\alpha}{2}\right)^2}{20}} \\ & =\frac{1}{2 \sqrt{10}} e^{-\frac{\alpha^2}{80}}
    \end{align*}
    For the Fourier transform of $h(x)$ we use a change of variables:
    \begin{align*}
        \mathfrak{F}(h(x))(\alpha)
        &= 
        \frac{1}{\sqrt{2\pi}}
        \int_{-\infty}^{+\infty} e^{-5(x-3)^2} e^{ - \imath x \alpha } \;dx
        \\&= 
        \frac{1}{\sqrt{2\pi}}
        \int_{-\infty}^{+\infty} e^{-5x^2} e^{ - \imath (x + 3) \alpha } \;dx
        \\&= 
        \frac{1}{\sqrt{2\pi}}
        \int_{-\infty}^{+\infty} e^{-5x^2} e^{ - \imath x \alpha } e^{ - \imath 3 \alpha }\;dx
        \\&= 
        \frac{ e^{ - \imath 3 \alpha }}{\sqrt{2\pi}}
        \int_{-\infty}^{+\infty} e^{-5x^2} e^{ - \imath x \alpha }\;dx
        \\&= 
        e^{ - \imath 3 \alpha } 
        \mathfrak{F}(f(x))(\alpha)
        \\&= 
        \frac{e^{ - \imath 3 \alpha }}{\sqrt{10}} e^{-\frac{\alpha^2}{20}}.
        \\&= 
        \frac{1}{\sqrt{10}} e^{-\frac{\alpha^2}{20} - \imath 3 \alpha }
        .
    \end{align*}
\end{solution}






































\begin{exercise}
    Find $f : \mathbb R \to \mathbb R$ such that 
    \begin{align*}
        \hat f(\alpha) = \frac{3}{1+\alpha^{2}} + \frac{-1}{1+4\alpha^{2}} + \frac{ \sin( 4 \alpha + 3 ) }{ 4 \alpha + 3 }
    \end{align*}
\end{exercise}
\begin{solution}   Because of the linearity property we can treat the inverse of each term in seperately. Therefore 
    First,
    \begin{align*} 
        \mathfrak{F}^{-1}\left(\frac{\sin (4 \alpha+3)}{4 a r+3}\right)(x) & =\frac{\frac{1}{4}}{\frac{1}{4}} \mathfrak{F}^{-1}\left(\frac{\sin \left(\frac{\alpha+3 / 4}{\frac{1}{4}}\right)}{\frac{\alpha+\frac{3}{4}}{\frac{1}{4}}}\right)(x) \\ & =\frac{1}{4} e^{-i \frac{3}{4} x} \sqrt{\frac{\pi}{2}} \mathfrak{F}^{-1}\left(\sqrt{\frac{2}{\pi}} \frac{\sin (\alpha)}{\alpha}\right)\left(\frac{x}{4}\right) \\ & = \begin{cases}\frac{1}{4} \sqrt{\frac{\pi}{2}} e^{-i \frac{3}{4} x} & \text { if }-4 \leq x \leq 4 \\ 0 & \text { else }\end{cases} 
    \end{align*}
    Second,
    \begin{align*} 
        \mathfrak{F}^{-1}\left(\frac{3}{1+\alpha^2}\right)(x) & =3 \sqrt{\frac{\pi}{2}} \mathfrak{F}^{-1}\left(\sqrt{\frac{2}{\pi}} \frac{1}{1+a^2}\right)(x) \\ & =3 \sqrt{\frac{\pi}{2}} e^{-|x|}
    \end{align*}
    Lastly,
    \begin{align*} 
        \mathfrak{F}^{-1}\left(\frac{-1}{1+4 \alpha^2}\right)(x) & =-\sqrt{\frac{\pi}{2}} \mathfrak{F}^{-1}\left(\sqrt{\frac{2}{\pi}} \frac{1}{1+(2 \alpha)^2}\right)(x) \\ & =-\sqrt{\frac{\pi}{2}} \frac{\frac{1}{2}}{\frac{1}{2}} \mathfrak{F}^{-1}\left(\sqrt{\frac{2}{\pi}} \frac{1}{1+\left(\frac{\alpha}{\frac{1}{2}}\right)^2}\right)(x \mid \\ & =-\sqrt{\frac{\pi}{2}} \frac{1}{2} \mathfrak{F}^{-1}\left(\sqrt{\frac{2}{\pi}} \frac{1}{1+\alpha^2}\right)\left(\frac{x}{2}\right) \\ & =-\frac{1}{2} \sqrt{\frac{\pi}{2}} e^{-\left|\frac{x}{2}\right|}
    \end{align*}
    Altogether this gives
    \begin{align*} 
        f(x) = \begin{cases}-\frac{1}{2} \sqrt{\frac{\pi}{2}} e^{-\left|\frac{x}{2}\right|}+3 \sqrt{\frac{\pi}{2}} e^{-|x|}+\frac{1}{4} \sqrt{\frac{\pi}{2}} e^{-i \frac{3}{4} x} & -4 \leq x \leq 4 \\ -\frac{1}{2} \sqrt{\frac{\pi}{2}} e^{-\left|\frac{x}{2}\right|}+3 \sqrt{\frac{\pi}{2}} e^{-|x|} & \text { else }\end{cases}
    \end{align*}
    
\end{solution}


















\begin{exercise}
    We consider the Poisson problem with Dirichlet boundary conditions over the interval $[0,L]$:
    \begin{gather*}
        - \Delta u(x) = x^2, \quad 0 < x < L,
        \\
        u(0) = 1, \quad u(L) = 2
    \end{gather*}
    \begin{itemize}
        \item Solve this problem directly. The solution is a polynomial of order $4$.
        \item Extend the right-hand side $f(x) = x^2$ to an odd function with period $2L$ and compute its Fourier coefficients.
        \item How do you use the superposition principle to split the problem? Using these coefficients, find the Fourier series of the solution $u^f$,
        which solves 
        \begin{gather*}
            - \Delta u^f(x) = x^2, \quad 0 < x < L,
            \\
            u^f(0) = 0, \quad u^f(L) = 0.
        \end{gather*}
    \end{itemize}
\end{exercise}
\begin{solution}     
\end{solution}

\begin{exercise}
    Directly compute the solution of the problem
    \begin{gather*}
        - \Delta u(x) = x^2, \quad 0 < x < L,
        \\
        u(0) = 0, \quad u(L) = 0,
    \end{gather*}
    using elementary analysis. Then find its Fourier coefficients. Compare this with the function $u^f$ from the previous exercise.
\end{exercise}
\begin{solution}     
\end{solution}



















\begin{exercise}
    We solve the Poisson problem with homogeneous Dirichlet boundary conditions over $[0,1]$:
    \begin{gather*}
        - \Delta u(x) = \left\{\begin{array}{ll} x & \text{ if } 0 < x \leq 0.5 \\ 0 & \text{ if } 0.5 < x \leq 1 \end{array}\right., \qquad 0 < x < 1,
        \\
        u(0) = 0, \quad u(1) = 0
    \end{gather*}
    \begin{itemize}
        \item Extend the right-hand side $f(x)$ to an odd function with period $2$ and compute its Fourier coefficients.
        \item Using these coefficients, find the Fourier series of the solution $u$. Verify that the boundary condition $u(0) = u(1) = 0$ is satisfied.
    \end{itemize}
\end{exercise}
\begin{solution}     
\end{solution}














\begin{exercise}
    Consider the following Fourier sine series for two functions $u, f : \bbR \to \bbR$ with period $T$:
    \begin{align*}
        u(x) = \sum_{n=1}^{\infty} b_{n}^{u} \sin\left( \frac{2\pi n}{T} x \right)
        ,
        \quad 
        f(x) = \sum_{n=1}^{\infty} b_{n}^{f} \sin\left( \frac{2\pi n}{T} x \right)
        .
    \end{align*}
    Express the Fourier coefficients of $f$ by the Fourier coefficients of $u$, and vice-versa, express the Fourier coefficients of $u$ by the Fourier coefficients of $f$,
    in the following cases:
    \begin{enumerate}[label=(\alph*)]
     \item 
     $-u''(x) = f(x)$.
     \item 
     $u''''(x) = f(x)$.
     \item 
     $-u''(x) + 25 u(x) = f(x)$.
     \item 
     $u''''(x) - u''(x) = f(x)$.
     \item 
     $-u''(x) + \gamma \cdot u''(x) = f(x)$.
    \end{enumerate}
    In the last item, $\gamma \in \bbR$ is a constant. For which values of $\gamma$ can you always find a solution?
\end{exercise}
\begin{solution}     
\end{solution}





\begin{exercise}
    Recall the definition of the Fourier transform:
    \begin{align}
        \mathfrak{F}[f](\alpha) = \frac{1}{\sqrt{2\pi}} \int_{-\infty}^{+\infty} f(t) e^{-it\alpha} \ d\alpha
    \end{align}
    Let $a,b,c \in \mathbb R$ be real parameters with $a \neq 0$. Compute the Fourier transforms of the following:
    \begin{align}
        g(t) = f( a t ),
        \\
        h(t) = e^{-i t b} f(t),
        \\
        m(t) = f( t - c ).
    \end{align}
    \textit{Hint: } you have the first two in the lecture and in textbook. You can compute them using results from the lecture or via some standard integral manipulations. 
\end{exercise}

\begin{solution}
We write \( {\hat f}(\alpha) = \mathfrak{F}[f](\alpha) \) for the Fourier transform of $\alpha$. 
To compute the Fourier transforms of the given functions \( g(t) \), \( h(t) \), and \( m(t) \), we proceed as follows:
\begin{itemize}
\item 
Let \( {\hat g}(\alpha) = \mathfrak{F}[g](\alpha) \). By definition,
\begin{gather*}
    {\hat g}(\alpha) 
    = 
    \frac{1}{\sqrt{2\pi}} \int_{-\infty}^{+\infty} g(t) e^{-it\alpha} \, dt
    =
    \frac{1}{\sqrt{2\pi}} \int_{-\infty}^{+\infty} f(at) e^{-it\alpha} \, dt.
\end{gather*}
We perform a substitution \( u = at \), so \( du = |a| \, dt \) and \( t = \frac{u}{a} \). This gives
\begin{gather*}
    {\hat g}(\alpha) 
    = 
    \frac{1}{\sqrt{2\pi}} \int_{-\infty}^{+\infty} f(u) e^{-i \frac{u}{a} \alpha} \frac{du}{a}
    =
    \frac{1}{\sqrt{2\pi}} \cdot \frac{1}{a} \int_{-\infty}^{+\infty} f(u) e^{-i \frac{\alpha}{a} u} \, du.
\end{gather*}
Recognizing the integral as the Fourier transform of \( f(t) \), we find:
\begin{gather*}
    {\hat g}(\alpha) = \frac{1}{|a|} {\hat f}\left(\frac{\alpha}{a}\right),
\end{gather*}
\item 
Let \( {\hat h}(\alpha) = \mathfrak{F}[h](\alpha) \). Using definitions. 
\begin{align*}
    {\hat h}(\alpha) 
    &
    = 
    \frac{1}{\sqrt{2\pi}} \int_{-\infty}^{+\infty} h(t) e^{-it\alpha} \, dt
    \\&
    =
    \frac{1}{\sqrt{2\pi}} \int_{-\infty}^{+\infty} e^{-ibt} f(t) e^{-it\alpha} \, dt
    =
    \frac{1}{\sqrt{2\pi}} \int_{-\infty}^{+\infty} f(t) e^{-it(\alpha+b)} \, dt.
\end{align*}
We recognizing the integral as the Fourier transform of \( f(t) \). Thus, 
\begin{gather*}
    {\hat h}(\alpha) = {\hat f}(\alpha + b).
\end{gather*}
\item 
Let \( {\hat m}(\alpha) = \mathfrak{F}[m](\alpha) \). By definition,
\begin{gather*}
    {\hat m}(\alpha) 
    = 
    \frac{1}{\sqrt{2\pi}} \int_{-\infty}^{+\infty} m(t) e^{-it\alpha} \, dt
    =
    \frac{1}{\sqrt{2\pi}} \int_{-\infty}^{+\infty} f(t - c) e^{-it\alpha} \, dt.
\end{gather*}
We substitute \( u = t - c \), so \( du = dt \) and \( t = u + c \). This gives
\begin{gather*}
    {\hat m}(\alpha) 
    = 
    \frac{1}{\sqrt{2\pi}} \int_{-\infty}^{+\infty} f(u) e^{-i(u+c)\alpha} \, du
    =
    \frac{1}{\sqrt{2\pi}} \int_{-\infty}^{+\infty} f(u) e^{-iu\alpha} e^{-ic\alpha} \, du
    .
\end{gather*}
We pull out the factor out \( e^{-ic\alpha} \),
\begin{gather*}
    {\hat m}(\alpha) = e^{-ic\alpha} \cdot \frac{1}{\sqrt{2\pi}} \int_{-\infty}^{+\infty} f(u) e^{-iu\alpha} \, du,
\end{gather*}
and recognize the last integral as the Fourier transform of \( f(t) \). In summary, 
\begin{gather*}
    {\hat m}(\alpha) = e^{-ic\alpha} {\hat f}(\alpha).
\end{gather*}
\end{itemize}
\end{solution}









\end{document}
