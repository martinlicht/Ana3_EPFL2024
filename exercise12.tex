\documentclass[11pt]{article}
% \def\hidesolutions{}
%%%%%%%%%%% SET MARGINS
\setlength{\textheight}{20cm}
\setlength{\topmargin}{-0.5cm}
\setlength{\oddsidemargin}{+0cm}
\setlength{\textwidth}{16.3cm}
%\setlength{\parskip}{6pt}
\setlength{\parindent}{0pt}

%%%%%%%%%%% PACKAGES
\usepackage{amsmath}
\usepackage{amssymb}
\usepackage{amsfonts}
%\usepackage{a4wide}
\usepackage{graphicx}
\usepackage{color}
\usepackage[normalem]{ulem}
\usepackage{enumitem}
\usepackage{capt-of}
\usepackage{float}
\usepackage{amsmath}
\usepackage{listings}
\definecolor{mygreen}{RGB}{28,172,0} % color values Red, Green, Blue
\definecolor{mylilas}{RGB}{170,55,241}
\usepackage{empheq}
\usepackage[ruled]{algorithm2e}
\usepackage{mathrsfs}
\usepackage{datetime}
\usepackage{subcaption}

% TODO: combine the two package lists and reduce redundancies 
\usepackage{mathtools}
\usepackage{nicefrac}
\usepackage{hyperref}
\usepackage{url}
\usepackage{amsmath,amssymb,amsfonts}
\usepackage{a4wide}
\usepackage{graphicx}
\usepackage{color}
\usepackage[normalem]{ulem}
\usepackage{capt-of}
\usepackage{float}
\usepackage[ruled]{algorithm2e}
\usepackage{amsmath,amssymb,amsfonts}
\usepackage{a4wide}
\usepackage{graphicx}
\usepackage{color}
\usepackage[normalem]{ulem}
\usepackage{capt-of}
\usepackage{float}
\usepackage[ruled]{algorithm2e}
\usepackage{mathrsfs}







\newcommand{\Lc}[2]{{\color{blue} \sout{#1} } \textcolor{red}{#2}}
\newcommand{\La}[1]{\textcolor{red}{#1}}
\newcommand{\lh}{\mathscr{L}_h}
\newcommand{\cl}{\mathscr{L}}
\newcommand{\cf}{\mathscr{F}}
\newcommand{\dx}{dx}
\newcommand{\ltn}{\mathscr{l}^2}
\newcommand{\bbR}{\mathbb{R}}
\newcommand{\Rset}{\mathbb{R}}
\newcommand{\Nset}{\mathbb{N}}
\newcommand{\scL}{\mathcal{L}}
\newcommand{\xx}{\mathbf{x}}
\newcommand{\norm}[1]{\|{#1}\|}
\newcommand{\yy}{\mathbf{y}}
\newcommand{\at}[1]{\big|_{#1}}
\renewcommand{\div}{\mathrm{div}}
\newcommand{\divergence}{\mathrm{div}}
\newcommand{\cp}[1]{\textcolor{blue}{#1}}

\newcommand{\FF}{\texttt{FreeFem++ }}
\newcommand{\FFns}{\texttt{FreeFem++}}
\newcommand{\FFfull}{\texttt{FreeFem++-x11}}
\newcommand{\cmd}[1]{ \medskip \noindent \texttt{#1} \medskip}
\newcommand{\incmd}[1]{\texttt{#1}}
\newcommand{\shrinkitems}{\addtolength{\itemsep}{-0.5\baselineskip}}
\newcommand{\mtt}[1]{\mathtt{#1}}
\newcommand{\ML}{\texttt{Matlab }}

\newcommand{\bb}{\mathbf{b}}
\newcommand{\nn}{\mathbf{n}}
\newcommand{\vecA}{\vec{A}}
\newcommand{\vecB}{\vec{B}}


\newcommand{\mesh}{\mathcal{T}_h}
\newcommand{\refel}{\widehat{K}}
\newcommand{\ver}{\mathbf{a}}
\newcommand{\refver}{\widehat{\mathbf{a}}}
\newcommand{\grad}{\nabla}
\newcommand{\refgrad}{\widehat{\nabla}}
\newcommand{\refu}{\widehat{u}}
\newcommand{\refbasis}{\widehat{\varphi}}
\newcommand{\refxx}{\widehat{\xx}}
\newcommand{\refx}{\widehat{x}}
\newcommand{\refy}{\widehat{y}}
\newcommand{\refrho}{\widehat{\rho}}
\newcommand{\refh}{\widehat{h}}






% For typesetting Python code
\newcommand{\matlab}{{\sc Matlab}\xspace}
\usepackage{listings}
\lstloadlanguages{Python}
\lstloadlanguages{csh}%
\definecolor{MyDarkGreen}{rgb}{0.0,0.4,0.0}
\definecolor{purple}{rgb}{0.58,0,0.82}
\lstset{language=Python,                    % Use Python
	%frame=single,                          % Single frame around code
	basicstyle=\ttfamily\footnotesize\color{black},
	keywordstyle=[1]\color{blue}\bf,        % Python functions bold and blue
	keywordstyle=[2]\color{purple},         % Python function arguments purple
	keywordstyle=[3]\color{red}\underbar,   % User functions underlined and blue
	commentstyle=\usefont{T1}{pcr}{m}{sl}\color{MyDarkGreen}\small,
	stringstyle=\color{purple},
	showstringspaces=false,                 % Don't put marks in string spaces
	tabsize=3,                              % 5 spaces per tab
	morekeywords={xlim,ylim,var,alpha,factorial,poissrnd,normpdf,normcdf},
	morecomment=[l][\color{blue}]{...},
	breaklines=true,
	breakatwhitespace=true,
	emptylines=1,
	mathescape=true,
	xleftmargin=0ex,
	emphstyle=\bfseries\color{red}
}





%%%%%%%%%%% MACROS NAMES
\newcommand{\lecturername}{Martin Licht}
% \newcommand{\assistantnamea}{Jochen Hinz}
% \newcommand{\assistantnameb}{Ivan Bioli}
\newcommand{\semestername}{Winter Semester 2023}
\newcommand{\lecturename}{Analysis III - 202(c)}
\DeclarePairedDelimiter\floor{\lfloor}{\rfloor}

%%%%%%%%%%% HEADER
\newdateformat{yeardate}{\THEYEAR}
\newcommand{\exsheet}[3] % input is the number of the session and the day TODO What's that
{\clearpage

	\begin{center}
		{\Large \textbf{\lecturename}}\\[2ex]
		\semestername
	\end{center}

	% \vspace{2ex}
	% \lecturername

	\vspace{2ex}
	{\Large Session #1: #3\,#2, \yeardate\today}
	%\hfill
	%{\Large EPF Lausanne}

	\hrulefill
}





\usepackage{comment}

\newtheorem{exercise}{Exercise}
\newtheorem{solutionenv}{Solution}

\newboolean{hide_solution}
\ifx\hidesolutions\undefined
\newenvironment{solution}{\begin{solutionenv}}{\end{solutionenv}}
\setboolean{hide_solution}{false}
\else
\excludecomment{solution}
\setboolean{hide_solution}{true}
\fi

\newcommand{\ifnotsolution}[1]{\ifthenelse{\boolean{hide_solution}}{#1}{}}
\newcommand{\ifsolution}[1]{\ifthenelse{\boolean{hide_solution}}{}{#1}}








\allowdisplaybreaks

\begin{document}
\exsheet{12}{5}{December} % parameters are the number of the session and the day






\begin{exercise}
    Consider the following functions with period $T$:
    \begin{itemize}
        \item A function $f$ with period $T = 1$ such that 
        \begin{align*}
            f(x) = 
            \left\{\begin{array}{ll}
                1 & \text{if } 0 \leq x < 0.5 \\
                0 & \text{if } 0.5 \leq x \leq 1
            \end{array}\right.
        \end{align*}
        \item A function $g$ with period $T = 2\pi$ such that 
        \begin{align*}
            g(x) = 
            \left\{\begin{array}{ll}
                x & \text{if } 0 \leq x < \pi \\
                2\pi - x & \text{if } \pi \leq x \leq 2\pi
            \end{array}\right.
        \end{align*}
        \item A function $h$ with period $T = 1$ such that 
        \begin{align*}
            h(x) = -x \text{ if } 0 \leq x < 1.
        \end{align*}
    \end{itemize}
    Find the Fourier coefficients of the Fourier series of these functions. 
    \textit{You can modify the functions seen in the lecture to get the coefficients.}
\end{exercise}

\begin{solution}     
    \begin{itemize}
    \item As seen in the lecture we know the Fourier coefficients of a square wave,
    $$
    l(x)= \begin{cases}1, & \text { if } 0 \leq x<0.5 \\ -1, & \text { if } 0.5 \leq x<2 \pi\end{cases},
    $$

    are given by:

    $$
    a_n = 0\quad \text{for } n \geq 0, \quad b_n = \begin{cases}0, & \text { if n is odd} \\ \frac{4}{n\pi}, & \text{ if n is even}\end{cases}
    $$

    We can express the function $f(x)$ in terms of $l(x)$ as follows:

    $$
    f(x) = \frac{1}{2} + \frac{1}{2}l(x)
    $$

    therefore the Fourier coefficients of $f$ are given by:

    $$
    a_n = \begin{cases}\frac{1}{2}, & \text { if n} = 0 \\0,& \text{ if n }>0\end{cases}, \quad b_n = \begin{cases}0, & \text { if n is odd} \\ \frac{2}{n\pi}, & \text{ if n is even}\end{cases}
    $$
    \item  As seen in the lecture we know the Fourier coefficients of a triangle wave,
    $$
    m(x)= \begin{cases}2 x, & \text { if } 0 \leq x<0.5 \\ 2(1-x), & \text { if } 0.5 \leq x<1\end{cases}
    $$

    are given by:

    $$
    a_0 = \frac{1}{2}, \quad a_n = \begin{cases}-\frac{4}{n^2\pi^2}, & \text { if n is odd} \\ 0, & \text{ if n is even}\end{cases}, \quad b_n = 0 \text{ for } n > 0
    $$

    We can express the function $g(x)$ in terms of $m(x)$ as follows:

    $$
    g(x) = \pi m(\frac{x}{2\pi})
    $$

    therefore the Fourier coefficients of $g$ are given by:

    $$
    a_0 = \frac{\pi}{2}, \quad a_n = \begin{cases}-\frac{4}{n^2\pi}, & \text { if n is odd} \\ 0, & \text{ if n is even}\end{cases}, \quad b_n = 0 \text{ for } n > 0
    $$
    \item As seen in the lecture we know the Fourier coefficients of a sawtooth wave,
    $$
    n(x)=x \text { for } 0 \leq x<1
    $$

    are given by:

    $$
    a_0 = \frac{1}{2}, \quad a_n = 0 \text{ for } n > 0,  \quad a_n = -\frac{1}{n\pi} \text{ for } n > 0,
    $$

    We can express the function $h(x)$ in terms of $n(x)$ as follows:

    $$
    h(x) = -n(x)
    $$

    therefore the Fourier coefficients of $h$ are given by:

    $$
    a_0 = -\frac{1}{2}, \quad a_n = 0 \text{ for } n > 0,  \quad a_n = \frac{1}{n\pi} \text{ for } n > 0,
    $$
    \end{itemize}
\end{solution}







\begin{exercise}
    Explicitly write down the coefficients $a_n$ and $b_n$ and the periods of the following Fourier series:
    \begin{align*}
        f(x) &= \sum_{n=1}^{\infty} \frac{1}{(2n+1)^2}\sin(2\pi (2n+1) x),
        \\
        g(x) &= \sum_{n=1}^{\infty} \frac{(-1)^n}{(2n-1)^3}\cos(2\pi (2n-1) x),
        \\
        h(x) &= \frac \pi 3 + \sum_{n=1}^{\infty} \frac{(-1)^{n+1}}{n^2}\cos(\pi n x).
    \end{align*}
    Determine the Fourier coefficients of the derivatives of those functions. 
\end{exercise}
\begin{solution}     
    \begin{itemize}
    \item We substitute $m = 2n+1$ to obtain 
    $$
            f(x) = \sum_{\substack{m=3\\m \text{ is odd}}}^{\infty} \frac{1}{m^2}\cos(2\pi m x),
    $$

    therefore the Fourier coefficients are given by 
    $$
    a_m = 0 \text{ for } m \geq 0, \quad b_m = \begin{cases}\frac{1}{m^2}, & \text { if m is odd and }m \geq 3 \\ 0, & \text{ if m is even and } m \geq 3\\ 0 & \text{ if } 0< m < 3  \end{cases} 
    $$

    and the period is $T = 1$. The derivative of $f(x)$ is given by:
    $$
    f'(x) = \sum_{\substack{m=3\\m \text{ is odd}}}^{\infty} \frac{2\pi}{m}\sin(2\pi m x),
    $$

    therefore the Fourier coefficients of the derivative of $f(x)$ are given by 

    $$
    b_m = 0 \text{ for } m \geq 0, \quad a_m = \begin{cases}\frac{2\pi}{m}, & \text { if m is odd and }m \geq 3 \\ 0, & \text{ if m is even and } m \geq 3\\ 0 & \text{ if } 0< m < 3  \end{cases} 
    $$

    \item We substitute $m = 2n-1$ to obtain 
    $$
        g(x) = \sum_{\substack{ m=1 \\ m \text{ is odd}}}^{\infty} \frac{(-1)^{\frac{m+1}{2}}}{m^3}\cos(2\pi m x),
    $$

    therefore the Fourier coefficients are given by 
    $$
    b_m = 0 \text{ for } m >0, \quad a_m = \begin{cases}\frac{(-1)^{\frac{m+1}{2}}}{m^3}, & \text { if m is odd and }m \geq 1 \\ 0, & \text{ if m is even and } m \geq 1\\ 0 & \text{ if } m = 0  \end{cases} 
    $$

    and the period is $T = 1$. The derivative of $g(x)$ is given by:
    $$
    g'(x) = \sum_{\substack{m=1\\m \text{ is odd}}}^{\infty} -\frac{2\pi (-1)^{\frac{m+1}{2}}}{m^2}\sin(2\pi m x),
    $$

    therefore the Fourier coefficients of the derivative of $g(x)$ are given by 
    $$
    a_m = 0 \text{ for } m \geq 0, \quad b_m = \begin{cases}-\frac{2\pi (-1)^{\frac{m+1}{2}}}{m^2}, & \text { if m is odd and }m \geq 1 \\ 0, & \text{ if m is even and } m \geq 1\\ 0 & \text{ if } m = 0  \end{cases} 
    $$
    \item 
    the Fourier coefficients are given by 
    $$
    b_n = 0 \text{ for } n >0, \quad a_n = \begin{cases}\frac{\pi}{3}, & \text { if }n = 0 \\ -\frac{(-1)^n}{n^2}, & \text{ if } n > 1 \end{cases} 
    $$

    and the period $T = 2$. The derivative of $h(x)$ is given by:
    $$
    h'(x) =\sum_{n=1}^{\infty} -\frac{\pi(-1)^{n+1}}{n}\sin(\pi n x),
    $$

    therefore the Fourier coefficients of the derivative of $h(x)$ are given by 
    $$
    a_n = 0 \text{ for } n \geq b, \quad b_n = -\frac{\pi(-1)^{n+1}}{n},\quad \text{for }n > 0
    $$
    \end{itemize}
\end{solution}




\begin{exercise}
    Let \( f(x) \) be a periodic function with period \( T \), represented by its Fourier series:
    \begin{align*}
        f(x) = \frac{a_0}{2} + \sum_{n=1}^\infty \left( a_n \cos\left(\frac{2\pi n x}{T}\right) + b_n \sin\left(\frac{2\pi n x}{T}\right) \right).
    \end{align*}
    As explained in the lecture, 
    \begin{align*}
        \int_{x_0}^{x}
        f(x) 
        dx
        = 
        \frac{a_0}{2} ( x - x_0 )
        + 
        \sum_{n=1}^{\infty} 
        a_n 
        \int_{x_0}^{x}
        \cos\left(\frac{2\pi n x}{T}\right)
        dx 
        +
        b_n
        \int_{x_0}^{x}
        \sin\left(\frac{2\pi n x}{T}\right)
        dx
        .
    \end{align*}
    Complete this discussion and find the Fourier series of a function $g(x)$ such that 
    \begin{align}
        \int_{x_0}^{x} f(x) = c x + g(x)
    \end{align}
    for some $c \in \mathbb R$. 
\end{exercise}
\begin{solution}
    We first compute 
    \begin{gather*}
        \int_{x_0}^x \frac{a_0}{2} \, dt = \frac{a_0}{2} (x - x_0).
    \end{gather*}
    We integrate the sign and cosine terms, using the fundamental theorem of calculus:
    \begin{gather*}
        \int_{x_0}^x \cos\left(\frac{2\pi n t}{T}\right) \, dt 
        = 
        \frac{T}{2\pi n} \left( \sin\left(\frac{2\pi n x}{T}\right) - \sin\left(\frac{2\pi n x_0}{T}\right) \right).
    \end{gather*}
    \begin{gather*}
        \int_{x_0}^x \sin\left(\frac{2\pi n t}{T}\right) \, dt 
        = 
        -\frac{T}{2\pi n} \left( \cos\left(\frac{2\pi n x}{T}\right) - \cos\left(\frac{2\pi n x_0}{T}\right) \right).
    \end{gather*}
    Combining all terms, the integral \( F(x) \) becomes:
    \begin{gather*}
        F(x) 
        = 
        \frac{a_0}{2} (x - x_0) 
        + 
        \sum_{n=1}^\infty 
        a_n \frac{T}{2\pi n} \left( \sin\left(\frac{2\pi n x}{T}\right) - \sin\left(\frac{2\pi n x_0}{T}\right) \right) 
        - 
        b_n \frac{T}{2\pi n} \left( \cos\left(\frac{2\pi n x}{T}\right) - \cos\left(\frac{2\pi n x_0}{T}\right) \right).
    \end{gather*}
    We rewrite this once again and obtain
    \begin{align*}
        F(x) 
        &
        = 
        \frac{a_0}{2} x 
        - 
        \frac{a_0 x_0 }{2}
        + 
        \sum_{n=1}^\infty 
        \left( 
        (-a_n) \frac{T}{2\pi n} \sin\left(\frac{2\pi n x_0}{T}\right) 
        - 
        b_n \frac{T}{2\pi n} \cos\left(\frac{2\pi n x_0}{T}\right) 
        \right)
        \\&\qquad\qquad\qquad\qquad 
        + 
        \sum_{n=1}^\infty 
        (-b_n) \frac{T}{2\pi n} \cos\left(\frac{2\pi n x}{T}\right)
        + 
        a_n \frac{T}{2\pi n} \sin\left(\frac{2\pi n x}{T}\right) 
        .
    \end{align*}
    We thus obtain $c = \frac{a_0}{2}$ and the Fourier coefficients of the function $g$:
    \begin{align*}
        \frac{A_0}{2} &:= \frac{a_0 x_0 }{2}
        + 
        \sum_{n=1}^\infty 
        \left( 
        (-a_n) \frac{T}{2\pi n} \sin\left(\frac{2\pi n x_0}{T}\right) 
        - 
        b_n \frac{T}{2\pi n} \cos\left(\frac{2\pi n x_0}{T}\right) 
        \right)
        \\
        A_n := (-b_n) \frac{T}{2\pi n} 
        \\
        B_n := a_n \frac{T}{2\pi n} 
    \end{align*}
    This completes the discussion. 
\end{solution}


\end{document}
