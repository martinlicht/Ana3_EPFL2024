\documentclass[11pt]{article}
\def\hidesolutions{}
%%%%%%%%%%% SET MARGINS
\setlength{\textheight}{20cm}
\setlength{\topmargin}{-0.5cm}
\setlength{\oddsidemargin}{+0cm}
\setlength{\textwidth}{16.3cm}
%\setlength{\parskip}{6pt}
\setlength{\parindent}{0pt}

%%%%%%%%%%% PACKAGES
\usepackage{amsmath}
\usepackage{amssymb}
\usepackage{amsfonts}
%\usepackage{a4wide}
\usepackage{graphicx}
\usepackage{color}
\usepackage[normalem]{ulem}
\usepackage{enumitem}
\usepackage{capt-of}
\usepackage{float}
\usepackage{amsmath}
\usepackage{listings}
\definecolor{mygreen}{RGB}{28,172,0} % color values Red, Green, Blue
\definecolor{mylilas}{RGB}{170,55,241}
\usepackage{empheq}
\usepackage[ruled]{algorithm2e}
\usepackage{mathrsfs}
\usepackage{datetime}
\usepackage{subcaption}

% TODO: combine the two package lists and reduce redundancies 
\usepackage{mathtools}
\usepackage{nicefrac}
\usepackage{hyperref}
\usepackage{url}
\usepackage{amsmath,amssymb,amsfonts}
\usepackage{a4wide}
\usepackage{graphicx}
\usepackage{color}
\usepackage[normalem]{ulem}
\usepackage{capt-of}
\usepackage{float}
\usepackage[ruled]{algorithm2e}
\usepackage{amsmath,amssymb,amsfonts}
\usepackage{a4wide}
\usepackage{graphicx}
\usepackage{color}
\usepackage[normalem]{ulem}
\usepackage{capt-of}
\usepackage{float}
\usepackage[ruled]{algorithm2e}
\usepackage{mathrsfs}







\newcommand{\Lc}[2]{{\color{blue} \sout{#1} } \textcolor{red}{#2}}
\newcommand{\La}[1]{\textcolor{red}{#1}}
\newcommand{\lh}{\mathscr{L}_h}
\newcommand{\cl}{\mathscr{L}}
\newcommand{\cf}{\mathscr{F}}
\newcommand{\dx}{dx}
\newcommand{\ltn}{\mathscr{l}^2}
\newcommand{\bbR}{\mathbb{R}}
\newcommand{\Rset}{\mathbb{R}}
\newcommand{\Nset}{\mathbb{N}}
\newcommand{\scL}{\mathcal{L}}
\newcommand{\xx}{\mathbf{x}}
\newcommand{\norm}[1]{\|{#1}\|}
\newcommand{\yy}{\mathbf{y}}
\newcommand{\at}[1]{\big|_{#1}}
\renewcommand{\div}{\mathrm{div}}
\newcommand{\divergence}{\mathrm{div}}
\newcommand{\cp}[1]{\textcolor{blue}{#1}}

\newcommand{\FF}{\texttt{FreeFem++ }}
\newcommand{\FFns}{\texttt{FreeFem++}}
\newcommand{\FFfull}{\texttt{FreeFem++-x11}}
\newcommand{\cmd}[1]{ \medskip \noindent \texttt{#1} \medskip}
\newcommand{\incmd}[1]{\texttt{#1}}
\newcommand{\shrinkitems}{\addtolength{\itemsep}{-0.5\baselineskip}}
\newcommand{\mtt}[1]{\mathtt{#1}}
\newcommand{\ML}{\texttt{Matlab }}

\newcommand{\bb}{\mathbf{b}}
\newcommand{\nn}{\mathbf{n}}
\newcommand{\vecA}{\vec{A}}
\newcommand{\vecB}{\vec{B}}


\newcommand{\mesh}{\mathcal{T}_h}
\newcommand{\refel}{\widehat{K}}
\newcommand{\ver}{\mathbf{a}}
\newcommand{\refver}{\widehat{\mathbf{a}}}
\newcommand{\grad}{\nabla}
\newcommand{\refgrad}{\widehat{\nabla}}
\newcommand{\refu}{\widehat{u}}
\newcommand{\refbasis}{\widehat{\varphi}}
\newcommand{\refxx}{\widehat{\xx}}
\newcommand{\refx}{\widehat{x}}
\newcommand{\refy}{\widehat{y}}
\newcommand{\refrho}{\widehat{\rho}}
\newcommand{\refh}{\widehat{h}}






% For typesetting Python code
\newcommand{\matlab}{{\sc Matlab}\xspace}
\usepackage{listings}
\lstloadlanguages{Python}
\lstloadlanguages{csh}%
\definecolor{MyDarkGreen}{rgb}{0.0,0.4,0.0}
\definecolor{purple}{rgb}{0.58,0,0.82}
\lstset{language=Python,                    % Use Python
	%frame=single,                          % Single frame around code
	basicstyle=\ttfamily\footnotesize\color{black},
	keywordstyle=[1]\color{blue}\bf,        % Python functions bold and blue
	keywordstyle=[2]\color{purple},         % Python function arguments purple
	keywordstyle=[3]\color{red}\underbar,   % User functions underlined and blue
	commentstyle=\usefont{T1}{pcr}{m}{sl}\color{MyDarkGreen}\small,
	stringstyle=\color{purple},
	showstringspaces=false,                 % Don't put marks in string spaces
	tabsize=3,                              % 5 spaces per tab
	morekeywords={xlim,ylim,var,alpha,factorial,poissrnd,normpdf,normcdf},
	morecomment=[l][\color{blue}]{...},
	breaklines=true,
	breakatwhitespace=true,
	emptylines=1,
	mathescape=true,
	xleftmargin=0ex,
	emphstyle=\bfseries\color{red}
}





%%%%%%%%%%% MACROS NAMES
\newcommand{\lecturername}{Martin Licht}
% \newcommand{\assistantnamea}{Jochen Hinz}
% \newcommand{\assistantnameb}{Ivan Bioli}
\newcommand{\semestername}{Winter Semester 2023}
\newcommand{\lecturename}{Analysis III - 202(c)}
\DeclarePairedDelimiter\floor{\lfloor}{\rfloor}

%%%%%%%%%%% HEADER
\newdateformat{yeardate}{\THEYEAR}
\newcommand{\exsheet}[3] % input is the number of the session and the day TODO What's that
{\clearpage

	\begin{center}
		{\Large \textbf{\lecturename}}\\[2ex]
		\semestername
	\end{center}

	% \vspace{2ex}
	% \lecturername

	\vspace{2ex}
	{\Large Session #1: #3\,#2, \yeardate\today}
	%\hfill
	%{\Large EPF Lausanne}

	\hrulefill
}





\usepackage{comment}

\newtheorem{exercise}{Exercise}
\newtheorem{solutionenv}{Solution}

\newboolean{hide_solution}
\ifx\hidesolutions\undefined
\newenvironment{solution}{\begin{solutionenv}}{\end{solutionenv}}
\setboolean{hide_solution}{false}
\else
\excludecomment{solution}
\setboolean{hide_solution}{true}
\fi

\newcommand{\ifnotsolution}[1]{\ifthenelse{\boolean{hide_solution}}{#1}{}}
\newcommand{\ifsolution}[1]{\ifthenelse{\boolean{hide_solution}}{}{#1}}








\allowdisplaybreaks

\begin{document}
\exsheet{12}{5}{December} % parameters are the number of the session and the day




\begin{exercise}
    Compute the Fourier transform of the function
    \begin{gather*}
        f(x) = \left\{\begin{array}{ll}
            x   & \text{ if $0 \leq x < 1$ }
            \\
            0   & \text{ otherwise }
          \end{array}\right.
    \end{gather*}
    You can either directly use the complex exponential, or you can express it in terms of the sine and cosine function. 

    (Interpretation: the function $f(x)$ describes a localized signal: it is zero at $x=0$, then it rises linearly up to $1$, and then it jumps back to zero and remains zero from there on. The signal is not periodic.)
\end{exercise}
\begin{solution}     
    We write down the solution in two different ways, either using the complex exponential directly, or writing it as a sum of sine and cosine.
    \begin{align*}
        \mathfrak{F}(f)(\alpha)&=\frac{1}{\sqrt{2 \pi}} \int_{-\infty}^{\infty} f(x) e^{-i \alpha x} d x\\
        &=\frac{1}{\sqrt{2 \pi}} \int_{0}^{1} x e^{-i \alpha x} d x\\
        &=\frac{1}{\sqrt{2 \pi}} \left[ \frac{x}{-i\alpha} e^{-i \alpha x} \right]_0^1 + \frac{1}{\sqrt{2 \pi}} \int_{0}^{1} \frac{1}{i\alpha} e^{-i \alpha x} d x\\
        &=\frac{1}{\sqrt{2 \pi}} \left[ \frac{x}{-i\alpha} e^{-i \alpha x} \right]_0^1 + \frac{1}{\sqrt{2 \pi}} \left[ \frac{1}{\alpha^2} e^{-
        i \alpha x} \right]_{0}^1\\
        &=\frac{1}{\sqrt{2 \pi}} \frac{1}{-i\alpha} e^{-i \alpha } + \frac{1}{\sqrt{2 \pi}} \left( \frac{1}{\alpha^2} e^{-i \alpha} -\frac{1}{\alpha^2}\right)\\
        &=\frac{1}{\sqrt{2 \pi}}\left( \frac{i}{\alpha} e^{-i \alpha } + \frac{1}{\alpha^2} e^{-i \alpha} -\frac{1}{\alpha^2}\right)\\ 
        &=\frac{1}{\sqrt{2 \pi}}\left( \frac{i}{\alpha} \cos\alpha  + \frac{1}{\alpha} \sin\alpha + \frac{1}{\alpha^2} \cos\alpha - \frac{i}{\alpha^2} \sin\alpha -\frac{1}{\alpha^2}\right)\\ 
    \end{align*}
    Next we do it in terms of sine and cosine functions.
    $$
    \begin{aligned}
    \mathfrak{F}(f)(\alpha)&=\frac{1}{\sqrt{2 \pi}} \int_{-\infty}^{\infty} f(x) e^{-i \alpha x} d x\\
    &=\frac{1}{\sqrt{2 \pi}} \int_{-\infty}^{\infty} f(x)(\cos \alpha x - i\sin\alpha x) d x\\
    & = \frac{1}{\sqrt{2 \pi}} \int_{-\infty}^{\infty} f(x) \cos\alpha x d x - i\frac{1}{\sqrt{2 \pi}} \int_{-\infty}^{\infty} f(x) \sin\alpha x d x
    \end{aligned}
    $$
    We evaluate each integral seperately.
    $$\begin{aligned} \int_0^1 x \cos (\alpha x) d x & =\left[\frac{x}{\alpha} \sin \alpha x\right]_0^1-\int_0^1 \frac{1}{\alpha} \sin \alpha x d x \\ & =\left[\frac{x}{\alpha} \sin \alpha x\right]_0^1+\left[\frac{1}{\alpha^2} \cos \alpha x\right]_0^1 \\ & =\frac{1}{\alpha} \sin \alpha+\frac{1}{\alpha^2} \cos \alpha-\frac{1}{\alpha^2}\end{aligned}$$
    $$\begin{aligned} \int_0^1 x \sin (\alpha x) d x & =\left[-\frac{x}{\alpha} \cos (\alpha x)\right]_0^1+\int_0^1 \frac{1}{\alpha} \cos (\alpha(x) d x \\ & =\left[-\frac{x}{\alpha} \cos (\alpha x)\right]_0^1+\left[\frac{1}{\alpha^2} \sin \alpha x\right]_0^1 \\ & =-\frac{1}{\alpha} \cos \alpha+\frac{1}{\alpha^2} \sin \alpha\end{aligned}$$
    All together we have:
    $$
        \mathfrak{F}(f)(\alpha)=\frac{1}{\sqrt{2 \pi}} \int_{-\infty}^{\infty} f(x) e^{-i \alpha x} d x = \frac{1}{\sqrt{2\pi}}\left(\frac{1}{\alpha} \sin \alpha+\frac{1}{\alpha^2} \cos \alpha-\frac{1}{\alpha^2} +i\frac{1}{\alpha} \cos \alpha-i\frac{1}{\alpha^2} \sin \alpha\right)
    $$
\end{solution}






\begin{exercise}
    The Fourier transform of 
    \begin{align*}
        f(x) = e^{-5x^2}
    \end{align*}
    is the function 
    \begin{align*}
        \hat f(x) = \frac{1}{\sqrt{10}} e^{-\frac{\alpha^2}{20}}.
    \end{align*}
    Find the Fourier transforms of 
    \begin{align*}
     f', \quad f'', \quad f''', \quad f'''', \quad g(x) = f(2x), \quad h(x) = f(x-3).
    \end{align*}
\end{exercise}
\begin{solution}  
    To evaluate the Fourier transform of the derivative of a function we will use the following identity from the lecture:
    \[
        \mathfrak{F}\left(f^{(n)}\right)(\alpha)=(i \alpha)^n \mathfrak{F}(f)(\alpha)
    \]
    We then find 
    \begin{itemize}
    \item $\mathfrak{F}\left(f' \right)(\alpha) = i \alpha \mathfrak{F}(f)(\alpha) = \frac{i \alpha}{\sqrt{10}} e^{-\frac{\alpha^2}{20}}$
    \item $\mathfrak{F}\left(f'' \right)(\alpha) = -\alpha^2 \mathfrak{F}(f)(\alpha) = -\frac{ \alpha^2}{\sqrt{10}} e^{-\frac{\alpha^2}{20}}$
    \item $\mathfrak{F}\left(f' \right)(\alpha) = -i \alpha^3 \mathfrak{F}(f)(\alpha) = \frac{-i \alpha^3}{\sqrt{10}} e^{-\frac{\alpha^2}{20}}$
    \item $\mathfrak{F}\left(f' \right)(\alpha) = \alpha^4 \mathfrak{F}(f)(\alpha) = \frac{ \alpha^4}{\sqrt{10}} e^{-\frac{\alpha^2}{20}}$
    \end{itemize}
    For the Fourier transform of $g(x)$ we use the following identity from the lecture: 
    $g(x)=e^{-i b x} f(a x)$ has the Fourier transform
    $\hat{g}(\alpha)=\frac{1}{|a|} \hat{f}\left(\frac{\alpha+b}{a}\right)$,
    consequently:
    \begin{align*}
        \mathfrak{F}(g(x))(\alpha) & =\mathfrak{F}(f(2 x))(\alpha) \\ & =\frac{1}{2} \mathfrak{F}(f(x))\left(\frac{\alpha}{2}\right) \\ & =\frac{1}{2 \sqrt{10}} e^{-\frac{\left(\frac{\alpha}{2}\right)^2}{20}} \\ & =\frac{1}{2 \sqrt{10}} e^{-\frac{\alpha^2}{80}}
    \end{align*}
    For the Fourier transform of $h(x)$ we use a change of variables:
    \begin{align*}
        \mathfrak{F}(h(x))(\alpha)
        &= 
        \frac{1}{\sqrt{2\pi}}
        \int_{-\infty}^{+\infty} e^{-5(x-3)^2} e^{ - \imath x \alpha } \;dx
        \\&= 
        \frac{1}{\sqrt{2\pi}}
        \int_{-\infty}^{+\infty} e^{-5x^2} e^{ - \imath (x + 3) \alpha } \;dx
        \\&= 
        \frac{1}{\sqrt{2\pi}}
        \int_{-\infty}^{+\infty} e^{-5x^2} e^{ - \imath x \alpha } e^{ - \imath 3 \alpha }\;dx
        \\&= 
        \frac{ e^{ - \imath 3 \alpha }}{\sqrt{2\pi}}
        \int_{-\infty}^{+\infty} e^{-5x^2} e^{ - \imath x \alpha }\;dx
        \\&= 
        e^{ - \imath 3 \alpha } 
        \mathfrak{F}(f(x))(\alpha)
        \\&= 
        \frac{e^{ - \imath 3 \alpha }}{\sqrt{10}} e^{-\frac{\alpha^2}{20}}.
        \\&= 
        \frac{1}{\sqrt{10}} e^{-\frac{\alpha^2}{20} - \imath 3 \alpha }
        .
    \end{align*}
\end{solution}







\begin{exercise}
    Find the Fourier transform of 
    \begin{align*}
        f(x) = \left\{\begin{array}{ll}
                \sin(x) & \text{ if } 0 \leq x \leq 2 \pi 
                \\
                0 & \text{ otherwise }
               \end{array}
               \right.
    \end{align*}
\end{exercise}
\begin{solution} 
$$\begin{aligned} \mathcal{F}(f(x))(\alpha) & =\frac{1}{\sqrt{2 \pi}} \int_0^{2 \pi} \sin x e^{-i \alpha x} d x \\ & =\frac{1}{\sqrt{2 \pi}}\left[\sin x \frac{e^{-i \alpha x}}{-i \alpha}\right]_0^{2 \pi}+\frac{1}{\sqrt{2 \pi}} \int_0^{2 \pi} \frac{\cos x}{i \alpha} e^{-\alpha i x} d x \\ & =\frac{1}{\sqrt{2 \pi}}\left[\frac{\cos x}{a^2} e^{-i \alpha x}\right]_0^{2 \pi}+\frac{1}{\sqrt{2 \pi}} \int_0^{2 \pi} \frac{\sin x}{\alpha^2} e^{-i \alpha x} d x \\ & =\frac{1}{\sqrt{2 \pi} \alpha^2}\left(e^{-2 \pi i \alpha}-1\right)+\frac{1}{\alpha^2} \frac{1}{\sqrt{2 \pi}} \int_0^{2 \pi} \sin x e^{-i \alpha x} d x\end{aligned}$$
$$
\begin{aligned} \Rightarrow \frac{1}{\sqrt{2 \pi}} \int_0^{2 \pi} \sin x e^{-i \alpha x} d x & =\left(\frac{1}{1-\frac{1}{\alpha^2}}\right) \frac{\left(e^{-2 \pi i \alpha}-1\right)}{\sqrt{2 \pi} \alpha^2} \\ & =\frac{\alpha^2}{\alpha^2-1} \frac{\left(e^{-2 \pi i \alpha}-1\right)}{\alpha^2 \sqrt{2 \pi}} \\ & =\frac{\left(e^{-2 \pi i \alpha}-1\right)}{\sqrt{2 \pi}\left(\alpha^2-1\right)}\end{aligned}
$$
\end{solution}


















\begin{exercise}
    Find the Laplace transform $F(z)$ of 
    \begin{align*}
        f : \bbR_{0}^{+} \to \bbR, \quad t \mapsto t^{2}.
    \end{align*}
\end{exercise}
\begin{solution}     
We compute 
\begin{align*}
    F(z)&=\int_0^{\infty} t^2 e^{-z t} d t=\left[-\frac{t^2}{z} e^{-z t}\right]_0^{\infty}+\int_0^{\infty} \frac{2 t}{z} e^{-z t} d t \\ &=\left[\frac{2 t}{-z^2} e^{-z t}\right]_0^{\infty}+\int_0^{\infty} \frac{2}{z^2} e^{-z t} d t \\ &=\left[-\frac{2}{z^3} e^{-z t}\right]_0^{\infty} \\ &=0+\frac{2}{z^3} \\ &=\frac{2}{z^3}
    .
\end{align*}
Hence 
\begin{align*}
    F(z)=\frac{2}{z^3} \quad \text { for } \mathrm{R}(z)>0.
\end{align*}
\end{solution}

\begin{exercise}[Extra]
    We have introduced the Fourier transform 
    \begin{align*}
        \frakF(f)(\alpha) := \frac{1}{\sqrt{2\pi}}\int_{-\infty}^{\infty} f(x) e^{-i \alpha x} dx
    \end{align*}
    Different authors define the Fourier transform alternatively by:
    \begin{align*}
        \frakF_2(f)(\xi) 
        := \int_{-\infty}^{\infty} f(x) e^{-i 2\pi \xi x} dx
        ,
        \quad 
        \frakF_3(f)(\omega) 
        := \int_{-\infty}^{\infty} f(x) e^{-i \omega x} dx
        .
    \end{align*}
    Express $\frakF_{2}(f)$ and $\frakF_{3}(f)$ in terms of $\frakF(f)$.
\end{exercise}
\begin{solution}    
    Obviously, 
    \begin{align*}
        \frakF_{3}(f)(\omega) = \sqrt{2\pi} \frakF(f)(\omega).
    \end{align*}
    For the other transformation:
    \begin{align*}
        \frakF_2(f)(\xi) 
        &
        = 
        \int_{-\infty}^{\infty} f(x) e^{-i 2\pi \xi x} dx
        = 
        \sqrt{2\pi} 
        \cdot 
        \frac{1}{\sqrt{2\pi}}
        \int_{-\infty}^{\infty} f(x) e^{-i 2\pi \xi x} dx
        = 
        \sqrt{2\pi} 
        \cdot 
        \frakF(f)( 2\pi \xi )
        .
    \end{align*}
\end{solution}
















\begin{exercise}
    Find $f : \mathbb R \to \mathbb R$ such that 
    \begin{align*}
        \hat f(\alpha) = \frac{3}{1+\alpha^{2}} + \frac{-1}{1+4\alpha^{2}} + \frac{ \sin( 4 \alpha + 3 ) }{ 4 \alpha + 3 }
    \end{align*}
\end{exercise}
\begin{solution}   Because of the linearity property we can treat the inverse of each term in seperately. Therefore 
 \begin{itemize}
    \item $\begin{aligned} \mathfrak{F}^{-1}\left(\frac{\sin (4 \alpha+3)}{4 a r+3}\right)(x) & =\frac{\frac{1}{4}}{\frac{1}{4}} \mathfrak{F}^{-1}\left(\frac{\sin \left(\frac{\alpha+3 / 4}{\frac{1}{4}}\right)}{\frac{\alpha+\frac{3}{4}}{\frac{1}{4}}}\right)(x) \\ & =\frac{1}{4} e^{-i \frac{3}{4} x} \sqrt{\frac{\pi}{2}} \mathfrak{F}^{-1}\left(\sqrt{\frac{2}{\pi}} \frac{\sin (\alpha)}{\alpha}\right)\left(\frac{x}{4}\right) \\ & = \begin{cases}\frac{1}{4} \sqrt{\frac{\pi}{2}} e^{-i \frac{3}{4} x} & \text { if }-4 \leq x \leq 4 \\ 0 & \text { else }\end{cases} \end{aligned}$
    \item $\begin{aligned} \mathfrak{F}^{-1}\left(\frac{3}{1+\alpha^2}\right)(x) & =3 \sqrt{\frac{\pi}{2}} \mathfrak{F}^{-1}\left(\sqrt{\frac{2}{\pi}} \frac{1}{1+a^2}\right)(x) \\ & =3 \sqrt{\frac{\pi}{2}} e^{-|x|}\end{aligned}$
    \item $\begin{aligned} \mathfrak{F}^{-1}\left(\frac{-1}{1+4 \alpha^2}\right)(x) & =-\sqrt{\frac{\pi}{2}} \mathfrak{F}^{-1}\left(\sqrt{\frac{2}{\pi}} \frac{1}{1+(2 \alpha)^2}\right)(x) \\ & =-\sqrt{\frac{\pi}{2}} \frac{\frac{1}{2}}{\frac{1}{2}} \mathfrak{F}^{-1}\left(\sqrt{\frac{2}{\pi}} \frac{1}{1+\left(\frac{\alpha}{\frac{1}{2}}\right)^2}\right)(x \mid \\ & =-\sqrt{\frac{\pi}{2}} \frac{1}{2} \mathfrak{F}^{-1}\left(\sqrt{\frac{2}{\pi}} \frac{1}{1+\alpha^2}\right)\left(\frac{x}{2}\right) \\ & =-\frac{1}{2} \sqrt{\frac{\pi}{2}} e^{-\left|\frac{x}{2}\right|}\end{aligned}$
 \end{itemize}
 All together this gives
    $$
    f(x) = \begin{cases}-\frac{1}{2} \sqrt{\frac{\pi}{2}} e^{-\left|\frac{x}{2}\right|}+3 \sqrt{\frac{\pi}{2}} e^{-|x|}+\frac{1}{4} \sqrt{\frac{\pi}{2}} e^{-i \frac{3}{4} x} & -4 \leq x \leq 4 \\ -\frac{1}{2} \sqrt{\frac{\pi}{2}} e^{-\left|\frac{x}{2}\right|}+3 \sqrt{\frac{\pi}{2}} e^{-|x|} & \text { else }\end{cases}
    $$
\end{solution}


















\begin{exercise}
    Draw the complex exponentials $e^{zt}$ in the complex plane for $t=0,1,2,3,4$, 
    where $z$ is one of the following complex numbers:
    \begin{gather*}
        z_1 = \frac 1 2,
        \quad 
        z_2 = - \frac 1 2,
        \quad 
        z_3 = 0.2i,
        \quad 
        z_4 = -0.2i,
        \quad 
        z_3 = - \frac 1 2 + 0.2i
        .
    \end{gather*}
    \textit{You may use a calculater.}
\end{exercise}

\begin{solution}     
\end{solution}

\begin{exercise}
    We consider the Poisson problem with Dirichlet boundary conditions over the interval $[0,L]$:
    \begin{gather*}
        - \Delta u(x) = x^2, \quad 0 < x < L,
        \\
        u(0) = 1, \quad u(L) = 2
    \end{gather*}
    \begin{itemize}
        \item Solve this problem directly. The solution is a polynomial of order $4$.
        \item Extend the right-hand side $f(x) = x^2$ to an odd function with period $2L$ and compute its Fourier coefficients.
        \item How do you use the superposition principle to split the problem? Using these coefficients, find the Fourier series of the solution $u^f$.
    \end{itemize}
\end{exercise}
\begin{solution}     
\end{solution}

\begin{exercise}[Extra]
    Directly compute the solution of the problem
    \begin{gather*}
        - \Delta u(x) = x^2, \quad 0 < x < L,
        \\
        u(0) = 0, \quad u(L) = 0
    \end{gather*}
    and find its Fourier coefficients. Compare this with the function $u^f$ from the previous exercise.
\end{exercise}
\begin{solution}     
\end{solution}

\begin{exercise}
    We solve the Poisson problem with homogeneous Dirichlet boundary conditions over $[0,1]$:
    \begin{gather*}
        - \Delta u(x) = \left\{\begin{array}{ll} x & \text{ if } 0 < x \leq 0.5 \\ 0 & \text{ if } 0.5 < x \leq 1 \end{array}\right., \qquad 0 < x < 1,
        \\
        u(0) = 0, \quad u(1) = 0
    \end{gather*}
    \begin{itemize}
        \item Extend the right-hand side $f(x)$ to an odd function with period $2L$ and compute its Fourier coefficients.
        \item Using these coefficients, find the Fourier series of the solution $u$. Verify that the boundary condition $u(0) = u(L) = 0$ is satisfied.
    \end{itemize}
\end{exercise}
\begin{solution}     
\end{solution}

\begin{exercise}
    We want to find a solution to the boundary value problem 
    \begin{gather*}
        - \Delta u(x) + u(x) = x, \quad 0 < x < L,
        \\
        u(0) = 0, \quad u(L) = 0.
    \end{gather*}
    \begin{itemize}
        \item Extend the right-hand side $f(x) = x$ to an odd function with period $2L$ and compute its Fourier coefficients.
        \item Using these coefficients, find the Fourier series of the solution $u$. Verify that the boundary condition $u(0) = u(L) = 0$ is satisfied.
    \end{itemize}
\end{exercise}
\begin{solution}     
\end{solution}



\end{document}
