\documentclass[11pt]{article}
%\def\hidesolutions{}
%%%%%%%%%%% SET MARGINS
\setlength{\textheight}{20cm}
\setlength{\topmargin}{-0.5cm}
\setlength{\oddsidemargin}{+0cm}
\setlength{\textwidth}{16.3cm}
%\setlength{\parskip}{6pt}
\setlength{\parindent}{0pt}

%%%%%%%%%%% PACKAGES
\usepackage{amsmath}
\usepackage{amssymb}
\usepackage{amsfonts}
%\usepackage{a4wide}
\usepackage{graphicx}
\usepackage{color}
\usepackage[normalem]{ulem}
\usepackage{enumitem}
\usepackage{capt-of}
\usepackage{float}
\usepackage{amsmath}
\usepackage{listings}
\definecolor{mygreen}{RGB}{28,172,0} % color values Red, Green, Blue
\definecolor{mylilas}{RGB}{170,55,241}
\usepackage{empheq}
\usepackage[ruled]{algorithm2e}
\usepackage{mathrsfs}
\usepackage{datetime}
\usepackage{subcaption}

% TODO: combine the two package lists and reduce redundancies 
\usepackage{mathtools}
\usepackage{nicefrac}
\usepackage{hyperref}
\usepackage{url}
\usepackage{amsmath,amssymb,amsfonts}
\usepackage{a4wide}
\usepackage{graphicx}
\usepackage{color}
\usepackage[normalem]{ulem}
\usepackage{capt-of}
\usepackage{float}
\usepackage[ruled]{algorithm2e}
\usepackage{amsmath,amssymb,amsfonts}
\usepackage{a4wide}
\usepackage{graphicx}
\usepackage{color}
\usepackage[normalem]{ulem}
\usepackage{capt-of}
\usepackage{float}
\usepackage[ruled]{algorithm2e}
\usepackage{mathrsfs}







\newcommand{\Lc}[2]{{\color{blue} \sout{#1} } \textcolor{red}{#2}}
\newcommand{\La}[1]{\textcolor{red}{#1}}
\newcommand{\lh}{\mathscr{L}_h}
\newcommand{\cl}{\mathscr{L}}
\newcommand{\cf}{\mathscr{F}}
\newcommand{\dx}{dx}
\newcommand{\ltn}{\mathscr{l}^2}
\newcommand{\bbR}{\mathbb{R}}
\newcommand{\Rset}{\mathbb{R}}
\newcommand{\Nset}{\mathbb{N}}
\newcommand{\scL}{\mathcal{L}}
\newcommand{\xx}{\mathbf{x}}
\newcommand{\norm}[1]{\|{#1}\|}
\newcommand{\yy}{\mathbf{y}}
\newcommand{\at}[1]{\big|_{#1}}
\renewcommand{\div}{\mathrm{div}}
\newcommand{\divergence}{\mathrm{div}}
\newcommand{\cp}[1]{\textcolor{blue}{#1}}

\newcommand{\FF}{\texttt{FreeFem++ }}
\newcommand{\FFns}{\texttt{FreeFem++}}
\newcommand{\FFfull}{\texttt{FreeFem++-x11}}
\newcommand{\cmd}[1]{ \medskip \noindent \texttt{#1} \medskip}
\newcommand{\incmd}[1]{\texttt{#1}}
\newcommand{\shrinkitems}{\addtolength{\itemsep}{-0.5\baselineskip}}
\newcommand{\mtt}[1]{\mathtt{#1}}
\newcommand{\ML}{\texttt{Matlab }}

\newcommand{\bb}{\mathbf{b}}
\newcommand{\nn}{\mathbf{n}}
\newcommand{\vecA}{\vec{A}}
\newcommand{\vecB}{\vec{B}}


\newcommand{\mesh}{\mathcal{T}_h}
\newcommand{\refel}{\widehat{K}}
\newcommand{\ver}{\mathbf{a}}
\newcommand{\refver}{\widehat{\mathbf{a}}}
\newcommand{\grad}{\nabla}
\newcommand{\refgrad}{\widehat{\nabla}}
\newcommand{\refu}{\widehat{u}}
\newcommand{\refbasis}{\widehat{\varphi}}
\newcommand{\refxx}{\widehat{\xx}}
\newcommand{\refx}{\widehat{x}}
\newcommand{\refy}{\widehat{y}}
\newcommand{\refrho}{\widehat{\rho}}
\newcommand{\refh}{\widehat{h}}






% For typesetting Python code
\newcommand{\matlab}{{\sc Matlab}\xspace}
\usepackage{listings}
\lstloadlanguages{Python}
\lstloadlanguages{csh}%
\definecolor{MyDarkGreen}{rgb}{0.0,0.4,0.0}
\definecolor{purple}{rgb}{0.58,0,0.82}
\lstset{language=Python,                    % Use Python
	%frame=single,                          % Single frame around code
	basicstyle=\ttfamily\footnotesize\color{black},
	keywordstyle=[1]\color{blue}\bf,        % Python functions bold and blue
	keywordstyle=[2]\color{purple},         % Python function arguments purple
	keywordstyle=[3]\color{red}\underbar,   % User functions underlined and blue
	commentstyle=\usefont{T1}{pcr}{m}{sl}\color{MyDarkGreen}\small,
	stringstyle=\color{purple},
	showstringspaces=false,                 % Don't put marks in string spaces
	tabsize=3,                              % 5 spaces per tab
	morekeywords={xlim,ylim,var,alpha,factorial,poissrnd,normpdf,normcdf},
	morecomment=[l][\color{blue}]{...},
	breaklines=true,
	breakatwhitespace=true,
	emptylines=1,
	mathescape=true,
	xleftmargin=0ex,
	emphstyle=\bfseries\color{red}
}





%%%%%%%%%%% MACROS NAMES
\newcommand{\lecturername}{Martin Licht}
% \newcommand{\assistantnamea}{Jochen Hinz}
% \newcommand{\assistantnameb}{Ivan Bioli}
\newcommand{\semestername}{Winter Semester 2023}
\newcommand{\lecturename}{Analysis III - 202(c)}
\DeclarePairedDelimiter\floor{\lfloor}{\rfloor}

%%%%%%%%%%% HEADER
\newdateformat{yeardate}{\THEYEAR}
\newcommand{\exsheet}[3] % input is the number of the session and the day TODO What's that
{\clearpage

	\begin{center}
		{\Large \textbf{\lecturename}}\\[2ex]
		\semestername
	\end{center}

	% \vspace{2ex}
	% \lecturername

	\vspace{2ex}
	{\Large Session #1: #3\,#2, \yeardate\today}
	%\hfill
	%{\Large EPF Lausanne}

	\hrulefill
}





\usepackage{comment}

\newtheorem{exercise}{Exercise}
\newtheorem{solutionenv}{Solution}

\newboolean{hide_solution}
\ifx\hidesolutions\undefined
\newenvironment{solution}{\begin{solutionenv}}{\end{solutionenv}}
\setboolean{hide_solution}{false}
\else
\excludecomment{solution}
\setboolean{hide_solution}{true}
\fi

\newcommand{\ifnotsolution}[1]{\ifthenelse{\boolean{hide_solution}}{#1}{}}
\newcommand{\ifsolution}[1]{\ifthenelse{\boolean{hide_solution}}{}{#1}}








\allowdisplaybreaks

\begin{document}
\exsheet{4}{17}{October} % parameters are the number of the session and the day






\begin{exercise}
    We work over the quadratic domain 
    \[
        \Omega := \left\{ (x_1,x_2) \in \bbR^{2} \suchthat* -1 < x_1 < 1, \; -1 < x_2 < 1 \right\}.
    \]
    Compute the integral $\iint_\Omega \divergence \vec F \;dx_1dx_2$, where 
    \[
        \vec F(x_1,x_2) = \left( \sin(x_1) x_2, \left( x_1^2 + x_2 \right)^5 \right)
    \]
\end{exercise}
\begin{solution}
    % Express this volume integral as a curve integral.
 We make use of the Divergence theorem to express the volume integral as a curve integral.

\begin{align*}
\int \int_{\Omega} \nabla \cdot \vec{F} \;d x_1 \:d x_2 =&\int_{\partial \Omega}\vec{F}\cdot\vec{n} \;d\ell,\\
=&\int_{-1}^1 \vec{F}(x_1,-1)\cdot\begin{pmatrix}0\\-1\end{pmatrix} \;dx_1 + \int_{-1}^1 \vec{F}(x_1,1)\cdot\begin{pmatrix}0\\1\end{pmatrix} \;dx_1,\\
& + \int_{-1}^1 \vec{F}(-1,x_2)\cdot\begin{pmatrix}-1\\0\end{pmatrix} \;dx_2 + \int_{-1}^1 \vec{F}(1,x_2)\cdot\begin{pmatrix}1\\0\end{pmatrix} \;dx_2,\\
=&\int_{-1}^1 (1-x_1^ 2)^ 5 \;dx_1 + \int_{-1}^1 (x_1^2 + 1)^ 5 \;dx_1 + \int_{-1}^1 \sin(1)x_2\;dx_2 +  \int_{-1}^1 -\sin(-1)x_2\;dx_2,\\
=&  \left[\frac{1}{6}(1-x_1^ 2)^ 6 \frac{1}{-2x_1} \right]_{x_1 = -1}^{x_1 = 1} + \left[\frac{1}{6}(x_1^ 2 + 1)^ 6 \frac{1}{2x_1} \right]_{x_1 = -1}^{x_1 = 1},\\
=& \frac{1}{6} 2 \frac{1}{2} - \frac{1}{6} 2 \frac{-1}{2} = \frac{1}{3},
\end{align*}

where we have used that $x_2$ and $x_2$ are odd functions in the fourth line. 
\end{solution}




\begin{exercise}
    We have the vector field 
    \[
        \vec F( x_1, x_2 ) = \left( x_1 x_2, x_2^{2} \right).
    \]
    and the domains 
    \begin{align*}
        \Omega_1 &:= \left\{ (x_1,x_2) \in \bbR^{2} \suchthat* x_1^{2} + x_2^{2} < 1 \right\},
        \\
        \Omega_2 &:= \left\{ (x_1,x_2) \in \bbR^{2} \suchthat* x_1^{2} + x_2^{2} < 1, \; x_1 > 0 \right\},
     \end{align*}
    Verify Green's theorem for the vector field $\vec F$ with the domains $\Omega_1$ and $\Omega_2$. 
    \textit{You need to find parameterizations of the boundary first.}
\end{exercise}
\begin{solution}   
   Green's theorem is given by $$\int_{\partial \Omega} \vec{F} \cdot \;d\vec{s} = \int \int_{\Omega} \frac{\partial F_2}{\partial x_1} - \frac{\partial F_1}{\partial x_2} \;d \Omega$$
We consider domain $\Omega_1$. A parameterisation for the boundary is given by $\gamma: [0,2\pi] \mapsto \mathbb{R}^2, \quad t\mapsto  (\cos t, \sin t).$

\begin{align*}
\int_{\partial \Omega} \vec{F} \cdot \;d\vec{s} =& \int_0^ {2\pi} \begin{pmatrix}\sin t \cos t\\ -\sin^2 t \cos t \end{pmatrix} \cdot \begin{pmatrix}-\sin t \\ \cos t \end{pmatrix} \; d t \\
=& \int_0^{2\pi} -\sin^2 t\cos t + \sin^ 2 t \cos t \;dt = 0
\end{align*}

\begin{align*}
\int \int_{\Omega} \frac{\partial F_2}{\partial x_1} - \frac{\partial F_1}{\partial x_2} \;d \Omega =& \int \int_{\Omega} -x_1 \; d\Omega\\
=& \int_0^ {2\pi} \int_0^1 -r\cos \theta r \;d\theta\;dr\\
=& -\int_0^ {2\pi} \cos \theta \;d \theta \int_0^1 r^ 2 \;d r = 0,
\end{align*}

where we have used that integrating the cosine function over a period is $0$. Boths integrals give the same value thereby verifying Green's theorem.\\

We consider domain $\Omega_2$. A parameterisation for the boundary is given by two seperate curves. For the circular arc we have $\gamma: [-\pi/2,2\pi] \mapsto \mathbb{R}^2, \quad t\mapsto  (\cos t, \sin t)$ and for the vertical line we have  $\gamma: [1,-1] \mapsto \mathbb{R}^2, \quad t\mapsto  (0, t)$. 

\begin{align*}
\int_{\partial \Omega} \vec{F} \cdot \;d\vec{s} =& \int_{-\frac{\pi}{2}}^ {\frac{\pi}{2}} \begin{pmatrix}\sin t \cos t\\ -\sin^2 t \cos t \end{pmatrix} \cdot \begin{pmatrix}-\sin t \\ \cos t \end{pmatrix} \; d t + \int_1^ {-1} \begin{pmatrix}0\\t^ 2\end{pmatrix}\cdot \begin{pmatrix}0\\1\end{pmatrix}\\
=& \int_0^{2\pi} -\sin^2 t\cos t + \sin^ 2 t \cos t \;dt + \int_{1}^ {-1} t^ 2 \;d t\\
=& \left[\frac{1}{3} t^ 3\right]_1^ {-1} = \frac{2}{3}
\end{align*}

\begin{align*}
\int \int_{\Omega} \frac{\partial F_2}{\partial x_1} - \frac{\partial F_1}{\partial x_2} \;d \Omega =& \int \int_{\Omega} -x_1 \; d\Omega\\
=& \int_{-\frac{\pi}{2}}^ {\frac{\pi}{2}} \int_0^1 -r\cos \theta r \;d\theta\;dr\\
=& -\int_{-\frac{\pi}{2}}^ {\frac{\pi}{2}} \cos \theta \;d \theta \int_0^1 r^ 2 \;d r\\
=& \left[\sin \theta\right]_{-\frac{\pi}{2}}^ {\frac{\pi}{2}} \left[-\frac{1}{3}r^ 3\right]_0^{1} = -\frac{2}{3}
\end{align*}

Boths integrals give the same value thereby verifying Green's theorem.\\

\end{solution}





\begin{exercise}
    Consider the triangle domain 
    \[
        T := \left\{ (x_1,x_2) \in \bbR^{2} \suchthat* x_1 > 0, \; x_2 > 0, \; x_1+x_2 < 1 \right\}.
    \]
    and the vector field 
    \[
        \vec F(x_1,x_2) = \left( x_1x_2 + \frac{x_1}{1+x_1^2+x_2^2}, x_1x_2 + \frac{x_2}{1+x_1^2+x_2^2} \right)
    \]
    Find the curve integral of $\vec F$ along the boundary of $T$ using Green's theorem.
\end{exercise}
\begin{solution}
\begin{align*}
    \int_{\partial \Omega} \vec{F} \cdot \; d \vec{s} &= \int\int_{\Omega} \frac{\partial F_2}{\partial x_1} - \frac{\partial F_1}{\partial x_2}\; d \Omega\\
&= \int\int_{\Omega} x_2 + \frac{2x_1x_2}{(1+x_1^ 2 + x_2^ 2)^ {\frac{3}{2}}} - x_1 - \frac{2x_1x_2}{(1+x_1^ 2 + x_2^ 2)^ {\frac{3}{2}}} \; d \Omega\\
&= \int\int_{\Omega} x_2  - x_1\; d \Omega\\
&= \int_0^ 1\int_0^ {1-x_1} x_2 - x_1 \;dx_2\;dx_1\\
&= \int_0^ 1 \frac{1}{2}(1-x_1)^ 2 - x_1(1-x_1) \;d x\\
&= \left[ \frac{1}{6}(1-x_1)^ 3 - \frac{1}{2} x_1^2 + \frac{1}{3}x_1^ 3 \right]_0^1 = -\frac{1}{2} + \frac{1}{3} - \frac{1}{6} = -\frac{2}{6}
\end{align*}
\end{solution}
















\begin{exercise}
    Consider the parabolic arc 
    \[
        \Gamma := \left\{ (x_1,x_2) \in \bbR^{2} \suchthat* -1 < x_1 < 1, \; x_2 = 3 ( 1 - x_1^2 ) \right\}.
    \]
    Find the curve integral $\int_\Gamma \vec F \;dl$, where 
    \[
        \vec F(x_1,x_2) = \left( x_1 ( 2 - \cos(x_1x_2)^{2} ), x_2 ( 2 + \cos(x_1x_2)^{2} ) \right)
    \]
\end{exercise}
\begin{solution}
    % Draw a line between arc endpoints to enclose a domain.
    % The divergence of the vector field is constant.
    % Compute the integral along the bottom line and the integrate the divergence over the domain
    % We just need the area of the domain 

The Idea is to make use of Divergence theorem. We do this by closing the curve $\gamma$ by introducing the curve $\delta: [-1,1] \mapsto \mathbb{R}^2, \quad t\mapsto  (t,0)$. This allows us to reformulate the problem as follows:

$$
\int_{\Gamma} \vec{F} \cdot \; d\vec{s} = \int_{\Gamma\cup \Delta} \vec{F} \cdot 
\; d\vec{s}-\int_{\Delta} \vec{F}\cdot \;d \vec{s}
$$

For the first integral we use the Divergence theorem:

\begin{align*}
\int_{\Gamma\cup\Delta} \vec{F} \cdot 
\; d\vec{s} &= \int_{\Omega} \nabla\cdot\vec{F}\;d \Omega\\
&=\int_{\Omega} 4\;d \Omega\\
&=\int_{-1}^1\int_{0}^{3(1-x_1^2)} 4 \;d x_2 ;\d x_1\\
&=\int_{-1}^1\left[4x_2\right]_{0}^{3(1-x_1^2)}  \;d x_1\\
&=\int_{-1}^1 12( 1-x_1^2)\;d x_1\\
&=\left[12( 1-x_1^2)\right]_{-1}^1 = 23\frac{1}{3}\\
\end{align*}

and the value of the second integral is given by:

\begin{align*}
\int_{\Delta} \vec{F} \cdot 
\; d\vec{s} &= \int_{-1}^1 x_1 
\; dx_1 =0\\
\end{align*}

We conclude that:

$$
\int_{\Gamma} \vec{F} \cdot \; d\vec{s} = \int_{\Gamma\cup \Delta} \vec{F} \cdot 
\; d\vec{s} = 23\frac{1}{3}
$$

\end{solution}

\begin{exercise}
    Find the tangential vector $\dot\gamma(t)$, the unit tangential vector $\vec\tau$ and the unit normal $\vec n$ of the simple closed curve 
    \begin{align*}
        \gamma : [0,2\pi] \to \bbR^{2}, \quad t \mapsto ( \cos(t), \sin(t)( 1 + \sin(2t)^2 ) ).
    \end{align*}
    Find the values of $\gamma$ and $\vec\tau$ for a few values of $t \in [0,2\pi]$, such as $t = \frac{\pi}{4}, 2 \frac{\pi}{4}, 3 \frac{\pi}{4}, \dots, 7 \frac{\pi}{4}$
\end{exercise}
\begin{solution}
    Some standard calculations show that 
    \begin{align*}
        \dot\gamma(t) 
        &= 
        \left( \sin(t), \cos(t) + \cos(t) \sin(2t)^{2} + \sin(t) 2 \sin(2t) \cos(2t) 2 \right)
        \\&= 
        \left( \sin(t), \cos(t) + \cos(t) \sin(2t)^{2} + 4 \sin(t) \sin(2t) \cos(2t) \right)
    \end{align*}
    Hence,{\scriptsize
    \begin{align*}
        &
        |\dot\gamma(t)|
        \\&=
        \sqrt{
            \sin(t)^{2}
            +
            \cos(t)^{2} + \cos(t)^{2} \sin(2t)^{4} + 16 \sin(t)^{2} \sin(2t)^{2} \cos(2t)^{2}
            +
            2\cos(t) \sin(2t)^{2} + 2\cos(t) 4 \sin(t) \sin(2t) \cos(2t) + 2 \sin(2t)^{2} 4 \sin(t) \sin(2t) \cos(2t)
        }
        \\&
        =
        \sqrt{
            \sin(t)^{2}
            +
            \cos(t)^{2} + \cos(t)^{2} \sin(2t)^{4} + 16 \sin(t)^{2} \sin(2t)^{2} \cos(2t)^{2}
            +
            2\cos(t) \sin(2t)^{2} + 2\cos(t) 4 \sin(t) \sin(2t) \cos(2t) + 2 \sin(2t)^{2} 4 \sin(t) \sin(2t) \cos(2t)
        }
    \end{align*}
    }

    \begin{tikzpicture}
        \begin{axis}[
            domain=0:360,
            xmin=-2,
            xmax=2,
            ymin=-2,
            ymax=2,
            xlabel={x},
            ylabel={y}
        ]
        \addplot[blue, variable=\t, samples=100] ({cos(\t)}, {sin(\t)* ( 1 + 0.5 * sin(2*\t)*sin(2*\t) )});
        \end{axis}
    \end{tikzpicture}
\end{solution}


\end{document}
