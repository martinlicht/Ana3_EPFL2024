\documentclass[11pt]{article}
\def\hidesolutions{}
%%%%%%%%%%% SET MARGINS
\setlength{\textheight}{20cm}
\setlength{\topmargin}{-0.5cm}
\setlength{\oddsidemargin}{+0cm}
\setlength{\textwidth}{16.3cm}
%\setlength{\parskip}{6pt}
\setlength{\parindent}{0pt}

%%%%%%%%%%% PACKAGES
\usepackage{amsmath}
\usepackage{amssymb}
\usepackage{amsfonts}
%\usepackage{a4wide}
\usepackage{graphicx}
\usepackage{color}
\usepackage[normalem]{ulem}
\usepackage{enumitem}
\usepackage{capt-of}
\usepackage{float}
\usepackage{amsmath}
\usepackage{listings}
\definecolor{mygreen}{RGB}{28,172,0} % color values Red, Green, Blue
\definecolor{mylilas}{RGB}{170,55,241}
\usepackage{empheq}
\usepackage[ruled]{algorithm2e}
\usepackage{mathrsfs}
\usepackage{datetime}
\usepackage{subcaption}

% TODO: combine the two package lists and reduce redundancies 
\usepackage{mathtools}
\usepackage{nicefrac}
\usepackage{hyperref}
\usepackage{url}
\usepackage{amsmath,amssymb,amsfonts}
\usepackage{a4wide}
\usepackage{graphicx}
\usepackage{color}
\usepackage[normalem]{ulem}
\usepackage{capt-of}
\usepackage{float}
\usepackage[ruled]{algorithm2e}
\usepackage{amsmath,amssymb,amsfonts}
\usepackage{a4wide}
\usepackage{graphicx}
\usepackage{color}
\usepackage[normalem]{ulem}
\usepackage{capt-of}
\usepackage{float}
\usepackage[ruled]{algorithm2e}
\usepackage{mathrsfs}







\newcommand{\Lc}[2]{{\color{blue} \sout{#1} } \textcolor{red}{#2}}
\newcommand{\La}[1]{\textcolor{red}{#1}}
\newcommand{\lh}{\mathscr{L}_h}
\newcommand{\cl}{\mathscr{L}}
\newcommand{\cf}{\mathscr{F}}
\newcommand{\dx}{dx}
\newcommand{\ltn}{\mathscr{l}^2}
\newcommand{\bbR}{\mathbb{R}}
\newcommand{\Rset}{\mathbb{R}}
\newcommand{\Nset}{\mathbb{N}}
\newcommand{\scL}{\mathcal{L}}
\newcommand{\xx}{\mathbf{x}}
\newcommand{\norm}[1]{\|{#1}\|}
\newcommand{\yy}{\mathbf{y}}
\newcommand{\at}[1]{\big|_{#1}}
\renewcommand{\div}{\mathrm{div}}
\newcommand{\divergence}{\mathrm{div}}
\newcommand{\cp}[1]{\textcolor{blue}{#1}}

\newcommand{\FF}{\texttt{FreeFem++ }}
\newcommand{\FFns}{\texttt{FreeFem++}}
\newcommand{\FFfull}{\texttt{FreeFem++-x11}}
\newcommand{\cmd}[1]{ \medskip \noindent \texttt{#1} \medskip}
\newcommand{\incmd}[1]{\texttt{#1}}
\newcommand{\shrinkitems}{\addtolength{\itemsep}{-0.5\baselineskip}}
\newcommand{\mtt}[1]{\mathtt{#1}}
\newcommand{\ML}{\texttt{Matlab }}

\newcommand{\bb}{\mathbf{b}}
\newcommand{\nn}{\mathbf{n}}
\newcommand{\vecA}{\vec{A}}
\newcommand{\vecB}{\vec{B}}


\newcommand{\mesh}{\mathcal{T}_h}
\newcommand{\refel}{\widehat{K}}
\newcommand{\ver}{\mathbf{a}}
\newcommand{\refver}{\widehat{\mathbf{a}}}
\newcommand{\grad}{\nabla}
\newcommand{\refgrad}{\widehat{\nabla}}
\newcommand{\refu}{\widehat{u}}
\newcommand{\refbasis}{\widehat{\varphi}}
\newcommand{\refxx}{\widehat{\xx}}
\newcommand{\refx}{\widehat{x}}
\newcommand{\refy}{\widehat{y}}
\newcommand{\refrho}{\widehat{\rho}}
\newcommand{\refh}{\widehat{h}}






% For typesetting Python code
\newcommand{\matlab}{{\sc Matlab}\xspace}
\usepackage{listings}
\lstloadlanguages{Python}
\lstloadlanguages{csh}%
\definecolor{MyDarkGreen}{rgb}{0.0,0.4,0.0}
\definecolor{purple}{rgb}{0.58,0,0.82}
\lstset{language=Python,                    % Use Python
	%frame=single,                          % Single frame around code
	basicstyle=\ttfamily\footnotesize\color{black},
	keywordstyle=[1]\color{blue}\bf,        % Python functions bold and blue
	keywordstyle=[2]\color{purple},         % Python function arguments purple
	keywordstyle=[3]\color{red}\underbar,   % User functions underlined and blue
	commentstyle=\usefont{T1}{pcr}{m}{sl}\color{MyDarkGreen}\small,
	stringstyle=\color{purple},
	showstringspaces=false,                 % Don't put marks in string spaces
	tabsize=3,                              % 5 spaces per tab
	morekeywords={xlim,ylim,var,alpha,factorial,poissrnd,normpdf,normcdf},
	morecomment=[l][\color{blue}]{...},
	breaklines=true,
	breakatwhitespace=true,
	emptylines=1,
	mathescape=true,
	xleftmargin=0ex,
	emphstyle=\bfseries\color{red}
}





%%%%%%%%%%% MACROS NAMES
\newcommand{\lecturername}{Martin Licht}
% \newcommand{\assistantnamea}{Jochen Hinz}
% \newcommand{\assistantnameb}{Ivan Bioli}
\newcommand{\semestername}{Winter Semester 2023}
\newcommand{\lecturename}{Analysis III - 202(c)}
\DeclarePairedDelimiter\floor{\lfloor}{\rfloor}

%%%%%%%%%%% HEADER
\newdateformat{yeardate}{\THEYEAR}
\newcommand{\exsheet}[3] % input is the number of the session and the day TODO What's that
{\clearpage

	\begin{center}
		{\Large \textbf{\lecturename}}\\[2ex]
		\semestername
	\end{center}

	% \vspace{2ex}
	% \lecturername

	\vspace{2ex}
	{\Large Session #1: #3\,#2, \yeardate\today}
	%\hfill
	%{\Large EPF Lausanne}

	\hrulefill
}





\usepackage{comment}

\newtheorem{exercise}{Exercise}
\newtheorem{solutionenv}{Solution}

\newboolean{hide_solution}
\ifx\hidesolutions\undefined
\newenvironment{solution}{\begin{solutionenv}}{\end{solutionenv}}
\setboolean{hide_solution}{false}
\else
\excludecomment{solution}
\setboolean{hide_solution}{true}
\fi

\newcommand{\ifnotsolution}[1]{\ifthenelse{\boolean{hide_solution}}{#1}{}}
\newcommand{\ifsolution}[1]{\ifthenelse{\boolean{hide_solution}}{}{#1}}








\allowdisplaybreaks

\begin{document}
\exsheet{4}{3}{October} % parameters are the number of the session and the day



\begin{exercise}
    We review notions of potentials and conservative vector fields. Let $\Omega \subseteq \bbR^n$ be open.
    Suppose we have a vector field $F = (F_1,\dots,F_n) \in C^1(\Omega,\bbR^n)$.
	Recall that we have introduced the condition
    \begin{align}\label{math:symmetry_of_jacobian}
        \partial_i F_j = \partial_j F_i, \qquad 1 \leq i,j \leq n.
    \end{align}
    \begin{itemize}
     \item Suppose that $n=2$. Show that $F$ satisfies \eqref{math:symmetry_of_jacobian} if and only if it is curl-free: $\operatorname{curl} F = 0$.
     \item Suppose that $n=3$. Show that $F$ satisfies \eqref{math:symmetry_of_jacobian} if and only if it is curl-free: $\operatorname{curl} F = 0$.
     \item Suppose that $n=1$. Show that $F$ satisfies \eqref{math:symmetry_of_jacobian}.
     \item Suppose that $F$ admits a potential $f \in C^1(\Omega,\bbR)$, so that $\nabla f = F$. 
     Show that if $\gamma : [a,b] \rightarrow \Omega$ is a simple regular curve, then 
     \begin{align}
        \int_\gamma F \;dl = f(\gamma(b)) - f(\gamma(a)).
     \end{align}
     Show that if $\gamma$ is closed, then 
     \begin{align}
        \int_\gamma F \;dl = 0.
     \end{align}
    \end{itemize}
\end{exercise}
\begin{solution}
    
\end{solution}

\begin{exercise}[vector analysis in 1D]
    Let $\Omega \subseteq (a,b)$ be an open interval in one-dimensional space.
    \begin{itemize}
     \item Explain why there cannot be a simple closed curve in $\Omega$.
     \item When $\Omega = (-10,10)$, compute the integral of the scalar field 
     \[
        f(x) = \frac{x}{\sqrt{1+x^2}}
     \]
     along the curves 
     \[
        \gamma_1 : [0,1] \rightarrow \Omega, \quad t \mapsto (2t - 1),
        \\
        \gamma_2 : [-1,1] \rightarrow \Omega, \quad t \mapsto (t),
        \\
        \gamma_3 : [0,1] \rightarrow \Omega, \quad t \mapsto (1-2t),
        \\
        \gamma_4 : [0,1] \rightarrow \Omega, \quad t \mapsto (-1 + 2t^5),
     \]
     Compute the tangent vectors $\dot\gamma(t)$.
    \end{itemize}
    \item 
    Compute the integral of the vector field 
    \begin{align}
        F(x) = \left( x e^{x^2} \right)
    \end{align}
    along the curve $\gamma_4$. Find a potential for this vector field, and write down the general form of all potentials.
\end{exercise}
\begin{solution}
    This is obviously content of Analysis 1 but repackaged in the manner of vector analysis. 
\end{solution}



\begin{exercise}
    Compute the line integral of the vector field $\vec{F}$ along the curve $\gamma$, where 
    \[
        \vec F : \bbR^2 \to \bbR^2, \quad (x_1,x_2) \mapsto ( x_2, 0 ), \qquad \gamma : [0,2\pi]  \to \bbR^2, \quad t \mapsto ( \cos(t), \sin(t) )
    \]
    Compute the line integral of the vector field $\vec{G}$ along the curve $\delta$, where 
    \[
        \vec G : \bbR^3 \to \bbR^3, \quad (x_1,x_2,x_3) \mapsto ( 2, 3, -1 ), \qquad \delta : [0,4]  \to \bbR^2, \quad t \mapsto  ( t^2, \cos(t), e^t )
    \]
\end{exercise}
\begin{solution}
    \begin{align*}
        \int_{\Gamma} \vec{F}(x_1,x_2) d\ell &= \int_{0}^{2\pi} (\vec{F}\circ \gamma)(t)\cdot \dot{\gamma}(t)dt
        \\&= \int_{0}^{2\pi} \begin{pmatrix}\sin t\\ 0\end{pmatrix}\cdot \begin{pmatrix} -\sin(t)\\ \cos(t)\end{pmatrix}dt
        \\&= \int_{0}^{2\pi}  -\sin^2(t)dt
        \\&= \int_{0}^{2\pi}  -\frac{1}{2}+\frac{1}{2}\cos(2t)dt
        \\&= \left[-\frac{1}{2}t + \frac{1}{4}\sin(2t)\right]_{0}^{2\pi} = -\pi
    \end{align*}

    \begin{align*}
        \int_{\Gamma} \vec{G}(x_1,x_2,x_3) d\ell &= \int_{0}^{4} (\vec{G}\circ \delta)(t)\cdot \dot{\delta}(t)dt
        \\&= \int_{0}^{4}\begin{pmatrix} 2\\ 3\\ -1\end{pmatrix} \cdot  \begin{pmatrix}2t\\ -\sin(t) \\ e^t \end{pmatrix}dt
        \\&= \int_{0}^{4} 4t - 3\sin(t) - e^t dt
        \\&= \left[2t^2 + 3\cos(t) - e^{t} \right]_{0}^{4}
        \\&= 32 + 3\cos4-e^4 - \left(3 - 1\right) = 30+3\cos4 -e^ 4
    \end{align*}
        % TODO: we need of those cosine half angle formulas to turn the square of sin into a cos 2t      
        % TODO: the second one is easy
\end{solution}


\begin{exercise}
    Compute the line integral of the vector field $\vec{F}$ along the curve $\gamma$, where 
    \[
        \vec F : \bbR^2 \to \bbR^2, \quad (x_1,x_2) \mapsto ( e^{ x_1x_2 } x_2, e^{ x_1 x_2 } x_1 )
    \]
    and 
    \[
        \gamma : [0,1] \to \bbR^2, \quad t \mapsto \left( \arctan( \cos(\pi t)^2 - \sin(\pi t)^2 ), 1 + \sqrt[2]{\arctan(t)} \right)
    \]
\end{exercise}
\begin{solution}
    The vector field is the gradient of $e^{ x_1 x_2 }$, and so we can use the formula for conservative vector fields.
    We find that
    \[
        \int_\Gamma \vec F d\ell = \int_\Gamma \grad f d\ell = f(\gamma(1)) - f(\gamma(0))
    \]
    Now we only need to compute the function values of $f$. Explicitly,
    \[
        \gamma(0) = \left( \arctan( \cos(\pi 0)^2 - \sin(\pi 0)^2 ), 1 + \sqrt[2]{\arctan(0)} \right)
        =
        \left( \arctan( 1 ), 1 + \sqrt[2]{0} \right) 
        =
        \left( \pi/4, 1 \right)
    \]
    \[
        \gamma(1) = \left( \arctan( \cos(\pi 1)^2 - \sin(\pi 1)^2 ), 1 + \sqrt[2]{\arctan(1)} \right)
        =
        \left( \arctan( 1 ), 1 + \sqrt[2]{\pi/4} \right)
        =
        \left( \pi/4, 1 + \sqrt[2]{\pi/4} \right)
    \]
    We then compute
    \[
        f(\gamma(1)) - f(\gamma(0)) 
        =
        \exp\left( \frac \pi 4 ( 1 + \sqrt[2]{\pi/4} ) \right) - \exp\left( \pi/4 \right).
    \]
\end{solution}

\begin{exercise}
    What is the length of the graph of the function $g(x) = x^{\frac 3 2}$ over the interval $[0,2]$?
    Simplify as much as reasonable.
\end{exercise}
\begin{solution}
    We can express the graph as a curve 
    \[
        \gamma : [0,2] \to \mathbb R^2, \quad t \mapsto ( t, t^{\frac 3 2} ) 
    \]
    As mentioned in the lecture, the integral of $1$ along that curve gives the length:
    \[
        \int_\Gamma 1 \;dl = \int_0^2 \left| ( 1, \frac 3 2 t^{\frac 1 2} \right| \;dt = \int_0^2 \sqrt{ 1 + \frac 9 4 t } \;dt
    \]
    We integrate the last expression. The integrand is a derivative, up to some numerical factor:
    \begin{align*}
        \int_0^2 \sqrt{ 1 + \frac 9 4 t } \;dt 
        &= 
        \frac 2 3 \cdot \frac 4 9 \int_0^2 \frac 3 2 \cdot \frac 9 4 \left( 1 + \frac 9 4 t \right)^{\frac 1 2} \;dt 
        \\&= 
        \frac 2 3 \cdot \frac 4 9 \int_0^2 \partial_t \left( 1 + \frac 9 4 t \right)^{\frac 3 2} \;dt 
        \\&= 
        \frac 2 3 \cdot \frac 4 9 \left[ \left( 1 + \frac 9 4 t \right)^{\frac 3 2} \right]_0^2
        \\&= 
        \frac 2 3 \cdot \frac 4 9 \left( \left( 1 + \frac 9 2 \right)^{\frac 3 2} - 1 \right)
        \\&= 
        \frac{8}{27} \left( \left( \frac {11} 2 \right)^{\frac 3 2} - 1 \right)
        .
    \end{align*}
    (You can also use \textit{Intégration par changement de variable}, which is the same idea in a different formalism.)
    % \begin{align*}
    %     \int_0^2 \sqrt{1 + \left[\frac{\;d}{\;dx} (x^2\sin x)\right]^2}dx &= \int_0^2 \sqrt{1 + (2x\sin x + x^2\cos x)^2}\;dx\\
    % &=\int_0^2 \sqrt{1 + 4x^2\sin^2 x + x^4\cos^2 x + 4x^3\cos x\sin x}\;dx
    % \end{align*}
\end{solution}



\begin{exercise}
    Take a look at the functions 
    \[
        f(x_1,x_2) = 1, \quad g(x_1,x_2) = x_2.
    \]
    We have the following curves:
    \begin{align*}
        &\alpha : [0,1] \to \bbR^2, \quad t \mapsto (t,t),
        \\
        &\beta  : [0,2] \to \bbR^2, \quad t \mapsto \begin{cases} (t,0) & 0 \leq t < 1, \\ (0,t-1) & 1 \leq t \leq 2, \end{cases}
        \\
        &\gamma : [0,1] \to \bbR^2, \quad t \mapsto (t^2,t).
%         \\
%         &\delta : [0,1] \to \bbR^2, \quad t \mapsto (1-t^2,1-t).
    \end{align*}
    Compute:
    \[
        \int_A f \;d\ell, \quad \int_B f \;d\ell, \quad 
        \int_A g \;d\ell, \quad \int_B g \;d\ell, \quad 
        \int_\Gamma g \;d\ell
%         \quad \int_\Delta g \;d\ell
        .
    \]
\end{exercise}
\begin{solution}  
     We compute these integrals as follows:
     \begin{align*}
        \int_A f \;d\ell
        =
        \int_0^1 \sqrt{2} \;dt = \left[\sqrt{2}t\right]_0^1 = \sqrt{2}
        ,
        \quad 
        \int_B f \;d\ell
        = 
        \int_0^2 \;dt = \left[t \right]_0^2  = 2
        %
    \end{align*}  
    \begin{align*}
        \int_A g \;d\ell
        &= \int_0^1 t \sqrt{2} \;dt = \left[\frac{\sqrt{2}}{2}t^2 \right]_0^1 = \frac{\sqrt{2}}{2}
        %
    \end{align*}
    \begin{align*}
        \int_B g \;d\ell
        &=
        \int_0^1 (0) \;dt + \int_1^2 (t-1) \;dt  
        = 
        \int_1^2 (t-1) \;dt 
        = 
        \left[ \frac{1}{2} (t-1)^2 \right]_1^2 = \frac{1}{2}
        %
    \end{align*}
    For the interal of $g$ over the third curve, we proceed as follows:
    \begin{align*}
        \int_\Gamma g \;d\ell
        &=
        \int_0^1 t \sqrt{ 4t^2 + 1 } \;dt
        \\&=
        \int_0^1 t \left( 4t^2 + 1 \right)^{\frac 1 2} \;dt
        \\&=
        \frac 2 3 \cdot \frac 1 8 \cdot 
        \int_0^1 \frac 3 2 \cdot 8t \left( 4t^2 + 1 \right)^{\frac 1 2} \;dt
        \\&=
        \frac 2 3 \cdot \frac 1 8 \cdot 
        \int_0^1 \partial_t \left( 4t^2 + 1 \right)^{\frac 3 2} \;dt
        =
        \frac 2 3 \cdot \frac 1 8 \cdot 
        \left( 5^{\frac 3 2} - 1 \right)
        = 
        \frac{ 5^{\frac 3 2} - 1 }{12}
        .
    \end{align*}
%     For the integral of $g$ over the last curve, we take a sharp look and integrate by parts:
%     \begin{align*}
%         \int_\Delta g \;d\ell
%         &=
%         \int_0^1 t^2 \left( 1 + 4t^2 \right)^{\frac 1 2} \;dt
%         =
%         \int_0^1 t^2 \left( 2t + 1 \right)^{\frac 1 2} \;dt
%         \\&=
%         \int_0^1 t^2 \cdot \partial_t\left( \frac 4 3 \left( 1 + 2t \right)^{\frac 3 2} \right) \;dt
%         \\&=
%         \frac 4 3 \left[ t^2 \left( 2t + 1 \right)^{\frac 3 2} \right]_0^1
%         -
%         \frac 4 3 \int_0^1 2t \left( 2t + 1 \right)^{\frac 3 2} \;dt
%         =
%         3^{\frac 3 2}
%         -
%         \frac 4 3 \int_0^1 2t \left( 2t + 1 \right)^{\frac 3 2} \;dt
%         .
%     \end{align*}
%     We integrate by parts once more:
%     \begin{align*}
%         \int_0^1 2t \cdot \partial_t\left( \frac 4 5 \left( 2t + 1 \right)^{\frac 5 2} \right) \;dt
%         &=
%         \left[ 2t \cdot \frac 4 5 \left( 2t + 1 \right)^{\frac 5 2} \right]_0^1
%         -
%         \int_0^1 2 \cdot \frac 4 5 \left( 2t + 1 \right)^{\frac 5 2} \;dt
%         \\&=
%         \frac 8 5 \cdot 3^{\frac 5 2}
%         -
%         \int_0^1 2 \cdot \frac 4 5 \left( 2t + 1 \right)^{\frac 5 2} \;dt
%         \\&=
%         \frac 8 5 \cdot 3^{\frac 5 2}
%         -
%         \frac 8 5 
%         \int_0^1 \left( 2t + 1 \right)^{\frac 5 2} \;dt
%         ,
%     \end{align*}
%     and then we calculate, using the substitution $u = 2t + 1$, which means $du = 2 dt$
%     \begin{align*}
%         \int_0^1 \left( 2t + 1 \right)^{\frac 5 2} \;dt
%         &=
%         \frac 1 2 
%         \int_1^3 \left( u \right)^{\frac 5 2} \;du
%         =
%         \frac 1 2 
%         \left[ \frac 2 7 u^{\frac 7 2}\right]_1^3
%         =
%         \frac 1 7 
%         \left( 3^{\frac 7 2} - 1 \right)
%         .
%     \end{align*}
%     In combination,
%     \begin{align}
%         \int_\Delta g \;d\ell 
%         &= 
%         3^{\frac 3 2} - \frac 4 3 \cdot \frac 8 5 \cdot 3^{\frac 5 2} + \frac 4 3 \cdot \frac 8 5 \cdot \frac 1 7 \left( 3^{\frac 7 2} - 1 \right)
%         \\&=
%         3^{\frac 3 2} - \frac{32}{15} 3^{\frac 5 2} + \frac{32}{105} \left( 3^{\frac 7 2} - 1 \right)
%         .
%     \end{align}
    Note that the curves all have the same starting and end points, $(0,0)$ and $(1,1)$, 
    but the curve integrals of the function may (generally) differ from curve to curve.  
\end{solution}





\begin{exercise}
    We want to verify that the curve integrals do not depend on the direction of the parameterization.
    Consider the curve parameterizations
    \begin{align}
        \gamma_+ : [0,1] \rightarrow \bbR^2, \quad t \mapsto ( t, 1 + \frac 1 2 t^2 ),
        \\
        \gamma_- : [0,1] \rightarrow \bbR^2, \quad s \mapsto ( 1-s, 1 + \frac 1 2 (1-s)^2 ),
    \end{align}
    \begin{itemize}
     \item Given the scalar field 
     \[
        f(x,y) 
        = x ( y - \frac 1 2 x^2 )
        = x y - \frac 1 2 x^3
        ,
     \]
     show that 
     \begin{align}
        \int_{\gamma_+} f \;dl = \int_{\gamma_-} f \;dl
     \end{align}
     \item Compute the tangent and unit tangent vectors of each curve,
     and show that $\dot\gamma_+(t) = - \dot\gamma_-(1-t)$.
     Show that the curve integrals along $\gamma_+$ and $\gamma_-$ of the vector field
     \begin{align}
        F(x,y) = (-y,x)
     \end{align}
     are identical.
    \end{itemize}
\end{exercise}
\begin{solution}
    \begin{itemize}
     \item We compute the first integral directly:
     \begin{align}
        \int_{\gamma_+} f \;dl
        &=
        \int_{0}^1 f( t, 1 + \frac 1 2 t^2 ) |\dot{\gamma}_+(t)|dt
        \\&
        =
        \int_{0}^1 t \sqrt{ 1 + t^2 }
        \\&
        =
        \frac 1 2 \int_{0}^1 2t \sqrt{ 1 + t^2 }
        \\&
        =
        \frac 1 2 \int_1^2 \sqrt{ u } \;du\\&
        =
        \frac 1 2 \left[ \frac 2 3 u^{\frac 3 2} \right]_1^2 = \frac 1 3 \left( 2 \sqrt{2} - 1 \right),
     \end{align}
     where we used the substitution $u = 1 + t^2$.
     The second integral can be computed directly as well, or with a change of variables. For the latter, we use
     \begin{align}
        \int_{\gamma_-} f \;dl
        &=
        \int_{0}^1 f( 1 - s, 1 + \frac 1 2 (1 - s)^2 ) |\dot{\gamma}_-(s)|ds
        \\&
        =-\int_1^0 f(t, 1 + \frac 1 2 t^2) |\dot{\gamma}_-(1-t)|dt
        \\&
        = \int_0^1 t \sqrt{1 + (1 - (1 - t))^2} dt = \int_0^1 t \sqrt{1 + t^2} dt
        \\&
        = \int_{\gamma_+}f dl,
     \end{align}
     where we used the substitution $t = 1 - s$ with $dt = -ds$.

        \item We compute the tangent vectors:
        \begin{align}
            \dot\gamma_+(t) = \begin{pmatrix} 1 \\ t \end{pmatrix}, \quad |\dot \gamma_+(t)| = \sqrt{1 + t^2},
            \\
            \dot\gamma_-(s) = \begin{pmatrix} -1 \\ s - 1 \end{pmatrix}, \quad |\dot \gamma_-(s)| = \sqrt{1 + (s-1)^2}.
        \end{align}
        Hence, it is clear that $\dot\gamma_+(t) = - \dot\gamma_-(1-t)$. Furthermore, it is clear that $\gamma_+(t) = \gamma_-(1 - t)$. Finally, for the curve integral we find again using a change of variables that
        \begin{align}
            \int_{\gamma_+} F \;dl &= \int_0^1 F(\gamma_+(t)) \cdot \dot\gamma_+(t) \;dt
            \\&
            = -\int_0^1 F(\gamma_-(1 - t)) \cdot \dot\gamma_-(1 - t) \;dt
            \\&
            = \int_0^1 F(\gamma_-(s)) \cdot \dot\gamma_-(s) \;ds
            \\&
            = \int_{\gamma_-} F \;dl,
        \end{align}
        where we have used the substitution $s = 1 - t$ with $dt = -ds$.
    \end{itemize}
\end{solution}












\begin{exercise}
    Consider the open cube 
    \[
        \Omega = (0,1)^2 = \left\{ (x,y) \in \mathbb R^2 : 0 < x,y < 1 \right\}. 
    \]
    \begin{itemize}
        \item Show that $\Omega$ is convex.
        \item Let $z \in \Omega$. Show that $\Omega$ is star-shaped with respect to $z$.
        \item Show that $\Omega$ is simply-connected.
    \end{itemize}
    For the last part, show that any two continuous curves with the same endpoints are homotopic relative to endpoints. 
\end{exercise}
\begin{solution}
    \item
    \item 
    \item 
    Given two continuous curves $\gamma_0, \gamma_1 : [a,b] \rightarrow \Omega$ with $\gamma_0(a) = \gamma_1(a) =: g_a$ and $\gamma_0(b) = \gamma_1(b) =: g_b$, we define 
    \[
        \gamma : [a,b] \times [0,1] \rightarrow \bbR^2
    \]
    by setting 
    \[
        \gamma(t,s) = (1-s) \cdot \gamma_0(t) + s \cdot \gamma_1(t).
    \]
    For any $t \in [a,b]$ we have $\gamma_0(t), \gamma_1(t) \in \Omega$,
    and since $\Omega$ is convex, we have $(1-s) \cdot \gamma_0(t) + s \cdot \gamma_1(t) \in \Omega$ for all $s \in [0,1]$.
    Hence $\gamma$ defines a function 
    \[
        \gamma : [a,b] \times [0,1] \rightarrow \Omega
    \]
    Furthermore, $\gamma$ is obviously continuous in both variables. 
    
    We verify the remaing conditions. For all $t \in [a,b]$ we have 
    \begin{align}
        \gamma(t,0) = (1-0) \cdot \gamma_0(t) + 0 \cdot \gamma_1(t) = \gamma_0(t)
    \end{align}
    and 
    \begin{align}
        \gamma(t,1) = (1-1) \cdot \gamma_0(t) + 1 \cdot \gamma_1(t) = \gamma_1(t)
        .
    \end{align}
    Additionally, for all $s \in [0,1]$, we have 
    \begin{align}
        \gamma(a,s) = (1-s) \cdot \gamma_0(a) + s \cdot \gamma_1(a) = (1-s) \cdot g_a + s \cdot g_a = g_a
    \end{align}
    and 
    \begin{align}
        \gamma(b,s) = (1-s) \cdot \gamma_0(b) + s \cdot \gamma_1(b) = (1-s) \cdot g_b + s \cdot g_b = g_b
        .
    \end{align}
    This shows that $\gamma$ satisfies the required axioms.
\end{solution}






































\end{document}
