\documentclass[11pt]{article}
%%%%%%%%%%% SET MARGINS
\setlength{\textheight}{20cm}
\setlength{\topmargin}{-0.5cm}
\setlength{\oddsidemargin}{+0cm}
\setlength{\textwidth}{16.3cm}
%\setlength{\parskip}{6pt}
\setlength{\parindent}{0pt}

%%%%%%%%%%% PACKAGES
\usepackage{amsmath}
\usepackage{amssymb}
\usepackage{amsfonts}
%\usepackage{a4wide}
\usepackage{graphicx}
\usepackage{color}
\usepackage[normalem]{ulem}
\usepackage{enumitem}
\usepackage{capt-of}
\usepackage{float}
\usepackage{amsmath}
\usepackage{listings}
\definecolor{mygreen}{RGB}{28,172,0} % color values Red, Green, Blue
\definecolor{mylilas}{RGB}{170,55,241}
\usepackage{empheq}
\usepackage[ruled]{algorithm2e}
\usepackage{mathrsfs}
\usepackage{datetime}
\usepackage{subcaption}

% TODO: combine the two package lists and reduce redundancies 
\usepackage{mathtools}
\usepackage{nicefrac}
\usepackage{hyperref}
\usepackage{url}
\usepackage{amsmath,amssymb,amsfonts}
\usepackage{a4wide}
\usepackage{graphicx}
\usepackage{color}
\usepackage[normalem]{ulem}
\usepackage{capt-of}
\usepackage{float}
\usepackage[ruled]{algorithm2e}
\usepackage{amsmath,amssymb,amsfonts}
\usepackage{a4wide}
\usepackage{graphicx}
\usepackage{color}
\usepackage[normalem]{ulem}
\usepackage{capt-of}
\usepackage{float}
\usepackage[ruled]{algorithm2e}
\usepackage{mathrsfs}







\newcommand{\Lc}[2]{{\color{blue} \sout{#1} } \textcolor{red}{#2}}
\newcommand{\La}[1]{\textcolor{red}{#1}}
\newcommand{\lh}{\mathscr{L}_h}
\newcommand{\cl}{\mathscr{L}}
\newcommand{\cf}{\mathscr{F}}
\newcommand{\dx}{dx}
\newcommand{\ltn}{\mathscr{l}^2}
\newcommand{\bbR}{\mathbb{R}}
\newcommand{\Rset}{\mathbb{R}}
\newcommand{\Nset}{\mathbb{N}}
\newcommand{\scL}{\mathcal{L}}
\newcommand{\xx}{\mathbf{x}}
\newcommand{\norm}[1]{\|{#1}\|}
\newcommand{\yy}{\mathbf{y}}
\newcommand{\at}[1]{\big|_{#1}}
\renewcommand{\div}{\mathrm{div}}
\newcommand{\divergence}{\mathrm{div}}
\newcommand{\cp}[1]{\textcolor{blue}{#1}}

\newcommand{\FF}{\texttt{FreeFem++ }}
\newcommand{\FFns}{\texttt{FreeFem++}}
\newcommand{\FFfull}{\texttt{FreeFem++-x11}}
\newcommand{\cmd}[1]{ \medskip \noindent \texttt{#1} \medskip}
\newcommand{\incmd}[1]{\texttt{#1}}
\newcommand{\shrinkitems}{\addtolength{\itemsep}{-0.5\baselineskip}}
\newcommand{\mtt}[1]{\mathtt{#1}}
\newcommand{\ML}{\texttt{Matlab }}

\newcommand{\bb}{\mathbf{b}}
\newcommand{\nn}{\mathbf{n}}
\newcommand{\vecA}{\vec{A}}
\newcommand{\vecB}{\vec{B}}


\newcommand{\mesh}{\mathcal{T}_h}
\newcommand{\refel}{\widehat{K}}
\newcommand{\ver}{\mathbf{a}}
\newcommand{\refver}{\widehat{\mathbf{a}}}
\newcommand{\grad}{\nabla}
\newcommand{\refgrad}{\widehat{\nabla}}
\newcommand{\refu}{\widehat{u}}
\newcommand{\refbasis}{\widehat{\varphi}}
\newcommand{\refxx}{\widehat{\xx}}
\newcommand{\refx}{\widehat{x}}
\newcommand{\refy}{\widehat{y}}
\newcommand{\refrho}{\widehat{\rho}}
\newcommand{\refh}{\widehat{h}}






% For typesetting Python code
\newcommand{\matlab}{{\sc Matlab}\xspace}
\usepackage{listings}
\lstloadlanguages{Python}
\lstloadlanguages{csh}%
\definecolor{MyDarkGreen}{rgb}{0.0,0.4,0.0}
\definecolor{purple}{rgb}{0.58,0,0.82}
\lstset{language=Python,                    % Use Python
	%frame=single,                          % Single frame around code
	basicstyle=\ttfamily\footnotesize\color{black},
	keywordstyle=[1]\color{blue}\bf,        % Python functions bold and blue
	keywordstyle=[2]\color{purple},         % Python function arguments purple
	keywordstyle=[3]\color{red}\underbar,   % User functions underlined and blue
	commentstyle=\usefont{T1}{pcr}{m}{sl}\color{MyDarkGreen}\small,
	stringstyle=\color{purple},
	showstringspaces=false,                 % Don't put marks in string spaces
	tabsize=3,                              % 5 spaces per tab
	morekeywords={xlim,ylim,var,alpha,factorial,poissrnd,normpdf,normcdf},
	morecomment=[l][\color{blue}]{...},
	breaklines=true,
	breakatwhitespace=true,
	emptylines=1,
	mathescape=true,
	xleftmargin=0ex,
	emphstyle=\bfseries\color{red}
}





%%%%%%%%%%% MACROS NAMES
\newcommand{\lecturername}{Martin Licht}
% \newcommand{\assistantnamea}{Jochen Hinz}
% \newcommand{\assistantnameb}{Ivan Bioli}
\newcommand{\semestername}{Winter Semester 2023}
\newcommand{\lecturename}{Analysis III - 202(c)}
\DeclarePairedDelimiter\floor{\lfloor}{\rfloor}

%%%%%%%%%%% HEADER
\newdateformat{yeardate}{\THEYEAR}
\newcommand{\exsheet}[3] % input is the number of the session and the day TODO What's that
{\clearpage

	\begin{center}
		{\Large \textbf{\lecturename}}\\[2ex]
		\semestername
	\end{center}

	% \vspace{2ex}
	% \lecturername

	\vspace{2ex}
	{\Large Session #1: #3\,#2, \yeardate\today}
	%\hfill
	%{\Large EPF Lausanne}

	\hrulefill
}





\usepackage{comment}

\newtheorem{exercise}{Exercise}
\newtheorem{solutionenv}{Solution}

\newboolean{hide_solution}
\ifx\hidesolutions\undefined
\newenvironment{solution}{\begin{solutionenv}}{\end{solutionenv}}
\setboolean{hide_solution}{false}
\else
\excludecomment{solution}
\setboolean{hide_solution}{true}
\fi

\newcommand{\ifnotsolution}[1]{\ifthenelse{\boolean{hide_solution}}{#1}{}}
\newcommand{\ifsolution}[1]{\ifthenelse{\boolean{hide_solution}}{}{#1}}








\allowdisplaybreaks

\begin{document}
\exsheet{3}{3}{October} % parameters are the number of the session and the day
\def\hidesolutions{}

\begin{exercise}
    We have focused on scalar and vector fields that are defined on the entire space. 
    But sometimes we are also interested in fields with singularities. 
    Compute the gradient and the Laplacian of the function 
    \[
        f : \bbR^3 \setminus \{0\} \rightarrow \bbR, \quad (x_1,x_2,x_3) \mapsto \frac{1}{\sqrt[root]{x_1^2+x_2^2+x_3^1}}
    \]
    where the scalar field is twice differentiable. 
    % Sketch the gradient field.
\end{exercise}
\begin{solution}     
\end{solution}

\begin{exercise}
    Consider the curves 
    \[
        \gamma : [-1,1] \rightarrow \bbR^2, \quad t \mapsto ( t^2, t^3 )
    \]
    \[
        \delta : [-1,1] \rightarrow \bbR^2, \quad t \mapsto ( \cos(t), \sin(2t) )
    \]
    Draw them. Are they simple, closed, differentiable, or regular? What are their tangents?
\end{exercise}
\begin{solution}     
\end{solution}

\begin{exercise}
    Compute the line integral of $f : \bbR^2 \rightarrow \bbR$ along the two curves $\gamma$ and $\delta$, 
    where 
    \[
        f : \bbR^2 \to \bbR, \quad (x_1,x_2) \mapsto 1 + x_1^2 + x_2^2
    \]
    \[
        \gamma : [-1,1] \rightarrow \bbR^2, \quad t \mapsto ( -\cos(t), \sin(2t) )
    \]
    \[
        \delta : [-1,1] \rightarrow \bbR^2, \quad t \mapsto (2t-1,0)
    \]
    Compare the results. What are the endpoints of the two curves.  
\end{exercise}
\begin{solution}     
\end{solution}

\begin{exercise}
    Compute the line integral of $f : \bbR^2 \rightarrow \bbR$ 
    along the circle $C$ with radius $3$ centered at the origin, where 
    \[
        f : \bbR^2 \to \bbR, \quad (x_1,x_2) \mapsto 3 x_2^2 + x_2^3
    \]
    Hint: you must first find a parameterization of $C$. 
\end{exercise}
\begin{solution}     
    % Pick a parameterization, use that one integrand is odd and therefore vanishes, and use the cosine half-angle formula for the other term.
\end{solution}

\begin{exercise}
    Compute the line integral of $f : \bbR^2 \rightarrow \bbR$ 
    along the circle $C$ with radius $3$ centered at the origin, where 
    \[
        f : \bbR^2 \to \bbR, \quad (x_1,x_2) \mapsto 3 x_2^2 + x_2^3
    \]
    Hint: you must first find a parameterization of $C$. 
\end{exercise}
\begin{solution}     
    % Pick a parameterization, use that one integrand is odd and therefore vanishes, and use the cosine half-angle formula for the other term.
\end{solution}

\begin{exercise}
    Compute the line integral of the vector field $\vec{F}$ along the curve $\gamma$, where 
    \[
        \vec F : \bbR^2 \to \bbR^2, \quad (x_1,x_2) \mapsto ( x_2, 0 ), \qquad \gamma : [0,2\pi]  \to \bbR^2, \quad t \mapsto ( \cos(t), \sin(t) )
    \]
    Compute the line integral of the vector field $\vec{G}$ along the curve $\delta$, where 
    \[
        \vec G : \bbR^2 \to \bbR^2, \quad (x_1,x_2) \mapsto ( 2, 3, -1 ), \qquad \delta : [0,4]  \to \bbR^2, \quad t \mapsto  ( t^2, \cos(t), e^x )
    \]
\end{exercise}
\begin{solution}
    % TODO: we need of those cosine half angle formulas to turn the square of sin into a cos 2t      
    % TODO: the second one is easy
\end{solution}

\end{document}