\documentclass[11pt]{article}
% % \def\hidesolutions{}
%%%%%%%%%%% SET MARGINS
\setlength{\textheight}{20cm}
\setlength{\topmargin}{-0.5cm}
\setlength{\oddsidemargin}{+0cm}
\setlength{\textwidth}{16.3cm}
%\setlength{\parskip}{6pt}
\setlength{\parindent}{0pt}

%%%%%%%%%%% PACKAGES
\usepackage{amsmath}
\usepackage{amssymb}
\usepackage{amsfonts}
%\usepackage{a4wide}
\usepackage{graphicx}
\usepackage{color}
\usepackage[normalem]{ulem}
\usepackage{enumitem}
\usepackage{capt-of}
\usepackage{float}
\usepackage{amsmath}
\usepackage{listings}
\definecolor{mygreen}{RGB}{28,172,0} % color values Red, Green, Blue
\definecolor{mylilas}{RGB}{170,55,241}
\usepackage{empheq}
\usepackage[ruled]{algorithm2e}
\usepackage{mathrsfs}
\usepackage{datetime}
\usepackage{subcaption}

% TODO: combine the two package lists and reduce redundancies 
\usepackage{mathtools}
\usepackage{nicefrac}
\usepackage{hyperref}
\usepackage{url}
\usepackage{amsmath,amssymb,amsfonts}
\usepackage{a4wide}
\usepackage{graphicx}
\usepackage{color}
\usepackage[normalem]{ulem}
\usepackage{capt-of}
\usepackage{float}
\usepackage[ruled]{algorithm2e}
\usepackage{amsmath,amssymb,amsfonts}
\usepackage{a4wide}
\usepackage{graphicx}
\usepackage{color}
\usepackage[normalem]{ulem}
\usepackage{capt-of}
\usepackage{float}
\usepackage[ruled]{algorithm2e}
\usepackage{mathrsfs}







\newcommand{\Lc}[2]{{\color{blue} \sout{#1} } \textcolor{red}{#2}}
\newcommand{\La}[1]{\textcolor{red}{#1}}
\newcommand{\lh}{\mathscr{L}_h}
\newcommand{\cl}{\mathscr{L}}
\newcommand{\cf}{\mathscr{F}}
\newcommand{\dx}{dx}
\newcommand{\ltn}{\mathscr{l}^2}
\newcommand{\bbR}{\mathbb{R}}
\newcommand{\Rset}{\mathbb{R}}
\newcommand{\Nset}{\mathbb{N}}
\newcommand{\scL}{\mathcal{L}}
\newcommand{\xx}{\mathbf{x}}
\newcommand{\norm}[1]{\|{#1}\|}
\newcommand{\yy}{\mathbf{y}}
\newcommand{\at}[1]{\big|_{#1}}
\renewcommand{\div}{\mathrm{div}}
\newcommand{\divergence}{\mathrm{div}}
\newcommand{\cp}[1]{\textcolor{blue}{#1}}

\newcommand{\FF}{\texttt{FreeFem++ }}
\newcommand{\FFns}{\texttt{FreeFem++}}
\newcommand{\FFfull}{\texttt{FreeFem++-x11}}
\newcommand{\cmd}[1]{ \medskip \noindent \texttt{#1} \medskip}
\newcommand{\incmd}[1]{\texttt{#1}}
\newcommand{\shrinkitems}{\addtolength{\itemsep}{-0.5\baselineskip}}
\newcommand{\mtt}[1]{\mathtt{#1}}
\newcommand{\ML}{\texttt{Matlab }}

\newcommand{\bb}{\mathbf{b}}
\newcommand{\nn}{\mathbf{n}}
\newcommand{\vecA}{\vec{A}}
\newcommand{\vecB}{\vec{B}}


\newcommand{\mesh}{\mathcal{T}_h}
\newcommand{\refel}{\widehat{K}}
\newcommand{\ver}{\mathbf{a}}
\newcommand{\refver}{\widehat{\mathbf{a}}}
\newcommand{\grad}{\nabla}
\newcommand{\refgrad}{\widehat{\nabla}}
\newcommand{\refu}{\widehat{u}}
\newcommand{\refbasis}{\widehat{\varphi}}
\newcommand{\refxx}{\widehat{\xx}}
\newcommand{\refx}{\widehat{x}}
\newcommand{\refy}{\widehat{y}}
\newcommand{\refrho}{\widehat{\rho}}
\newcommand{\refh}{\widehat{h}}






% For typesetting Python code
\newcommand{\matlab}{{\sc Matlab}\xspace}
\usepackage{listings}
\lstloadlanguages{Python}
\lstloadlanguages{csh}%
\definecolor{MyDarkGreen}{rgb}{0.0,0.4,0.0}
\definecolor{purple}{rgb}{0.58,0,0.82}
\lstset{language=Python,                    % Use Python
	%frame=single,                          % Single frame around code
	basicstyle=\ttfamily\footnotesize\color{black},
	keywordstyle=[1]\color{blue}\bf,        % Python functions bold and blue
	keywordstyle=[2]\color{purple},         % Python function arguments purple
	keywordstyle=[3]\color{red}\underbar,   % User functions underlined and blue
	commentstyle=\usefont{T1}{pcr}{m}{sl}\color{MyDarkGreen}\small,
	stringstyle=\color{purple},
	showstringspaces=false,                 % Don't put marks in string spaces
	tabsize=3,                              % 5 spaces per tab
	morekeywords={xlim,ylim,var,alpha,factorial,poissrnd,normpdf,normcdf},
	morecomment=[l][\color{blue}]{...},
	breaklines=true,
	breakatwhitespace=true,
	emptylines=1,
	mathescape=true,
	xleftmargin=0ex,
	emphstyle=\bfseries\color{red}
}





%%%%%%%%%%% MACROS NAMES
\newcommand{\lecturername}{Martin Licht}
% \newcommand{\assistantnamea}{Jochen Hinz}
% \newcommand{\assistantnameb}{Ivan Bioli}
\newcommand{\semestername}{Winter Semester 2023}
\newcommand{\lecturename}{Analysis III - 202(c)}
\DeclarePairedDelimiter\floor{\lfloor}{\rfloor}

%%%%%%%%%%% HEADER
\newdateformat{yeardate}{\THEYEAR}
\newcommand{\exsheet}[3] % input is the number of the session and the day TODO What's that
{\clearpage

	\begin{center}
		{\Large \textbf{\lecturename}}\\[2ex]
		\semestername
	\end{center}

	% \vspace{2ex}
	% \lecturername

	\vspace{2ex}
	{\Large Session #1: #3\,#2, \yeardate\today}
	%\hfill
	%{\Large EPF Lausanne}

	\hrulefill
}





\usepackage{comment}

\newtheorem{exercise}{Exercise}
\newtheorem{solutionenv}{Solution}

\newboolean{hide_solution}
\ifx\hidesolutions\undefined
\newenvironment{solution}{\begin{solutionenv}}{\end{solutionenv}}
\setboolean{hide_solution}{false}
\else
\excludecomment{solution}
\setboolean{hide_solution}{true}
\fi

\newcommand{\ifnotsolution}[1]{\ifthenelse{\boolean{hide_solution}}{#1}{}}
\newcommand{\ifsolution}[1]{\ifthenelse{\boolean{hide_solution}}{}{#1}}








\allowdisplaybreaks

\begin{document}
\exsheet{5}{10}{October} % parameters are the number of the session and the day





\begin{exercise}
    Consider a curve $u$ in one-dimensional space with
    \begin{align*}
        u : [1,2] \rightarrow \bbR, \qquad t \mapsto t^2 + t
    \end{align*}
    Verify that the curve is simple, differentiable, and regular. Compute the curve integral $\int_u f \;dl$, where 
    \begin{align*}
        f : \bbR \rightarrow \bbR, \qquad x \rightarrow 3x^3
    \end{align*}
    is the scalar field.
\end{exercise}
\begin{solution}
    This is just a fancy way of describing integration by substitution from Analysis I.
    We first verify that the curve is simple, differentiable, and regular.
    \begin{itemize}
        \item 
        The curve is simple if it does not intersect itself. This means that $u(t_1) \neq u(t_2)$ for $t_1 \neq t_2$. 
        In this case, the curve $u(t) = t^2 + t$ is a strictly increasing function on the interval $[1,2]$.
        We conclude that the curve is simple. 
        \item 
        We see that the curve is differentiable because its only component is differentiable. 
        \item 
        The curve is regular because the derivative $\dot u(t) = 2t + 1$ is not zero over the interval $(1,2)$. 
    \end{itemize}
    The curve integral of $f$ over $\Gamma$ is given by
    \[
        \int_u f \, dl = \int_1^2 f(u(t)) \, |\dot u(t)| \, dt 
        = 
        \int_1^2 3(u(t))^3 \, |2t+1| \, dt 
        = 
        \int_1^2 3(t^2 + t)^3 \, |2t+1| \, dt 
        .
    \]
    We have $|2t+1| = 2t+1$ over the interval $[1,2]$. Hence
    \[
        \int_u f \, dl = 3 \int_1^2 (t^2 + t)^3 (2t + 1) \, dt.
    \]
    This integral can be computed using standard methods of integration. 
    One straight-forward but technical solution is to just expand the polynomial that we integrate. 
	However, a simpler method uses substitution:
    \begin{align}
        \int_u f \, dl 
        = 
        \frac 3 4 \int_1^2 4(t^2 + t)^3 (2t + 1) \, dt
        = 
        \frac 3 4 \int_1^2 \partial_t (t^2 + t)^4 \, dt
        = 
        \frac 3 4 (t^2 + t)^4 |_{t=1}^{t=2}
        = 
        \frac 3 4 \left( 6^4 - 2^4 \right)
        .
    \end{align}
    This simplifies to 
    \begin{align}
        \int_u f \, dl 
        = 
        \frac 3 4 \cdot 2^4 \left( 3^4 - 1 \right) = \frac 3 4 \cdot 16 \cdot 80 = 3 \cdot 320 = 960
        .
    \end{align}
\end{solution}


\begin{exercise}[vector analysis in 1D]
    Let $\Omega \subseteq (a,b)$ be an open interval in one-dimensional space.
    \begin{itemize}
     \item Explain why there cannot be a simple closed continuous curve in $\Omega$.
     \item When $\Omega = (-10,10)$, compute the integral of the scalar field 
     \begin{gather*}
        f(x) = \frac{x}{\sqrt{1+x^2}}
     \end{gather*}
     along the curves 
     \begin{gather*}
        \gamma_1 : [0,1] \rightarrow \Omega, \quad t \mapsto (2t - 1),
        \\
        \gamma_2 : [-1,1] \rightarrow \Omega, \quad t \mapsto (t),
        \\
        \gamma_3 : [0,1] \rightarrow \Omega, \quad t \mapsto (1-2t),
        \\
        \gamma_4 : [0,1] \rightarrow \Omega, \quad t \mapsto (-1 + 2t^5),
    \end{gather*}
    Compute the tangent vectors $\dot\gamma(t)$.
    \item 
    Compute the integral of the vector field 
    \begin{align}
        F(x) = \left( x e^{x^2} \right)
    \end{align}
    along the curve $\gamma_4$. Find a potential for this vector field, and write down the general form of all potentials.
    \end{itemize}
\end{exercise}
\begin{solution}
    This is content of Analysis 1 but repackaged in the manner of vector analysis. 
    \begin{itemize}
        \item It is visually clear that any closed continuous curve in $\Omega$ would have to intersect itself.
		Formally, one can use the intermediate value theorem.
        \item Before we compute all the line integrals, we note that the curves $\gamma_1, \gamma_3$, and $\gamma_4$ map the interval $[0, 1]$ to the interval $[-1, 1]$.
        Hence, we can solve the corresponding line integrals by a change of variables in the line integral for $\gamma_2$.
        Furthermore, we note that $\gamma_3$ is the reparameterization of $\gamma_1$ in the opposite direction. Hence, we conclude from Exercise 5 on Exercise Sheet 4,
        that the corresponding line integrals must be equal. Let us now start the computations with $\gamma_2$:
        \begin{align}
            \int_{\gamma_2} f \;dl 
            &= 
            \int_{-1}^1 f(t) \;dt 
            = 
            \int_{-1}^1 \frac{t}{\sqrt{1+t^2}} \;dt 
            = 0,
        \end{align}
        since the integrand is odd around $t = 0$. For $\gamma_1$, we have
        \begin{align}
            \int_{\gamma_1} f \;dl 
            &= 
            \int_{0}^1 f(\gamma_1(t)) |\dot \gamma_1(t)| \;dt 
            = 
            2\int_{0}^1 f(2t - 1) \;dt 
            = \int_{-1}^1 f(s) \;ds = 0 = \int_{\gamma_2} f \;dl = 0.
        \end{align}
        As already pointed out previously, we must have $\int_{\gamma_3} f \;dl = \int_{\gamma_1} f \;dl = 0$. 
        Finally, we find for $\gamma_4$ that $\dot \gamma_4(t) = 10t^4 \geq 0$ for $t \in [0,1]$
        and thus
        \begin{align}
            \int_{\gamma_4} f \;dl 
            = 
            \int_{0}^1 f(\gamma_4(t)) \dot \gamma_4(t) \;dt 
            = 
            \int_{-1}^1 f(u) \;du = \int_{\gamma_2} f \;dl = 0,
        \end{align}
        where we have used the substitution $u = \gamma_4(t) = -1 + 2t^5$ with $du = \dot \gamma_4(t) \;dt$.
        \item The general form of a potential for $F$ is $f(x) = \frac 1 2 e^{x^2} + C$, where $C$ is an arbitrary constant. We set
        $C = -1/2$ such that $f(0) = 0$. For the curve integral of $F$ along $\gamma_4$, we thus find that
        \begin{align}
            \int_{\gamma_4} F dl = f(\gamma_4(1)) - f(\gamma_4(0)) = f(1) - f(-1) = \frac 1 2 (e - e^{(-1)^2}) = 0.
        \end{align}
    \end{itemize}
\end{solution}

\begin{exercise}
    We review notions of potentials and conservative vector fields. Let $\Omega \subseteq \bbR^n$ be open.
    Suppose we have a vector field $F = (F_1,\dots,F_n) \in C^1(\Omega,\bbR^n)$.
	Recall that we have introduced the condition
    \begin{align}
    \label{eq:conservative}
        \partial_i F_j = \partial_j F_i, \qquad 1 \leq i,j \leq n.
    \end{align}
    \begin{itemize}
     \item Suppose that $n=2$. Show that $F$ satisfies \eqref{eq:conservative} if and only if it is curl-free: $\operatorname{curl} F = 0$.
     \item Suppose that $n=3$. Show that $F$ satisfies \eqref{eq:conservative} if and only if it is curl-free: $\operatorname{curl} F = 0$.
     \item Suppose that $n=1$. Show that $F$ satisfies \eqref{eq:conservative}.
     \item Suppose that $F$ admits a potential $f \in C^1(\Omega,\bbR)$, so that $\nabla f = F$. 
     Show that if $\gamma : [a,b] \rightarrow \Omega$ is a simple regular curve, then 
     \begin{align}
        \int_\gamma F \;dl = f(\gamma(b)) - f(\gamma(a)).
     \end{align}
     Show that if $\gamma$ is closed, then 
     \begin{align}
        \int_\gamma F \;dl = 0.
     \end{align}
    \end{itemize}
\end{exercise}
\begin{solution}
\begin{itemize}
    \item We recall that the curl of a vector field $F = (F_1,F_2)$ is given by
    \begin{align}
        \operatorname{curl} F = \left( \partial_1 F_2 - \partial_2 F_1 \right).
    \end{align}
    Hence, it is clear that $\operatorname{curl} F = 0$ if and only if $F$ satisfies \eqref{eq:conservative}.
    \item We recall that the curl of a vector field $F = (F_1,F_2,F_3)$ is given by
    \begin{align}
        \operatorname{curl} F = 
        \begin{pmatrix}
            \partial_2 F_3 - \partial_3 F_2
            \\
            \partial_3 F_1 - \partial_1 F_3
            \\
            \partial_1 F_2 - \partial_2 F_1
        \end{pmatrix}.
    \end{align}
    Hence, it is clear that $\operatorname{curl} F = 0$ if and only if $F$ satisfies \eqref{eq:conservative}. 
    Also, recall Exercise 6 from Exercise Sheet 3. On the one hand, we noticed there that the curl of a vector field is given by the off-diagonal entries of the
    anti-symmetric part of the Jacobian matrix. On the other hand, condition \eqref{eq:conservative} is equivalent to the statement that the Jacobian matrix of $F$ is symmetric.
    We therefore conclude that the curl vanishes if and only if the Jacobian matrix is symmetric.
    \item For $n=1$, we have $F = (F_1)$ and the condition \eqref{eq:conservative} is trivially satisfied.
    \item We have $\nabla f = F$. Hence, we can write the line integral as
    \begin{align}
        \int_\gamma F \;dl &= \int_a^b F(\gamma(t)) \cdot \dot \gamma(t) \;dt \\
        &= \int_a^b \nabla f(\gamma(t)) \cdot \dot \gamma(t) \;dt = \int_a^b \frac{d}{dt} f(\gamma(t)) \;dt = f(\gamma(b)) - f(\gamma(a)).
    \end{align}
    If $\gamma$ is closed, then $\gamma(a) = \gamma(b)$ and we find that the line integral vanishes.
\end{itemize}
\end{solution}







\begin{exercise}
    We introduce the following curves:
    \begin{gather*}
        \gamma : [0,1] \rightarrow \bbR^{3}, \qquad t \mapsto \left( 3, t^2, 4t \right),
        \\
        \delta : [1,\infty) \rightarrow \bbR^{2}, \qquad t \mapsto \left( 5, e^{-t} \right)
    \end{gather*}
    For each curve
    \begin{itemize}
        \item compute the tangent vector
        \item compute the speed of the curve
        \item find the unit tangent vector
        \item for $\delta$, find the unit normal along the curve that is the 90 degree clockwise rotation of unit tangent
        \item argue why it is a regular curve
        \item and compute the length of the curve.
    \end{itemize}
\end{exercise}
\begin{solution}
    \begin{itemize}
        \item We compute the tangent vectors:
        \begin{align}
            \dot \gamma(t) = \begin{pmatrix} 0 \\ 2t \\ 4 \end{pmatrix}, \quad \dot \delta(t) = \begin{pmatrix} 0 \\ -e^{-t} \end{pmatrix}.
        \end{align}
        \item The speed of the curve is given by the norm of the tangent vector:
        \begin{align}
            |\dot \gamma(t)| = \sqrt{4t^2 + 16} = 2 \sqrt{t^2 + 4}, \quad |\dot \delta(t)| = e^{-t}.
        \end{align}
        \item The unit tangent vector is given by
        \begin{align}
            \hat t_\gamma(t) &= \frac{\dot \gamma(t)}{|\dot \gamma(t)|}  = \frac{1}{\sqrt{t^2 + 4}} \begin{pmatrix} 0 \\ t \\ 2 \end{pmatrix},\\
            \hat t_\delta(t) &= \frac{\dot \delta(t)}{|\dot \delta(t)|} = \begin{pmatrix} 0 \\ -1 \end{pmatrix}.
        \end{align}
        Note that the unit tangent vector of $\delta$ is constant!
        \item The unit normal vector is given by the 90 degree clockwise rotation of the unit tangent vector. The corresponding rotation matrix is
        \begin{align}
            R = \begin{pmatrix} 0 & 1 \\ -1 & 0 \end{pmatrix}.
        \end{align}
        Hence, we find
        \begin{align}
            \hat n_\delta(t) = R \hat t_\delta(t) = \begin{pmatrix} -1 \\ 0 \end{pmatrix}.
        \end{align}
        \item The curves are regular because their tangent vectors never vanish.
        \item The length of the curves is given by the integral of the speed: For $\gamma$, we find
        \begin{align}
            \int_0^1 |\dot \gamma(t)| dt &= 2 \int_0^1 \sqrt{t^2 + 4} dt = 4 \int_{0}^1 \sqrt{(t/2)^2 + 1} dt\\
            &= 8 \int_0^{1/2} \sqrt{u^2 + 1} du =  4(\sqrt{2} + \sinh^{-1}(1/2)),\\
        \end{align}
        where we have used the substitution $u = t/2$ with $du = dt/2$. The last integral can be found with \emph{trigonometric substitution}.
        
        For $\delta$, we find
        \begin{align}
            \int_1^\infty |\dot \delta(t)| dt = \int_1^\infty e^{-t} dt = e^{-1}.
        \end{align}
        Note that the curve is parametrized for $t \in [1,\infty)$, but still has a finite length!
    \end{itemize}
\end{solution}



\begin{exercise}
	We consider the vector field 
	\[
		F : \bbR^2 \rightarrow \bbR^2, \quad (x,y) \mapsto \left( x^3, y^3 \right)
	\]
	We want to find a potential over the domain $\Omega = \bbR^{2}$. 
	Fix a constant of integration at $(0,0)$ and define a potential via the integral of the vector field $F$ 
	along a simple regular curve going from $(0,0)$ to $(x,y)$.
\end{exercise}
\begin{solution}
	The most simple among the simple regular curves from $(0,0)$ to $(x,y)$ is the straight line:
	\begin{gather*}
        \gamma : [0,1] \rightarrow \bbR^{2}, \qquad t \mapsto \left( tx, ty \right).
    \end{gather*}
	We compute that $\dot\gamma(t) = (x,y)$. 
	We fix some constant of integration $f(0,0) = C$ for our yet-to-be-found potential $f \in C^1(\bbR^2,\bbR)$.
	For any $(x,y) \in \bbR^2$, we now compute 
    \begin{align}
        &
        f(x,y) - f(0,0)
		=
		\int_{0}^{1} \left( t^{3}x^3, t^{3}y^{3} \right) \cdot \left( x,y \right) \;dt
        =
        \left( x^{4} + y^{4} \right) \int_{0}^{1} t^{3} \;dt
        =
        \frac 1 4 \left( x^{4} + y^{4} \right) 
        .
    \end{align}
	Therefore,
	\begin{align}
        f(x,y) = \frac 1 4 \left( x^{4} + y^{4} \right) + C
		.
    \end{align}
	Indeed, one easily verifies that $\nabla f = F$.
\end{solution}


\begin{exercise}
    The closed curve 
    \begin{align*}
        \gamma(t) = \left( \sin(t)(1+0.5 \sin(2t)), \cos(t)(1+0.5 \sin(2t)) \right)
    \end{align*}
    encircles a domain $\Omega$ in counterclockwise direction. 
	Find the tangent $\dot\gamma(t)$, the unit tangent $\tau(t)$ and the outward pointing unit normal $\vec n(t)$.
    Only simplify as much as reasonable. 
\end{exercise}
\begin{solution}
    We calculate:
    \begin{align*}
        \dot\gamma(t) 
        = \left( 
            \cos(t)( 1+0.5 \sin(2t) ) + \sin(t)\cos(2t)
            ,
            -\sin(t)( 1+0.5 \sin(2t) ) + \cos(t)\cos(2t)
        \right)
    \end{align*}
    With that:
    \begin{align*}
        |\dot\gamma(t)|^{2}
        &= 
        \cos(t)^{2}( 1+0.5 \sin(2t) )^{2} + \sin(t)^{2}\cos(2t)^{2} + 2 \cos(t)( 1+0.5 \sin(2t) ) \sin(t)\cos(2t)
        \\&\quad \quad 
        +
        \sin(t)^{2}( 1+0.5 \sin(2t) )^{2} + \cos(t)^{2}\cos(2t)^{2} - 2\sin(t)( 1+0.5 \sin(2t) ) \cos(t)\cos(2t)
        \\&= 
        ( 1+0.5 \sin(2t) )^{2} + \cos(2t)^{2} + 2 \cos(t)( 1+0.5 \sin(2t) ) \sin(t)\cos(2t)
        \\&\quad \quad 
        - 2 \sin(t)( 1+0.5 \sin(2t) ) \cos(t)\cos(2t)
        \\&= 
        ( 1+0.5 \sin(2t) )^{2} + \cos(2t)^{2}
        .
    \end{align*}
    Hence 
    \begin{align*}
        |\dot\gamma(t)|
        &= 
        \sqrt{ ( 1+0.5 \sin(2t) )^{2} + \cos(2t)^{2} }
        .
    \end{align*}
    We want to compute the tangent vector:
    \begin{align*}
        \tau(t) 
        = 
        \left( 
            \frac{\cos(t)( 1+0.5 \sin(2t) ) + \sin(t)\cos(2t)}{\sqrt{ ( 1+0.5 \sin(2t) )^{2} + \cos(2t)^{2} }}
            , 
            \frac{-\sin(t)( 1+0.5 \sin(2t) ) + \cos(t)\cos(2t)}{\sqrt{ ( 1+0.5 \sin(2t) )^{2} + \cos(2t)^{2} }} 
        \right)
    \end{align*}
    Accordingly, the normal vector is:
    \begin{align*}
        \vec n(t) 
        = 
        \left( 
            \frac{-\sin(t)( 1+0.5 \sin(2t) ) + \cos(t)\cos(2t)}{\sqrt{ ( 1+0.5 \sin(2t) )^{2} + \cos(2t)^{2} }} 
            , 
            - \frac{\cos(t)( 1+0.5 \sin(2t) ) + \sin(t)\cos(2t)}{\sqrt{ ( 1+0.5 \sin(2t) )^{2} + \cos(2t)^{2} }}
        \right)
    \end{align*}
    

    \begin{tikzpicture}
        \begin{axis}[
            domain=0:360,
            xmin=-2,
            xmax=2,
            ymin=-2,
            ymax=2,
            xlabel={x},
            ylabel={y}
        ]
        \addplot[blue, variable=\t, samples=100] ({sin(\t)*( 1 + 0.5 * sin(2*\t) )}, {cos(\t)*( 1 + 0.5 * sin(2*\t) )});
        \end{axis}
    \end{tikzpicture}
\end{solution}














\begin{exercise}
    We work over the quadratic domain 
    \begin{gather*}
        \Omega := \left\{ (x_1,x_2) \in \bbR^{2} \suchthat* -1 < x_1 < 1, \; -1 < x_2 < 1 \right\}.
    \end{gather*}
    Compute the integral $\iint_\Omega \divergence \vec F \;dx_1dx_2$, where 
    \begin{gather*}
        \vec F(x_1,x_2) = \left( \sin(x_1) x_2, \left( x_1^2 + x_2 \right)^5 \right)
    \end{gather*}
\end{exercise}
\begin{solution}
    % Express this volume integral as a curve integral.
    We make use of the Divergence theorem to express the volume integral as a curve integral.
    \begin{align*}
        \int \int_{\Omega} \nabla \cdot \vec{F} \;d x_1 \:d x_2 
        =
        &
        \int_{\partial \Omega}\vec{F}\cdot\vec{n} \;d\ell,
        \\&=
        \int_{-1}^1 \vec{F}(x_1,-1)\cdot\begin{pmatrix}0\\-1\end{pmatrix} \;dx_1 + \int_{-1}^1 \vec{F}(x_1,1)\cdot\begin{pmatrix}0\\1\end{pmatrix} \;dx_1,
        \\
        & + \int_{-1}^1 \vec{F}(-1,x_2)\cdot\begin{pmatrix}-1\\0\end{pmatrix} \;dx_2 + \int_{-1}^1 \vec{F}(1,x_2)\cdot\begin{pmatrix}1\\0\end{pmatrix} \;dx_2,s
        \\&=
        \int_{-1}^1 (1-x_1^ 2)^ 5 \;dx_1 + \int_{-1}^1 (x_1^2 + 1)^ 5 \;dx_1 + \int_{-1}^1 \sin(1)x_2\;dx_2 +  \int_{-1}^1 -\sin(-1)x_2\;dx_2,\\
    \end{align*}
    Since $x_2$ is an odd function, so the last two integrals are 0. To evaluate the first two integrals,
    we can either compute the quintic powers manually, which is a lot of work, or we utilize the binomial theorem:
    \begin{gather*}
        (x+y)^n=\sum_{k=0}^n\left(\begin{array}{l}
        n \\
        k
        \end{array}\right) x^{n-k} y^k
    \end{gather*}
    With that, we can simplify the calculation as follows: 
    \begin{align*}
        \int \int_{\Omega} \nabla \cdot \vec{F} \;d x_1 \:d x_2 
        &=
        \int_{\partial \Omega}\vec{F}\cdot\vec{n} \;d\ell,
        \\&=
        \int_{-1}^1 (1-x_1^2)^5 \;dx_1 + \int_{-1}^1 (x_1^2 + 1)^ 5 \;dx_1
        \\&=
        \sum_{k = 0}^5 \binom{5}{k} (-1)^{k} \int_{-1}^1 (x)^{2k} \;dx_1 + \sum_{k = 0}^5 \binom{5}{k}\int_{-1}^1 (x)^{2k} \;dx_1
        \\&=
        \sum_{k = 0}^5 \binom{5}{k} (-1)^{k} \left[\frac{x^{2k+1}}{2k+1}\right]_{-1}^1 + \sum_{k = 0}^5 \binom{5}{k} \left[\frac{x^{2k+1}}{2k+1}\right]_{-1}^1
        \\&=
        \sum_{k = 0}^5 \binom{5}{k} \left( (-1)^{k} + 1 \right) 
        \left[\frac{x^{2k+1}}{2k+1}\right]_{-1}^1
    \end{align*}
    For $k = 1, 3, 5$, we get $\left( (-1)^{k} + 1 \right) = 0$. 
    So we only need the terms with $k = 0, 2, 4$.
    Thus
    \begin{align*}
        &
        \sum_{k = 0}^5 \binom{5}{k} \left( (-1)^{k} + 1 \right) 
        \left[\frac{x^{2k+1}}{2k+1}\right]_{-1}^1
        \\&=
        2
        \binom{5}{0}
        \left[\frac{x^{0+1}}{0+1}\right]_{-1}^1
        +
        2
        \binom{5}{2}
        \left[\frac{x^{4+1}}{4+1}\right]_{-1}^1
        +
        2
        \binom{5}{4}
        \left[\frac{x^{8+1}}{8+1}\right]_{-1}^1
        \\&=
        2
        \binom{5}{0}
        \left[\frac{x^{1}}{1}\right]_{-1}^1
        +
        2
        \binom{5}{2}
        \left[\frac{x^{5}}{5}\right]_{-1}^1
        +
        2
        \binom{5}{4}
        \left[\frac{x^{9}}{9}\right]_{-1}^1
        \\&=
        4
        \binom{5}{0}
        +
        \frac 4 5
        \binom{5}{2}
        +
        \frac 4 9
        \binom{5}{4}
        \\&=
        \frac{128}{9}
        .
    \end{align*}
\end{solution}

















\end{document}
