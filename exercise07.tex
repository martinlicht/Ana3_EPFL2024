\documentclass[11pt]{article}
% % \def\hidesolutions{}
%%%%%%%%%%% SET MARGINS
\setlength{\textheight}{20cm}
\setlength{\topmargin}{-0.5cm}
\setlength{\oddsidemargin}{+0cm}
\setlength{\textwidth}{16.3cm}
%\setlength{\parskip}{6pt}
\setlength{\parindent}{0pt}

%%%%%%%%%%% PACKAGES
\usepackage{amsmath}
\usepackage{amssymb}
\usepackage{amsfonts}
%\usepackage{a4wide}
\usepackage{graphicx}
\usepackage{color}
\usepackage[normalem]{ulem}
\usepackage{enumitem}
\usepackage{capt-of}
\usepackage{float}
\usepackage{amsmath}
\usepackage{listings}
\definecolor{mygreen}{RGB}{28,172,0} % color values Red, Green, Blue
\definecolor{mylilas}{RGB}{170,55,241}
\usepackage{empheq}
\usepackage[ruled]{algorithm2e}
\usepackage{mathrsfs}
\usepackage{datetime}
\usepackage{subcaption}

% TODO: combine the two package lists and reduce redundancies 
\usepackage{mathtools}
\usepackage{nicefrac}
\usepackage{hyperref}
\usepackage{url}
\usepackage{amsmath,amssymb,amsfonts}
\usepackage{a4wide}
\usepackage{graphicx}
\usepackage{color}
\usepackage[normalem]{ulem}
\usepackage{capt-of}
\usepackage{float}
\usepackage[ruled]{algorithm2e}
\usepackage{amsmath,amssymb,amsfonts}
\usepackage{a4wide}
\usepackage{graphicx}
\usepackage{color}
\usepackage[normalem]{ulem}
\usepackage{capt-of}
\usepackage{float}
\usepackage[ruled]{algorithm2e}
\usepackage{mathrsfs}







\newcommand{\Lc}[2]{{\color{blue} \sout{#1} } \textcolor{red}{#2}}
\newcommand{\La}[1]{\textcolor{red}{#1}}
\newcommand{\lh}{\mathscr{L}_h}
\newcommand{\cl}{\mathscr{L}}
\newcommand{\cf}{\mathscr{F}}
\newcommand{\dx}{dx}
\newcommand{\ltn}{\mathscr{l}^2}
\newcommand{\bbR}{\mathbb{R}}
\newcommand{\Rset}{\mathbb{R}}
\newcommand{\Nset}{\mathbb{N}}
\newcommand{\scL}{\mathcal{L}}
\newcommand{\xx}{\mathbf{x}}
\newcommand{\norm}[1]{\|{#1}\|}
\newcommand{\yy}{\mathbf{y}}
\newcommand{\at}[1]{\big|_{#1}}
\renewcommand{\div}{\mathrm{div}}
\newcommand{\divergence}{\mathrm{div}}
\newcommand{\cp}[1]{\textcolor{blue}{#1}}

\newcommand{\FF}{\texttt{FreeFem++ }}
\newcommand{\FFns}{\texttt{FreeFem++}}
\newcommand{\FFfull}{\texttt{FreeFem++-x11}}
\newcommand{\cmd}[1]{ \medskip \noindent \texttt{#1} \medskip}
\newcommand{\incmd}[1]{\texttt{#1}}
\newcommand{\shrinkitems}{\addtolength{\itemsep}{-0.5\baselineskip}}
\newcommand{\mtt}[1]{\mathtt{#1}}
\newcommand{\ML}{\texttt{Matlab }}

\newcommand{\bb}{\mathbf{b}}
\newcommand{\nn}{\mathbf{n}}
\newcommand{\vecA}{\vec{A}}
\newcommand{\vecB}{\vec{B}}


\newcommand{\mesh}{\mathcal{T}_h}
\newcommand{\refel}{\widehat{K}}
\newcommand{\ver}{\mathbf{a}}
\newcommand{\refver}{\widehat{\mathbf{a}}}
\newcommand{\grad}{\nabla}
\newcommand{\refgrad}{\widehat{\nabla}}
\newcommand{\refu}{\widehat{u}}
\newcommand{\refbasis}{\widehat{\varphi}}
\newcommand{\refxx}{\widehat{\xx}}
\newcommand{\refx}{\widehat{x}}
\newcommand{\refy}{\widehat{y}}
\newcommand{\refrho}{\widehat{\rho}}
\newcommand{\refh}{\widehat{h}}






% For typesetting Python code
\newcommand{\matlab}{{\sc Matlab}\xspace}
\usepackage{listings}
\lstloadlanguages{Python}
\lstloadlanguages{csh}%
\definecolor{MyDarkGreen}{rgb}{0.0,0.4,0.0}
\definecolor{purple}{rgb}{0.58,0,0.82}
\lstset{language=Python,                    % Use Python
	%frame=single,                          % Single frame around code
	basicstyle=\ttfamily\footnotesize\color{black},
	keywordstyle=[1]\color{blue}\bf,        % Python functions bold and blue
	keywordstyle=[2]\color{purple},         % Python function arguments purple
	keywordstyle=[3]\color{red}\underbar,   % User functions underlined and blue
	commentstyle=\usefont{T1}{pcr}{m}{sl}\color{MyDarkGreen}\small,
	stringstyle=\color{purple},
	showstringspaces=false,                 % Don't put marks in string spaces
	tabsize=3,                              % 5 spaces per tab
	morekeywords={xlim,ylim,var,alpha,factorial,poissrnd,normpdf,normcdf},
	morecomment=[l][\color{blue}]{...},
	breaklines=true,
	breakatwhitespace=true,
	emptylines=1,
	mathescape=true,
	xleftmargin=0ex,
	emphstyle=\bfseries\color{red}
}





%%%%%%%%%%% MACROS NAMES
\newcommand{\lecturername}{Martin Licht}
% \newcommand{\assistantnamea}{Jochen Hinz}
% \newcommand{\assistantnameb}{Ivan Bioli}
\newcommand{\semestername}{Winter Semester 2023}
\newcommand{\lecturename}{Analysis III - 202(c)}
\DeclarePairedDelimiter\floor{\lfloor}{\rfloor}

%%%%%%%%%%% HEADER
\newdateformat{yeardate}{\THEYEAR}
\newcommand{\exsheet}[3] % input is the number of the session and the day TODO What's that
{\clearpage

	\begin{center}
		{\Large \textbf{\lecturename}}\\[2ex]
		\semestername
	\end{center}

	% \vspace{2ex}
	% \lecturername

	\vspace{2ex}
	{\Large Session #1: #3\,#2, \yeardate\today}
	%\hfill
	%{\Large EPF Lausanne}

	\hrulefill
}





\usepackage{comment}

\newtheorem{exercise}{Exercise}
\newtheorem{solutionenv}{Solution}

\newboolean{hide_solution}
\ifx\hidesolutions\undefined
\newenvironment{solution}{\begin{solutionenv}}{\end{solutionenv}}
\setboolean{hide_solution}{false}
\else
\excludecomment{solution}
\setboolean{hide_solution}{true}
\fi

\newcommand{\ifnotsolution}[1]{\ifthenelse{\boolean{hide_solution}}{#1}{}}
\newcommand{\ifsolution}[1]{\ifthenelse{\boolean{hide_solution}}{}{#1}}








\allowdisplaybreaks

\begin{document}
\exsheet{7}{31}{October} % parameters are the number of the session and the day





















    








\begin{exercise}
    Verify the divergence theorem for the following vector field $\vec F$ and volume $V$:
    \begin{gather*}
        \vec F(x_1,x_2,x_3) := ( x_1 x_3, x_2, x_2 ),
        \qquad 
        V := \left\{\; (x_1,x_2,x_3) \in \mathbb R^3 \suchthat* x_1^2 + x_2^2 + x_3^2 < 1 \;\right\}
        .
    \end{gather*}
    Note that $V$ is just the three-dimensional unit ball. 
\end{exercise}
\begin{solution}    
    We have to show that the followig identity holds: 
    \[
    \iiint_{V}\nabla \cdot \vec{F}\;dV =  \oint \vec{F}\cdot \vec{n} \;d(\partial V)
    \] 
    We start with the volume integral:
    \begin{align*}
        &
        \iiint_{V}\nabla \cdot \vec{F}\;dV
        \\&
        =
        \iiint_{V}x_3 + 1\;dV
        \\&
        =
        \int_{0}^{2\pi}\int_0^{\pi}\int_0^ 1 (r\cos\phi + 1)r^ 2\sin\phi\;d r\;d\phi\;d\theta
        \\&
        =
        2\pi\int_0^1 r^3\;d r\int_0^{\pi}\cos\phi\sin\phi\;d \phi + 2\pi\int_0^1 r^2\;d r\int_0^{\pi}\sin\phi\;d \phi 
        \\&
        =
        2\pi\left[ \frac{1}{4}r^ 4 \right]_0^ {1} \left[-\frac{1}{4}\cos2\phi\right]_0^{\pi} + 2\pi\left[ \frac{1}{3}r^ 3 \right]_0^ {1} \left[-\cos\phi\right]_0^{\pi}
        \\&
        =
        2\pi\left(\frac{1}{4} - 0\right)\left(\frac{1}{4} - -\frac{1}{4}\right) + 2\pi\left(\frac{1}{3} - 0\right)\left(1 - - 1\right) = \frac{4\pi}{3}
        .
    \end{align*}
    Next we consider the surface integral. We need a parameterisation for the surface. One can consider:
    \[
    \Phi: [0,2\pi) \times (0,\pi) \mapsto S: (\theta,\phi)\mapsto (\cos\theta \sin\phi, \sin\theta\sin\phi,\cos\phi)
    \]
    with 
    \[
    \partial_{\theta}\Phi\times \partial_{\phi}\Phi  = \begin{pmatrix} -\sin\theta\sin\phi\\ \cos\theta\sin\phi \\0 \end{pmatrix} \times \begin{pmatrix} -\cos\theta\cos\phi\\ \sin\theta\cos\phi \\-\sin\phi \end{pmatrix} = -\begin{pmatrix} \cos\theta\sin^2\phi\\ \sin\theta\sin^2\phi \\\sin\phi\cos\phi \end{pmatrix} 
    \]
    Note this is inward pointing. Therefore we need an extra minus sign.
    \begin{align*}
        &
        \oint \vec{F}\cdot \vec{n} \;dS
        \\&
        =
        \int_{0}^{2\pi}\int_0^{\pi} \begin{pmatrix}\cos\theta\sin\phi\cos\phi\\ \sin\theta\sin\phi \\ \sin\theta\sin\phi \end{pmatrix}\cdot\begin{pmatrix} \cos\theta\sin^2\phi\\ \sin\theta\sin^2\phi \\\sin\phi\cos\phi \end{pmatrix} \;d\phi\;d\theta
        \\&
        =
        \int_{0}^{2\pi}\cos^2\theta\;d\theta\int_0^{\pi}\sin^3\phi\cos\phi\;d\phi + \int_{0}^{2\pi}\sin^2\theta\;d\theta\int_0^{\pi}\sin^3\phi\;d\phi +  \int_{0}^{2\pi}\sin\theta\;d\theta\int_0^{\pi}\sin^2\phi\cos\phi\;d\phi
        .
    \end{align*}
    The last term is zero due to the fact that the integral of $\sin\theta$ over 1 period is equal to $0$. 
    The other integrals are evaluated seperately, where we use the substitution $u = \cos\phi$ several times:
    \begin{align*}
        &
        \int_0^{\pi}\sin^3\phi\cos\phi\;d\phi 
        \\&
        = \int_0^{\pi}(1 - \cos^2\phi)\sin\phi\cos\phi\;d\phi 
        \\&
        = \int_0^{\pi}\cos\phi\sin\phi - \cos^3\phi\sin\phi\;d\phi 	
		\\&
        = \int_{0}^{\pi} \frac 1 2 \left[ \cos^2(\phi) \right]' + \frac 1 4 \left[ \cos^4(\phi) \right]' \;d u  
        \\&
        = \frac 1 2 \left[ \cos^2(\phi) \right]_{\phi=0}^{\phi=\pi} + \frac 1 4 \left[ \cos^4(\phi) \right]_{\phi=0}^{\phi=\pi} = 0
			% \\&
			% = \int_{-1}^1 u + u^3 \;d u  
			% \\&
			% = \left[\frac{1}{2} u^2 +\frac{1}{4}u^4\right]_{-1}^{1} = 0
    \end{align*}
    Now, 
    \begin{align*}
        &
        \int_0^{2\pi}\sin^2\theta\;d\theta
        \\&
        = \int_0^{2\pi}\frac{1}{2}(1-\cos2\theta)\;d\theta 
        \\&
        = \left[\frac{1}{2} \theta -\frac{1}{4}\sin2\theta\right]_{0}^{2\pi} = \pi
    \end{align*}
    Next, 
    \begin{align*}
        &
        \int_0^{\pi}\sin^3\phi\;d\phi 
        \\&
        = \int_0^{\pi}(1 - \cos^2\phi)\sin\phi\;d\phi 
        \\&
        = \int_{-1}^1 1 - u^2 \;d u  
        \\&
        = \left[ u - \frac{1}{3}u^3\right]_{-1}^{1} = \frac{4}{3}
    \end{align*}
    If we put everything together we get:
    \begin{align*}
        &
        \oint \vec{F}\cdot \vec{n} \;d(\partial \Omega)
        \\&
        =
        \int_{0}^{2\pi}\cos^2\theta\;d\theta\int_0^{\pi}\sin^3\phi\cos\phi\;d\phi + \int_{0}^{2\pi}\sin^2\theta\;d\theta\int_0^{\pi}\sin^3\phi\;d\phi 
        \\&
        =\int_{0}^{2\pi}\cos^2\theta\;d\theta \cdot 0+ \pi\frac{4}{3} = \frac{4\pi}{3}
        .
    \end{align*}
    We conclude that both integrals give the same value thereby verifying the divergence theorem.
\end{solution}

    





\begin{exercise} % 1
    Consider the volume 
    \begin{align}
     V := \left\{ \vec x \in \bbR^3 \suchthat* x_1^2 + x_2^2 + x_3^2 < 5 \right\},
    \end{align}
    \begin{itemize}
    \item
     What is the surface $S$ of this volume?
    \item 
     Find the outward pointing unit normal $\vec n$ along the surface $S$ of this volume. Write $\vec n$ in terms of $(x_1,x_2,x_3)$ at any point on the surface $S$. 
    \item
     Find a parameterization of the surface $S$.
    \item
     Find a vector field $\vec F$ such that $\vec F \cdot \vec n = x_1^2 + x_2 + x_3$ along the surface $S$.
    \item 
     Use the divergence theorem to compute 
     \begin{align}
        \iint_{S} x_1^2 + x_2 + x_3 dx_1 dx_2 dx_3.
     \end{align}
    \end{itemize}
\end{exercise}
\begin{solution}     
    \begin{itemize}
    \item $S := \left\{ \vec x \in \bbR^3 \suchthat* x_1^2 + x_2^2 + x_3^2 = 5 \right\}$
    \item We can write $G(x_1,x_2,x_3) = -5 + x_1^2 + x_2^2 + x_3^2$. Then since we know that the gradient is normal to the level curves we have that:
    $$
    \vec{n} = \frac{\nabla G}{||\nabla G||} = \frac{1}{\sqrt{5}}\begin{pmatrix}
    x_1\\x_2\\x_3
    \end{pmatrix}
    $$
    \item A possible parameterization is:
     \begin{align*}
        \Phi: [0,2\pi)\times[0,\pi) \mapsto S, (\theta,\phi) \mapsto (\sqrt{5}\sin\phi\cos\theta,\sqrt{5}\sin\phi\sin\theta,\sqrt{5}\cos\phi)
     \end{align*}
    \item $\vec{F} = \sqrt{5}\begin{pmatrix}x_1\\1\\1\end{pmatrix}$
    \item 
     We use the divergence theorem:
     \begin{align*}
        \iint_{S} x_1^2 + x_2 + x_3 dx_1 dx_2 dx_3 & = \oint_{S} \vec{F}\cdot\vec{n} d\sigma \\
		& = \iiint_V \nabla \cdot F dV\\
		& = \iiint_V \sqrt{5} dV = \sqrt{5} \cdot \iiint_V 1 dV = \frac{4}{3}\pi\sqrt{5}^4 =  \frac{4}{3}\pi\cdot 25\\
     \end{align*}
     In the last step, we used the formula for the volume of the ball with radius $r=\sqrt{5}$.
    \end{itemize}
\end{solution}

































\begin{exercise}
    Consider the volume  
    \begin{gather*}
        V := \left\{\; (x_1,x_2,x_3) \in \mathbb R^3 \suchthat* x_1^2 + x_2^2 < x_3 < 1 \;\right\}
    \end{gather*}
    Find the the boundary $S$ of this volume and compute its surface area. 
\end{exercise}
\begin{solution} 
    The boundary of this domain consist of two surfaces:
    \begin{align*}
        S_a =\left\{ (x_1,x_2,x_3) \in \mathbb R^3 \suchthat* x_1^2 + x_2^2 < 1, x_3 = 1 \right\} 
        ,
        \\
        S_b =\left\{ (x_1,x_2,x_3) \in \mathbb R^3 \suchthat* x_1^2 + x_2^2 = x_3, 0<x_3<1 \right\} 
    \end{align*}
    We have that $\partial V = \partial \Omega_a \cup \partial V_b$. 
    To calculate the area we need a parameterisation of both surfaces:
    \begin{align*}
        \Phi_a: [0,2\pi) \times (0,1) \mapsto S_a: (\theta,r)\mapsto (r\cos\theta, r\sin\theta,1),
        \\
        \Phi_b: [0,2\pi) \times (0,1) \mapsto S_b: (\theta,z)\mapsto (\sqrt{z}\cos\theta, \sqrt{z}\sin\theta,z).
    \end{align*}
    Moreover we have that
    \[
        \left\|\partial_{\theta}\Phi_{a}\times \partial_r\Phi_{a}\right\| 
        = 
        \left\|\begin{pmatrix} -r\sin\theta \\ r\cos\theta\\0 \end{pmatrix} \times \begin{pmatrix} \cos\theta \\ \sin\theta \\0 \end{pmatrix}\right\| 
        = 
        \left\|\begin{pmatrix} 0 \\ 0 \\ -r \end{pmatrix}\right\| 
        = 
        r
    \]
    and 
    \[
        \left\|\partial_{\theta}\Phi_{b}\times \partial_z\Phi_{b}\right\|  
        = 
        \left\|\begin{pmatrix} -\sqrt{z}\sin\theta \\ \sqrt{z}\cos\theta\\0 \end{pmatrix} \times \begin{pmatrix} \frac{1}{2\sqrt{z}}\cos\theta \\  \frac{1}{2\sqrt{z}}\sin\theta \\1 \end{pmatrix}\right\| 
        = 
        \left\|\begin{pmatrix} \sqrt{z}\sin\theta \\ \sqrt{z}\cos\theta\\-\frac{1}{2} \end{pmatrix}\right\| 
        = 
        \sqrt{z+\frac{1}{4}}
    \]
    Now the area is given by:
    \begin{align*}
        \iint_{\partial\Omega} 1 \;d (\partial \Omega)
        &
        =
        \iint_{\partial\Omega_a} 1 \;dS_a + 	\iint_{\partial\Omega_b} 1 \;dS_b
        \\&
        = \int_0^{2\pi} \int_0^1 r \;d r\;d\theta + \int_0^{2\pi} \int_0^1\sqrt{z+\frac{1}{4}}\;d z \;d \theta
        \\&
        = 
        2\pi\left[\frac{1}{2}r^2\right]_0^1 + 2\pi\left[\frac{2}{3}\left(z+\frac{1}{4}\right)^{\frac{3}{2}}\right]_0^1
        \\&
        = 
        \pi + \frac{4\pi}{3}\left(\left(\frac{5}{4}\right)^{\frac{3}{2}} - \left(\frac{1}{4}\right)^{\frac{3}{2}}\right)
        = 
        \pi + \pi \frac{5\sqrt{5} - 1}{6}
        = 
        5\pi \frac{1 + \sqrt{5}}{6}
        .
    \end{align*}
\end{solution}















\begin{exercise}
    Find a regular parameterization $\Phi(s,t)$ of the surface 
    \begin{gather*}
        S := \left\{\; (x_1,x_2,x_3) \in \mathbb R^3 \suchthat* 0 < x_3 < 1, \; x_1^2 + x_2^2 = 1 + x_3^2 \;\right\}
    \end{gather*}
    Compute cross product $\partial_s \Phi(s,t) \times \partial_t \Phi(s,t)$ and its norm. 
\end{exercise}
\begin{solution}     
    We define the following parameterisation for the surface $S$:
    \[
        \Phi: [0,2\pi) \times (0,1) \mapsto S,
        \quad 
        (s,t) \mapsto \left( \sqrt{1+t^{2}}\cos s, \sqrt{1+t^{2}}\sin s, t \right)
    \]
    \[
        \partial_s \Phi(s,t) \times \partial_t \Phi(s,t) 
        = 
        \begin{pmatrix} 
            -\sqrt{1+t^2}\sin s
            \\ 
            \sqrt{1+t^2}\cos s
            \\
            0 
        \end{pmatrix} 
        \times 
        \begin{pmatrix} 
            \frac{- t}{\sqrt{1+t^2}}\cos s 
            \\ 
            \frac{- t}{\sqrt{1+t^2}}\sin s 
            \\
            1 
        \end{pmatrix} 
        = 
        \begin{pmatrix} 
            \sqrt{1+t^2}\cos s 
            \\ 
            \sqrt{1+t^2}\sin s
            \\ 
            t
        \end{pmatrix} 
    \]
    \[
        \left\|
        \partial_s \Phi(s,t) \times \partial_t \Phi(s,t)
        \right\| 
        = 
        \left\| 
        \begin{pmatrix} 
            \sqrt{1+t^2}\cos s 
            \\ 
            \sqrt{1+t^2}\sin s
            \\ 
            t
        \end{pmatrix} \right\| 
        = 
        \sqrt{1+2t^2}
    \]
\end{solution}
















\begin{exercise} % 2
    Consider the volume 
    \begin{align}
     V := \left\{ \vec x \in \bbR^3 \suchthat* x_1 + x_2 + x_3 < 1, \; x_1, x_2, x_3 > 0 \right\},
    \end{align}
    Suppose we have a vector field 
    \begin{align}
        \vec F(x_1,x_2,x_3) = \left( x_1^2 x_2, 3 x_2^{2} x_3, 9 x_3^{2} x_1 \right)
    \end{align}
    Use the divergence theorem to compute the surface integral 
    \begin{align}
        \iint_{S} \vec F d\sigma.
    \end{align}
\end{exercise}
\begin{solution}     
    Given any scalar function $g : V \to \bbR$, we have 

    \begin{align*}
        \oint_{S} \vec F d\sigma & = \iiint_V \nabla \cdot F dV\\
		&= \iiint_V 2x_1x_2 + 6x_2x_3 + 18x_3x_1 dV\\
		&= \int_{0}^{1}\int_{0}^{1 - x_3}\int_{0}^{1 - x_2 - x_3} 2x_1x_2 + 6x_2x_3 + 18x_3x_1 dx_1 dx_2 dx_3
    \end{align*}

    We solve each integral seperately. Note that we apply integration by parts in between steps. 
    \begin{align*}
        \int_{0}^{1}\int_{0}^{1 - x_3}\int_{0}^{1 - x_2 - x_3} 2x_1x_2 dx_1 dx_2 dx_3
        &= \int_{0}^{1}\int_{0}^{1 - x_3} (1-x_2-x_3)^2 x_2 dx_2 dx_3\\
        &= \int_{0}^{1}\int_{0}^{1 - x_3} \frac{1}{3}(1-x_2-x_3)^3 dx_2 dx_3\\
        &= \int_{0}^{1} \frac{1}{12}(1-x_3)^4 dx_3\\
        &= \left[ \frac{-1}{60}(1-x_3)^5\right]_0^1 = \frac{1}{60}
    \end{align*}
    \begin{align*}
        \int_{0}^{1}\int_{0}^{1 - x_3}\int_{0}^{1 - x_2 - x_3} 6x_2x_3  dx_1 dx_2 dx_3 &= \int_{0}^{1}\int_{0}^{1 - x_3} 6(1 - x_2 - x_3) x_2 x_3   dx_2 dx_3 \\
	&= \int_{0}^{1}\int_{0}^{1 - x_3} 3(1 - x_2 - x_3)^2 x_2 x_3   dx_2 dx_3 \\
	&= \int_{0}^{1} (1 - x_3)^3 x_3   dx_3 \\
	&= \int_{0}^{1} \frac{1}{4}(1 - x_3)^4  dx_3 \\
	&= \left[-\frac{1}{20}(1 - x_3)^5 \right]_0^1 = \frac{1}{20}
    \end{align*}
	\begin{align*}
         \int_{0}^{1}\int_{0}^{1 - x_3}\int_{0}^{1 - x_2 - x_3} 18x_3x_1 dx_1 dx_2 dx_3&= \int_{0}^{1}\int_{0}^{1 - x_3} 9x_3(1-x_2-x_3)^2   dx_2 dx_3 \\
        &=\int_{0}^{1} 3x_3(1-x_3)^3  dx_3 \\
        &= \int_{0}^{1} \frac{3}{4}(1-x_3)^4  dx_3 \\
        &= \left[-\frac{3}{20}(1-x_3)^5  \right]_0^1 =\frac{3}{20}
    \end{align*}
    We conclude that:
    $$
    \int_{0}^{1}\int_{0}^{1 - x_3}\int_{0}^{1 - x_2 - x_3} 2x_1x_2 + 6x_2x_3 + 18x_3x_1 dx_1 dx_2 dx_3 = \frac{1}{60} + \frac{1}{20} + \frac{3}{20} = \frac{13}{60}
    $$
\end{solution}











\begin{exercise}
    Given the curve
    \begin{gather*}
        \gamma : [0,\pi] \to \bbR^2, \quad t \mapsto ( 3\cos(2t), \sin(2t) )
    \end{gather*}
    and a function
    \begin{gather*}
        f : \bbR^2 \to \bbR, \quad (x_1,x_2) \mapsto \left( x_1^2 + 81 x_2^2 \right)^{\frac 3 2},
    \end{gather*}
    compute the integrals 
    \begin{align*}
        \int_\Gamma f \;d\ell, \quad \int_\Gamma \nabla f \;d\ell.
    \end{align*}
\end{exercise}
\begin{solution}     
    The integral of the gradient along this simple closed regular curve is zero.
    With regard to the integral of the scalar field along the curve, we have 
    \begin{gather*}
        \dot\gamma(t) = \left( -6 \sin(2t), 2\cos(2t) \right),
        \\ 
        |\dot\gamma(t)| 
        = 
        \sqrt{ 36 \sin^2(2t) + 4 \cos^2(2t) }
        = 
        2 \sqrt{ 9 \sin^2(2t) + \cos^2(2t) }
        .
    \end{gather*}
    We also see that 
    \begin{align*}
        f(\gamma(t))
        =
        \left( 9\cos^2(2t) + 81 \sin^2(2t) \right)^{\frac 3 2}
        &=
        9^{\frac 3 2 }   \left( \cos^2(2t) + 9 \sin^2(2t) \right)^{\frac 3 2}
        \\&=
        27 \left( \cos^2(2t) + 9 \sin^2(2t) \right)^{\frac 3 2}
        ,
    \end{align*}
    In combination, 
    \begin{align*}
       \int_\Gamma f d\ell
       &=
       \int_0^{\pi} f(\gamma(t)) \cdot |\dot\gamma(t)| \;dt
       \\&=
       \int_0^{\pi} 27 \left( \cos^2(2t) + 9\sin^2(2t) \right)^{\frac 3 2} \cdot 2\sqrt{ \cos^2(2t) + 9\sin^2(2t) } \;dt
       \\&=
       54 \int_0^{\pi} \left( \cos^2(2t) + 9 \sin^2(2t) \right)^{2} \;dt
       .
    \end{align*}
    Our life will be easier if we substitute $u = 2t$. We also use a trigonometric identity. Then
    \begin{align*}
        &
        \int_0^{\pi} \left( \cos^2(2t) + 9 \sin^2(2t) \right)^{2} \;dt
        \\&=
        \frac {1} 2 \int_0^{2\pi} \left( \cos^2(u) + 9 \sin^2(u) \right)^{2} \;du
        \\&=
        \frac {1} 2 \int_0^{2\pi} \left( 1 - \sin^2(u) + 9 \sin^2(u) \right)^{2} \;du
        \\&=
        \frac {1} 2 \int_0^{2\pi} \left( 1 + 8 \sin^2(u) \right)^{2} \;du
        \\&=
        \frac {1} 2 \int_0^{2\pi} 1 + 16 \sin^2(u) + 64 \sin^4(u) \;du
        \\&=
        \int_0^{\pi} 1 + 16 \sin^2(u) + 64 \sin^4(u) \;du
        \\&=
        \pi + 16 \int_0^{\pi} \sin^2(u) \;du + 64 \int_0^{\pi} \sin^4(u) \;du
        .
    \end{align*}
    We use the half-angle formula:
    \begin{align*}
        \sin(u)^2 = \frac 1 2 \left( 1 - \cos(2u) \right).
    \end{align*}
    Thus
    \begin{align*}
        \int_0^{\pi} \sin^2(u) \;du
        &=
        \frac 1 2 \int_0^{\pi} 1 - \cos(2u) \;du
        \\&=
        \frac \pi 2 - \int_0^{\pi} \frac 1 2 \cos(2u) \;du
        \\&=
        \frac \pi 2 - \int_0^{\pi} \frac 1 4 (\sin(2u))' \;du
        \\&=
        \frac \pi 2 - \frac 1 4 \left[ (\sin(2u))' \right]_0^\pi
        =
        \frac \pi 2
        .
    \end{align*}
    Similarly, using the half-angle formula twice: 
    \begin{align*}
        \int_0^{\pi} \sin^4(u) \;du
        &=
        \frac 1 4 \int_0^{\pi} \left( 1 - \cos(2u) \right)^2 \;du
        \\&=
        \frac 1 4 \int_0^{\pi} 1 - 2 \cos(2u) + \cos(2u)^2 \;du
        \\&=
        \frac 1 4 \int_0^{\pi} 1 - 2 \cos(2u) + \frac 1 2 \left( 1 - \cos(4u) \right) \;du
        \\&=
        \frac 1 4 \int_0^{\pi} \frac 3 2 - 2 \cos(2u) - \frac 1 2 \cos(4u) \;du
        \\&=
        \frac 1 4 \int_0^{\pi} \frac 3 2 \;du - \frac 1 2 \int_0^{\pi} \cos(2u) \;du - \frac 1 8 \int_0^{\pi} \cos(4u) \;du
        \\&=
        \frac 3 8 \pi
        .
    \end{align*}
    The integrals of $\cos(2u)$ and $\cos(4u)$ from $0$ to $\pi$ vanish, by the the same argument as above. 
    In total,
    \begin{align*}
        \int_\Gamma f d\ell
        =
        54
        \left(
            \pi + 16 \cdot \frac \pi 2 + 64 \cdot \frac 3 8 \pi
        \right)
        =
        54
        \left(
        \pi + 8\pi + 24 \pi
        \right)
        =
        54 \cdot 33 \pi
        =
        1782 \pi
        .
    \end{align*}
\end{solution}








\end{document}
