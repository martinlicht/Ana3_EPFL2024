\documentclass[11pt]{article}
% \def\hidesolutions{}
%%%%%%%%%%% SET MARGINS
\setlength{\textheight}{20cm}
\setlength{\topmargin}{-0.5cm}
\setlength{\oddsidemargin}{+0cm}
\setlength{\textwidth}{16.3cm}
%\setlength{\parskip}{6pt}
\setlength{\parindent}{0pt}

%%%%%%%%%%% PACKAGES
\usepackage{amsmath}
\usepackage{amssymb}
\usepackage{amsfonts}
%\usepackage{a4wide}
\usepackage{graphicx}
\usepackage{color}
\usepackage[normalem]{ulem}
\usepackage{enumitem}
\usepackage{capt-of}
\usepackage{float}
\usepackage{amsmath}
\usepackage{listings}
\definecolor{mygreen}{RGB}{28,172,0} % color values Red, Green, Blue
\definecolor{mylilas}{RGB}{170,55,241}
\usepackage{empheq}
\usepackage[ruled]{algorithm2e}
\usepackage{mathrsfs}
\usepackage{datetime}
\usepackage{subcaption}

% TODO: combine the two package lists and reduce redundancies 
\usepackage{mathtools}
\usepackage{nicefrac}
\usepackage{hyperref}
\usepackage{url}
\usepackage{amsmath,amssymb,amsfonts}
\usepackage{a4wide}
\usepackage{graphicx}
\usepackage{color}
\usepackage[normalem]{ulem}
\usepackage{capt-of}
\usepackage{float}
\usepackage[ruled]{algorithm2e}
\usepackage{amsmath,amssymb,amsfonts}
\usepackage{a4wide}
\usepackage{graphicx}
\usepackage{color}
\usepackage[normalem]{ulem}
\usepackage{capt-of}
\usepackage{float}
\usepackage[ruled]{algorithm2e}
\usepackage{mathrsfs}







\newcommand{\Lc}[2]{{\color{blue} \sout{#1} } \textcolor{red}{#2}}
\newcommand{\La}[1]{\textcolor{red}{#1}}
\newcommand{\lh}{\mathscr{L}_h}
\newcommand{\cl}{\mathscr{L}}
\newcommand{\cf}{\mathscr{F}}
\newcommand{\dx}{dx}
\newcommand{\ltn}{\mathscr{l}^2}
\newcommand{\bbR}{\mathbb{R}}
\newcommand{\Rset}{\mathbb{R}}
\newcommand{\Nset}{\mathbb{N}}
\newcommand{\scL}{\mathcal{L}}
\newcommand{\xx}{\mathbf{x}}
\newcommand{\norm}[1]{\|{#1}\|}
\newcommand{\yy}{\mathbf{y}}
\newcommand{\at}[1]{\big|_{#1}}
\renewcommand{\div}{\mathrm{div}}
\newcommand{\divergence}{\mathrm{div}}
\newcommand{\cp}[1]{\textcolor{blue}{#1}}

\newcommand{\FF}{\texttt{FreeFem++ }}
\newcommand{\FFns}{\texttt{FreeFem++}}
\newcommand{\FFfull}{\texttt{FreeFem++-x11}}
\newcommand{\cmd}[1]{ \medskip \noindent \texttt{#1} \medskip}
\newcommand{\incmd}[1]{\texttt{#1}}
\newcommand{\shrinkitems}{\addtolength{\itemsep}{-0.5\baselineskip}}
\newcommand{\mtt}[1]{\mathtt{#1}}
\newcommand{\ML}{\texttt{Matlab }}

\newcommand{\bb}{\mathbf{b}}
\newcommand{\nn}{\mathbf{n}}
\newcommand{\vecA}{\vec{A}}
\newcommand{\vecB}{\vec{B}}


\newcommand{\mesh}{\mathcal{T}_h}
\newcommand{\refel}{\widehat{K}}
\newcommand{\ver}{\mathbf{a}}
\newcommand{\refver}{\widehat{\mathbf{a}}}
\newcommand{\grad}{\nabla}
\newcommand{\refgrad}{\widehat{\nabla}}
\newcommand{\refu}{\widehat{u}}
\newcommand{\refbasis}{\widehat{\varphi}}
\newcommand{\refxx}{\widehat{\xx}}
\newcommand{\refx}{\widehat{x}}
\newcommand{\refy}{\widehat{y}}
\newcommand{\refrho}{\widehat{\rho}}
\newcommand{\refh}{\widehat{h}}






% For typesetting Python code
\newcommand{\matlab}{{\sc Matlab}\xspace}
\usepackage{listings}
\lstloadlanguages{Python}
\lstloadlanguages{csh}%
\definecolor{MyDarkGreen}{rgb}{0.0,0.4,0.0}
\definecolor{purple}{rgb}{0.58,0,0.82}
\lstset{language=Python,                    % Use Python
	%frame=single,                          % Single frame around code
	basicstyle=\ttfamily\footnotesize\color{black},
	keywordstyle=[1]\color{blue}\bf,        % Python functions bold and blue
	keywordstyle=[2]\color{purple},         % Python function arguments purple
	keywordstyle=[3]\color{red}\underbar,   % User functions underlined and blue
	commentstyle=\usefont{T1}{pcr}{m}{sl}\color{MyDarkGreen}\small,
	stringstyle=\color{purple},
	showstringspaces=false,                 % Don't put marks in string spaces
	tabsize=3,                              % 5 spaces per tab
	morekeywords={xlim,ylim,var,alpha,factorial,poissrnd,normpdf,normcdf},
	morecomment=[l][\color{blue}]{...},
	breaklines=true,
	breakatwhitespace=true,
	emptylines=1,
	mathescape=true,
	xleftmargin=0ex,
	emphstyle=\bfseries\color{red}
}





%%%%%%%%%%% MACROS NAMES
\newcommand{\lecturername}{Martin Licht}
% \newcommand{\assistantnamea}{Jochen Hinz}
% \newcommand{\assistantnameb}{Ivan Bioli}
\newcommand{\semestername}{Winter Semester 2023}
\newcommand{\lecturename}{Analysis III - 202(c)}
\DeclarePairedDelimiter\floor{\lfloor}{\rfloor}

%%%%%%%%%%% HEADER
\newdateformat{yeardate}{\THEYEAR}
\newcommand{\exsheet}[3] % input is the number of the session and the day TODO What's that
{\clearpage

	\begin{center}
		{\Large \textbf{\lecturename}}\\[2ex]
		\semestername
	\end{center}

	% \vspace{2ex}
	% \lecturername

	\vspace{2ex}
	{\Large Session #1: #3\,#2, \yeardate\today}
	%\hfill
	%{\Large EPF Lausanne}

	\hrulefill
}





\usepackage{comment}

\newtheorem{exercise}{Exercise}
\newtheorem{solutionenv}{Solution}

\newboolean{hide_solution}
\ifx\hidesolutions\undefined
\newenvironment{solution}{\begin{solutionenv}}{\end{solutionenv}}
\setboolean{hide_solution}{false}
\else
\excludecomment{solution}
\setboolean{hide_solution}{true}
\fi

\newcommand{\ifnotsolution}[1]{\ifthenelse{\boolean{hide_solution}}{#1}{}}
\newcommand{\ifsolution}[1]{\ifthenelse{\boolean{hide_solution}}{}{#1}}








\allowdisplaybreaks

\begin{document}
\exsheet{7}{31}{October} % parameters are the number of the session and the day




\begin{exercise} % 1
    Consider the volume 
    \begin{align}
     V := \left\{ \vec x \in \bbR^3 \suchthat* x_1^2 + x_2^2 + x_3^2 < 5 \right\},
    \end{align}
    \begin{itemize}
    \item
     What is the surface $S$ of this volume?
    \item 
     Find the outward pointing unit normal $\vec n$ along the surface $S$ of this volume. Write $\vec n$ in terms of $(x_1,x_2,x_3)$ at any point on the surface $S$. 
    \item
     Find a parameterization of the surface $S$.
    \item
     Find a vector field $\vec F$ such that $\vec F \cdot \vec n = x_1^2 + x_2 + x_3$ along the surface $S$.
    \item 
     Use the divergence theorem to compute 
     \begin{align}
        \iint_{S} x_1^2 + x_2 + x_3 dx_1 dx_2 dx_3.
     \end{align}
    \end{itemize}
\end{exercise}
\begin{solution}     
    \begin{itemize}
    \item $S := \left\{ \vec x \in \bbR^3 \suchthat* x_1^2 + x_2^2 + x_3^2 = 5 \right\}$
    \item We can write $G(x_1,x_2,x_3) = -5 + x_1^2 + x_2^2 + x_3^2$. Then since we know that the gradient is normal to the level curves we have that:
    $$
    \vec{n} = \frac{\nabla G}{||\nabla G||} = \frac{1}{\sqrt{5}}\begin{pmatrix}
    x_1\\x_2\\x_3
    \end{pmatrix}
    $$
    \item A possible parameterization is:
     \begin{align*}
        \Phi: [0,2\pi)\times[0,\pi) \mapsto S, (\theta,\phi) \mapsto (\sqrt{5}\sin\phi\cos\theta,\sqrt{5}\sin\phi\sin\theta,\sqrt{5}\cos\phi)
     \end{align*}
    \item $\vec{F} = \sqrt{5}\begin{pmatrix}x_1\\1\\1\end{pmatrix}$
    \item 
     We use the divergence theorem:
     \begin{align*}
        \iint_{S} x_1^2 + x_2 + x_3 dx_1 dx_2 dx_3 & = \oint_{S} \vec{F}\cdot\vec{n} d\sigma \\
		& = \iiint_V \nabla \cdot F dV\\
		& = \iiint_V \sqrt{5} dV = \sqrt{5} \cdot \iiint_V 1 dV = \frac{4}{3}\pi\sqrt{5}^4 =  \frac{4}{3}\pi\cdot 25\\
     \end{align*}
     In the last step, we used the formula for the volume of the ball with radius $r=\sqrt{5}$.
    \end{itemize}
\end{solution}


\begin{exercise} % 2
    Consider the volume 
    \begin{align}
     V := \left\{ \vec x \in \bbR^3 \suchthat* x_1 + x_2 + x_3 < 1, \; x_1, x_2, x_3 > 0 \right\},
    \end{align}
    Suppose we have a vector field 
    \begin{align}
        \vec F(x_1,x_2,x_3) = \left( x_1^2 x_2, 3 x_2^{2} x_3, 9 x_3^{2} x_1 \right)
    \end{align}
    Use the divergence theorem to compute the surface integral 
    \begin{align}
        \iint_{S} \vec F d\sigma.
    \end{align}
\end{exercise}
\begin{solution}     
    Given any scalar function $g : V \to \bbR$, we have 

    \begin{align*}
        \oint_{S} \vec F d\sigma & = \iiint_V \nabla \cdot F dV\\
		&= \iiint_V 2x_1x_2 + 6x_2x_3 + 18x_3x_1 dV\\
		&= \int_{0}^{1}\int_{0}^{1 - x_3}\int_{0}^{1 - x_2 - x_3} 2x_1x_2 + 6x_2x_3 + 18x_3x_1 dx_1 dx_2 dx_3
    \end{align*}

    We solve each integral seperately. Note that we apply integration by parts in between steps. 
    \begin{align*}
        \int_{0}^{1}\int_{0}^{1 - x_3}\int_{0}^{1 - x_2 - x_3} 2x_1x_2 dx_1 dx_2 dx_3
        &= \int_{0}^{1}\int_{0}^{1 - x_3} (1-x_2-x_3)^2 x_2 dx_2 dx_3\\
        &= \int_{0}^{1}\int_{0}^{1 - x_3} \frac{1}{3}(1-x_2-x_3)^3 dx_2 dx_3\\
        &= \int_{0}^{1} \frac{1}{12}(1-x_3)^4 dx_3\\
        &= \left[ \frac{-1}{60}(1-x_3)^5\right]_0^1 = \frac{1}{60}
    \end{align*}
    \begin{align*}
        \int_{0}^{1}\int_{0}^{1 - x_3}\int_{0}^{1 - x_2 - x_3} 6x_2x_3  dx_1 dx_2 dx_3 &= \int_{0}^{1}\int_{0}^{1 - x_3} 6(1 - x_2 - x_3) x_2 x_3   dx_2 dx_3 \\
	&= \int_{0}^{1}\int_{0}^{1 - x_3} 3(1 - x_2 - x_3)^2 x_2 x_3   dx_2 dx_3 \\
	&= \int_{0}^{1} (1 - x_3)^3 x_3   dx_3 \\
	&= \int_{0}^{1} \frac{1}{4}(1 - x_3)^4  dx_3 \\
	&= \left[-\frac{1}{20}(1 - x_3)^5 \right]_0^1 = \frac{1}{20}
    \end{align*}
	\begin{align*}
         \int_{0}^{1}\int_{0}^{1 - x_3}\int_{0}^{1 - x_2 - x_3} 18x_3x_1 dx_1 dx_2 dx_3&= \int_{0}^{1}\int_{0}^{1 - x_3} 9x_3(1-x_2-x_3)^2   dx_2 dx_3 \\
        &=\int_{0}^{1} 3x_3(1-x_3)^3  dx_3 \\
        &= \int_{0}^{1} \frac{3}{4}(1-x_3)^4  dx_3 \\
        &= \left[-\frac{3}{20}(1-x_3)^5  \right]_0^1 =\frac{3}{20}
    \end{align*}
    We conclude that:
    $$
    \int_{0}^{1}\int_{0}^{1 - x_3}\int_{0}^{1 - x_2 - x_3} 2x_1x_2 + 6x_2x_3 + 18x_3x_1 dx_1 dx_2 dx_3 = \frac{1}{60} + \frac{1}{20} + \frac{3}{20} = \frac{13}{60}
    $$
\end{solution}



\begin{exercise} % 3
    Let $S$ be a surface with parameterization $\Phi : \Omega \to S$ as follows:
    \begin{gather}
     S := \left\{ \vec x \in \bbR^3 \suchthat* x_1 + x_2 + x_3 = 1, \; x_1, x_2, x_3 > 0 \right\},
     \\
     \Omega = \left\{ (s,t) \in \bbR^2 \suchthat* s + t < 1, \; s, t > 0 \right\},
     \\ 
     \Phi : \Omega \to S, \quad (s,t) \mapsto (s,t,1-s-t).
    \end{gather}
    We consider the vector fields 
    \begin{align}
        \vec F(x_1,x_2,x_3) := (x_2 x_3, -x_1 x_3, x_1 x_2), \quad \vec G(x_1,x_2,x_3) := (-x_2, x_1,x_3)
    \end{align}
    \begin{itemize}
    \item
     Find the unit normal $\vec n$ given associated with that parameterization.
    \item
     Compute the scalar fields $\vec F \cdot \vec n$ and $\vec G \cdot \vec n$ and calculate the integrals of these scalar fields over $S$.
    \item
     Compute the integrals using the formula 
     \begin{align}
        \int_\Omega \vec F \cdot ( \partial_S \Phi \times \partial_t \Phi ) dsdt
     \end{align}
     and compare the result. 
    \item 
     Find a counterclockwise parameterization of $\partial\Omega$ and use it build a parameterization of the curve $C = \partial S$.
    \item 
     Compute the integrals 
     \begin{align}
        \oint_C \vec F dl, \quad \oint_C \vec G dl.
     \end{align}
    \item 
     Can you exclude that $\vec F$ or $\vec G$ are gradients of a scalar field?
    \end{itemize}
\end{exercise}
\begin{solution}     
    \begin{itemize}
    \item  In a similar fashion as exercise 1 subquestion 2 we obtain the following normal vector:
	\begin{align*}
		\vec{n} = \frac{1}{\sqrt{3}} \begin{pmatrix}1\\1\\1\\\end{pmatrix}
	\end{align*}
    \item
	\begin{align*}
        \int_{\Omega} \vec{F}(\Phi) \cdot  ( \partial_S \Phi \times \partial_t \Phi ) dsdt &=\int_{\Omega} \vec{F}(\Phi) \cdot  ( \partial_S \Phi \times \partial_t \Phi )\frac{||\partial_S \Phi \times \partial_t \Phi ||}{||\partial_S \Phi \times \partial_t \Phi ||} dsdt \\
        &=\int_{\Omega} \vec{F}(\Phi) \cdot \vec{n}||\partial_S \Phi \times \partial_t \Phi || dsdt \\
        &=\int_S \vec{F}\cdot \vec{n} d\sigma
	\end{align*}
    \item
	\begin{align*}
        \partial_s\Phi\times\partial_t\Phi &= \begin{pmatrix}1\\0\\-1\\\end{pmatrix}\times\begin{pmatrix}0\\1\\-1\\\end{pmatrix} = \begin{pmatrix}1\\1\\1\\\end{pmatrix}\\
    \end{align*}
	\begin{align*}
        \int_S \vec{F}(\Phi) \cdot  ( \partial_S \Phi \times \partial_t \Phi ) dsdt &= \int_0^1 \int_0^{1-t} \begin{pmatrix}t(1-s-t)\\-s(1-s-t)\\st\end{pmatrix}\cdot  \begin{pmatrix}1\\1\\1\end{pmatrix} ds dt\\
        &= \int_0^1 \int_0^{1-t} t(1-s-t)-s(1-s-t)+st ds dt\\
	\end{align*}
	\begin{align*}
        \int_0^1 \int_0^{1-t} t(1-s-t) ds dt &= \int_0^1 \frac{1}{2}t -t^2 + \frac{1}{2}t^3 dt = \frac{1}{24}\\
        \int_0^1 \int_0^{1-t} s(1-s-t) ds dt &=\int_0^1 \int_0^{1-t} \frac{1}{2}(1-s-t)^2 ds dt =\int_0^1 \frac{1}{6}(1-t)^3 dt  =  \frac{1}{24}\\
        \int_0^1 \int_0^{1-t} st ds dt &=\int_0^1 \frac{1}{2}(1-t)^2t dt =\int_0^1 \frac{1}{6}(1-t)^3 dt  =  \frac{1}{24}\\
	\end{align*}
	\begin{align*}
    	\int_0^1 \int_0^{1-t} t(1-s-t)-s(1-s-t)+st ds dt = \frac{1}{24} - \frac{1}{24} + \frac{1}{24} =  \frac{1}{24}\\
	\end{align*}
	\begin{align*}
	    \int_S \vec{G}(\Phi) \cdot  ( \partial_S \Phi \times \partial_t \Phi ) dsdt &= \int_0^1 \int_0^{1-t} \begin{pmatrix}-t\\s\\1-s-t\end{pmatrix}\cdot  \begin{pmatrix}1\\1\\1\end{pmatrix} ds dt\\
        &= \int_0^1 \int_0^{1-t} 1-2t ds dt\\
        &= \int_0^1 (1-2t)(1-t) dt\\
        &= \left[-\frac{1}{2}(1-2t)(1-t)\right]_0^1 -\int_0^1 (1-t)^2 dt \\
        &= \frac{1}{2} -\left[-\frac{1}{3}(1-t)^3\right]_0^1 = \frac{1}{2} - \frac{1}{3} = \frac{1}{6}\\
	\end{align*}
    \item The boundary of $\Omega$ consists of three lines which can be parameterized in a counter clockwise way as follows:
	\begin{align*}
	&\gamma_1 : [0,1]\mapsto \partial \Omega_1, (r) \mapsto (r,0),\\
	&\gamma_2 : [1,0]\mapsto \partial \Omega_2, (r) \mapsto (r,1-r),\\
	&\gamma_3 : [1,0]\mapsto \partial \Omega_3, (r) \mapsto (0,r),
	\end{align*}
	which we can use to parameterize the curve $C$ which consist of three lines:
	\begin{align*}
	&\Phi\circ\gamma_1 : [0,1]\mapsto C_1, (r) \mapsto (r,0,1-r),\\
	&\Phi\circ\gamma_2 : [1,0]\mapsto C_2, (r) \mapsto (r,1-r,0),\\
	&\Phi\circ\gamma_3 : [1,0]\mapsto C_3, (r) \mapsto (0,r,1-r),
	\end{align*}
    \item 
	\begin{align*}
	\oint_C \vec F dl &= \oint_{C_1 \cup C_2 \cup C_3} \vec F dl\\
	&= \int_{C_1} \vec F(\Phi\circ \gamma_1) \cdot \frac{d(\Phi\circ \gamma_1)}{dr}dl + \int_{C_2} \vec F(\Phi\circ \gamma_2) \cdot \frac{d(\Phi\circ \gamma_2)}{dr}dl + \int_{C_3} \vec F(\Phi\circ \gamma_3) \cdot \frac{d(\Phi\circ \gamma_3)}{dr}dl\\
	&= \int_0^1 \begin{pmatrix}0\\-r+r^2\\0\end{pmatrix}\cdot\begin{pmatrix}1\\0\\-1\end{pmatrix}dr + \int_1^0 \begin{pmatrix}0\\0\\r-r^2\end{pmatrix}\cdot\begin{pmatrix}1\\-1\\0\end{pmatrix}dr + \int_1^0 \begin{pmatrix}r-r^2\\0\\0\end{pmatrix}\cdot\begin{pmatrix}0\\1\\-1\end{pmatrix}dr =\\
	&=0
	\end{align*}
	\begin{align*}
	\oint_C \vec G dl &= \oint_{C_1 \cup C_2 \cup C_3} \vec G dl\\
	&= \int_{C_1} \vec G(\Phi\circ \gamma_1) \cdot \frac{d(\Phi\circ \gamma_1)}{dr}dl + \int_{C_2} \vec G\Phi\circ \gamma_2) \cdot \frac{d(\Phi\circ \gamma_2)}{dr}dl + \int_{C_3} \vec G(\Phi\circ \gamma_3) \cdot \frac{d(\Phi\circ \gamma_3)}{dr}dl\\
	&= \int_0^1 \begin{pmatrix}0\\r\\1-r\end{pmatrix}\cdot\begin{pmatrix}1\\0\\-1\end{pmatrix}dr + \int_1^0 \begin{pmatrix}r-1\\r\\0\end{pmatrix}\cdot\begin{pmatrix}1\\-1\\0\end{pmatrix}dr + \int_1^0 \begin{pmatrix}-r\\0\\1-r\end{pmatrix}\cdot\begin{pmatrix}0\\1\\-1\end{pmatrix}dr =\\
	&= \int_0^1 r-1dr + \int_1^0 rdr + \int_1^0 r-1dr\\
	&= \int_1^0 rdr = -\frac{1}{2}
	\end{align*}
    \item 
     Both $\vec F$ and $\vec G$ have non-zero curl, so they cannot be gradients of a scalar field. 
    \end{itemize}
\end{solution}

\begin{exercise} % 4
    Let $S$ be the surface with parameterization $\Phi : \Omega \to S$ as follows:
    \begin{align}
     S := \left\{ \vec x \in \bbR^3 \suchthat* x_1^2 + x_2^2 = 1, \; -1 < x_3 < 1 \right\},
    \end{align}
    Suppose we have a vector field 
    \begin{align}
        \vec F(x_1,x_2,x_3) := (x_2 x_3, -x_1 x_3, 0 ).
    \end{align}
    \begin{itemize}
    \item
     Find a parameterization of the unit normal $\vec n$ given associated with that parameterization.
    \item
     Compute the scalar field $\vec F \cdot \vec n$ and calculate the integrals of this scalar fields over $S$.
    \item
     Compute the integral using the formula 
     \begin{align}
        \int_\Omega \vec F \cdot ( \partial_S \Phi \times \partial_t \Phi ) dsdt
     \end{align}
     and compare the result. 
    \item 
     Find a counterclockwise parameterization of $\partial\Omega$ and use it build a parameterization of the curve $C = \partial S$.
    \item 
     Compute the integrals 
     \begin{align}
        \oint_C \vec F dl.
     \end{align}

    \end{itemize}
\end{exercise}
\begin{solution}     
    \begin{itemize}
    \item
    In a similar fashion as Exercise 1 Subquestion 2 we obtain the following normal vector:
		$$\vec{n} = \begin{pmatrix} x_1\\x_2\\0 \end{pmatrix}$$
    \item 
	\begin{align*}
	\vec F \cdot \vec n & = \begin{pmatrix}x_2x_3 \\ -x_1x_3\\0\end{pmatrix}\cdot \begin{pmatrix}x_1 \\ x_2\\0\end{pmatrix} = x_1x_2x_3 - x_1x_2x_3 = 0
	\end{align*}
	\begin{align*}
        \int_S \vec F \cdot \vec{n}d\sigma = \int_S 0 d\sigma = 0
     \end{align*}
    \item
    We define the parameterisation: 
	$$
	\Phi: [0,2\pi) \times [-1,1] \mapsto S, (\theta ,z) \mapsto (\cos\theta, \sin\theta ,z)
	$$
	\begin{align*}
        \int_\Omega \vec F(\Phi) \cdot ( \partial_S \Phi \times \partial_t \Phi ) dsdt &=\int_0^{2\pi} \int_{-1}^1 \begin{pmatrix}z\sin\theta\\ -z\cos\theta \\ 0\end{pmatrix} \cdot \begin{pmatrix}\cos\theta\\ \sin\theta \\ 0\end{pmatrix} dsdt \\
	    & = \int_0^{2\pi} \int_{-1}^1 0  dsdt  = 0\\
     \end{align*}
    \item 
    A counter clockwise parameterisation of $\partial \Omega$ is given by:
	\begin{align*}
	&\gamma_1: [0,2\pi) \mapsto \partial \Omega_1, (r) \mapsto (r, 1)\\
	&\gamma_2: [1,-1) \mapsto \partial \Omega_2, (r) \mapsto (2\pi, r)\\
	&\gamma_3: [2\pi,0) \mapsto \partial \Omega_3, (r) \mapsto (r, -1)\\
	&\gamma_4: [-1,1) \mapsto \partial \Omega_4, (r) \mapsto (0, r)\\
	\end{align*}
	which we can use to build a parameterization of the curve $C = \partial S$:
	\begin{align*}
	&\Phi \circ \gamma_1: [0,2\pi) \mapsto C_1, (r) \mapsto (\cos r, \sin r, 1)\\
	&\Phi \circ \gamma_2: [1,-1)  \mapsto C_2, (r) \mapsto (1, 0, r)\\
	&\Phi \circ \gamma_3: [2\pi,0) \mapsto C_3, (r) \mapsto (\cos r, \sin r, -1)\\
	&\Phi \circ \gamma_4: [-1,1) \mapsto C_4, (r) \mapsto (1, 0, r)\\
	\end{align*}
    \item 
	\begin{align*}
	\oint_C \vec F dl &= \oint_{C_1 \cup C_2 \cup C_3\cup C_4} \vec F dl\\
	&= \sum_{i = 1}^4\int_{C_i} \vec F(\Phi\circ \gamma_i) \cdot \frac{d(\Phi\circ \gamma_i)}{dr}dl \\
	&= \int_0^{2\pi} \begin{pmatrix}\sin r\\-\cos r\\0\end{pmatrix}\cdot\begin{pmatrix}-\sin r\\\cos r\\ 0\end{pmatrix}dr + \int_{1}^{-1} \begin{pmatrix}0\\-r\\0\end{pmatrix}\cdot\begin{pmatrix}0\\0\\1\end{pmatrix}dr + \int_{2\pi}^0 \begin{pmatrix}-\sin r\\ \cos r\\0\end{pmatrix}\cdot\begin{pmatrix}-\sin r\\\cos r\\ 0\end{pmatrix}dr +\\
	&+ \int_{-1}^{1} \begin{pmatrix}0\\-r\\0\end{pmatrix}\cdot\begin{pmatrix}0\\0\\-1\end{pmatrix}dr\\
	&=2\int_0^{2\pi} \begin{pmatrix}\sin r\\-\cos r\\0\end{pmatrix}\cdot\begin{pmatrix}-\sin r\\\cos r\\ 0\end{pmatrix}dr \\
	&=-2 \int_0^{2\pi} dr = -4\pi
	\end{align*}
    \item 
     $\vec F$ has  non-zero curl, so it cannot be a gradient of a scalar field. 
    \end{itemize}
\end{solution}

\begin{exercise} % 5
    Consider the surface 
    \begin{align*}
        S := \left\{ \vec x \in \bbR^{3} \suchthat* \|x\| = 1, x_3 > 0 \right\}
    \end{align*}
    \begin{itemize}
    \item Find a parameterization of $S$ and find the unit normal corresponding to that parameterization.
    \item Find a parameterization of $C = \partial S$.
    \item Verify Stokes theorem with the vector field 
     \begin{align*}
        \vec F(x_1,x_2,x_3) = ( -x_2, x_1, x_3 ).
     \end{align*}   
    \end{itemize}
\end{exercise}
\begin{solution}     
    \begin{itemize}
    \item $$\Phi: \Omega:= [0,2\pi)\times[0,\frac{\pi}{2}) \mapsto S: (\theta,\phi)\mapsto (\cos\theta\sin\phi,\sin\theta\sin\phi,\cos\phi) $$
			$$\vec{n} = \frac{\partial_{\theta}\Phi\times \partial_{\phi}\Phi}{||\partial_{\theta}\Phi\times \partial_{\phi}\Phi||} = \begin{pmatrix}\cos\theta\sin\phi \\ \sin\theta\sin\phi\\ \cos\phi\end{pmatrix}$$

    \item We first parameterize the boundary of the parameter space $\Omega$ 

$$\gamma:[0,2\pi) \mapsto \partial \Omega : \theta \mapsto(\theta, \frac{\pi}{2}),$$
			
which gives us the parameterization of the boundary as follows:

$$\Phi\circ \gamma:[0,2\pi) \mapsto \partial C : \theta \mapsto(\cos\theta,\sin\theta,0).$$

    \item 
        We have to show that:

$$
\iint_S \nabla \times \vec{F} d\sigma = \oint_C \vec{F} d\ell
$$

     \begin{align*}
\iint_S \nabla \times \vec{F} d\sigma &= \iint_S \nabla \times \vec{F}\cdot \vec{n}||\partial_{\theta}\Phi\times \partial_{\phi}\Phi|| d\sigma\\
 &= \iint_S \begin{pmatrix} 0\\ 0\\ 2 \end{pmatrix}\cdot \begin{pmatrix}\cos\theta\sin\phi \\ \sin\theta\sin\phi\\ \cos\phi\end{pmatrix}\sin\phi d\sigma\\
&= \int_0^{2\pi}\int_0^{\frac{\pi}{2}} 2\cos\phi\sin\phi d\phi d\theta\\
&= \int_0^{2\pi}\int_0^{\frac{\pi}{2}} \sin2\phi d\phi d\theta\\
&= 2\pi \left[ -\frac{1}{2}\cos2\phi \right]_0^{\frac{\pi}{2}} = 2\pi\\
     \end{align*}   

\begin{align*}
\oint_C \vec{F} d\ell &= \int_0^{2\pi} \vec{F}(\Phi\circ \gamma)\cdot \frac{d(\Phi\circ \gamma)}{d\theta}  d \theta\\
 &= \int_0^{2\pi} \begin{pmatrix} -\sin\theta\\ \cos\theta\\ 0 \end{pmatrix}\cdot \begin{pmatrix}-\sin\theta \\ \cos\phi \\ 0\end{pmatrix} d \theta\\
&= \int_0^{2\pi} 1 d \theta = 2\pi\\
     \end{align*}   
    \end{itemize}
\end{solution}



\end{document}
