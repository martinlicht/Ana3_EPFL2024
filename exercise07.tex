\documentclass[11pt]{article}
\def\hidesolutions{}
%%%%%%%%%%% SET MARGINS
\setlength{\textheight}{20cm}
\setlength{\topmargin}{-0.5cm}
\setlength{\oddsidemargin}{+0cm}
\setlength{\textwidth}{16.3cm}
%\setlength{\parskip}{6pt}
\setlength{\parindent}{0pt}

%%%%%%%%%%% PACKAGES
\usepackage{amsmath}
\usepackage{amssymb}
\usepackage{amsfonts}
%\usepackage{a4wide}
\usepackage{graphicx}
\usepackage{color}
\usepackage[normalem]{ulem}
\usepackage{enumitem}
\usepackage{capt-of}
\usepackage{float}
\usepackage{amsmath}
\usepackage{listings}
\definecolor{mygreen}{RGB}{28,172,0} % color values Red, Green, Blue
\definecolor{mylilas}{RGB}{170,55,241}
\usepackage{empheq}
\usepackage[ruled]{algorithm2e}
\usepackage{mathrsfs}
\usepackage{datetime}
\usepackage{subcaption}

% TODO: combine the two package lists and reduce redundancies 
\usepackage{mathtools}
\usepackage{nicefrac}
\usepackage{hyperref}
\usepackage{url}
\usepackage{amsmath,amssymb,amsfonts}
\usepackage{a4wide}
\usepackage{graphicx}
\usepackage{color}
\usepackage[normalem]{ulem}
\usepackage{capt-of}
\usepackage{float}
\usepackage[ruled]{algorithm2e}
\usepackage{amsmath,amssymb,amsfonts}
\usepackage{a4wide}
\usepackage{graphicx}
\usepackage{color}
\usepackage[normalem]{ulem}
\usepackage{capt-of}
\usepackage{float}
\usepackage[ruled]{algorithm2e}
\usepackage{mathrsfs}







\newcommand{\Lc}[2]{{\color{blue} \sout{#1} } \textcolor{red}{#2}}
\newcommand{\La}[1]{\textcolor{red}{#1}}
\newcommand{\lh}{\mathscr{L}_h}
\newcommand{\cl}{\mathscr{L}}
\newcommand{\cf}{\mathscr{F}}
\newcommand{\dx}{dx}
\newcommand{\ltn}{\mathscr{l}^2}
\newcommand{\bbR}{\mathbb{R}}
\newcommand{\Rset}{\mathbb{R}}
\newcommand{\Nset}{\mathbb{N}}
\newcommand{\scL}{\mathcal{L}}
\newcommand{\xx}{\mathbf{x}}
\newcommand{\norm}[1]{\|{#1}\|}
\newcommand{\yy}{\mathbf{y}}
\newcommand{\at}[1]{\big|_{#1}}
\renewcommand{\div}{\mathrm{div}}
\newcommand{\divergence}{\mathrm{div}}
\newcommand{\cp}[1]{\textcolor{blue}{#1}}

\newcommand{\FF}{\texttt{FreeFem++ }}
\newcommand{\FFns}{\texttt{FreeFem++}}
\newcommand{\FFfull}{\texttt{FreeFem++-x11}}
\newcommand{\cmd}[1]{ \medskip \noindent \texttt{#1} \medskip}
\newcommand{\incmd}[1]{\texttt{#1}}
\newcommand{\shrinkitems}{\addtolength{\itemsep}{-0.5\baselineskip}}
\newcommand{\mtt}[1]{\mathtt{#1}}
\newcommand{\ML}{\texttt{Matlab }}

\newcommand{\bb}{\mathbf{b}}
\newcommand{\nn}{\mathbf{n}}
\newcommand{\vecA}{\vec{A}}
\newcommand{\vecB}{\vec{B}}


\newcommand{\mesh}{\mathcal{T}_h}
\newcommand{\refel}{\widehat{K}}
\newcommand{\ver}{\mathbf{a}}
\newcommand{\refver}{\widehat{\mathbf{a}}}
\newcommand{\grad}{\nabla}
\newcommand{\refgrad}{\widehat{\nabla}}
\newcommand{\refu}{\widehat{u}}
\newcommand{\refbasis}{\widehat{\varphi}}
\newcommand{\refxx}{\widehat{\xx}}
\newcommand{\refx}{\widehat{x}}
\newcommand{\refy}{\widehat{y}}
\newcommand{\refrho}{\widehat{\rho}}
\newcommand{\refh}{\widehat{h}}






% For typesetting Python code
\newcommand{\matlab}{{\sc Matlab}\xspace}
\usepackage{listings}
\lstloadlanguages{Python}
\lstloadlanguages{csh}%
\definecolor{MyDarkGreen}{rgb}{0.0,0.4,0.0}
\definecolor{purple}{rgb}{0.58,0,0.82}
\lstset{language=Python,                    % Use Python
	%frame=single,                          % Single frame around code
	basicstyle=\ttfamily\footnotesize\color{black},
	keywordstyle=[1]\color{blue}\bf,        % Python functions bold and blue
	keywordstyle=[2]\color{purple},         % Python function arguments purple
	keywordstyle=[3]\color{red}\underbar,   % User functions underlined and blue
	commentstyle=\usefont{T1}{pcr}{m}{sl}\color{MyDarkGreen}\small,
	stringstyle=\color{purple},
	showstringspaces=false,                 % Don't put marks in string spaces
	tabsize=3,                              % 5 spaces per tab
	morekeywords={xlim,ylim,var,alpha,factorial,poissrnd,normpdf,normcdf},
	morecomment=[l][\color{blue}]{...},
	breaklines=true,
	breakatwhitespace=true,
	emptylines=1,
	mathescape=true,
	xleftmargin=0ex,
	emphstyle=\bfseries\color{red}
}





%%%%%%%%%%% MACROS NAMES
\newcommand{\lecturername}{Martin Licht}
% \newcommand{\assistantnamea}{Jochen Hinz}
% \newcommand{\assistantnameb}{Ivan Bioli}
\newcommand{\semestername}{Winter Semester 2023}
\newcommand{\lecturename}{Analysis III - 202(c)}
\DeclarePairedDelimiter\floor{\lfloor}{\rfloor}

%%%%%%%%%%% HEADER
\newdateformat{yeardate}{\THEYEAR}
\newcommand{\exsheet}[3] % input is the number of the session and the day TODO What's that
{\clearpage

	\begin{center}
		{\Large \textbf{\lecturename}}\\[2ex]
		\semestername
	\end{center}

	% \vspace{2ex}
	% \lecturername

	\vspace{2ex}
	{\Large Session #1: #3\,#2, \yeardate\today}
	%\hfill
	%{\Large EPF Lausanne}

	\hrulefill
}





\usepackage{comment}

\newtheorem{exercise}{Exercise}
\newtheorem{solutionenv}{Solution}

\newboolean{hide_solution}
\ifx\hidesolutions\undefined
\newenvironment{solution}{\begin{solutionenv}}{\end{solutionenv}}
\setboolean{hide_solution}{false}
\else
\excludecomment{solution}
\setboolean{hide_solution}{true}
\fi

\newcommand{\ifnotsolution}[1]{\ifthenelse{\boolean{hide_solution}}{#1}{}}
\newcommand{\ifsolution}[1]{\ifthenelse{\boolean{hide_solution}}{}{#1}}








\allowdisplaybreaks

\begin{document}
\exsheet{7}{31}{October} % parameters are the number of the session and the day




\begin{exercise} % 1
    Consider the volume 
    \begin{align}
     V := \left\{ \vec x \in \bbR^3 \suchthat* x_1^2 + x_2^2 + x_3^2 < 5 \right\},
    \end{align}
    \begin{itemize}
     \item
     What is the surface $S$ of this volume?
     \item 
     Find the outward pointing unit normal $\vec n$ along the surface $S$ of this volume. Write $\vec n$ in terms of $(x_1,x_2,x_3)$ at any point on the surface $S$. 
     \item
     Find a parameterization of the surface $S$.
     \item
     Find a vector field $\vec F$ such that $\vec F \cdot \vec n = x_1^2 + x_2 + x_3$ along the surface $S$.
     \item 
     Use the divergence theorem to compute 
     \begin{align}
        \iint_{S} x_1^2 + x_2 + x_3 dx_1 dx_2 dx_3.
     \end{align}
    \end{itemize}
\end{exercise}
\begin{solution}     
    \begin{itemize}
     \item
     \item 
     \item
     \item
     \item 
    \end{itemize}
\end{solution}


\begin{exercise} % 2
    Consider the volume 
    \begin{align}
     V := \left\{ \vec x \in \bbR^3 \suchthat* x_1 + x_2 + x_3 < 1, \; x_1, x_2, x_3 > 0 \right\},
    \end{align}
    Suppose we have a vector field 
    \begin{align}
        \vec F(x_1,x_2,x_3) = \left( x_1^2 x_2, 3 x_2^{2} x_3, 9 x_3^{2} x_1 \right)
    \end{align}
    Use Gauss theorem to compute the surface integral 
    \begin{align}
        \iint_{S} \vec F d\sigma.
    \end{align}
\end{exercise}
\begin{solution}     
    Given any scalar function $g : V \to \bbR$, we have 
    \begin{align}
     \iiint_V g(x_1,x_2,x_3) dx_1dx_2dx_3
     =
     \int_{0}^{1}
     \int_{0}^{1 - x_2}
     \int_{0}^{1 - x_2 - x_3}
     ...
     g(x_1,x_2,x_3) dx_1dx_2dx_3
     .
    \end{align}
\end{solution}



\begin{exercise} % 3
    Let $S$ be a surface with parameterization $\Phi : \Omega \to S$ as follows:
    \begin{align}
     S := \left\{ \vec x \in \bbR^3 \suchthat* x_1 + x_2 + x_3 < 1, \; x_1, x_2, x_3 > 0 \right\},
     \\
     \Omega = \left\{ (s,t) \in \bbR^3 \suchthat* s + t < 1, \; s, t > 0 \right\},
     \quad 
     \Phi : \Omega \to S, \quad (s,t) \mapsto (s,t,1-s-t).
    \end{align}
    We consider the vector fields 
    \begin{align}
        \vec F(x_1,x_2,x_3) := (x_2 x_3, -x_1 x_3, x_1 x_2), \quad \vec G(x_1,x_2,x_3) := (-x_2, x_1,x_3)
    \end{align}
    \begin{itemize}
     \item
     Find the unit normal $\vec n$ given associated with that parameterization.
     \item
     Compute the scalar fields $\vec F \cdot \vec n$ and $\vec G \cdot \vec n$ and calculate the integrals of these scalar fields over $S$.
     \item
     Compute the integrals using the formula 
     \begin{align}
        \int_\Omega \vec F \cdot ( \partial_S \Phi \times \partial_t \Phi ) dsdt
     \end{align}
     and compare the result. 
     \item 
     Find a counterclockwise parameterization of $\partial\Omega$ and use it build a parameterization of the curve $C = \partial S$.
     \item 
     Compute the integrals 
     \begin{align}
        \oint_C \vec F dl, \quad \oint_C \vec G dl.
     \end{align}
     \item 
     Can you exclude that $\vec F$ or $\vec G$ are gradients of a scalar field?
    \end{itemize}
\end{exercise}
\begin{solution}     
    \begin{itemize}
     \item
     \item
     \item
     \item 
     \item 
     \item 
     Both $\vec F$ and $\vec G$ have non-zero curl, so they cannot be gradients of a scalar field. 
    \end{itemize}
\end{solution}

\begin{exercise} % 4
    Let $S$ be the surface with parameterization $\Phi : \Omega \to S$ as follows:
    \begin{align}
     S := \left\{ \vec x \in \bbR^3 \suchthat* x_1^2 + x_2^2 = 1, \; -1 < x_3 < 1 \right\},
    \end{align}
    Suppose we have a vector field 
    \begin{align}
        \vec F(x_1,x_2,x_3) := (x_2 x_3, -x_1 x_3, 0 ).
    \end{align}
    \begin{itemize}
     \item
     Find a parameterization of the unit normal $\vec n$ given associated with that parameterization.
     \item
     Compute the scalar field $\vec F \cdot \vec n$ and calculate the integrals of this scalar fields over $S$.
     \item
     Compute the integral using the formula 
     \begin{align}
        \int_\Omega \vec F \cdot ( \partial_S \Phi \times \partial_t \Phi ) dsdt
     \end{align}
     and compare the result. 
     \item 
     Find a counterclockwise parameterization of $\partial\Omega$ and use it build a parameterization of the curve $C = \partial S$.
     \item 
     Compute the integrals 
     \begin{align}
        \oint_C \vec F dl.
     \end{align}
    \end{itemize}
\end{exercise}
\begin{solution}     
    \begin{itemize}
     \item
     \item
     \item
     \item 
     \item 
     \item 
     Both $\vec F$ and $\vec G$ have non-zero curl, so they cannot be gradients of a scalar field. 
    \end{itemize}
\end{solution}

\begin{exercise} % 5
    Consider the surface 
    \begin{align*}
        S := \left\{ \vec x \in \bbR^{3} \suchthat* \|x\| = 1, x_3 > 0 \right\}
    \end{align*}
    \begin{itemize}
     \item Find a parameterization of $S$ and find the unit normal corresponding to that parameterization.
     \item Find a parameterization of $C = \partial S$.
     \item Verify Stokes theorem with the vector field 
     \begin{align*}
        \vec F(x_1,x_2,x_3) = ( -x_2, x_1, x_3 ).
     \end{align*}   
    \end{itemize}
\end{exercise}
\begin{solution}     
    \begin{itemize}
     \item Find a parameterization of $S$ and find the unit normal corresponding to that parameterization.
     \item Find a parameterization of $C = \partial S$.
     \item Verify Stokes theorem with the vector field 
     \begin{align*}
        \vec F(x_1,x_2,x_3) = ( -x_2, x_1, x_3 ).
     \end{align*}   
    \end{itemize}
\end{solution}



\end{document}
